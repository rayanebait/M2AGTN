\documentclass[a4paper,12pt]{book}
\usepackage{amsmath,  amsthm,enumerate}
\usepackage{csquotes}
\usepackage[provide=*,french]{babel}
\usepackage[dvipsnames]{xcolor}
\usepackage{quiver, tikz}

%symbole caligraphique
\usepackage{mathrsfs}

%hyperliens
\usepackage{hyperref}

%pseudo-code
\usepackage{algorithm}
\usepackage{algpseudocode}

\usepackage{fancyhdr}

\pagestyle{fancy}
\addtolength{\headwidth}{\marginparsep}
\addtolength{\headwidth}{\marginparwidth}
\renewcommand{\chaptermark}[1]{\markboth{#1}{}}
\renewcommand{\sectionmark}[1]{\markright{\thesection\ #1}}
\fancyhf{}
\fancyfoot[C]{\thepage}
\fancyhead[LO]{\textit \leftmark}
\fancyhead[RE]{\textit \rightmark}
\renewcommand{\headrulewidth}{0pt} % and the line
\fancypagestyle{plain}{%
    \fancyhead{} % get rid of headers
}

%bibliographie
\usepackage[
backend=biber,
style=alphabetic,
sorting=ynt
]{biblatex}

\addbibresource{bib.bib}

\usepackage{appendix}
\renewcommand{\appendixpagename}{Annexe}

\definecolor{wgrey}{RGB}{148, 38, 55}

\setlength\parindent{24pt}

\newcommand{\Z}{\mathbb{Z}}
\newcommand{\R}{\mathbb{R}}
\newcommand{\rel}{\omathcal{R}}
\newcommand{\Q}{\mathbb{Q}}
\newcommand{\C}{\mathbb{C}}
\newcommand{\N}{\mathbb{N}}
\newcommand{\K}{\mathbb{K}}
\newcommand{\A}{\mathbb{A}}
\newcommand{\B}{\mathcal{B}}
\newcommand{\Or}{\mathcal{O}}
\newcommand{\F}{\mathscr F}
\newcommand{\Hom}{\textrm{Hom}}
\newcommand{\disc}{\textrm{disc}}
\newcommand{\Pic}{\textrm{Pic}}
\newcommand{\End}{\textrm{End}}
\newcommand{\Spec}{\textrm{Spec}}
\newcommand{\Supp}{\textrm{Supp}}
\renewcommand{\Im}{\textrm{Im}}
\newcommand{\m}{\mathfrak{m}}
\renewcommand{\P}{\mathbb{P}}
\newcommand{\p}{\mathfrak{p}}


\newcommand{\cL}{\mathscr{L}}
\newcommand{\G}{\mathscr{G}}
\newcommand{\D}{\mathscr{D}}
\newcommand{\E}{\mathscr{E}}
\newcommand{\Po}{\mathscr{P}}
\renewcommand{\H}{\mathscr{H}}

\makeatletter
\newcommand{\colim@}[2]{%
  \vtop{\m@th\ialign{##\cr
    \hfil$#1\operator@font colim$\hfil\cr
    \noalign{\nointerlineskip\kern1.5\ex@}#2\cr
    \noalign{\nointerlineskip\kern-\ex@}\cr}}%
}
\newcommand{\colim}{%
  \mathop{\mathpalette\colim@{\rightarrowfill@\scriptscriptstyle}}\nmlimits@
}
\renewcommand{\varprojlim}{%
  \mathop{\mathpalette\varlim@{\leftarrowfill@\scriptscriptstyle}}\nmlimits@
}
\renewcommand{\varinjlim}{%
  \mathop{\mathpalette\varlim@{\rightarrowfill@\scriptscriptstyle}}\nmlimits@
}
\makeatother

\theoremstyle{plain}
\newtheorem{thm}[subsection]{Théoreme}
\newtheorem{lem}[subsection]{Lemme}
\newtheorem{prop}[subsection]{Proposition}
\newtheorem{cor}[subsection]{Corollaire}
\newtheorem{heur}{Heuristique}
\newtheorem{rem}{Remarque}
\newtheorem{note}{Note}

\theoremstyle{definition}
\newtheorem{conj}{Conjecture}
\newtheorem{prob}{Problème}
\newtheorem{quest}{Question}
\newtheorem{prot}{Protocole}
\newtheorem{algo}{Algorithme}
\newtheorem{defn}[subsection]{Définition}
\newtheorem{exmp}[subsection]{Exemples}
\newtheorem{exo}[subsection]{Exercices}
\newtheorem{ex}[subsection]{Exemple}
\newtheorem{exs}[subsection]{Exemples}
\newtheorem{res}{Résumé}
\newtheorem{rep}{Réponse}
\newtheorem{cons}{Conséquence}

\theoremstyle{remark}

\definecolor{wgrey}{RGB}{148, 38, 55}
\definecolor{wgreen}{RGB}{100, 200,0} 
\hypersetup{
    colorlinks=true,
    linkcolor=wgreen,
    urlcolor=wgrey,
    filecolor=wgrey
}

\title{Géométrie algébrique}
\date{}

\begin{document}
\maketitle
\tableofcontents



Idées des développements : Cadre $\to$ comment se ramener à ce cadre.
Aussi comparer le cas quasi-projectif au cas général.

\chapter{Exemples de variétés}
\section{Quotients}
Étudier des morphismes $\A^n(k)\to \A^n(k)/G$. Ils sont finis par 
le cours, le revoir. Aussi $X/G$.

\section{Fibrés et variétés projectives}
Systèmes linéaires. Variétés ayant des fibrés amples. Diviseur
canonique ample. Riemann-Roch.

\section{Variétés rationnelles}
C'est des variétés birationnelles à $\P^n(k)$. On peut en dire 
beaucoup de choses.
\subsection*{Le cas de $\P^1(k)$}
Être birationnel à $\P^1_k$ ça force à être de dimension $1$ et 
irréductible.


\section{Variété de drapeaux}
\section{Grassmaniennes}
\section{Variétés abéliennes}
Ça c'est bcp plus dur, mais on sait qu'elles sont projectives.
\section{Groupes algébriques}

\chapter{Variétés}
Ici le but c'est de décrire les propriétés des variétés elles-même,
une bonne ambition ce serait de décrire aussi comment sont préservées
ces propriétés.
\section{Variétés normales et morphismes finis}
Ici c'est 
\[f\colon X\to Y\]
\chapter{Morphismes de variétés}


\chapter{Catégories de variétés}
Le but de cette section c'est de se déplacer dans des catégories 
de variétés pour voir ce qu'y se passe. 

Les marches aléatoires ça pourrait mdr, dans l'ensemble des coniques
sur $\Q$ par exemple mdr. Ou sur un corps fini, y'a des probabilités
claires là!
\section{Tentative de marches pas aléatoires dans \textbf{k-Var}}
Une marche aléatoire sur un corps algébriquement clos ca semble assez
difficile à définir mdr. Par contre y'a des variétés qui sont cooles
à atteindre. Typiquement les
\[\A^n_k,\P^n_k.\]
\subsection*{Diviseurs et fonctions régulières}
Si on se met dans \textbf{k-Var}, la catégorie des variétés,
on peut toujours trouver 
\[X\to \P^1\]
dominante (sauf si $dim(X)=0$) via l'exemple canonique 
$f\in \Or_X(U)\ne k$ alors 
\[\bar f\colon X\to \P^1\]
par contre 
\[X\to \A^1\]
dominante c'est pas toujours possible vu que $X$ peut-être propre par
exemple. 
Bon ce qu'on peut faire dans le cas quasi-projectif


\section{Variétés rationnelles}







\printbibliography
\end{document}


\documentclass[a4paper,12pt]{book}
\usepackage{amsmath,  amsthm,enumerate}
\usepackage{csquotes}
\usepackage[provide=*,french]{babel}
\usepackage[dvipsnames]{xcolor}
\usepackage{quiver, tikz}

%symbole caligraphique
\usepackage{mathrsfs}

%hyperliens
\usepackage{hyperref}

%pseudo-code
\usepackage{algorithm}
\usepackage{algpseudocode}

\usepackage{fancyhdr}

\pagestyle{fancy}
\addtolength{\headwidth}{\marginparsep}
\addtolength{\headwidth}{\marginparwidth}
\renewcommand{\chaptermark}[1]{\markboth{#1}{}}
\renewcommand{\sectionmark}[1]{\markright{\thesection\ #1}}
\fancyhf{}
\fancyfoot[C]{\thepage}
\fancyhead[LO]{\textit \leftmark}
\fancyhead[RE]{\textit \rightmark}
\renewcommand{\headrulewidth}{0pt} % and the line
\fancypagestyle{plain}{%
    \fancyhead{} % get rid of headers
}

%bibliographie
\usepackage[
backend=biber,
style=alphabetic,
sorting=ynt
]{biblatex}

\addbibresource{bib.bib}

\usepackage{appendix}
\renewcommand{\appendixpagename}{Annexe}

\definecolor{wgrey}{RGB}{148, 38, 55}

\setlength\parindent{24pt}

\newcommand{\Z}{\mathbb{Z}}
\newcommand{\R}{\mathbb{R}}
\newcommand{\rel}{\omathcal{R}}
\newcommand{\Q}{\mathbb{Q}}
\newcommand{\C}{\mathbb{C}}
\newcommand{\N}{\mathbb{N}}
\newcommand{\K}{\mathbb{K}}
\newcommand{\A}{\mathbb{A}}
\newcommand{\B}{\mathcal{B}}
\newcommand{\Or}{\mathcal{O}}
\newcommand{\F}{\mathscr F}
\newcommand{\Hom}{\textrm{Hom}}
\newcommand{\disc}{\textrm{disc}}
\newcommand{\Pic}{\textrm{Pic}}
\newcommand{\End}{\textrm{End}}
\newcommand{\Spec}{\textrm{Spec}}
\newcommand{\Supp}{\textrm{Supp}}
\renewcommand{\Im}{\textrm{Im}}


\newcommand{\m}{\mathfrak{m}}
\newcommand{\p}{\mathfrak{p}}


\newcommand{\cL}{\mathscr{L}}
\newcommand{\G}{\mathscr{G}}
\newcommand{\D}{\mathscr{D}}
\newcommand{\E}{\mathscr{E}}
\renewcommand{\Pr}{\mathbb{P}}
\renewcommand{\P}{\mathscr{P}}
\renewcommand{\H}{\mathscr{H}}

\makeatletter
\newcommand{\colim@}[2]{%
  \vtop{\m@th\ialign{##\cr
    \hfil$#1\operator@font colim$\hfil\cr
    \noalign{\nointerlineskip\kern1.5\ex@}#2\cr
    \noalign{\nointerlineskip\kern-\ex@}\cr}}%
}
\newcommand{\colim}{%
  \mathop{\mathpalette\colim@{\rightarrowfill@\scriptscriptstyle}}\nmlimits@
}
\renewcommand{\varprojlim}{%
  \mathop{\mathpalette\varlim@{\leftarrowfill@\scriptscriptstyle}}\nmlimits@
}
\renewcommand{\varinjlim}{%
  \mathop{\mathpalette\varlim@{\rightarrowfill@\scriptscriptstyle}}\nmlimits@
}
\makeatother

\theoremstyle{plain}
\newtheorem{thm}[subsection]{Théoreme}
\newtheorem{lem}[subsection]{Lemme}
\newtheorem{prop}[subsection]{Proposition}
\newtheorem{cor}[subsection]{Corollaire}
\newtheorem{heur}{Heuristique}
\newtheorem{rem}{Remarque}
\newtheorem{note}{Note}

\theoremstyle{definition}
\newtheorem{conj}{Conjecture}
\newtheorem{prob}{Problème}
\newtheorem{quest}{Question}
\newtheorem{prot}{Protocole}
\newtheorem{algo}{Algorithme}
\newtheorem{defn}[subsection]{Définition}
\newtheorem{exmp}[subsection]{Exemples}
\newtheorem{exo}[subsection]{Exercices}
\newtheorem{ex}[subsection]{Exemple}
\newtheorem{exs}[subsection]{Exemples}

\theoremstyle{remark}

\definecolor{wgrey}{RGB}{148, 38, 55}
\definecolor{wgreen}{RGB}{100, 200,0} 
\hypersetup{
    colorlinks=true,
    linkcolor=wgreen,
    urlcolor=wgrey,
    filecolor=wgrey
}

\title{Dimension et nombre d'équations}
\date{}

\begin{document}
\maketitle
\tableofcontents
\[\ldots\]   

\chapter{Les définitions}
\section{Définition topologique}
Concrètement pour $X=\cup_{i=1}^n X_i$ une variété algébrique décomposée
en composantes irréductibles :
\[\dim(X)=\max(\dim(X_i))=\sup_{U\subset X~ouvert} \dim(U)\]
et pour une variété irréductible c'est le sup des longueurs de chaines
\[Y_0\subsetneq Y_1\subsetneq \ldots \subsetneq Y_d=X\]

\section{Définition par les corps de fonctions}
Ducoup dans le cas intègre, les restrictions du faisceau sont 
injectives et le faisceau est approximable par les ouverts principaux
affines ! En particulier 
\[k(X)\simeq k(U)\]
et \[k(X)\simeq Frac(\Or_X(U_0))\]
pour un affine ouvert $U_0\subset X$ quelconque. On peut montrer que
\[dim(X)= degtr_k k(X)\]
et c'est bien défini :
\begin{enumerate}
    \item Si on prends deux familles algébriquement 
indépendantes et $K$ algébriques sur les deux, on peut montrer qu'elles
ont la même cardinalité. 
    \item On peut se réduire au cas affine.
    \item On conclut par l'injection de Noether dans $A(X)$ qui fixe la
	dimension en passant au corps de fractions!
\end{enumerate}

\section{À rajouter : codimension, voir notes sur la partiel}

\section{Et donc ?}
Bon tout ça montre qu'on peut tjr se ramener au cas affine, ouvert
qui nous arrange. Lesquels ? (je pense à l'exemple des équations
locales qui utilise des ouverts particuliers)

\chapter{Équations}
\section{Hypersurfaces}
\subsection{Cas affine}
Essentiellement, y'a cette suite d'arguments :
\begin{enumerate}
    \item La dimension est invariante par extension d'anneaux entiers.
	(Y'a pas mal d'algèbre là dedans, j'en parlerai ailleurs)
    \item Par Noether si $F\in k[T_1,\ldots, T_n]-k$ alors 
	\[\dim k[T_1,\ldots, T_n]/(F)=\dim k[T_1,\ldots, T_{n-1}]\]
    \item Ensuite $\dim k[T_1,\ldots,T_n]=n$ par récurrence et 
	l'argument d'avant (faut faire un tout petit peu attention).
    \item Automatiquement, si $F\in k[T_1,\ldots, T_n]$ alors
	$\dim(Z(F))=n-1$.
\end{enumerate}
\subsection{Cas intègre}
Ça c'était le cas affine, maintenant le cas intègre : Étant donné
$f\in \Or_X(X)$ on a
\[\dim(Z(f))=\dim(X)-1\]
La preuve consiste à dire 
\begin{enumerate}
    \item $\dim(U)=\dim(X)$ en utilisant $k(U)\simeq k(X)$ d'où
	on se ramène au cas affine.
    \item On à une injection entière finie 
	\[k[T_1,\ldots, T_n]/fA\cap k[T_1,\ldots, T_n]\hookrightarrow
	A(X)/fA(X)\]
	où $fA\cap k[T_1,\ldots, T_n]$ c'est juste en identifiant avec
	l'image.
    \item Puis on a 
	\[fA\cap k[T_1,\ldots, T_n]\subset \sqrt{N_{k(X)/k(\A^n)}(f)}
	\subset \sqrt{fA\cap k[T_1,\ldots, T_n]}\]
    et on conclut par Noether.
\end{enumerate}
\begin{rem}
    Les anneaux de polynômes sont factoriels donc intégralement clos.
    D'où la norme fonctionne bien là.
\end{rem}
\begin{rem}
    Je sais vraiment pas si on est obligés d'utiliser $k(U)\simeq k(X)$
    mdr. À méditer. Si $F_1\cap U \subsetneq F_2\cap U$ et $U$ dense
    dans $X$, alors $\bar F_1 = \bar F_2$ implique $F_1$ dense dans 
    $F_2$ d'où y sont égaux dans $U$ car fermés ? 
\end{rem}
\section{Nombre d'équations d'un fermé (à finir)}
Tout irréductible affine $Z$ de dimension $s$ dans $\A^n$ est une 
composante d'un
\[Z\subseteq Z(f_1,\ldots, f_{n-s})\]
dont toutes les composantes ont dimension $s$. On peut le faire
par récurrence sur $0\leq n-s\leq n$. On a 
\begin{enumerate}
    \item $n-s=0 :$ On est sur $\A^n$.
    \item $n-s=1 :$ C'est un $Z(f)$, le fait que les composantes
	aient dimension $n-1$ c'est juste que c'est des 
	$Z(P_i)$ avec $\prod P_i^{e_i}=f$.
    \item On prends $Z$ de dimension $s$ dans $Z_1$ de dimension
	$s+1$, alors $Z_1\subset Z(f_1,\ldots, f_{n-(s+)})$ avec
	les hypothèses.
    \item On baisser la dimension de toutes les composantes de $1$
	on peut intersecter avec $Z\subset Z(f)$. Donc on prends
	$f\in I(Z)$ et on veut que $Z_i\nsubseteq Z(f)$ donc 
	$f\notin I(Z_i)$. Donc $f\in I(Z)-\cup_i I(Z_i)$. Et ça
	par lemme d'évitement.
\end{enumerate}

À l'inverse
\[Z(f_1,\ldots, f_{s})\subset \A^n(k)\]
est de dimension $\geq n-s$.
\subsection{Dimension des fibres}
On en déduit que si $f\colon X\to Y$ est dominant alors
\[\dim(f^{-1}(y))\geq \dim(X)-\dim(Y)\]
parce que $f^{-1}(y)\subseteq Z(f_*\m_y)$ et $\m_y$ est défini
par $\dim(Y)$ équations ! 

pour l'ouvert, factoriser un morphisme fini et là c'est clair
que $dim f^{-1}(Z).= dim Z$ si $f$ fini dominant.

\printbibliography
\end{document}


\documentclass[a4paper,12pt]{book}
\usepackage{amsmath,  amsthm,enumerate}
\usepackage{csquotes}
\usepackage[provide=*,french]{babel}
\usepackage[dvipsnames]{xcolor}
\usepackage{quiver, tikz}

%symbole caligraphique
\usepackage{mathrsfs}

%hyperliens
\usepackage{hyperref}

%pseudo-code
\usepackage{algorithm}
\usepackage{algpseudocode}

\usepackage{fancyhdr}

\pagestyle{fancy}
\addtolength{\headwidth}{\marginparsep}
\addtolength{\headwidth}{\marginparwidth}
\renewcommand{\chaptermark}[1]{\markboth{#1}{}}
\renewcommand{\sectionmark}[1]{\markright{\thesection\ #1}}
\fancyhf{}
\fancyfoot[C]{\thepage}
\fancyhead[LO]{\textit \leftmark}
\fancyhead[RE]{\textit \rightmark}
\renewcommand{\headrulewidth}{0pt} % and the line
\fancypagestyle{plain}{%
    \fancyhead{} % get rid of headers
}

%bibliographie
\usepackage[
backend=biber,
style=alphabetic,
sorting=ynt
]{biblatex}

\addbibresource{bib.bib}

\usepackage{appendix}
\renewcommand{\appendixpagename}{Annexe}

\definecolor{wgrey}{RGB}{148, 38, 55}

\setlength\parindent{24pt}

\newcommand{\Z}{\mathbb{Z}}
\newcommand{\R}{\mathbb{R}}
\newcommand{\rel}{\omathcal{R}}
\newcommand{\Q}{\mathbb{Q}}
\newcommand{\C}{\mathbb{C}}
\newcommand{\N}{\mathbb{N}}
\newcommand{\K}{\mathbb{K}}
\newcommand{\A}{\mathbb{A}}
\newcommand{\B}{\mathcal{B}}
\newcommand{\Or}{\mathcal{O}}
\newcommand{\F}{\mathscr F}
\newcommand{\Hom}{\textrm{Hom}}
\newcommand{\disc}{\textrm{disc}}
\newcommand{\Pic}{\textrm{Pic}}
\newcommand{\End}{\textrm{End}}
\newcommand{\Spec}{\textrm{Spec}}
\newcommand{\Supp}{\textrm{Supp}}
\renewcommand{\Im}{\textrm{Im}}


\newcommand{\m}{\mathfrak{m}}
\newcommand{\n}{\mathfrak{n}}
\newcommand{\p}{\mathfrak{p}}


\newcommand{\cL}{\mathscr{L}}
\newcommand{\G}{\mathscr{G}}
\newcommand{\D}{\mathscr{D}}
\newcommand{\E}{\mathscr{E}}
\renewcommand{\Pr}{\mathbb{P}}
\renewcommand{\P}{\mathscr{P}}
\renewcommand{\H}{\mathscr{H}}

\makeatletter
\newcommand{\colim@}[2]{%
  \vtop{\m@th\ialign{##\cr
    \hfil$#1\operator@font colim$\hfil\cr
    \noalign{\nointerlineskip\kern1.5\ex@}#2\cr
    \noalign{\nointerlineskip\kern-\ex@}\cr}}%
}
\newcommand{\colim}{%
  \mathop{\mathpalette\colim@{\rightarrowfill@\scriptscriptstyle}}\nmlimits@
}
\renewcommand{\varprojlim}{%
  \mathop{\mathpalette\varlim@{\leftarrowfill@\scriptscriptstyle}}\nmlimits@
}
\renewcommand{\varinjlim}{%
  \mathop{\mathpalette\varlim@{\rightarrowfill@\scriptscriptstyle}}\nmlimits@
}
\makeatother

\theoremstyle{plain}
\newtheorem{thm}[subsection]{Théoreme}
\newtheorem{lem}[subsection]{Lemme}
\newtheorem{prop}[subsection]{Proposition}
\newtheorem{cor}[subsection]{Corollaire}
\newtheorem{heur}{Heuristique}
\newtheorem{rem}{Remarque}
\newtheorem{note}{Note}

\theoremstyle{definition}
\newtheorem{conj}{Conjecture}
\newtheorem{prob}{Problème}
\newtheorem{quest}{Question}
\newtheorem{prot}{Protocole}
\newtheorem{algo}{Algorithme}
\newtheorem{defn}[subsection]{Définition}
\newtheorem{exmp}[subsection]{Exemples}
\newtheorem{exo}[subsection]{Exercices}
\newtheorem{ex}[subsection]{Exemple}
\newtheorem{rep}{Réponse}
\newtheorem{exs}[subsection]{Exemples}

\theoremstyle{remark}

\definecolor{wgrey}{RGB}{148, 38, 55}
\definecolor{wgreen}{RGB}{100, 200,0} 
\hypersetup{
    colorlinks=true,
    linkcolor=wgreen,
    urlcolor=wgrey,
    filecolor=wgrey
}

\title{Normalisation}
\date{}

\begin{document}
\maketitle
\tableofcontents
\[\ldots\]   
\chapter{Caractérisation}
On dit qu'une variété algébrique intègre $X$ est normale si pour
tout ouvert $U\subset X$, $\Or_X(U)$ est intégralement clos.
\section{Localisation et cloture intégrale}
À savoir que la localisation commute avec clôture intégrale.
À prouver (c'est pas dur).

Pour $S$ une partie multiplicative, si $A$ est intégralement clos,
et $x^n+\sum a_i x^i=0$ avec $a_i\in S^{-1}A$. Il existe $s$ tel
que $sx\in A$ puis $x\in S^{-1}A$. Si $A$ est pas intégralement
clos c'est pareil en fait.

\section{Cas affine}
Il suffit que $A(X)$ soit intégralement clos ! Car pour tout 
$U=\cup D(f_i)$ on a \[\Or_X(U)=\cap_i A(D(f_i))\]
et que cette intersection est intégralement close car les
$A(X)_{f_i}$ sont intégralement clos par la propriété de 
commutativité du dessus.

\section{Cas général}
\begin{rem}
    Étant donné un ouvert affine $U$, on a $\Or_{X,x}=A(U)_{\p_x}$,
    en particulier par propriété de commutativité du dessus
    c'est encore intégralement clos.
\end{rem}
On peut prouver que dans une variété intègre on a toujours
\[\Or_X(U)=\cap_{x\in U} \Or_{X,x}\]
en particulier toutes les sous-variétés de $X$ sont normales si
$X$ est normale. Ce sera utile pour la normalisation.

\chapter{Normalisation}
Une/la normalisation c'est un 
\[\pi\colon X'\to X\]
birationnel fini avec $X'$ normale. On peut aussi normaliser dans 
une extension $L$ de $k(X)$. La première est la normalisation
dans $k(X)$. C'est unique à isomorphisme canonique près.
\section{Clôture intégrale dans une extension finie, cas des
$k$-algèbres}
\begin{thm}
    Si $A$ est une $k$-algèbre de t.f intègre, typiquement $A(X)$,
    et $L/k(X)$ finie. Alors $\tilde A=B$ dans $L$ est finie sur
    $A$.
\end{thm}
C'est étonnant ducoup vu qu'on assume rien sur l'extension et 
la caractéristique. Faut utiliser la normalisation de Noether
pour la partie purement inséparable.

\section{Construction}
Étant donné une $k$-algèbre de type fini intègre, suffit de
savoir que sa clôture intégrale dans un corps est de type 
fini sur $k$ intègre. Puis on trouve une variété affine ! Ensuite
on peut recoller.
\subsection*{Cas affine}
C'est direct.

\subsection*{Unicité}
Donc en gros de
\[\Or_{X'}(\pi^{-1}U)=\cap_{x\in \pi^{-1}U}\Or_{X',x}\]
car $X'$ normale implique intègre par déf. On déduit que 
\[\pi^{-1}U\]
est normale, en plus par finitude de $\pi$. C'est affine si $U$
est affine et c'est forcément la clôture de $A(U)$ dans $k(X')$.
D'où l'unicité canonique. 
\subsection*{Cas général}
Y suffit de prendre les normalisations affines
\[\pi_i\colon X_i'\to X_i\]
et de remarquer que $\pi_i^{-1}(X_i\cap X_j)$ et 
$\pi_j^{-1}(X_i\cap X_j)$ sont deux normalisations de $X_i\cap X_j$
d'où par l'unicité on peut recoller les $X_i'$!

\subsection*{Normalisation simple}
Donc si on normalise dans $k(X)$, on peut remarquer que la 
birationalité est pas trop dure. Car 
\[A(X)\subset\tilde{A(X)}\subset k(X)\]
et $\tilde{A(X)}$ de type fini sur $k$ force 
\[\tilde{A(X)}=k[f_i/g_i,i=1,\ldots, r]\] avec $f_i,g_i\in A(X)$.
En particulier, \[A(X)_{g_1\ldots g_r}=\tilde{A(X)}.\] Ça montre
que pour tout $x\in D(g_1\ldots g_r)$, $\Or_{X,x}$ est 
intégralement clos. D'où $D(g_1\ldots g_r)$ est normale puis
la birationalité par unicité.




\printbibliography
\end{document}


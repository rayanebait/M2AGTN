\documentclass[a4paper,12pt]{book}
\usepackage{amsmath,  amsthm,enumerate}
\usepackage{csquotes}
\usepackage[provide=*,french]{babel}
\usepackage[dvipsnames]{xcolor}
\usepackage{quiver, tikz}

%symbole caligraphique
\usepackage{mathrsfs}

%hyperliens
\usepackage{hyperref}

%pseudo-code
\usepackage{algorithm}
\usepackage{algpseudocode}

\usepackage{fancyhdr}

\pagestyle{fancy}
\addtolength{\headwidth}{\marginparsep}
\addtolength{\headwidth}{\marginparwidth}
\renewcommand{\chaptermark}[1]{\markboth{#1}{}}
\renewcommand{\sectionmark}[1]{\markright{\thesection\ #1}}
\fancyhf{}
\fancyfoot[C]{\thepage}
\fancyhead[LO]{\textit \leftmark}
\fancyhead[RE]{\textit \rightmark}
\renewcommand{\headrulewidth}{0pt} % and the line
\fancypagestyle{plain}{%
    \fancyhead{} % get rid of headers
}

%bibliographie
\usepackage[
backend=biber,
style=alphabetic,
sorting=ynt
]{biblatex}

\addbibresource{bib.bib}

\usepackage{appendix}
\renewcommand{\appendixpagename}{Annexe}

\definecolor{wgrey}{RGB}{148, 38, 55}

\setlength\parindent{24pt}

\newcommand{\Z}{\mathbb{Z}}
\newcommand{\R}{\mathbb{R}}
\newcommand{\rel}{\omathcal{R}}
\newcommand{\Q}{\mathbb{Q}}
\newcommand{\C}{\mathbb{C}}
\newcommand{\N}{\mathbb{N}}
\newcommand{\K}{\mathbb{K}}
\newcommand{\A}{\mathbb{A}}
\newcommand{\B}{\mathcal{B}}
\newcommand{\Or}{\mathcal{O}}
\newcommand{\F}{\mathscr F}
\newcommand{\Hom}{\textrm{Hom}}
\newcommand{\disc}{\textrm{disc}}
\newcommand{\Pic}{\textrm{Pic}}
\newcommand{\End}{\textrm{End}}
\newcommand{\Spec}{\textrm{Spec}}
\newcommand{\Supp}{\textrm{Supp}}
\renewcommand{\Im}{\textrm{Im}}
\newcommand{\m}{\mathfrak{m}}


\newcommand{\cL}{\mathscr{L}}
\newcommand{\G}{\mathscr{G}}
\newcommand{\D}{\mathscr{D}}
\newcommand{\E}{\mathscr{E}}
\renewcommand{\Pr}{\mathbb{P}}
\renewcommand{\P}{\mathscr{P}}
\renewcommand{\H}{\mathscr{H}}

\makeatletter
\newcommand{\colim@}[2]{%
  \vtop{\m@th\ialign{##\cr
    \hfil$#1\operator@font colim$\hfil\cr
    \noalign{\nointerlineskip\kern1.5\ex@}#2\cr
    \noalign{\nointerlineskip\kern-\ex@}\cr}}%
}
\newcommand{\colim}{%
  \mathop{\mathpalette\colim@{\rightarrowfill@\scriptscriptstyle}}\nmlimits@
}
\renewcommand{\varprojlim}{%
  \mathop{\mathpalette\varlim@{\leftarrowfill@\scriptscriptstyle}}\nmlimits@
}
\renewcommand{\varinjlim}{%
  \mathop{\mathpalette\varlim@{\rightarrowfill@\scriptscriptstyle}}\nmlimits@
}
\makeatother

\theoremstyle{plain}
\newtheorem{thm}[subsection]{Théoreme}
\newtheorem{lem}[subsection]{Lemme}
\newtheorem{prop}[subsection]{Proposition}
\newtheorem{cor}[subsection]{Corollaire}
\newtheorem{heur}{Heuristique}
\newtheorem{rem}{Remarque}
\newtheorem{note}{Note}

\theoremstyle{definition}
\newtheorem{conj}{Conjecture}
\newtheorem{prob}{Problème}
\newtheorem{quest}{Question}
\newtheorem{prot}{Protocole}
\newtheorem{algo}{Algorithme}
\newtheorem{defn}[subsection]{Définition}
\newtheorem{exmp}[subsection]{Exemples}
\newtheorem{exo}[subsection]{Exercices}
\newtheorem{ex}[subsection]{Exemple}
\newtheorem{exs}[subsection]{Exemples}

\theoremstyle{remark}

\definecolor{wgrey}{RGB}{148, 38, 55}
\definecolor{wgreen}{RGB}{100, 200,0} 
\hypersetup{
    colorlinks=true,
    linkcolor=wgreen,
    urlcolor=wgrey,
    filecolor=wgrey
}

\title{Notes perso : Géométrie algébrique}
\date{}

\begin{document}
\maketitle
\tableofcontents

\chapter*{Introduction}
En suivant le I. R. Shafarevich.

\chapter{Premières propriétés}
\section{Variété produit}
Ce qui est assez clair c'est le produit de variétés affines. L'idée
en général c'est de construire 
\[X\times Y\to \Pr^n(k)\]
un isomorphisme sur son image qu'on peut reconstruire localement de 
sorte à ce qu'il soit unique. 
\begin{note}
    Page 54.
\end{note}
Concrètement dans le cas affine, les fermés de $\A^{n+m}$
qui sont fermés de $\A^n\times \A^m$ c'est les fermés du type
$V(F(X_1,\ldots, X_n)), G(X_{n+1},\ldots, X_m))$. Pour le projectif
c'est assez similaire. Le truc c'est que brutalement définir
\[\Pr^n\times \Pr^m\to \Pr^{n+1+m+1-1}\]
ca marche pas. Un tuple du produit $(u_0,\ldots, u_n, v_0,\ldots, v_m)$
est doublement homogène donc définit pas un point de $\Pr^{n+m+1}$. 
Ducoup faut déf plutôt
\begin{align*}
    \phi\colon\Pr^n\times \Pr^m&\to \Pr^{(n+1)(m+1)-1}\\
([u_0,\ldots, u_n],[v_0,\ldots, v_m])&\mapsto (u_iv_j)_{i,j}
\end{align*}
Et là on a clairement 
\[\phi(\Pr^n\times \Pr^m)=\bigcap_{i,j}V\left(\begin{vmatrix}
X_{ij}& X_{il}\\ X_{kj}&X_{kl}\end{vmatrix}\right)\]
En fait ce qui est cool c'est la remarque suivante.
\begin{rem}
    Si on regarde les mineures $2x2$ de $(X_{ij})$ et le lieu de leur
    zéros communs. On peut prouver que $\phi$ se surjecte dedans. En
    particulier la matrice $(w_{ij})=(u_iv_j)$ est de rang $1$ et en plus
    c'est un produit de matrices $1\times (n+1)$ par $(m+1)\times 1$.
\end{rem}
En fait c'est non trivial le produit projectif mdr. Déjà à faire :
\begin{exo}
    Les fermés de $\Pr^n\times \Pr^m$ sont les zéros de polynômes 
    bihomogènes. Réponse : L'idée c'est que si on prend un polynôme 
    homogènes en $(X_{ij})$ et qu'on regarde \[X_{ij}\mapsto X_iY_j\]. 
    On obtient un polynôme bihomogène en les $X_i$ et les $Y_j$ de 
    mêmes degrés. En fait inversement si on a un polynôme bihomogène
    $G$ de degré $r$ en $X_i$ et $s$ en $Y_j$ avec $r>s$, on peut 
    remarquer que $V(G)$ dans $\A^{n+1}\times \A^{m+1}$ est égal à 
    à \[\bigcup V(Y_j^{r-s} G)\]
    (à vérifier, ça a l'air de marcher justement parce qu'on regarde
    dans $\A^{n+1}\times \A^{m+1}$ et pas $\A^{n+1+m+1}$.)
\end{exo}



\printbibliography
\end{document}


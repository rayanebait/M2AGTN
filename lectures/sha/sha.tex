\documentclass[a4paper,12pt]{book}
\usepackage{amsmath,  amsthm,enumerate}
\usepackage{csquotes}
\usepackage[provide=*,french]{babel}
\usepackage[dvipsnames]{xcolor}
\usepackage{quiver, tikz}

%symbole caligraphique
\usepackage{mathrsfs}

%hyperliens
\usepackage{hyperref}

%pseudo-code
\usepackage{algorithm}
\usepackage{algpseudocode}

\usepackage{fancyhdr}

\pagestyle{fancy}
\addtolength{\headwidth}{\marginparsep}
\addtolength{\headwidth}{\marginparwidth}
\renewcommand{\chaptermark}[1]{\markboth{#1}{}}
\renewcommand{\sectionmark}[1]{\markright{\thesection\ #1}}
\fancyhf{}
\fancyfoot[C]{\thepage}
\fancyhead[LO]{\textit \leftmark}
\fancyhead[RE]{\textit \rightmark}
\renewcommand{\headrulewidth}{0pt} % and the line
\fancypagestyle{plain}{%
    \fancyhead{} % get rid of headers
}

%bibliographie
\usepackage[
backend=biber,
style=alphabetic,
sorting=ynt
]{biblatex}

\addbibresource{bib.bib}

\usepackage{appendix}
\renewcommand{\appendixpagename}{Annexe}

\definecolor{wgrey}{RGB}{148, 38, 55}

\setlength\parindent{24pt}

\newcommand{\Z}{\mathbb{Z}}
\newcommand{\R}{\mathbb{R}}
\newcommand{\rel}{\omathcal{R}}
\newcommand{\Q}{\mathbb{Q}}
\newcommand{\C}{\mathbb{C}}
\newcommand{\N}{\mathbb{N}}
\newcommand{\K}{\mathbb{K}}
\newcommand{\A}{\mathbb{A}}
\newcommand{\B}{\mathcal{B}}
\newcommand{\Or}{\mathcal{O}}
\newcommand{\F}{\mathscr F}
\newcommand{\Hom}{\textrm{Hom}}
\newcommand{\disc}{\textrm{disc}}
\newcommand{\Pic}{\textrm{Pic}}
\newcommand{\End}{\textrm{End}}
\newcommand{\Spec}{\textrm{Spec}}
\newcommand{\Supp}{\textrm{Supp}}
\renewcommand{\Im}{\textrm{Im}}
\newcommand{\m}{\mathfrak{m}}


\newcommand{\cL}{\mathscr{L}}
\newcommand{\G}{\mathscr{G}}
\newcommand{\D}{\mathscr{D}}
\newcommand{\E}{\mathscr{E}}
\renewcommand{\Pr}{\mathbb{P}}
\renewcommand{\P}{\mathscr{P}}
\renewcommand{\H}{\mathscr{H}}

\makeatletter
\newcommand{\colim@}[2]{%
  \vtop{\m@th\ialign{##\cr
    \hfil$#1\operator@font colim$\hfil\cr
    \noalign{\nointerlineskip\kern1.5\ex@}#2\cr
    \noalign{\nointerlineskip\kern-\ex@}\cr}}%
}
\newcommand{\colim}{%
  \mathop{\mathpalette\colim@{\rightarrowfill@\scriptscriptstyle}}\nmlimits@
}
\renewcommand{\varprojlim}{%
  \mathop{\mathpalette\varlim@{\leftarrowfill@\scriptscriptstyle}}\nmlimits@
}
\renewcommand{\varinjlim}{%
  \mathop{\mathpalette\varlim@{\rightarrowfill@\scriptscriptstyle}}\nmlimits@
}
\makeatother

\theoremstyle{plain}
\newtheorem{thm}[subsection]{Théoreme}
\newtheorem{lem}[subsection]{Lemme}
\newtheorem{prop}[subsection]{Proposition}
\newtheorem{cor}[subsection]{Corollaire}
\newtheorem{heur}{Heuristique}
\newtheorem{rem}{Remarque}
\newtheorem{note}{Note}

\theoremstyle{definition}
\newtheorem{conj}{Conjecture}
\newtheorem{prob}{Problème}
\newtheorem{quest}{Question}
\newtheorem{prot}{Protocole}
\newtheorem{algo}{Algorithme}
\newtheorem{defn}[subsection]{Définition}
\newtheorem{exmp}[subsection]{Exemples}
\newtheorem{exo}[subsection]{Exercices}
\newtheorem{ex}[subsection]{Exemple}
\newtheorem{exs}[subsection]{Exemples}

\theoremstyle{remark}

\definecolor{wgrey}{RGB}{148, 38, 55}
\definecolor{wgreen}{RGB}{100, 200,0} 
\hypersetup{
    colorlinks=true,
    linkcolor=wgreen,
    urlcolor=wgrey,
    filecolor=wgrey
}

\title{Notes perso : Géométrie algébrique}
\date{}

\begin{document}
\maketitle
\tableofcontents

\chapter*{Introduction}
En suivant le I. R. Shafarevich.

\chapter{Premières propriétés}
\section{Variété produit}
Ce qui est assez clair c'est le produit de variétés affines. L'idée
en général c'est de construire 
\[X\times Y\to \Pr^n(k)\]
un isomorphisme sur son image qu'on peut reconstruire localement de 
sorte à ce qu'il soit unique. 
\begin{note}
    Page 54.
\end{note}
Concrètement dans le cas affine, les fermés de $\A^{n+m}$
qui sont fermés de $\A^n\times \A^m$ c'est les fermés du type
$V(F(X_1,\ldots, X_n)), G(X_{n+1},\ldots, X_m))$. Pour le projectif
c'est assez similaire. Le truc c'est que brutalement définir
\[\Pr^n\times \Pr^m\to \Pr^{n+1+m+1-1}\]
ca marche pas. Un tuple du produit $(u_0,\ldots, u_n, v_0,\ldots, v_m)$
est doublement homogène donc définit pas un point de $\Pr^{n+m+1}$. 
Ducoup faut déf plutôt
\begin{align*}
    \phi\colon\Pr^n\times \Pr^m&\to \Pr^{(n+1)(m+1)-1}\\
([u_0,\ldots, u_n],[v_0,\ldots, v_m])&\mapsto (u_iv_j)_{i,j}
\end{align*}
Et là on a clairement 
\[\phi(\Pr^n\times \Pr^m)=\bigcap_{i,j}V\left(\begin{vmatrix}
X_{ij}& X_{il}\\ X_{kj}&X_{kl}\end{vmatrix}\right)\]
En fait ce qui est cool c'est la remarque suivante.
\begin{rem}
    Si on regarde les mineures $2x2$ de $(X_{ij})$ et le lieu de leur
    zéros communs. On peut prouver que $\phi$ se surjecte dedans. En
    particulier la matrice $(w_{ij})=(u_iv_j)$ est de rang $1$ et en plus
    c'est un produit de matrices $1\times (n+1)$ par $(m+1)\times 1$.
\end{rem}
En fait c'est non trivial le produit projectif mdr. Déjà à faire :
\begin{exo}
    Les fermés de $\Pr^n\times \Pr^m$ sont les zéros de polynômes 
    bihomogènes. Réponse : L'idée c'est que si on prend un polynôme 
    homogènes en $(X_{ij})$ et qu'on regarde \[X_{ij}\mapsto X_iY_j\]. 
    On obtient un polynôme bihomogène en les $X_i$ et les $Y_j$ de 
    mêmes degrés. En fait inversement si on a un polynôme bihomogène
    $G$ de degré $r$ en $X_i$ et $s$ en $Y_j$ avec $r>s$, on peut 
    remarquer que $V(G)$ dans $\A^{n+1}\times \A^{m+1}$ est égal à 
    à \[\bigcap V(Y_j^{r-s} G)\]
    (à vérifier, ça a l'air de marcher justement parce qu'on regarde
    dans $\A^{n+1}\times \A^{m+1}$ et pas $\A^{n+1+m+1}$.)
    \textbf{Meilleure réponse} : Si on regarde $Z(G)\subset \varphi(
    \Pr^n\times\Pr^m)$ on peut montrer que $G$ est bihomogène de mêmes
    degré. À l'inverse pour def des fermés de $\Pr^n$ on doit utiliser
    des générateurs homogènes et donc $\bigcap_i V(Y_j^{r-s}G)$ c'est des
    générateurs homogènes.
\end{exo}
Pour $\Pr^n\times \A^n$ on peut prendre les homogènes pour les premières.


\section{Graphes}
En gros étant donné la structure d'une variété projective $X$ on peut
se demander si la diagonale dans $X\times X$ est fermée. C'est assez
clair si on remarque que dire que deux points projectifs 
$u=[u_0:\ldots:_n]$ et $v=[v_0:\ldots:v_n]$ sont égaux ssi ils sont
proportionnels. Autrement dit si \[u_i/u_j=v_i/v_j\]
ou \[u_iv_j=u_jv_i.\] Ensuite étant donné une fonction 
\[f\colon Y\to X\]
on peut regarder \[id\times f\colon X\times Y\to X\times X\]
d'où le graphe est fermé.

\section{Les variétés projectives sont complètes}
Y'a la preuve de Shafarevich et la preuve de Dat qui a l'air plus 
parlante et qui dit qu'on peut "compléter" les courbes ponctionnées d'un
point sur une variété complète.

Pour la preuve de Shafarevich, on peut
regarder $Z(G)\subset X\times Y\to Y$ et donner des conditions pour que
$y\in p_2(Z(G))$ pour $G$ un polynôme bihomogène. Déjà on peut se ramener
à $X=\Pr^n$ car $X$ est projective et $Y=\A^n$. En fait, 
$y\in p_2(Z(G))$ ça veut juste dire que $G(\bar U, y)$ a un zéro 
projectif. Autrement dit, $I_s\nsubseteq (G(\bar U,y))$. Par cette 
condition, en notant $(M_{a})_a$ les monômes de degrés $s$ on obtient
\[M_a(\bar U)=F_a(\bar U)G(\bar U, y)\]
avec $\deg(F_a)=s-\deg(G)$. Enfin si on écrit $F_a$ en somme monomes
de degré $s-\deg(G)$ disons en $N_b$. Alors les $M_a$ sont combinaisons
linéaires des $N_bG$. Et on obtient une matrices entre la base canonique
des monômes de degré $s$ et les $N_bG$ à coefficients des polynomes en
$y$ à coefficient dans $k$! La condition $y\in p_2(Z(G))$ équivaut alors
au fait que le rang de la matrice soit $<$ au nombre de monôme de 
degré $s$ en $n$ variables. On prend $y$ qui annule les mineures de 
taille cette dimension!

\begin{cor}
    L'image d'une variété projective est fermée.
\end{cor}

Suffit de factoriser $X\to Y$ en $X\to X\times Y\to Y$, la première 
flèche est $x\mapsto (x, f(x))$ et envoie $X$ sur son graphe.


\section{Un peu de birationnalité (exo 5 (1.3)}
Si on prend $f_d,f_{d-1}\in k[X_1,\ldots, X_n]$ de degrés $d$ et $d-1$
tels que $f_d+f_{d-1}$ est irréductible, alors $Z(f_d+f_{d-1})$ est 
birationnelle à $\A^{n-1}$. 

Mon idée c'est que $f_d/f_{d-1}$ est de degré $1$, d'où 
$f_d/f_{d-1} + 1$ définit un hyperplan. Ducoup faut trouver un 
isomorphisme de $k(\bar X_1, \ldots, \bar X_n)$ vers 
$k(X_1, \ldots, X_{n-1})$. Pour ça deux manières de faire, si $f_d$ 
a un monôme qui étend $f_{d-1}$ on peut juste le factoriser en haut
et en haut de la fraction. On obtient une expression de la forme 
\[F_0.(X_k+F_1)\]
où $F_0$ est de degré $0$ et $F_1$ de degré $1$ qui a seulement pour 
dénominateur le monôme. On peut faire le changement de variable
$X_k\mapsto X_k/F_0 $, on obtient $X_k$ + $F_1(F_0, X_1,\ldots, X_n)$.

\textbf{Meilleure manière :} En fait on peut écrire $f_d/f_{d-1}$ comme
$1/F_0(f_d/M)$ où $F_0$ est une fraction rationnelle de degré $0$ et
$M$ un monôme. En gros on a factorisé en bas et en haut par $M$. Ensuite,
en notant $M(i)=\{M\in S_{d_1}| X_i\mid M\}$ et \[f_{d,i}= f_{d,i-1}-
X_{i-1}\sum_{M\in M(f_{d,i-1},i-1)} a_{M, i-1}M\] et $M(f_{d,i},i)=\{M\in M(i)| M\in Monome(f_{d,i})\}$
on peut écrire
\[f_d/M=\left(\sum_i X_i\sum_{N\in M(f_{d,i}, i)} a_{N, i} N\right)/M\]
en particulier les $N/M$ ont degré $0$. Enfin si on note $I$ le support 
de la somme de gauche, on peut déf 
\[\varphi\colon K[T_1,\ldots,T_n]\to K[Z(f_d+f_{d-1})]_{f_{d-1}, M}\]
par $T_i\mapsto X_i\sum_{N\in M(f_{d,i}, i)} a_{N, i} N/M$ si $i\in I$
et $T_i\mapsto X_i$ sinon. On peut remarquer que la flèche préserve la 
graduation, i.e. $\ker(\varphi)$ est engendré par des polynômes de degré
$1$. On obtient une flèche rationnelle dominante injective
\[Z(f_d+f_{d-1})\to Z(\ker(\varphi))\]
où $\ker(\varphi)$ définit un hyperplan de $\A^n$ d'où le résultat.

\printbibliography
\end{document}


\documentclass[a4paper,12pt]{book}
\usepackage{amsmath,  amsthm,enumerate}
\usepackage{csquotes}
\usepackage[provide=*,french]{babel}
\usepackage[dvipsnames]{xcolor}
\usepackage{quiver, tikz}

\usepackage{quantikz}

%symbole caligraphique
\usepackage{mathrsfs}

%hyperliens
\usepackage{hyperref}

%pseudo-code
\usepackage{algorithm}
\usepackage{algpseudocode}

\usepackage{fancyhdr}

\pagestyle{fancy}
\addtolength{\headwidth}{\marginparsep}
\addtolength{\headwidth}{\marginparwidth}
\renewcommand{\chaptermark}[1]{\markboth{#1}{}}
\renewcommand{\sectionmark}[1]{\markright{\thesection\ #1}}
\fancyhf{}
\fancyfoot[C]{\thepage}
\fancyhead[LO]{\textit \leftmark}
\fancyhead[RE]{\textit \rightmark}
\renewcommand{\headrulewidth}{0pt} % and the line
\fancypagestyle{plain}{%
    \fancyhead{} % get rid of headers
}


\usepackage{appendix}
\renewcommand{\appendixpagename}{Annexe}

\definecolor{wgrey}{RGB}{148, 38, 55}

\setlength\parindent{24pt}

\newcommand{\Z}{\mathbb{Z}}
\newcommand{\R}{\mathbb{R}}
\newcommand{\rel}{\omathcal{R}}
\newcommand{\Q}{\mathbb{Q}}
\newcommand{\C}{\mathbb{C}}
\newcommand{\N}{\mathbb{N}}
\newcommand{\K}{\mathbb{K}}
\newcommand{\A}{\mathbb{A}}
\newcommand{\B}{\mathcal{B}}
\newcommand{\Or}{\mathcal{O}}
\newcommand{\F}{\mathbb F}
\newcommand{\Hom}{\textrm{Hom}}
\newcommand{\disc}{\textrm{disc}}
\newcommand{\Pic}{\textrm{Pic}}
\newcommand{\End}{\textrm{End}}
\newcommand{\Spec}{\textrm{Spec}}

\newcommand{\cL}{\mathscr{L}}
\newcommand{\G}{\mathscr{G}}
\newcommand{\D}{\mathscr{D}}
\newcommand{\E}{\mathscr{E}}
\renewcommand{\H}{\mathscr{H}}

\theoremstyle{plain}
\newtheorem{thm}[subsection]{Théoreme}
\newtheorem{lem}[subsection]{Lemme}
\newtheorem{prop}[subsection]{Proposition}
\newtheorem{cor}[subsection]{Corollaire}
\newtheorem{heur}{Heuristique}
\newtheorem{rem}{Remarque}
\newtheorem{note}{Note}
\newtheorem{concl}{Conclusion}
\newtheorem{res}{Résumé}
\newtheorem{strat}{Stratégie}

\theoremstyle{definition}
\newtheorem{conj}{Conjecture}
\newtheorem{prob}{Problème}
\newtheorem{quest}{Question}
\newtheorem{prot}{Protocole}
\newtheorem{algo}{Algorithme}
\newtheorem{defn}[subsection]{Définition}
\newtheorem{exmp}[subsection]{Exemples}
\newtheorem{exo}[subsection]{Exercices}
\newtheorem{ex}[subsection]{Exemple}
\newtheorem{exs}[subsection]{Exemples}

\theoremstyle{remark}

\definecolor{wgrey}{RGB}{148, 38, 55}
\definecolor{wgreen}{RGB}{100, 200,0} 
\hypersetup{
    colorlinks=true,
    linkcolor=wgreen,
    urlcolor=wgrey,
    filecolor=wgrey
}

\title{Théorie des nombres algorithmique}
\date{2024-2025}

\begin{document}
\maketitle
\tableofcontents

\chapter{Algorithmes}
\section{$\frac{3}{5}|0>+\frac{4}{5}|1>$}

C'est un exo à la con mais c'est instructif, on regarde
\begin{align*}
    |0>&\to \frac{1}{\sqrt 2}(|0>+|1>)\\
       &\to \frac{1}{\sqrt 2}(|0>+e^{i\theta}|1>)\\
       &\to \frac{1}{2}(|0>(1+e^{i\theta})+|1>(1-e^{i\theta}))\\
       &\to \frac{1}{2}(e^{i\theta /2}(e^{-i\theta /2}+e^{i\theta /2})
       +e^{i\theta /2}|1>(e^{-i\theta /2}-e^{i\theta /2}))
\end{align*}
et là suffit d'ajuster theta puis de refaire des phases shifts.

\section{$\frac{3}{5}|0>+\frac{4}{5}|1>$ et $\frac{5}{13}|0>-\frac{12}{13}|1>$}
Déjà par rapport à la section d'avant, on aurait juste pu faire
\[
\begin{quantikz}
    &\gate{H}&\gate{R_\theta}&\gate{X}&\gate{R_{-\theta}}&\gate{H}&
\end{quantikz}
\]
est c'est fini, on obtient 
\[|0>\mapsto cos(\theta)|0>+sin(\theta)|1>\]
et
\[|1>\mapsto -sin(\theta)|0>-cos(\theta)|1>\]
maintenant si $|0>\mapsto \frac{3}{5}|0>+\frac{4}{5}|1>$
et $|1>\mapsto \frac{5}{13}|0>-\frac{12}{13}|1>$ on peut voir
que sur $|0>+|1>$ ça préserve pas la norme.

\section{Deutsch-Josza}
Donc l'algorithme permet de décider si $f\colon 2^n\to 2$ est constante
ou équilibrée (Comme un morphisme de groupes $(\Z/2\Z)^n\to \F_2$).
\newline

En gros le point crucial c'est que sur $|0^n>|1>$
Si on fait $H^{\otimes (n+1)}$, $U_f$ puis $H^{\otimes n}$ on obtient :
\begin{align*}
    |0^n>|1>&\to \sum_{x\in 2^n}|x>(|0>-|1>)\\
            &\to \sum_{x\in 2^n}|x>(|f(x)>-|1\oplus f(x)>\\
            &\to \sum_{y\in 2^n}|y>\sum_{x\in 2^n}(-1)^{f(x)}(-1)^{x.y}\\
\end{align*}
En particulier $||q_{0^n}||=\sum_{x\in 2^n} (-1)^{f(x)}/2^n$. D'où
si $f$ est constant on obtient $0^n$ avec proba $1$, sinon proba $0$ 
d'avoir $0^n$.

\section{Simon}
Cette fois c'est plus fun, si on prends 
\[f\colon (\Z/2\Z)^n\to X\]
avec $X$ un ensemble fini, et si $f$ vérifie $f(x)=f(y)$ ssi $x=y$ ou
$x=y+a$ on aimerait trouver $a$. Essentielemment, si $f$ passe au
quotient en $<a>$ on veut trouver le "noyau". On regarde
\begin{center}
\begin{quantikz}
    &\gate{H^{\otimes n}}&\gate[2]{Uf}&\gate{H^{\otimes n}}&\meter{}&\\
    &&&&&
\end{quantikz}
\end{center}
À nouveau on fait rentrer $|0^{n+m}>$, on obtient 
\begin{align*}
    |0^{n+m}>&\to \frac{1}{2^{n/2}}\sum_{x\in 2^n} |x>|f(x)>\\
             &\to \frac{1}{2^n}\sum_{y\in 2^n}|y>\sum_{x\in 2^n}(-1)^{x.y}
             |f(x)>\\
\end{align*}
et on à $q_y=\sum_{x\in 2^n} (-1)^{x.y}|f(x)>$. Le claim c'est qu'on
obtient un vecteur $v\in \F_2^n$ uniformément distribué orthogonal
à $a$ en sortie. Ça se voit direct en regardant $y.a\mod 2$ :
\begin{align*}
    q_y&=\sum_{\bar x\in (\Z/2\Z)^n/<a>} (1+(-1)^{a.y})((-1)^{x.y}|f(x)>\\
\end{align*}
\begin{align*}
    a.y=1\mod 2 \to q_y&=\sum_{\bar x\in (\Z/2\Z)^n/<a>} (1+(-1)^{a.y})((-1)^{x.y}|f(x)>\\
                &=0\\
    a.y=0\mod 2 \to q_y&=\sum_{\bar x\in (\Z/2\Z)^n/<a>} (1+(-1)^{a.y})((-1)^{x.y}|f(x)>\\
                       &=\sum_{\bar x\in (\Z/2\Z)^n/<a>} ((-1)^{x.y}|f(x)>\\
\end{align*}
En particulier, y'a que les $y.a=0\mod 2$ qui ont une proba de sortir.
L'uniforme distribution est claire.

Pour obtenir $a$, on peut lancer l'algorithme jusqu'à obtenir une base
de $<a>^T$.
\begin{rem}
    La proba d'avoir une base en $m$ étapes se calcule bien, regarder
    la matrice des $m$ vecteurs colonnes. Calculer sur le rang 
    sur les lignes! On obtient une proba 
    $P_{d+k}\geq 1-\frac{1}{q^k(q-1)}=1-\frac{1}{2^k}$ avec $q=2$ ici
    et $d=n-1=dim <a>^T$.
\end{rem}

\section{Transormée de Fourier quantique}
On a une nouvelle porte, \[QFT(|x>)=\frac{1}{2^{n/2}}\sum_{y\in 2^n}
\zeta_{2^n}^{x.y}|y>.\]
Et on peut l'utiliser pour trouver la période d'une fonction! Y'a
une nuance dans la suite :
\newline
\begin{center}
\textbf{On part de $\Z/2^n\Z$ et pas $(\Z/2\Z)^n$, puis
$xy$ est le produit dans $\Z/2^n\Z$ pas le produit scalaire}.
\end{center}
\subsection{Cas $r=2^d$}
\begin{prob}
    $f\colon \F_2^n\to X$ telle qu'il existe $d\leq n$ t.q :
    \begin{enumerate}
        \item $f$ est $2^d$-périodique.
        \item $f(x)=f(y)$ ssi $2^d\mid x-y$.
    \end{enumerate}
\end{prob}
Le circuit
\begin{center}
\begin{quantikz}
    &\gate{H^{\otimes n}}&\gate[2]{Uf}&\gate{QFT^{\otimes n}}&\meter{}&\\
    &&&&&
\end{quantikz}
\end{center}
mesure un $|y>$ uniforme divisible par $2^{n-d}$. En particulier,
on a une manière de chopper une période de la forme
$2^d$ en itérant. Donc en gros comme d'hab on regarde :
\[\sum_{x\in \Z/2^n\Z} |x>|f(x)>\]
et on applique $QFT$ cette fois pour obtenir :
\[\sum_{x\in \Z/2^n\Z}\sum_{y\in \Z/2^n\Z}\zeta_{2^n}^{xy}|y>|f(x)>\]
et on pose
\[q_y=1/2^n\sum_{x\in \Z/2^n\Z}\zeta_{2^n}^{xy}|f(x)>\]
puis on écrit $x=a+2^db$, alors $f(x)=f(a)$. On peut montrer que
\begin{enumerate}
    \item si $2^{n-d}\mid y$ alors $||q_y||^2=1/2^d$. 
    \item sinon $||q_y||^2=0$.
\end{enumerate}


\textbf{Fait :} En gros, en prenant la QFT comme boite noire,
j'ai revu deutsh josza et simon. Puis Shor pour trouver les périodes
de la forme $2^k$. Ensuite pour une période de la forme $r$ générale,
on peut l'estimer à partir du circuit du dessus.
\textbf{À faire :}  Comprendre l'estimation d'abord, via les fractions
continues. Puis comprendre la QFT sans boite noire mais ca c'est moins
important. Faut savoir construire $|x>\mapsto |0>+\zeta_{2^n}^{x}|1>$.

\textbf{Finalement :} Ça vaut le coup de comprendre comment marche
la QFT pour le td ? Ou au moins pour l'algorithmique.

\subsection{Construire la porte QFT}
En gros on utilise la factorisation :
\begin{align*}
    QFT(|x>)&=\frac{1}{2^{n/2}}\sum_{y\in 2^n}\zeta_{2^n}^{xy}|y>\\
            &=\frac{1}{2^{n/2}}\sum_{y_0,\ldots,y_{n-1}\in \{0,1\}}
            \prod_{i=0,\ldots,{n-1}}
        \zeta_{2^n}^{x2^iy_i}|y_i>\\
            &=\frac{1}{2^{n/2}}\prod_{i=0,\ldots,n-1}(|0>+\zeta_{2^n}^{x2^i}|1>)\\
\end{align*}
en particulier, y suffit de savoir construire 
\[\frac{|0>+\zeta_{2^{n-i}}^{x}|1>}{\sqrt{2}}\]
Mais ça on peut le faire avec un controlled phase shift :
\begin{align*}
    \frac{|0>+\zeta_{2^{n-i}}^{x}|1>}{\sqrt{2}}&=\frac{|0>+\zeta_{2^{n-i}}^{x_0+\ldots+2^{n-i}x_i}|1>}{\sqrt{2}}\\
                                               &=\frac{|0>+\prod_{j=0}^{n-i}\zeta_{2^{n-i-j}}^{x_j}|1>}{\sqrt{2}}\\
\end{align*}


Au sens où la c'est juste une suite de phase shifts là où 
$x_j=1$.
\begin{concl}
    Là on a construit un circuit $Q_i$ qui créé le qbit du dessus
    qui agit que sur la $n-i-1$-ème entrée. En particulier pour
    appliquer la QFT, on fait successivement $Q_0$ puis $Q_1$, $\ldots$.
\end{concl}
\subsection{Le cas $r$ général}
Bon là on savait trouver une période de la forme $2^d$. Maintenant
on veut trouver une période quelconque. On va faire exactement
la même chose et voir ce qu'y se passe.
\begin{res}
    En gros, la sortie de Shor est une approximation de $k/r$ une 
    fraction proche à $2^{n+1}$, i.e. :
    \[|y/2^n - j/r|<1/2^{n+1}\]
    On connaît par $j/r$ mais on peut montrer avec une telle 
    approximation que il existe un convergent de $y$, $p_n/q_n$ tel
    que $p_n/q_n = j/r$!
\end{res}
On peut montrer avec des théorèmes sur les fractions continues que 
pour une sortie de Shor $y$, il existe $j$ tel que 
$|y/2^n-j/r|<1/2^{n+1}$ et par la théorie des fractions continues, $j/r$
est un convergent de $y$! I.e. y'a dans la fraction continue on peut
trouver $p_n/q_n$ t.q $p_n/q_n=j/r$, wow.

\begin{strat}
    Ducoup la stratégie c'est : 
    \begin{enumerate}
        \item Lancer Shor et obtenir $y$.
        \item Calculer les convergents de $y/2^n$, $(p_i/q_i)$.
        \item Vérifier si $f(0)=f(q_i)$ pour un $i$, si oui
            retourner $q_i$.
    \end{enumerate}
    À la dernière étape ce sera vrai par la théorie parce que une
    condition c'est $2^n>r^2$ et on le choisit comme ça.
\end{strat}

\section{Factorisation et log discret}
Pour la factorisation, la page wiki explique bien!
\begin{strat}
    En gros
    \begin{enumerate}
        \item Prendre un entier plus petit que $n$ premier à $N$, 
    disons $a$ (si il est pas premier on a fini).
        \item Calculer son ordre avec l'algo de Shor (on a besoin de 
            l'ordre.
        \item Si $r$ est pair, alors $N|(a^{r/2}-1)(a^{r/2}+1)$
            mais peut diviser aucun des deux donc a deux facteurs !
        \item Il doit y avoir un $r$ pair, parce que sinon $N=2^k$.
            Donc itérer.
    \end{enumerate}
\end{strat}
\end{document}


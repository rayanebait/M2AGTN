\documentclass[a4paper,12pt]{book}
\usepackage{amsmath,  amsthm,enumerate}
\usepackage{csquotes}
\usepackage[provide=*,french]{babel}
\usepackage[dvipsnames]{xcolor}
\usepackage{quiver, tikz}

\usepackage{quantikz}

%symbole caligraphique
\usepackage{mathrsfs}

%hyperliens
\usepackage{hyperref}

%pseudo-code
\usepackage{algorithm}
\usepackage{algpseudocode}

\usepackage{fancyhdr}

\pagestyle{fancy}
\addtolength{\headwidth}{\marginparsep}
\addtolength{\headwidth}{\marginparwidth}
\renewcommand{\chaptermark}[1]{\markboth{#1}{}}
\renewcommand{\sectionmark}[1]{\markright{\thesection\ #1}}
\fancyhf{}
\fancyfoot[C]{\thepage}
\fancyhead[LO]{\textit \leftmark}
\fancyhead[RE]{\textit \rightmark}
\renewcommand{\headrulewidth}{0pt} % and the line
\fancypagestyle{plain}{%
    \fancyhead{} % get rid of headers
}


\usepackage{appendix}
\renewcommand{\appendixpagename}{Annexe}

\definecolor{wgrey}{RGB}{148, 38, 55}

\setlength\parindent{24pt}

\newcommand{\Z}{\mathbb{Z}}
\newcommand{\R}{\mathbb{R}}
\newcommand{\rel}{\omathcal{R}}
\newcommand{\Q}{\mathbb{Q}}
\newcommand{\C}{\mathbb{C}}
\newcommand{\N}{\mathbb{N}}
\newcommand{\K}{\mathbb{K}}
\newcommand{\A}{\mathbb{A}}
\newcommand{\B}{\mathcal{B}}
\newcommand{\Or}{\mathcal{O}}
\newcommand{\F}{\mathbb F}
\newcommand{\Hom}{\textrm{Hom}}
\newcommand{\disc}{\textrm{disc}}
\newcommand{\Pic}{\textrm{Pic}}
\newcommand{\End}{\textrm{End}}
\newcommand{\Spec}{\textrm{Spec}}

\newcommand{\cL}{\mathscr{L}}
\newcommand{\G}{\mathscr{G}}
\newcommand{\D}{\mathscr{D}}
\newcommand{\E}{\mathscr{E}}
\renewcommand{\H}{\mathscr{H}}

\theoremstyle{plain}
\newtheorem{thm}[subsection]{Théoreme}
\newtheorem{lem}[subsection]{Lemme}
\newtheorem{prop}[subsection]{Proposition}
\newtheorem{cor}[subsection]{Corollaire}
\newtheorem{heur}{Heuristique}
\newtheorem{rem}{Remarque}
\newtheorem{note}{Note}

\theoremstyle{definition}
\newtheorem{conj}{Conjecture}
\newtheorem{prob}{Problème}
\newtheorem{quest}{Question}
\newtheorem{prot}{Protocole}
\newtheorem{algo}{Algorithme}
\newtheorem{defn}[subsection]{Définition}
\newtheorem{exmp}[subsection]{Exemples}
\newtheorem{exo}[subsection]{Exercices}
\newtheorem{ex}[subsection]{Exemple}
\newtheorem{exs}[subsection]{Exemples}

\theoremstyle{remark}

\definecolor{wgrey}{RGB}{148, 38, 55}
\definecolor{wgreen}{RGB}{100, 200,0} 
\hypersetup{
    colorlinks=true,
    linkcolor=wgreen,
    urlcolor=wgrey,
    filecolor=wgrey
}

\title{Théorie des nombres algorithmique}
\date{2024-2025}

\begin{document}
\maketitle
\tableofcontents

\chapter{Algorithmes}
\section{$\frac{3}{5}|0>+\frac{4}{5}|1>$}

C'est un exo à la con mais c'est instructif, on regarde
\begin{align*}
    |0>&\to \frac{1}{\sqrt 2}(|0>+|1>)\\
       &\to \frac{1}{\sqrt 2}(|0>+e^{i\theta}|1>)\\
       &\to \frac{1}{2}(|0>(1+e^{i\theta})+|1>(1-e^{i\theta}))\\
       &\to \frac{1}{2}(e^{i\theta /2}(e^{-i\theta /2}+e^{i\theta /2})
       +e^{i\theta /2}|1>(e^{-i\theta /2}-e^{i\theta /2}))
\end{align*}
et là suffit d'ajuster theta puis de refaire des phases shifts.

\section{Deutsch-Josza}
Donc l'algorithme permet de décider si $f\colon 2^n\to 2$ est constante
ou équilibrée (Comme un morphisme de groupes $(\Z/2\Z)^n\to \F_2$).
\newline

En gros le point crucial c'est que sur $|0^n>|1>$
Si on fait $H^{\otimes (n+1)}$, $U_f$ puis $H^{\otimes n}$ on obtient :
\begin{align*}
    |0^n>|1>&\to \sum_{x\in 2^n}|x>(|0>-|1>)\\
            &\to \sum_{x\in 2^n}|x>(|f(x)>-|1\oplus f(x)>\\
            &\to \sum_{y\in 2^n}|y>\sum_{x\in 2^n}(-1)^{f(x)}(-1)^{x.y}\\
\end{align*}
En particulier $||q_{0^n}||=\sum_{x\in 2^n} (-1)^{f(x)}/2^n$. D'où
si $f$ est constant on obtient $0^n$ avec proba $1$, sinon proba $0$ 
d'avoir $0^n$.

\section{Simon}
Cette fois c'est plus fun, si on prends 
\[f\colon (\Z/2\Z)^n\to X\]
avec $X$ un ensemble fini, et si $f$ vérifie $f(x)=f(y)$ ssi $x=y$ ou
$x=y+a$ on aimerait trouver $a$. Essentielemment, si $f$ passe au
quotient en $<a>$ on veut trouver le "noyau". On regarde
\begin{center}
\begin{quantikz}
    &\gate{H^{\otimes n}}&\gate[2]{Uf}&\gate{H^{\otimes n}}&\meter{}&\\
    &&&&&
\end{quantikz}
\end{center}
À nouveau on fait rentrer $|0^{n+m}>$, on obtient 
\begin{align*}
    |0^{n+m}>&\to \frac{1}{2^{n/2}}\sum_{x\in 2^n} |x>|f(x)>\\
             &\to \frac{1}{2^n}\sum_{y\in 2^n}|y>\sum_{x\in 2^n}(-1)^{x.y}
             |f(x)>\\
\end{align*}
et on à $q_y=\sum_{x\in 2^n} (-1)^{x.y}|f(x)>$. Le claim c'est qu'on
obtient un vecteur $v\in \F_2^n$ uniformément distribué orthogonal
à $a$ en sortie. 





\end{document}


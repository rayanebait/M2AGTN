\documentclass[a4paper,12pt]{book}
\usepackage{amsmath,  amsthm,enumerate}
\usepackage{csquotes}
\usepackage[provide=*,french]{babel}
\usepackage[dvipsnames]{xcolor}
\usepackage{quiver, tikz}
\usepackage{quantikz}

%symbole caligraphique
\usepackage{mathrsfs}

%hyperliens
\usepackage{hyperref}

%pseudo-code
\usepackage{algorithm}
\usepackage{algpseudocode}

\usepackage{fancyhdr}

\pagestyle{fancy}
\addtolength{\headwidth}{\marginparsep}
\addtolength{\headwidth}{\marginparwidth}
\renewcommand{\chaptermark}[1]{\markboth{#1}{}}
\renewcommand{\sectionmark}[1]{\markright{\thesection\ #1}}
\fancyhf{}
\fancyfoot[C]{\thepage}
\fancyhead[LO]{\textit \leftmark}
\fancyhead[RE]{\textit \rightmark}
\renewcommand{\headrulewidth}{0pt} % and the line
\fancypagestyle{plain}{%
    \fancyhead{} % get rid of headers
}

%bibliographie
\usepackage[
backend=biber,
style=alphabetic,
sorting=ynt
]{biblatex}

\addbibresource{bib.bib}

\usepackage{appendix}
\renewcommand{\appendixpagename}{Annexe}

\definecolor{wgrey}{RGB}{148, 38, 55}

\setlength\parindent{24pt}

\newcommand{\Z}{\mathbb{Z}}
\newcommand{\R}{\mathbb{R}}
\newcommand{\rel}{\omathcal{R}}
\newcommand{\Q}{\mathbb{Q}}
\newcommand{\C}{\mathbb{C}}
\newcommand{\N}{\mathbb{N}}
\newcommand{\K}{\mathbb{K}}
\newcommand{\A}{\mathbb{A}}
\newcommand{\B}{\mathcal{B}}
\newcommand{\Or}{\mathcal{O}}
\newcommand{\F}{\mathbb F}
\newcommand{\Hom}{\textrm{Hom}}
\newcommand{\disc}{\textrm{disc}}
\newcommand{\Pic}{\textrm{Pic}}
\newcommand{\End}{\textrm{End}}
\newcommand{\Spec}{\textrm{Spec}}

\newcommand{\cL}{\mathscr{L}}
\newcommand{\G}{\mathscr{G}}
\newcommand{\D}{\mathscr{D}}
\newcommand{\E}{\mathscr{E}}
\renewcommand{\H}{\mathscr{H}}

\theoremstyle{plain}
\newtheorem{thm}[subsection]{Théoreme}
\newtheorem{lem}[subsection]{Lemme}
\newtheorem{prop}[subsection]{Proposition}
\newtheorem{cor}[subsection]{Corollaire}
\newtheorem{heur}{Heuristique}
\newtheorem{rem}{Remarque}
\newtheorem{note}{Note}

\theoremstyle{definition}
\newtheorem{conj}{Conjecture}
\newtheorem{prob}{Problème}
\newtheorem{quest}{Question}
\newtheorem{prot}{Protocole}
\newtheorem{algo}{Algorithme}
\newtheorem{defn}[subsection]{Définition}
\newtheorem{exmp}[subsection]{Exemples}
\newtheorem{exo}[subsection]{Exercices}
\newtheorem{ex}[subsection]{Exemple}
\newtheorem{exs}[subsection]{Exemples}

\theoremstyle{remark}

\definecolor{wgrey}{RGB}{148, 38, 55}
\definecolor{wgreen}{RGB}{100, 200,0} 
\hypersetup{
    colorlinks=true,
    linkcolor=wgreen,
    urlcolor=wgrey,
    filecolor=wgrey
}

\title{Théorie des nombres algorithmique}
\date{2024-2025}

\begin{document}
\maketitle
\tableofcontents
\chapter*{Introduction}

Le cours discute l'algorithmique quantique et le but c'est l'algo de 
Shor \cite{Shor_1997}!

\section*{Révolution du XXe en physique}
Experience de Young :
\begin{itemize}
    \item Un atome par un à travers les fentes de Young. Les atomes
        semblent interférer avec les autres anciencs atomes.
\end{itemize}
Invariance de la vitesse de la lumière par rapport
au référentiel (Boltzmann$\to$Einstein).

Expèrience de Dirac :
\begin{itemize}
    \item Polarisation des photons : plan polarisé agit comme un produit
        scalaire pour laisser passer la lumière.
\end{itemize}
\section*{Heisenberg-Schrödinger}
L'état d'un système physique est décrit par une fonction d'onde (vecteur
unitaire) d'un espace de Hilbert dépendant du système. Pour 
\begin{itemize}
    \item une seule
particule, l'espace de Hilbert associé à une particule est 
\[\H_1=L^2(\R^3,\C)\]
Où la probabilité de position dans l'espace de la particule.
    \item Pour deux particules
\[\H_2=L^2(\R^3\times \R^3,\C)\]
qui n'est pas isomorphe à $\H_1\times \H_1$, l'idée est que les deux 
particules peuvent être intriquées, par contre
\[\H_2=\H_1\otimes \H_2\]
    \item Pour une polarisation : $\H=\C^2=\C|\uparrow>\oplus \C|\to>$.
    \item Pour deux polarisations : $\H=\C^4=\C|\uparrow\uparrow>\oplus \C|\uparrow\to>\otimes \C|\to\uparrow>\otimes \C|\to\to>$ 
    \item Pour $n$ polarisations : $\C^{2^n}$ avec $2^n$ états.
\end{itemize}

Si on regarde maintenant l'équation de Schrödinger

\[\frac{d|\psi>}{dt}=H|\psi>\]
où $H$ est l'Hamiltonien $H\colon \H\to \H$ qui est linéaire 
auto-adjoint. On a la notion de mesure encodée par un observable
\[O\colon \H\to\H\]
auto-adjoint, on a $|\psi>\in\H=\bigoplus_{\lambda}\H_{\lambda}$ et
\begin{align*}
    |\psi>&=\sum_{\lambda}\psi_{\lambda}\\
          &=\sum_{\lambda}||\psi_{\lambda}^2||=1
\end{align*}
Ce que ca dit c'est que deux mesures du même système peuvent être 
différentes, et surtout, quand on \textit{mesure}, on est projetés
sur un état et les nouvelles mesures sont projetées au même endroit.
On obtient $\lambda$ en sortie avec $||\psi_{\lambda}||^2$. Après
la mesure, la fonction d'onde devient 
$\frac{\psi_{\lambda}}{||\psi_{\lambda}||^2}$.

Avec l'exemple d'une particule, $\H=L^2(\R^3,\C)$ :
\begin{itemize}
    \item La position en $x$ : $O\colon \H\to\H;f\mapsto x.f$.
    \item La vitesse en $x$ : $O\colon \H\to\H;f\mapsto \partial f/\partial x$.
\end{itemize}
!!! Pour avoir des mesures cohérentes faut que les observables commutent.
Et la par exemple les deux commutent pas!

Aussi, Einstein Podoloski ROsen (EPR) croyaient pas à ce formalisme.
La raison c'est la fonction à trappe $|\psi>=|\uparrow\uparrow>+|\to\to>$
qui au moment de la mesure d'une des deux polarisations on projette sur
un des deux facteurs de sorte que la deuxième polarisation doit être la
même! (Key agreement existe!!)


\section*{Références pour le cours}
Un livre : N. Mermin, Quantum Computer Science, an introduction.
\chapter{L'ordinateur quantique}
\section{Les qubits (q-bits)}
\begin{defn}
    Soit $n\geq 1$. Un $n$-qubit est une somme formelle de la forme
    \[w\in {0,1}^na_w|w>\]
    avec $a_w\in \C$.
    Il est normalisé si $\sum_{w\in\{0,1\}^n}|a_w|^2=1$.
\end{defn}
\begin{ex}
    Si $n=1$, on peut avoir $\alpha|0>+\beta|1>$, $\alpha,\beta\in\C$.
\end{ex}

L'ensemble des $n$-qubits est un espace de Hilbert de dimension $2^n$,
i.e. c'est un $\C$-espace vectoriel avec un produit hermitien : 
$<\sum_w a_w|w>,\sum_wb_w|w>>=\sum_w \bar a_wb_w$ où la base est supposée
orthogonale. On a aussi un produit tensoriel donné par la concaténation 
des chaines de bits.
\begin{rem}
    Le $2$-qubit \[|00>+|11>\] n'est pas un produit de $1$-qubits!
    Cet état est dit intriqué, l'idée est que sinon on peut réduire 
    le calcul à celui sur les $1$-qubits.
\end{rem}
\section{Les portes}
Pour pouvoir imaginer des portes utilisables en pratique, on doit avoir
des portes qui sont des isomorphismes et des isométries;
\begin{defn}
    La porte $X$ inverse $|0>$ et $|1>$.
\end{defn}
\begin{defn}
    La porte CX, où C est pour \textit{controlled} tel que
    \begin{itemize}
        \item $|00>\mapsto |00>$
        \item $|01>\mapsto |01>$
        \item $|10>\mapsto |11>$
        \item $|11>\mapsto |01>$
    \end{itemize}
\end{defn}
Faut imaginer que le premier bit agit sur le deuxième par une porte
$X$ si c'est $1$.
\begin{defn}
    La porte CCX,
    \begin{itemize}
        \item $|00x>\mapsto |00x>$
        \item $|01x>\mapsto |01x>$
        \item $|10x>\mapsto |11x>$
        \item $|110>\mapsto |111>$
        \item $|111>\mapsto |110>$
    \end{itemize}
\end{defn}
\noindent Pareil qu'avant mais avec les deux premiers bits. Apparemment 
on a tout
les algorithmes classiques avec ces trois portes. La prochaine est non
classique.
\begin{defn}
    La porte de Hadamard H, 
    \begin{itemize}
        \item $|0>\mapsto \frac{|0>+|1>}{2}$
        \item $|1>\mapsto \frac{|0>-|1>}{2}$
    \end{itemize}
    avec lequel on obtient des états intriqués.
\end{defn}
\begin{defn}
    La porte de changement de phase $R_{\theta}$ avec $\theta\in \R$,
    \begin{itemize}
        \item $|0>\mapsto |0>$
        \item $|1>\mapsto e^{i\theta}|1>$
    \end{itemize}
\end{defn}

\begin{defn}[\textbf{Porte booléenne}]
    Soit $f\colon \{0,1\}^{n}\to \{0,1\}^m$, on déf la porte 
    \[U_f(|x>|y>)=|x>|y\oplus f(x)>\]
    sur une base, puis on étend par linéarité.
\end{defn}
\begin{rem}
    C'est bien une bijection peu importe $f$.
\end{rem}

\section{Circuits}
Un fil est censé représenter une entrée, en pratique on met un fil
pour une entrée de $n$-bits mais vaudrait mieux en mettre $n$ pour les
portes CX par exemple.
\begin{exs}
    Un exemple avec la porte hadamard :\\
    \begin{center}
    \begin{quantikz}
        &\gate{H}& \ctrl[vertical]{1}& \\
        & & \gate{X}&\\
    \end{quantikz}
    \end{center}
    On a 
    \[|00>\mapsto \frac{|0>+|1>}{\sqrt 2}|0>\mapsto
    \frac{|00>+|11>}{\sqrt 2}\]
    Autrement dit on peut créer l'état EPR avec deux portes, c'est la
    force d'un ordinateur quantique! 
    La deuxième partie du calcul c'est parce que c'est une porte CX.
    On a aussi \[|01>\mapsto \frac{|01>+|10>}{\sqrt 2}.\] Les portes
    Hadamard, X, CX ont ordre $2$ et $R_{\theta}$ est additive.
\end{exs}
\begin{ex}
    Attention au diagramme :
    \begin{center}
    \begin{quantikz}
        &\gate{G}&& \\
        & & \gate{G'}&\\
    \end{quantikz}
    \end{center}
    qui équivaut à 
    \begin{center}
    \begin{quantikz}
        & & \gate{G}&\\
        &\gate{G'}&& \\
    \end{quantikz}
    \end{center}
    Le point c'est juste que $G\colon |x>|y>\mapsto G(|x>)|y>$ et pas
    $G\colon |x>|y>\mapsto |G(x)>|y>$ par exemple si G est la porte 
    Hadamard.
\end{ex}
\section{Mesures}
Ça a l'air compliqué mdr. On assume qu'on a un $n$-qubit normalisé
$q$. On veut mesurer le premier bit. Faut rappeler que 
\[q=\sum_{w\in \{0,1\}^n} a_w|w>\]
ducoup c'est pas clair \textit{mesurer le premier bit}. On réecrit
\[q=|0>\otimes q_0 + |1>\otimes q_1\], on peut le faire là on a juste
distingué les deux premierrs bits. On remarque qu'alors
\[||q_0||^2+||q_1||^2=||q||^2=1\]
La mesure est censée donner $0$ avec probabilité $||q_0||^2$ et donner
$1$ avec probabilité $||q_1||^2$.

À ce stade $q$ est détruit et on a remplacé $q$ par 
$|i>\otimes\frac{q_i}{||q_i||^2}$. Autrement dit, si on refait la mesure 
on obtient $i$ avec probabilité $1$.

\begin{defn}[Mesure]
    On écrit 
    \begin{center}
    \begin{quantikz}
        &\meter{}&& \\
        & & &\\
    \end{quantikz}
    \end{center}
    Pour dire qu'on fait une mesure.
\end{defn}
\begin{prop}]
    Les mesures de premier et deuxième bit
    \begin{center}
    \begin{quantikz}
        &\meter{}&& \\
        &&\meter{}& \\
        &&& \\
    \end{quantikz}
    \end{center}
    et
    \begin{center}
    \begin{quantikz}
        &&\meter{}& \\
        &\meter{}&& \\
        &&& \\
    \end{quantikz}
    \end{center}
    sont équivalentes.
\end{prop}
\begin{proof}[Preuve]
    Si on écrit $q=|00>\otimes q_{00} +|01>\otimes q_{01} + |10>\otimes q_{10}+ |11>\otimes q_{11}$, on peut écrire :\\
% https://q.uiver.app/#q=WzAsNyxbMiwwLCJxIl0sWzEsMSwiXFxmcmFje3wwMD5cXG90aW1lcyBxX3swMH0rfDAxPlxcb3RpbWVzIHFfezAxfX17XFxzcXJ0e3x8cV97MDB9fHxeMit8fHFfezAxfXx8XjJ9fSJdLFszLDEsIlxcZnJhY3t8MTA+XFxvdGltZXMgcV97MTB9K3wxMT5cXG90aW1lcyBxX3sxMX19e3tcXHNxcnR7fHxxX3sxMH18fF4yK3x8cV97MTF9fHxeMn19fSJdLFswLDIsIlxcZnJhY3t8MDA+XFxvdGltZXMgcV97MDB9fXt8fHFfezAwfXx8fSJdLFsxLDAsIlxcdGV4dHJte1Byb2JhIDp9IHx8cV97MDB9fHxeMit8fHFfezAxfXx8XjIiXSxbMCwxLCJcXHRleHRybXtQcm9iYSA6fSBcXGZyYWN7fHxxX3swMH18fH17fHxxX3swMH18fF4yK3x8cV97MDF9fHxeMn0iXSxbMiwyLCJcXGZyYWN7fDAwPlxcb3RpbWVzIHFfezAxfX17fHxxX3swMX18fH0iXSxbMCwxXSxbMCwyXSxbMSwzXSxbMSw2XSxbNCw3LCIiLDAseyJzaG9ydGVuIjp7InRhcmdldCI6MjB9LCJzdHlsZSI6eyJib2R5Ijp7Im5hbWUiOiJzcXVpZ2dseSJ9fX1dLFs1LDksIiIsMCx7InNob3J0ZW4iOnsidGFyZ2V0IjoyMH0sInN0eWxlIjp7ImJvZHkiOnsibmFtZSI6InNxdWlnZ2x5In19fV1d
\[\begin{tikzcd}[ampersand replacement=\&,cramped]
	\& {\textrm{Proba :} ||q_{00}||^2+||q_{01}||^2} \& q \\
	{\textrm{Proba :} \frac{||q_{00}||}{||q_{00}||^2+||q_{01}||^2}} \& {\frac{|00>\otimes q_{00}+|01>\otimes q_{01}}{\sqrt{||q_{00}||^2+||q_{01}||^2}}} \&\& {\frac{|10>\otimes q_{10}+|11>\otimes q_{11}}{{\sqrt{||q_{10}||^2+||q_{11}||^2}}}} \\
	{\frac{|00>\otimes q_{00}}{||q_{00}||}} \&\& {\frac{|00>\otimes q_{01}}{||q_{01}||}}
	\arrow[""{name=0, anchor=center, inner sep=0}, from=1-3, to=2-2]
	\arrow[from=1-3, to=2-4]
	\arrow[""{name=1, anchor=center, inner sep=0}, from=2-2, to=3-1]
	\arrow[from=2-2, to=3-3]
	\arrow[shorten >=3pt, Rightarrow, squiggly, from=1-2, to=0]
	\arrow[shorten >=4pt, Rightarrow, squiggly, from=2-1, to=1]
\end{tikzcd}\]

    En particulier, on a $ij$ avec proba $||q_{ij}||^2$!
\end{proof}

\begin{prop}
    Les deux circuits
    \begin{center}
    \begin{quantikz}
        &\ctrl[vertical]{1}&\meter{}& \\
        & \gate{G} &&\\
    \end{quantikz}
    \end{center}
    et
    \begin{center}
    \begin{quantikz}
        &\meter{}&& \\
        & & \gate{Gx} &\\
    \end{quantikz}
    \end{center}
    sont équivalents. Où $Gx$ est donnée par 
    \begin{center}
    \begin{quantikz}
        &  \gate{G} &\\
    \end{quantikz}
    \end{center}
    Si $x=1$ et 
    \begin{center}
    \begin{quantikz}
        &  &\\
    \end{quantikz}
    \end{center}
    sinon.
\end{prop}

\chapter{Premiers algorithmes quantiques}
\section{L'algorithme Deutsch-Jozsa}
\begin{prob}
    Étant donné $f\colon \{0,1\}\to \{0,1\}$ décider si $f(0)=f(1)$.
\end{prob}
L'idée c'est que dans le monde classique, on est obligés de calculer
$f(0)$ et $f(1)$ alors que dans le monde quantique on peut le faire en
une fois. La fonction $f$ est représentée par la porte 
\begin{center}
\begin{quantikz}
    && \gate[2]{U_f} &\\
    && &\\
\end{quantikz}
\end{center}
\begin{prop}[Circuit Deutsch]
    Le circuit 
    \begin{center}
        \begin{quantikz}
        & \gate{H}&\gate[2]{U_f}&\gate{H}&\meter{} &\\
        & \gate{H}&&&&\\
    \end{quantikz}
    \end{center}
résoud le problème précédent.
\end{prop}
\begin{proof}[Preuve]
    On fait rentrer $|0>$ sur les deux inputs. On a à l'étape $1$
    \begin{align*}
    |0>|0>\mapsto &\big(\frac{|0>+|1>}{\sqrt 2}\big)\otimes\big(\frac{|0>-|1>}{\sqrt 2}\big)\\
    =&\frac{1}{2}(|00>-|01>+|10>-|11>)
    \end{align*}
    Puis à l'étape $2$
    \begin{align*}
        U_f(|x>|y>)&=|x>|y\oplus f(x)>\\
                   &=\frac{1/2}(|0>|f(0)>-|0>|\bar f(0)>\\
                   &+|1>|f(1)>-|1>|\bar f(1)>)
    \end{align*}
    À l'étape $3$ on a un gros truc 
    \begin{align*}
        \frac{1}{2\sqrt 2}&(|0>|f(0)>+|1>|f(0)>\\
                       &-|0>|\bar f(0)>-|1>|\bar f(0)>\\
                       &+|0>|f(1)>-|1>|f(1)>\\
                       &-|0>|\bar f(1)>+|1>|\bar f(1)>)
    \end{align*} 
    Si $f(0)=f(1)=a$, on obtient 
    \begin{align*}
        \frac{1}{2\sqrt 2}&([0a>+|1a>-|0\bar a>-|1\bar a>\\
        &+|0a>-|1a>-|0\bar a>+|1\bar a>)\\
        &=\frac{1}{2\sqrt 2}(|0a>-|0\bar a>)
    \end{align*}
    Sinon avec $f(0)=a$ et $f(1)=\bar a$, on obtient 
    \[\frac{1}{\sqrt 2}(|1a>-|1\bar a>)\]
    On mesure donc le premier bit à $0$ avec probabilité $1$ si 
    $f(0)=f(1)$ et on mesure $1$ avec probabilité $1$ si 
    $f(0)=\bar(f(1))$.
\end{proof}
\begin{prob}
    Étant donné une fonction booléenne $f\colon \{0,1\}^n\to \{0,1\}$
    avec la promesse que $f$ est constante, ou équilibrée
    $\#f^{-1}(0)=\#f^{-1}(1)$. Décider si $f$ est équilibrée ou 
    constante.
\end{prob}
Dans le cas classique, on est plus ou moins obligé de tester $f$ 
sur la moitié des entrées. Dans le cas quantique, on peut le faire en
une évaluation.

\printbibliography

\end{document}


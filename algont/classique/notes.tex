\documentclass[a4paper,12pt]{book}
\usepackage{amsmath,  amsthm,enumerate}
\usepackage{csquotes}
\usepackage[provide=*,french]{babel}
\usepackage[dvipsnames]{xcolor}
\usepackage{quiver, tikz}

%symbole caligraphique
\usepackage{mathrsfs}

%hyperliens
\usepackage{hyperref}

%pseudo-code
\usepackage{algorithm}
\usepackage{algpseudocode}

\usepackage{fancyhdr}

\pagestyle{fancy}
\addtolength{\headwidth}{\marginparsep}
\addtolength{\headwidth}{\marginparwidth}
\renewcommand{\chaptermark}[1]{\markboth{#1}{}}
\renewcommand{\sectionmark}[1]{\markright{\thesection\ #1}}
\fancyhf{}
\fancyfoot[C]{\thepage}
\fancyhead[LO]{\textit \leftmark}
\fancyhead[RE]{\textit \rightmark}
\renewcommand{\headrulewidth}{0pt} % and the line
\fancypagestyle{plain}{%
    \fancyhead{} % get rid of headers
}

%bibliographie
\usepackage[
backend=biber,
style=alphabetic,
sorting=ynt
]{biblatex}

\addbibresource{bib.bib}

\usepackage{appendix}
\renewcommand{\appendixpagename}{Annexe}

\definecolor{wgrey}{RGB}{148, 38, 55}

\setlength\parindent{24pt}

\newcommand{\Z}{\mathbb{Z}}
\newcommand{\R}{\mathbb{R}}
\newcommand{\rel}{\omathcal{R}}
\newcommand{\Q}{\mathbb{Q}}
\newcommand{\C}{\mathbb{C}}
\newcommand{\N}{\mathbb{N}}
\newcommand{\K}{\mathbb{K}}
\newcommand{\A}{\mathbb{A}}
\newcommand{\B}{\mathcal{B}}
\newcommand{\Or}{\mathcal{O}}
\newcommand{\F}{\mathbb F}
\newcommand{\Hom}{\textrm{Hom}}
\newcommand{\disc}{\textrm{disc}}
\newcommand{\Pic}{\textrm{Pic}}
\newcommand{\End}{\textrm{End}}
\newcommand{\Spec}{\textrm{Spec}}

\newcommand{\cL}{\mathscr{L}}
\newcommand{\G}{\mathscr{G}}
\newcommand{\D}{\mathscr{D}}
\newcommand{\E}{\mathscr{E}}
\renewcommand{\H}{\mathscr{H}}

\theoremstyle{plain}
\newtheorem{thm}[subsection]{Théoreme}
\newtheorem{lem}[subsection]{Lemme}
\newtheorem{prop}[subsection]{Proposition}
\newtheorem{cor}[subsection]{Corollaire}
\newtheorem{heur}{Heuristique}
\newtheorem{rem}{Remarque}
\newtheorem{note}{Note}

\theoremstyle{definition}
\newtheorem{conj}{Conjecture}
\newtheorem{prob}{Problème}
\newtheorem{quest}{Question}
\newtheorem{prot}{Protocole}
\newtheorem{algo}{Algorithme}
\newtheorem{defn}[subsection]{Définition}
\newtheorem{exmp}[subsection]{Exemples}
\newtheorem{exo}[subsection]{Exercices}
\newtheorem{ex}[subsection]{Exemple}
\newtheorem{exs}[subsection]{Exemples}

\theoremstyle{remark}

\definecolor{wgrey}{RGB}{148, 38, 55}
\definecolor{wgreen}{RGB}{100, 200,0} 
\hypersetup{
    colorlinks=true,
    linkcolor=wgreen,
    urlcolor=wgrey,
    filecolor=wgrey
}

\title{Théorie des nombres algorithmique\\ \small{(Aspects classiques)}}
\date{2023-2024}

\begin{document}
\maketitle
\tableofcontents
\chapter*{Introduction}

Le cours discute l'algorithmique quantique et le but c'est l'algo de 
Shor \cite{Shor_1997}!

\chapter{Formalisme}
\section{Automates finis et langages}
Un alphabet est un ensemble de symboles, on regarde en général $\Sigma=
\{0,1\}$ les binaires. Ensuite y'a le langage élémentaire :
\[\{\emptyset, \in, 0, 1\}\]
À partir du langage élémentaire on construit les langages régulier,
par concaténations et unions finies.

\begin{defn}[Langage]
    On prends comme convention que les sous ensembles 
    \[L\subset\Sigma^*\]
    où $\Sigma = \{0,1\}$.
\end{defn}

Pour les automates on prends des $5$-tuples $(Q,\Sigma,\delta, q_0, F)$
où $Q$ est un ensemble d'états $\Sigma$ l'alphabet, 
\[\delta\colon Q\times \Sigma \to Q\]
une fonction de transition, $q_0$ l'état initial et $F$ l'ensemble des
états acceptés/terminaux. On étant ensuite $\delta$ en 
    \[\delta^*\colon Q\times \Sigma^*\to Q\]
    par $\delta^*(q, w)=\delta^*(\delta(q, w_n),w_0\ldots w_{n-1}$ avec
    $w_=w_0\ldots w_n$.
Le truc fun c'est qu'on peut déf le langage accepté par l'automate par :
\[L_{\delta}=\{w\in \Sigma^*| \delta^*(q_0,w)\in F\}.\]



\section{Machines de Turing}
En gros c'est un automate fini plus une tape infinie à droite et une
tête de lecture qui écrit et efface sur la tape.

\begin{defn}[Machine de Turing]
    Une machine de Turing est un tuple $(\Sigma, K, S,s)$
    avec $S\colon K\times \Sigma\to (\K\cup\{Y, N, H\}\times \Sigma \times \{\bullet,\leftarrow,\rightarrow\}$
    où on a les états... Un langage $L\subset \Sigma^*$ est accepté par $M$
    ssi $w\in L\leftrightarrow$ la machine s'arrête sur $Y$.
\end{defn}
\begin{defn}[Langage décidable]
    Un langage est décidable si il existe une machine de Turing qui 
    l'accepte.
\end{defn}
\begin{defn}[Fonction récursive]
    Fonction qui est calculable par une machine de Turing.
\end{defn}
Étant donné une fonction $f\colon \N\to \N$.
\begin{defn}
    On dit qu'une machine de Turing a complexité $O(f)$ si elle termine
    en temps $f(|n|)$ pour une entrée $n$ de taille $|n|$.
\end{defn}
\begin{defn}[PTIME]
    Dans l'ensemble des langages $2^{\Sigma^*}$ on regarde PTIME
    l'ensemble des langages décidables de complexité polynomial.
\end{defn}
\begin{defn}[FPTIME]
    Dans l'ensemble des fonctions $(\Sigma^*)^{\Sigma^*}$
    on déf $FPTIME$ l'analogue pour les fonctions.
\end{defn}

\section{Exponentiation rapide}
\begin{algorithm}
    \caption{Calcul de $a^e\mod N$}
\begin{algorithmic}[1]
    \State Écrire $e=\sum e_i 2^i$.
    \State Calculer et enregistrer $a^{2^i}\mod N$ en réduisant à
    chaque carré par $N$ pour les $e_i\ne 0$.
    \State Multiplier  $\prod_i a^{2^i}=a^e\mod N$.
\end{algorithmic}
\end{algorithm}

\chapter{Arithmétique et modules}
\begin{thm}
    Soit $r\leq 1$ et $M\leq \Z^r$ alors il existe $a_1,\ldots,a_s$ avec 
    $0\leq s\leq r$ et une base $v_1,\ldots, v_r$ de $\Z^r$ telle que 
    $a_1\mid\ldots\mid a_s$ et \[M=\bigoplus_i^s a_i v_i\]
\end{thm}

Grâce à lui on peut résoudre un système linéaire $AX=0$ en calculant
une base de $\ker(A)$.

\begin{thm}
    Soit $G$ un groupe abélien de type fini. Alors il existe
    $a_1\mid\ldots\mid a_s$ t.q \[G\simeq \Z^r\oplus \bigoplus_{i=1}^r
    (\Z/a_i\Z)\] et la décomposition est unique.
\end{thm}

\section{Miller-Rabin}
Un nombre $n\geq 2$ est pseudo-premier si 
$a^{n-1}\equiv 1\mod n$ pour tout $a\in (\Z/n\Z)^{\times}$.
On peut détecter si $n$ est composé par contre.

\begin{thm}
    Si $n$ est impair premier, alors $n-1=2^k.m$
    et pour tout $a\in(\Z/n\Z)^{\times}$ alors 
    $a^{m}=1$ ou $a^{m2^c}=-1$ pour un $c\leq k$.
\end{thm}

\begin{thm}
    Si $n\geq 15$ est composé et impair alors
    MR(n,a) est faux pour plus de $\varphi(n)/4$
    éléments de $(\Z/n\Z)^{\times}$.
\end{thm}

\section{Distributions}
Si $X\colon\Omega\to G$ une variable aléatoire de loi uniforme.
Et $Y$ une loi quelconque, alors $Z=X.Y$ suis une loi uniforme!

\section{Densité de premiers}
Soit $A\subset \R$ alors 
\[\pi(A)=\#\{n|1\leq n\leq A,~n\textrm{ est premier}\}\]
a un équivalent 
\[\pi(A)=\frac{A}{ln(A)}(1+o(1))\]
C'est Hadamard, de la vallée Poussin (1900).

\section{Racines carrées dans $\F_p^{\times}$}
On note $p-1=2*q$ si $(a,p)=1$ et $a^q=1$. On regarde le cas où 
$p\equiv 3\mod 4$. Si on note $l\equiv 2^{-1} \mod q$ alors on écrit
$q=2l-1$. Soit $S$ les carrés de $\F_p^{\times}$, $S$ est cyclique 
de taille $q$ impaire. Alors $[2]\colon S\to S$ est une bijection
d'inverse $[l]\colon S\to S$ (là $[n]$ c'est $[n].g=g^n$). Alors, 
on a $r\equiv a^l\mod p$ avec $l=\frac{q+1}{2}$ et $r^2=a\mod p$.
\section{Carrés dans $(\Z/n\Z)^{\times}$}
On note $n=\prod_i p_i^{e_i}$. Par le théorème chinois, $x\mod n$ est
un carré ssi $\forall i$ $x\mod p_i^{e_i}$ est un carré. 

Les cas particuliers $n=pq$ et $n=p^2q$ sont les cas RSA, et trouver
une racine carrées implique de factoriser $n$ avec notre méthode. En
fait à l'inverse si on peut calculer des racines carrées, on peut 
factoriser.

Pour le cas RSA on remarque que $S_n=S_p\times S_q$ et $\#\ker([2])=4$
de sorte que $\#S_n=\varphi(n)/4$.

\begin{rem}
    Si on a un oracle de racines carrées mod $n$, on peut prendre
    des éléments $y=x^2$ pour un $x$ aléatoire et demander une autre
    racine de $y$ mod $n$ a l'oracle. Comme $\mod n$ y'a plus que $2$
    racines carrées on obtient un $x'\ne \pm x$ avec bonne probabilité,
    en fait probabilité $1/2$ si $p\approx q$ via le fait que $x\equiv x'
    \mod p$ mais pas $\mod q$ ou inversement. Ensuite faut calculer
    $(x-x')\wedge n$.
\end{rem}

\section{Symbole de Jacobi}
\begin{thm}
    Soit $m,n\geq 3$ \textbf{impairs}. On a
    \[\left(\frac{m}{n}\right)\times \left(\frac{n}{m}\right)=(-1)^{(m-1)(n-1)/4}\]
    et $\left(\frac{-1}{m}\right)=(-1)^{(m-1)/2}$, $\left(\frac{2}{m}\right)=(-1)^{(m^2-1)/8}$.
\end{thm}
Dans le cas RSA $\left(\frac{x}{n}\right)$ vaut $1$ veut seulement
dire que $x$ est un carré ou $x$ n'est ni un carré mod $q$ ni mod $p$.
On déf les presques carrés $\{x|\left(\frac{x}{n}\right)=1\}$. Les
carrés sont d'indices $2$ dedans. Maintenant on peut faire du bit
commitment car chance de $1/2$ d'être un carré.

\section{Un protocole d'identification à divulgation nulle}
\textbf{Protocole d'identification :}
\begin{itemize}
    \item Alice construit son secret $n_A=p_A+q_A$ et choisit
        $r_A\in (\Z/n_A\Z)^{times}$, elle calcule ensuite 
        $r_A^2=s_A\mod n_A$ un carré aléatoire.
    \item Elle publie $(Alice,n_A,s_A)$.
    \item Bob contacte Alice et Alice prouve son identité. Alice
        choisit un élément aléatoire $u\in(\Z/n\Z)^{\times}$ et
        calcule son carré $u^2=v\mod n$ puis $t=v\times s_A$. (
        $s_A, v$ sont publiques tandis que $r_A,u$ sont secrets.
    \item Alice envoie $v$ et $t$ (Elle connaît la racine carrée
        de $v$ et $t=s_A v$.
    \item Bob génère un bit $\epsilon$ pour demander une racine
        carrée de $v$ ou $t$. (C'est le twist, on peut pas avoir une
        racine des deux sinon on peut obtenir $r_A$.)
    \item Alice répond et Bob vérifie.
    \item On peut répeter suffisamment de fois.
\end{itemize}

\chapter{Logarithmes discrets}
\section{Réseaux de relations}
Étant donné $G$ un groupe abélien, soit $g_1,g_2,\ldots, g_I\in G$.
Et soit $0\to R\to \Z^I\to G$ la suite associée, on appelle 
$R$ le réseau de relations des $g_i$. Comme $G$ est fini, $R$ est
automatiquement de rang plein!
\section{Algorithmes génériques}
\begin{thm}[Victor Shoup]
    Dans le cas générique, pas d'algorithmes meilleurs que 
    $\sqrt{\#G}$.
\end{thm}

Si $G=(\Z/n\Z)^{\times}$ pour $n$ premier, alors c'est plutôt
$exp(\sqrt{\log(n)\log\log(n)}\times k)$. Dans le cas des courbes
elliptiques sur les corps finis on sait pas.
\section{Automorphismes de groupe}
Si on regarde $G=(\Z/p\Z)^d$ alors $Aut(G)=GL_d(\F_p)$ de cardinal
$p^{d(d-1)/2}\times\prod_{i=1}^d(p^k-1)$. Pour $(\Z/n\Z, +)$, 
les automorphismes c'est juste $(\Z/n\Z)^{\times}$ via l'exponentiation.
Alors on peut reformuler le log discret en disant : étant donné
deux générateurs $g,h$  de $G$ cyclique trouver un automorphisme de $G$
qui envoie $g$ sur $h$.

\section{Un nouveau protocole d'identification à divulgation nulle}
\textbf{Protocole :}
\begin{itemize}
    \item Alice génère son secret : Elle choisit un groupe cyclique
        large, par exemple un s.g de $(\Z/p\Z)^{\times})$ ou 
        $E(\F_p)$.
    \item Elle construit un générateur $g_A$ de $G$ et choisit un 
        $a$ aléatoire de sorte à construire $g_A^a=h_A$.
    \item Elle publie ensuite $(G,\#G=e, g_A, h_A)$.
    \item Bob demande à Alice de s'authentifier, Alice choisit
        alors $a_r\in (\Z/e\Z)^{\ŧimes}$ et calcule $h_r=h_A^{r}$
        puis envoie $h_r$.
    \item Bob demande alors soit un exposant qui envoie $h_A$ vers $h_r$
        ou un exposant de $g_A$ vers $h_r$.
\end{itemize}


\printbibliography

\end{document}


\documentclass[a4paper,12pt]{book}
\usepackage{amsmath,  amsthm,enumerate}
\usepackage{csquotes}
\usepackage[provide=*,french]{babel}
\usepackage[dvipsnames]{xcolor}
\usepackage{quiver, tikz}

%symbole caligraphique
\usepackage{mathrsfs}

%hyperliens
\usepackage{hyperref}

%pseudo-code
\usepackage{algorithm}
\usepackage{algpseudocode}

\usepackage{fancyhdr}

\pagestyle{fancy}
\addtolength{\headwidth}{\marginparsep}
\addtolength{\headwidth}{\marginparwidth}
\renewcommand{\chaptermark}[1]{\markboth{#1}{}}
\renewcommand{\sectionmark}[1]{\markright{\thesection\ #1}}
\fancyhf{}
\fancyfoot[C]{\thepage}
\fancyhead[LO]{\textit \leftmark}
\fancyhead[RE]{\textit \rightmark}
\renewcommand{\headrulewidth}{0pt} % and the line
\fancypagestyle{plain}{%
    \fancyhead{} % get rid of headers
}

%bibliographie
\usepackage[
backend=biber,
style=alphabetic,
sorting=ynt
]{biblatex}

\addbibresource{bib.bib}

\usepackage{appendix}
\renewcommand{\appendixpagename}{Annexe}

\definecolor{wgrey}{RGB}{148, 38, 55}

\setlength\parindent{24pt}

\newcommand{\Z}{\mathbb{Z}}
\newcommand{\R}{\mathbb{R}}
\newcommand{\rel}{\omathcal{R}}
\newcommand{\Q}{\mathbb{Q}}
\newcommand{\C}{\mathbb{C}}
\newcommand{\Cat}{\mathcal{C}}
\newcommand{\Dat}{\mathcal{D}}
\newcommand{\Aat}{\mathcal{A}}
\newcommand{\N}{\mathbb{N}}
\newcommand{\K}{\mathbb{K}}
\newcommand{\A}{\mathbb{A}}
\newcommand{\B}{\mathcal{B}}
\newcommand{\Or}{\mathcal{O}}
\newcommand{\F}{\mathscr F}
\newcommand{\Hom}{\textrm{Hom}}
\newcommand{\disc}{\textrm{disc}}
\newcommand{\Pic}{\textrm{Pic}}
\newcommand{\End}{\textrm{End}}
\newcommand{\Spec}{\textrm{Spec}}
\newcommand{\Supp}{\textrm{Supp}}
\newcommand{\Ouv}{\textrm{Ouv}}
\newcommand{\im}{\textrm{im}}
\newcommand{\coker}{\textrm{coker}}
\newcommand{\coim}{\textrm{coim}}


\newcommand{\cL}{\mathscr{L}}
\newcommand{\G}{\mathscr{G}}
\newcommand{\D}{\mathscr{D}}
\newcommand{\E}{\mathscr{E}}
\renewcommand{\P}{\mathscr{P}}
\renewcommand{\H}{\mathscr{H}}

\makeatletter
\newcommand{\colim@}[2]{%
  \vtop{\m@th\ialign{##\cr
    \hfil$#1\operator@font colim$\hfil\cr
    \noalign{\nointerlineskip\kern1.5\ex@}#2\cr
    \noalign{\nointerlineskip\kern-\ex@}\cr}}%
}
\newcommand{\colim}{%
  \mathop{\mathpalette\colim@{\rightarrowfill@\scriptscriptstyle}}\nmlimits@
}
\renewcommand{\varprojlim}{%
  \mathop{\mathpalette\varlim@{\leftarrowfill@\scriptscriptstyle}}\nmlimits@
}
\renewcommand{\varinjlim}{%
  \mathop{\mathpalette\varlim@{\rightarrowfill@\scriptscriptstyle}}\nmlimits@
}
\makeatother

\theoremstyle{plain}
\newtheorem{thm}{Théoreme}
\newtheorem{lem}{Lemme}
\newtheorem{prop}{Proposition}
\newtheorem{cor}{Corollaire}
\newtheorem{heur}{Heuristique}
\newtheorem{rem}{Remarque}
\newtheorem{note}{Note}

\theoremstyle{definition}
\newtheorem{conj}{Conjecture}
\newtheorem{prob}{Problème}
\newtheorem{quest}{Question}
\newtheorem{prot}{Protocole}
\newtheorem{algo}{Algorithme}
\newtheorem{defn}{Définition}
\newtheorem{exmp}{Exemples}
\newtheorem{exo}{Exercices}
\newtheorem{ex}{Exemple}
\newtheorem{exs}{Exemples}

\theoremstyle{remark}

\definecolor{wgrey}{RGB}{148, 38, 55}
\definecolor{wgreen}{RGB}{100, 200,0} 
\hypersetup{
    colorlinks=true,
    linkcolor=wgreen,
    urlcolor=wgrey,
    filecolor=wgrey
}

\title{Penser avec les catégories}
\date{2024-2025}

\begin{document}
\maketitle
\tableofcontents

\chapter{Yoneda}


\chapter{Foncteurs}

\section{Foncteurs représentables}

\chapter{Objets universels dans les catégories abéliennes}

\section{Remarques sur la variance}
Le bi-foncteur $h_{\_}(\_)$ est celui qu'on a besoin pour
la cohomologie. Et on a $h_M(A)=h^A(M)$.
\subsection{Le plongement de Yoneda}
Le foncteur $I\mapsto h_I$ est covariant et pleinement fidèle,
via $I\to I'$ et $M\to I$ fournit $M\to I\to I'$. Mais ca c'est Yoneda.
À l'inverse $P\mapsto h^P$ est contravariant vu que étendre 
naturellement $P'\to M$ c'est par pullback. 

\subsection{Spécialisation, foncteur représentable}
Donc une fois $I$ fixé, $h_I$ est contravariant, vu que
$M\to I$ s'étend par pullback. Et $h^P$ est covariant
vu que $P\to A$ s'étend par pushforward. Et surtout que 
$h_I(M)\to h_I(M')=h^M(I)\to h^{M'}(I)$ donc on a la variance
de $h^{\_}$.

\section{Exactitude du foncteur Hom}
Ducoup, l'exactitude de $A\to B\to C\to 0$
via $h_M$ est toujours vraie. On obtient
\[0\to h_M(C)\to h_M(B)\to h_M(A)\]
et pour obtenir l'exactitude à droite, il
faut que $M=I$ soit un injectif.

On a deux suites exactes à traiter
\[0\to A\to B\to C\]
et 
\[A\to B\to C\to 0\]
c'est la deuxième qui m'interesse pour
la cohomologie via $h_I$. Je regarde que
\[A\to B\to C\to 0\]
On en déduit deux
suites exactes
\[0\to h_I(C)\to h_I(B)\to h_I(A)\]
et
\[0\to h^C\to h^B\to h^A\]
la deuxième donne pas la propriété des injectifs.
L'exactitude de la deuxième se vérifie terme à terme
donc c'est équivalent l'autre.

\subsection{Preuves du deuxième cas}

\subsubsection{Injectivité et surjectivité sur les
côtés}

Le cas $0\to h^C\to h^B$ est clair l'injectivité c'est le fait que
un épi $B\to C\to 0$ vient de $(C\to M)\mapsto (B\to C\to M)$
est injectif.
Et ça se traduit en mono dans la catégorie opposée où
cette fois un mono $0\to A\to B$ fournit un mono
\[0\to h_A\to h_B\]
par la variance et $M\to A\mapsto M\to A\to B$ uniquement.
\begin{rem}
  Marrant y'a ptet un truc philosophique sur cette
  asymétrie ?
\end{rem}
\subsubsection{Exactitude au milieu où $\ker$ et $\coker$}

\begin{rem}
Pour rappel un noyau $K$ est donné par une
injection exacte :
\[0\to h_K\to h_B\to h_C\]
et un conoyau par
\[0\to h^{coK}\to (h^B\to h^A)=\ker(h^B\to h^A)\]
l'image elle c'est
\[0\to h_{Im}\to h_{B}\to h_{coK(A\to B)}=\ker(h_B\to h_{coK(A\to B)})\]
\end{rem}
Maintenant on veut pas exactement montrer que $h_K=h_{\im}$
sachant $K=\im$ vu que c'est immédiat. On veut montrer
que $\ker(h_I(B)\to h_I(A))=\im(h_I(C)\to h_I(B))$ pour tout $I$
mais le problème c'est que ça se traduit en
\[\ker(h^B\to h^A)=\im(h^C\to h^B)\]
et donc on a pas a priori la propriété universelle direct. Le fait
que ce soit pas intuitif c'est que on demande que de $B\to M$
on ait 
$A\to B\to M=0$ ssi $A\to B\to M$ se factorise par $A\to B\to C\to
M$ i.e.
% https://q.uiver.app/#q=WzAsNCxbMSwxLCJNIl0sWzEsMCwiQiJdLFswLDAsIkEiXSxbMiwwLCJDIl0sWzEsMF0sWzIsMV0sWzEsM10sWzMsMCwiIiwxLHsic3R5bGUiOnsiYm9keSI6eyJuYW1lIjoiZGFzaGVkIn19fV1d
\[\begin{tikzcd}
  A & B & C \\
	& M
	\arrow[from=1-1, to=1-2]
	\arrow[from=1-2, to=1-3]
	\arrow[from=1-2, to=2-2]
	\arrow[dashed, from=1-3, to=2-2]
\end{tikzcd}\]
On dirait que $M$ est un projectif. Ça s'éclaircit en rappelant que
% https://q.uiver.app/#q=WzAsNCxbMSwxLCJNIl0sWzEsMCwiQiJdLFswLDAsIkEiXSxbMiwwLCJDIl0sWzEsMF0sWzIsMV0sWzEsM10sWzMsMCwiIiwxLHsic3R5bGUiOnsiYm9keSI6eyJuYW1lIjoiZGFzaGVkIn19fV1d
\[\begin{tikzcd}
  A & B & C &0\\
   & M & 
	\arrow[from=1-1, to=1-2]
	\arrow[from=1-2, to=1-3]
	\arrow[from=1-3, to=1-4]
	\arrow[from=1-2, to=2-2]
	\arrow[dashed, from=1-3, to=2-2]
\end{tikzcd}\]
et là on peut définir $C\to M$ parce que $B\to C$ est un épi 
intuitivement. Plus concrètement l'idée c'est une extension de
celle du théorème d'iso.
Étant donné $coim$ comment je déf $coim\to im$! Là c'est l'inverse!
En fait on prouve que \[C=\coker(A\to B)\] via
$coim(B\to C)\simeq C$ dans la catégorie abélienne et
\[\coker(A\to B)=\coker(\im(A\to B)\to B)=\coker(\ker(B\to C)\to B)=\coim(B\to C)\] par exactitude au milieu. 


\subsubsection{Exactitude au milieu en bref}
On veut que $\ker(h^B\to h^A)=\im(h^C\to h^B)$, ça revient à dire
que $C=\coker(A\to B)$. Et on utilise l'exactitude pour voir que
$\coker(A\to B)=\coim(B\to C)=\im(B\to C)=C$. En termes de 
foncteurs. La coimage vérifie 
\[0\to h^{\coim(B\to C)}\to h^B\to h^{\ker(B\to C)}\]
\[h^{\coim(B\to C)}\]
et on a un pont $h^{\coim(B\to C)}=h_{\im(B\to C)}$.

\subsubsection{Le théorème d'iso est un pont entre $h^{\_}$ et $h_{\_}$}


\subsection{Preuves du premier cas}
Pour l'exactitude au milieu, on utilise seulement que 
\[0\to A \to B\]
exact implique $A\simeq \im(A\to B)$ en particulier, $M\to B$
tel que $M\to B\to C = 0_{M,C}$ implique 
$M\to B=M\to \im(A\to B)\to B$ et via l'isomorphisme on obtient
naturellement $M\to A$ tel que $M\to B$ et $M\to A\to B$.

\section{Injectifs et projectifs}
Pour rappel $h_I(\_):=\Hom(\_,I)$ et 
$h^{P}(\_):=\Hom(P,\_)$. Être injectif $I$ ça
revient à ce que l'exactitude de 
\[0\to M'\to M\]
devienne l'exactitude de $h_I(M)\to h_I(M')\to 0$.
C'est un peu bizarre à intuiter au sens où le
diagramme est
% https://q.uiver.app/#q=WzAsMyxbMCwwLCJNJyJdLFsxLDAsIk0iXSxbMCwxLCJJIl0sWzAsMSwiIiwwLHsic3R5bGUiOnsidGFpbCI6eyJuYW1lIjoiaG9vayIsInNpZGUiOiJ0b3AifX19XSxbMCwyXSxbMSwyLCIiLDAseyJzdHlsZSI6eyJib2R5Ijp7Im5hbWUiOiJkYXNoZWQifX19XV0=
\[\begin{tikzcd}
	{M'} & M \\
	I
	\arrow[hook, from=1-1, to=1-2]
	\arrow[from=1-1, to=2-1]
	\arrow[dashed, from=1-2, to=2-1]
\end{tikzcd}\]
et que de $M'\to I$ on a une flèche qui induit l'existence
de $M\to M$. C'est une condition de surjectivité! I.e.
$h_I$ est exacte à droite (en passant à la catégorie opposée). 
À l'inverse pour les projectifs
% https://q.uiver.app/#q=WzAsMyxbMCwwLCJNJyJdLFsxLDAsIk0iXSxbMCwxLCJQIl0sWzEsMCwiIiwyLHsic3R5bGUiOnsiaGVhZCI6eyJuYW1lIjoiZXBpIn19fV0sWzIsMF0sWzIsMSwiIiwyLHsic3R5bGUiOnsiYm9keSI6eyJuYW1lIjoiZGFzaGVkIn19fV1d
\[\begin{tikzcd}
	{M'} & M \\
	P
	\arrow[two heads, from=1-2, to=1-1]
	\arrow[from=2-1, to=1-1]
	\arrow[dashed, from=2-1, to=1-2]
\end{tikzcd}\]

\section{Monomorphismes et épimorphismes}

En résumé $A\to B$ est un mono veut
dire que 
\[h_A\to h_B\]
est injectif via $i_*=h_i:=(M\to A)\mapsto (M\to A\to B)$.
(ca se vérif bien terme à terme) et pour la remarque, 
$h_A$ est un foncteur de $\Aat$ dans $Ab$
dans le cadre des catégories abéliennes donc
c'est une catégorie abélienne et l'injectivité 
se traduit en $0\to h_A\to h_B$.
\subsection{Détail}
On a $0\to A\to B$ est exacte c'est pareil que
\[0\to \Hom_{\Cat}(M,A)\to \Hom_{\Cat}(M,B)\]
est injective via le pushforward : 
$i\circ f =i\circ g\implies f=g$. En passant à
la catégorie opposée dans $\Cat$, on obtient les
épis par $B\to A\to 0$ et les flèches c'est :
\[\Hom_{\Cat^op}(B,M)\to \Hom_{\Cat^op}(A,M)\to 0\]
est surjective via $i_*^{op}$. D'où
\[\Hom_{\Cat}(M,B)\to \Hom_{\Cat}(M,A)\to 0\]
est surjective.
\begin{rem}
  C'est comme ça que les flèches s'inversent!
  La condition
\[\Hom_{\Cat}(B,M)\to \Hom_{\Cat}(A,M)\to 0\]
  via le pullback est bizarre. Ça veut dire que
  toutes les flèches $A\to M$ proviennent de 
  $B\to M$ sachant que $A\hookrightarrow B$ 
  s'injecte dans $B$. C'est pas exactement tout
  de suite en lien avec les injectifs.
\end{rem}
En résumé, 
\section{Ker et coker}
Un $\ker$ c'est ça :
% https://q.uiver.app/#q=WzAsNCxbMCwwLCJLIl0sWzEsMCwiQSJdLFsyLDAsIkIiXSxbMCwxLCJNIl0sWzAsMV0sWzEsMl0sWzMsMV0sWzMsMCwiIiwxLHsic3R5bGUiOnsiYm9keSI6eyJuYW1lIjoiZGFzaGVkIn19fV0sWzMsMiwiT197TSxCfSIsMSx7ImN1cnZlIjoxfV1d
\[\begin{tikzcd}
	K & A & B \\
	M
	\arrow[from=1-1, to=1-2]
	\arrow[from=1-2, to=1-3]
	\arrow[dashed, from=2-1, to=1-1]
	\arrow[from=2-1, to=1-2]
	\arrow["{O_{M,B}}"{description}, curve={height=6pt}, from=2-1, to=1-3]
\end{tikzcd}\]
Maintenant, on peut le traduire en 
\[h_{K}\to h_A\to h_B\]
est exacte. En plus, $K\to A$ est un mono par
le propriété universelle. D'où
\[0\to h_K\to h_A\to h_B\]
est exacte. 



\section{Snake Lemma}
Avec élément c'est assez clair comment on construit $\delta$ la
flèche de connexion. Sans élément ça l'es moins. J'ai eu une
idée et je me suis spoil la suite. On peut faire comme ça, donc
on regarde $\ker(w)\to C\to B\to B'\to A\to \coker(u)$. Et l'idée
c'est que on regarde des éléments dans $B$ tels que $B\to C\to C'$
est nulle, et pour qu'y soient bien définis, on les regarde modulo
$A$. En résumé, on regarde \[\ker(B\to C\to C')=B''\] on sait que
$(B''\to B\to B')\to C'=0$ donc on obtient $B''\to A'$. Pour 
conclure on a clairement $A\to B''$ et ça fournit
\[B''/A\to A'/A=\coker(u)\] en plus comme
$B\to C$ est un épi, $B''\to B$ est un mono, et $A\to B\to C$ est
exacte, on obtient que 
\[B''/A=B''/\ker(B\to C)\simeq\im( B''\to B\to C)=\ker(w)\] d'où
\[\ker(w)\to \coker(u)\]

\section{Obtenir des flèches}
Quelques tricks pour obtenir des flèches. Étant donné
$A\to B\to C$, on a $\ker(A\to B)\to \ker(A\to B\to C)$.
Parce que \[\ker( A\to B)\to A\to B\to C=(\ker(A\to B)\to A\to B)\to C\]
est nulle d'où la flèche. À l'inverse on a 
\[\coker(A\to B\to C)\to \coker(B\to C)\]
via 
\[A\to B\to C\to \coker(B\to C)=A\to (B\to C\to \coker(B\to C))=0\]
d'où la flèche. Aussi, on a 
\[\coker(A\to B)\to \coker(A\to B\to C)\]
via \[A\to B\to (C\to \coker(A\to B\to C))=(A\to B \to C\to \coker(A\to B\to C))=0\]
d'où la flèche. En particulier on obtient la flèche du théorème
d'isomorphisme.

\section{Théorème d'isomorphisme}
On regarde $\ker(A\to B)\to A\to B$, alors
\[\coim(A\to B)=\coker(\ker(A\to B)\to A)\to \coker((\ker(A\to B)\to A)\to B)=\coker(0\to B)=B\]
et la flèche est bien induite pas $A\to B$.



%\printbibliography
\end{document}


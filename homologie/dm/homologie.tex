\documentclass[a4paper,12pt]{article}
\usepackage{amsmath,  amsthm,enumerate}
\usepackage{csquotes}
\usepackage[provide=*,french]{babel}
\usepackage[dvipsnames]{xcolor}
\usepackage{quiver, tikz}

%symbole caligraphique
\usepackage{mathrsfs}

%hyperliens
\usepackage{hyperref}

%pseudo-code
\usepackage{algorithm}
\usepackage{algpseudocode}


%bibliographie
\usepackage[
backend=biber,
style=alphabetic,
sorting=ynt
]{biblatex}


\definecolor{wgrey}{RGB}{148, 38, 55}

\setlength\parindent{24pt}

\newcommand{\Z}{\mathbb{Z}}
\newcommand{\R}{\mathbb{R}}
\newcommand{\rel}{\omathcal{R}}
\newcommand{\Q}{\mathbb{Q}}
\newcommand{\C}{\mathbb{C}}
\newcommand{\Cat}{\mathcal{C}}
\newcommand{\Dat}{\mathcal{D}}
\newcommand{\N}{\mathbb{N}}
\newcommand{\K}{\mathbb{K}}
\newcommand{\A}{\mathbb{A}}
\newcommand{\B}{\mathcal{B}}
\newcommand{\Or}{\mathcal{O}}
\newcommand{\F}{\mathscr F}
\newcommand{\Hom}{\textrm{Hom}}
\newcommand{\disc}{\textrm{disc}}
\newcommand{\Pic}{\textrm{Pic}}
\newcommand{\End}{\textrm{End}}
\newcommand{\Spec}{\textrm{Spec}}
\newcommand{\Supp}{\textrm{Supp}}
\renewcommand{\Im}{\textrm{Im}}
\newcommand{\Ouv}{\textrm{Ouv}}
\newcommand{\im}{\textrm{im}}


\newcommand{\cL}{\mathscr{L}}
\newcommand{\G}{\mathscr{G}}
\newcommand{\D}{\mathscr{D}}
\newcommand{\E}{\mathscr{E}}
\renewcommand{\P}{\mathscr{P}}
\renewcommand{\H}{\mathscr{H}}

\makeatletter
\newcommand{\colim@}[2]{%
  \vtop{\m@th\ialign{##\cr
    \hfil$#1\operator@font colim$\hfil\cr
    \noalign{\nointerlineskip\kern1.5\ex@}#2\cr
    \noalign{\nointerlineskip\kern-\ex@}\cr}}%
}
\newcommand{\colim}{%
  \mathop{\mathpalette\colim@{\rightarrowfill@\scriptscriptstyle}}\nmlimits@
}
\renewcommand{\varprojlim}{%
  \mathop{\mathpalette\varlim@{\leftarrowfill@\scriptscriptstyle}}\nmlimits@
}
\renewcommand{\varinjlim}{%
  \mathop{\mathpalette\varlim@{\rightarrowfill@\scriptscriptstyle}}\nmlimits@
}
\makeatother

\theoremstyle{plain}
\newtheorem{thm}[subsection]{Théoreme}
\newtheorem{lem}[subsection]{Lemme}
\newtheorem{prop}[subsection]{Proposition}
\newtheorem{cor}[subsection]{Corollaire}
\newtheorem{heur}{Heuristique}
\newtheorem{rem}{Remarque}
\newtheorem{note}{Note}

\theoremstyle{definition}
\newtheorem{conj}{Conjecture}
\newtheorem{prob}{Problème}
\newtheorem{quest}{Question}
\newtheorem{prot}{Protocole}
\newtheorem{algo}{Algorithme}
\newtheorem{defn}[subsection]{Définition}
\newtheorem{exmp}[subsection]{Exemples}
\newtheorem{exo}[subsection]{Exercices}
\newtheorem{ex}[subsection]{Exemple}
\newtheorem{exs}[subsection]{Exemples}
\newtheorem{slog}{Slogan}

\theoremstyle{remark}

\definecolor{wgrey}{RGB}{148, 38, 55}
\definecolor{wgreen}{RGB}{100, 200,0} 
\hypersetup{
    colorlinks=true,
    linkcolor=wgreen,
    urlcolor=wgrey,
    filecolor=wgrey
}

\title{Exercices d'algèbre homologique}
\author{Rayane Bait}
\date{ }

\begin{document}
\maketitle

\section*{Exercice 0.1}
On utilise les notations de l'exercice. Soit 
% https://q.uiver.app/#q=WzAsNSxbMSwwLCJcXEYnIl0sWzAsMCwiMCJdLFsyLDAsIlxcRiJdLFszLDAsIlxcRicnIl0sWzQsMCwiKCopIl0sWzEsMF0sWzAsMiwiZiIsMl0sWzIsMywiZyIsMl1d
\[\begin{tikzcd}
	0 & {\F'} & \F & {\F''} & {(*)}
	\arrow[from=1-1, to=1-2]
	\arrow["f"', from=1-2, to=1-3]
	\arrow["g"', from=1-3, to=1-4]
\end{tikzcd}\]
une suite exacte dans $Sh(X)$ et $U\subseteq X$ un ouvert de $X$.
On doit montrer que 
% https://q.uiver.app/#q=WzAsNCxbMSwwLCJcXEYnKFUpIl0sWzAsMCwiMCJdLFsyLDAsIlxcRihVKSJdLFszLDAsIlxcRicnKFUpIl0sWzEsMF0sWzAsMiwiZihVKSIsMl0sWzIsMywiZyhVKSIsMl1d
\[\begin{tikzcd}
	0 & {\F'(U)} & {\F(U)} & {\F''(U)}
	\arrow[from=1-1, to=1-2]
	\arrow["{f(U)}"', from=1-2, to=1-3]
	\arrow["{g(U)}"', from=1-3, to=1-4]
\end{tikzcd}\]
est exacte dans $Ab$.
Autrement dit que $\ker(f(U))=0$ et $\im(f(U))=\ker(g(U))$ dans $Ab$.

On montre d'abord le premier point. Soit $s\in \ker(f(U))$ et
$x\in X$. Par commutativité de 
% https://q.uiver.app/#q=WzAsNSxbMSwwLCJcXEYnKFUpIl0sWzIsMCwiXFxGKFUpIl0sWzEsMSwiXFxGX3giXSxbMiwxLCJcXEYnX3giXSxbMCwxLCIwIl0sWzAsMSwiZihVKSIsMl0sWzAsMl0sWzIsM10sWzEsM10sWzQsMl1d
\[\begin{tikzcd}
	& {\F'(U)} & {\F(U)} \\
	0 & {\F_x} & {\F'_x}
	\arrow["{f(U)}"', from=1-2, to=1-3]
	\arrow["{f_x}"', from=2-2, to=2-3]
	\arrow[from=1-2, to=2-2]
	\arrow[from=1-3, to=2-3]
	\arrow[from=2-1, to=2-2]
	\arrow[from=2-2, to=2-3]
\end{tikzcd}\]
pour les flèches évidentes et par exactitude de la ligne du bas,
on a $(U,s)=(V_x,0)\in \F_x$ pour tout $x\in U$. Quitte à remplacer
$V_x$ par $U\cap V_x$, on peut supposer $V_x\subset U$. En particulier,
on obtient un recouvrement de $U$ par des ouverts $V_x$ tels que 
\[s|_{V_x}=0\]
pour tout $x\in U$. D'où $s=0\in \F(U)$ car $\F$ est un faisceau. On a 
montré que $f(U)$ est injective pour tout ouvert $U$ de $X$. 

On montre maintenant le second point par double inclusion, d'abord
$\ker(g(U))\subset \im(f(U))$ : Soit $s\in \ker(g(U))$ et $x\in U$, par
commutativité de 
% https://q.uiver.app/#q=WzAsNCxbMCwwLCJcXEYoVSkiXSxbMSwwLCJcXEYnJyhVKSJdLFswLDEsIlxcRl94Il0sWzEsMSwiXFxGJydfeCJdLFswLDEsImcoVSkiLDJdLFswLDJdLFsyLDMsImdfeCIsMl0sWzEsM11d
\[\begin{tikzcd}
	{\F(U)} & {\F''(U)} \\
	{\F_x} & {\F''_x}
	\arrow["{g(U)}"', from=1-1, to=1-2]
	\arrow[from=1-1, to=2-1]
	\arrow[from=1-2, to=2-2]
	\arrow["{g_x}"', from=2-1, to=2-2]
\end{tikzcd}\]
on a, $g_x((U,s))=0$. Maintenant par exactitude de
% https://q.uiver.app/#q=WzAsMyxbMCwwLCJcXEYnX3giXSxbMSwwLCJcXEZfeCJdLFsyLDAsIlxcRicnX3giXSxbMCwxLCJmX3giLDJdLFsxLDIsImdfeCIsMl1d
\[\begin{tikzcd}
	{\F'_x} & {\F_x} & {\F''_x}
	\arrow["{f_x}"', from=1-1, to=1-2]
	\arrow["{g_x}"', from=1-2, to=1-3]
\end{tikzcd}\]
il existe $(V_x, s'_x)\in \F'_x$ tel que $f_x((V_x,s'_x))=(U,s)$. Quitte
à prendre $V_x\cap U=V_x$ on peut supposer $V_x\subset U$. Maintenant
par commutativité de
% https://q.uiver.app/#q=WzAsOSxbMSwwLCJcXEYnKFUpIl0sWzIsMCwiXFxGKFUpIl0sWzEsMiwiXFxGJ194Il0sWzIsMiwiXFxGX3giXSxbMCwyLCIwIl0sWzAsMCwiMCJdLFswLDEsIjAgIl0sWzEsMSwiXFxGJyhWX3gpIl0sWzIsMSwiXFxGKFZfeCkiXSxbMCwxLCJmKFUpIiwyXSxbMiwzLCJmX3giLDJdLFs0LDJdLFs1LDBdLFs2LDddLFs3LDgsImYoVl94KSIsMl0sWzEsOF0sWzAsN10sWzcsMl0sWzgsM11d
\[\begin{tikzcd}
	0 & {\F'(U)} & {\F(U)} \\
	{0 } & {\F'(V_x)} & {\F(V_x)} \\
	0 & {\F'_x} & {\F_x}
	\arrow[from=1-1, to=1-2]
	\arrow["{f(U)}"', from=1-2, to=1-3]
	\arrow[from=1-2, to=2-2]
	\arrow[from=1-3, to=2-3]
	\arrow[from=2-1, to=2-2]
	\arrow["{f(V_x)}"', from=2-2, to=2-3]
	\arrow[from=2-2, to=3-2]
	\arrow[from=2-3, to=3-3]
	\arrow[from=3-1, to=3-2]
	\arrow["{f_x}"', from=3-2, to=3-3]
\end{tikzcd}\]
on obtient $f(V_x)(s'_x)=s|_{V_x}$ pour chaque $x\in U$ et 
$U=\cup_{x\in U} V_x$. Enfin, pour tout $x,x'\in U$, par commutativité
de
% https://q.uiver.app/#q=WzAsNixbMCwxLCJcXEYnKFZfeFxcY2FwIFZfe3gnfSkiXSxbMSwxLCJcXEYnKFZfeFxcY2FwIFZfe3gnfSkiXSxbMCwwLCJcXEYnKFZfeCkiXSxbMSwwLCJcXEYoVl94KSJdLFswLDIsIlxcRicoVl97eCd9KSJdLFsxLDIsIlxcRihWX3t4J30pIl0sWzAsMV0sWzIsM10sWzIsMF0sWzMsMV0sWzQsNV0sWzQsMF0sWzUsMV1d
\[\begin{tikzcd}
	{\F'(V_x)} & {\F(V_x)} \\
	{\F'(V_x\cap V_{x'})} & {\F'(V_x\cap V_{x'})} \\
	{\F'(V_{x'})} & {\F(V_{x'})}
	\arrow[from=1-1, to=1-2]
	\arrow[from=1-1, to=2-1]
	\arrow[from=1-2, to=2-2]
	\arrow[from=2-1, to=2-2]
	\arrow[from=3-1, to=2-1]
	\arrow[from=3-1, to=3-2]
	\arrow[from=3-2, to=2-2]
\end{tikzcd}\]
on a 
\[f(V_x\cap V_{x'})(s'_x|_{V_x\cap V_{x'}})=s|_{V_x\cap V_{x'}}=
f(V_x\cap V_{x'})(s'_{x'}|_{V_x\cap V_{x'}})\]
d'où par injectivité de $f(V_x\cap V_{x'})$ on a 
\[s'_x|_{V_x\cap V_{x'}}=s'_{x'}|_{V_x\cap V_{x'}}\]
comme $\F'$ est un faisceau on peut relever les $s'_x$ en un $s'\in U$.
Par commutativité du carré du haut dans l'avant dernier diagramme, comme
$\F$ est un faisceau et $f(V_x)(s'|_{V_x})=s|_{V_x}$ on obtient que
$f(U)(s')=s$ d'où $\ker(g(U))\subset \im(f(U))$. 

Soit maintenant $s=f(U)(s')$ pour $s'\in \F'(U)$. Pour tout $x\in U$,
par commutativité de 
% https://q.uiver.app/#q=WzAsNixbMCwxLCJcXEYnX3giXSxbMSwxLCJcXEZfeCJdLFsyLDEsIlxcRicnX3giXSxbMCwwLCJcXEYnKFUpIl0sWzEsMCwiXFxGKFUpIl0sWzIsMCwiXFxGJycoVSkiXSxbMCwxLCJmX3giLDJdLFsxLDIsImdfeCIsMl0sWzMsNF0sWzQsNV0sWzMsMF0sWzQsMV0sWzUsMl1d
\[\begin{tikzcd}
	{\F'(U)} & {\F(U)} & {\F''(U)} \\
	{\F'_x} & {\F_x} & {\F''_x}
	\arrow[from=1-1, to=1-2]
	\arrow[from=1-1, to=2-1]
	\arrow[from=1-2, to=1-3]
	\arrow[from=1-2, to=2-2]
	\arrow[from=1-3, to=2-3]
	\arrow["{f_x}"', from=2-1, to=2-2]
	\arrow["{g_x}"', from=2-2, to=2-3]
\end{tikzcd}\]
et par exactitude de la ligne du bas on a 
\[(V_x,0)=g_x(f_x((U,s')))=g_x((U,s)).\]
On obtient un recouvrement 
$U=\cup_x V_x$ de $U$ par les $V_x$ tel que 
$g(U)(s)|_{V_x}=g(V_x)(s|_{V_x})=0$.
Comme $\F''$ est un faisceau, on obtient $g(U)(s)=0$ d'où 
$\ker(g(U))=\im(f(U))$.

\section*{Exercice 0.2}










\end{document}


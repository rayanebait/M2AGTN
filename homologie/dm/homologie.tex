\documentclass[a4paper,12pt]{article}
\usepackage{amsmath,  amsthm,enumerate}
\usepackage{csquotes}
\usepackage[provide=*,french]{babel}
\usepackage[dvipsnames]{xcolor}
\usepackage{quiver, tikz}

%symbole caligraphique
\usepackage{mathrsfs}

%hyperliens
\usepackage{hyperref}

%pseudo-code
\usepackage{algorithm}
\usepackage{algpseudocode}


%bibliographie
\usepackage[
backend=biber,
style=alphabetic,
sorting=ynt
]{biblatex}


\definecolor{wgrey}{RGB}{148, 38, 55}

\setlength\parindent{24pt}

\newcommand{\Z}{\mathbb{Z}}
\newcommand{\R}{\mathbb{R}}
\newcommand{\rel}{\omathcal{R}}
\newcommand{\Q}{\mathbb{Q}}
\newcommand{\C}{\mathbb{C}}
\newcommand{\Cat}{\mathcal{C}}
\newcommand{\Dat}{\mathcal{D}}
\newcommand{\N}{\mathbb{N}}
\newcommand{\K}{\mathbb{K}}
\newcommand{\A}{\mathbb{A}}
\newcommand{\B}{\mathcal{B}}
\newcommand{\Or}{\mathcal{O}}
\newcommand{\F}{\mathscr F}
\newcommand{\Hom}{\textrm{Hom}}
\newcommand{\disc}{\textrm{disc}}
\newcommand{\Pic}{\textrm{Pic}}
\newcommand{\End}{\textrm{End}}
\newcommand{\Spec}{\textrm{Spec}}
\newcommand{\Supp}{\textrm{Supp}}
\renewcommand{\Im}{\textrm{Im}}
\newcommand{\Ouv}{\textrm{Ouv}}
\newcommand{\im}{\textrm{im}}
\newcommand{\coker}{\textrm{coker}}
\newcommand{\coim}{\textrm{coim}}


\newcommand{\cL}{\mathscr{L}}
\newcommand{\G}{\mathscr{G}}
\newcommand{\D}{\mathscr{D}}
\newcommand{\E}{\mathscr{E}}
\renewcommand{\P}{\mathscr{P}}
\renewcommand{\H}{\mathscr{H}}

\makeatletter
\newcommand{\colim@}[2]{%
  \vtop{\m@th\ialign{##\cr
    \hfil$#1\operator@font colim$\hfil\cr
    \noalign{\nointerlineskip\kern1.5\ex@}#2\cr
    \noalign{\nointerlineskip\kern-\ex@}\cr}}%
}
\newcommand{\colim}{%
  \mathop{\mathpalette\colim@{\rightarrowfill@\scriptscriptstyle}}\nmlimits@
}
\renewcommand{\varprojlim}{%
  \mathop{\mathpalette\varlim@{\leftarrowfill@\scriptscriptstyle}}\nmlimits@
}
\renewcommand{\varinjlim}{%
  \mathop{\mathpalette\varlim@{\rightarrowfill@\scriptscriptstyle}}\nmlimits@
}
\makeatother

\theoremstyle{plain}
\newtheorem{thm}[subsection]{Théoreme}
\newtheorem{lem}[subsection]{Lemme}
\newtheorem{prop}[subsection]{Proposition}
\newtheorem{cor}[subsection]{Corollaire}
\newtheorem{heur}{Heuristique}
\newtheorem{rem}{Remarque}
\newtheorem{note}{Note}

\theoremstyle{definition}
\newtheorem{conj}{Conjecture}
\newtheorem{prob}{Problème}
\newtheorem{quest}{Question}
\newtheorem{prot}{Protocole}
\newtheorem{algo}{Algorithme}
\newtheorem{defn}[subsection]{Définition}
\newtheorem{exmp}[subsection]{Exemples}
\newtheorem{exo}[subsection]{Exercices}
\newtheorem{ex}[subsection]{Exemple}
\newtheorem{exs}[subsection]{Exemples}
\newtheorem{slog}{Slogan}

\theoremstyle{remark}

\definecolor{wgrey}{RGB}{148, 38, 55}
\definecolor{wgreen}{RGB}{100, 200,0} 
\hypersetup{
    colorlinks=true,
    linkcolor=wgreen,
    urlcolor=wgrey,
    filecolor=wgrey
}

\title{Exercices d'algèbre homologique}
\author{Rayane Bait}
\date{ }

\begin{document}
\maketitle

\section*{Exercice 0.1}
On utilise les notations de l'exercice. Soit 
% https://q.uiver.app/#q=WzAsNSxbMSwwLCJcXEYnIl0sWzAsMCwiMCJdLFsyLDAsIlxcRiJdLFszLDAsIlxcRicnIl0sWzQsMCwiKCopIl0sWzEsMF0sWzAsMiwiZiIsMl0sWzIsMywiZyIsMl1d
\[\begin{tikzcd}
	0 & {\F'} & \F & {\F''} & {(*)}
	\arrow[from=1-1, to=1-2]
	\arrow["f"', from=1-2, to=1-3]
	\arrow["g"', from=1-3, to=1-4]
\end{tikzcd}\]
une suite exacte dans $Sh(X)$ et $U\subseteq X$ un ouvert de $X$.
On doit montrer que 
% https://q.uiver.app/#q=WzAsNCxbMSwwLCJcXEYnKFUpIl0sWzAsMCwiMCJdLFsyLDAsIlxcRihVKSJdLFszLDAsIlxcRicnKFUpIl0sWzEsMF0sWzAsMiwiZihVKSIsMl0sWzIsMywiZyhVKSIsMl1d
\[\begin{tikzcd}
	0 & {\F'(U)} & {\F(U)} & {\F''(U)}
	\arrow[from=1-1, to=1-2]
	\arrow["{f(U)}"', from=1-2, to=1-3]
	\arrow["{g(U)}"', from=1-3, to=1-4]
\end{tikzcd}\]
est exacte dans $Ab$.
Autrement dit que $\ker(f(U))=0$ et $\im(f(U))=\ker(g(U))$ dans $Ab$.

On montre d'abord le premier point. Soit $s\in \ker(f(U))$ et
$x\in X$. Par commutativité de 
% https://q.uiver.app/#q=WzAsNSxbMSwwLCJcXEYnKFUpIl0sWzIsMCwiXFxGKFUpIl0sWzEsMSwiXFxGX3giXSxbMiwxLCJcXEYnX3giXSxbMCwxLCIwIl0sWzAsMSwiZihVKSIsMl0sWzAsMl0sWzIsM10sWzEsM10sWzQsMl1d
\[\begin{tikzcd}
	& {\F'(U)} & {\F(U)} \\
	0 & {\F_x} & {\F'_x}
	\arrow["{f(U)}"', from=1-2, to=1-3]
	\arrow["{f_x}"', from=2-2, to=2-3]
	\arrow[from=1-2, to=2-2]
	\arrow[from=1-3, to=2-3]
	\arrow[from=2-1, to=2-2]
	\arrow[from=2-2, to=2-3]
\end{tikzcd}\]
pour les flèches évidentes et par exactitude de la ligne du bas,
on a $(U,s)=(V_x,0)\in \F_x$ pour tout $x\in U$. Quitte à remplacer
$V_x$ par $U\cap V_x$, on peut supposer $V_x\subset U$. En particulier,
on obtient un recouvrement de $U$ par des ouverts $V_x$ tels que 
\[s|_{V_x}=0\]
pour tout $x\in U$. D'où $s=0\in \F(U)$ car $\F$ est un faisceau. On a 
montré que $f(U)$ est injective pour tout ouvert $U$ de $X$. 

On montre maintenant le second point par double inclusion, d'abord
$\ker(g(U))\subset \im(f(U))$ : Soit $s\in \ker(g(U))$ et $x\in U$, par
commutativité de 
% https://q.uiver.app/#q=WzAsNCxbMCwwLCJcXEYoVSkiXSxbMSwwLCJcXEYnJyhVKSJdLFswLDEsIlxcRl94Il0sWzEsMSwiXFxGJydfeCJdLFswLDEsImcoVSkiLDJdLFswLDJdLFsyLDMsImdfeCIsMl0sWzEsM11d
\[\begin{tikzcd}
	{\F(U)} & {\F''(U)} \\
	{\F_x} & {\F''_x}
	\arrow["{g(U)}"', from=1-1, to=1-2]
	\arrow[from=1-1, to=2-1]
	\arrow[from=1-2, to=2-2]
	\arrow["{g_x}"', from=2-1, to=2-2]
\end{tikzcd}\]
on a, $g_x((U,s))=0$. Maintenant par exactitude de
% https://q.uiver.app/#q=WzAsMyxbMCwwLCJcXEYnX3giXSxbMSwwLCJcXEZfeCJdLFsyLDAsIlxcRicnX3giXSxbMCwxLCJmX3giLDJdLFsxLDIsImdfeCIsMl1d
\[\begin{tikzcd}
	{\F'_x} & {\F_x} & {\F''_x}
	\arrow["{f_x}"', from=1-1, to=1-2]
	\arrow["{g_x}"', from=1-2, to=1-3]
\end{tikzcd}\]
il existe $(V_x, s'_x)\in \F'_x$ tel que $f_x((V_x,s'_x))=(U,s)$. Quitte
à prendre $V_x\cap U=V_x$ on peut supposer $V_x\subset U$. Maintenant
par commutativité de
% https://q.uiver.app/#q=WzAsOSxbMSwwLCJcXEYnKFUpIl0sWzIsMCwiXFxGKFUpIl0sWzEsMiwiXFxGJ194Il0sWzIsMiwiXFxGX3giXSxbMCwyLCIwIl0sWzAsMCwiMCJdLFswLDEsIjAgIl0sWzEsMSwiXFxGJyhWX3gpIl0sWzIsMSwiXFxGKFZfeCkiXSxbMCwxLCJmKFUpIiwyXSxbMiwzLCJmX3giLDJdLFs0LDJdLFs1LDBdLFs2LDddLFs3LDgsImYoVl94KSIsMl0sWzEsOF0sWzAsN10sWzcsMl0sWzgsM11d
\[\begin{tikzcd}
	0 & {\F'(U)} & {\F(U)} \\
	{0 } & {\F'(V_x)} & {\F(V_x)} \\
	0 & {\F'_x} & {\F_x}
	\arrow[from=1-1, to=1-2]
	\arrow["{f(U)}"', from=1-2, to=1-3]
	\arrow[from=1-2, to=2-2]
	\arrow[from=1-3, to=2-3]
	\arrow[from=2-1, to=2-2]
	\arrow["{f(V_x)}"', from=2-2, to=2-3]
	\arrow[from=2-2, to=3-2]
	\arrow[from=2-3, to=3-3]
	\arrow[from=3-1, to=3-2]
	\arrow["{f_x}"', from=3-2, to=3-3]
\end{tikzcd}\]
on obtient $f(V_x)(s'_x)=s|_{V_x}$ pour chaque $x\in U$ et 
$U=\cup_{x\in U} V_x$. Enfin, pour tout $x,x'\in U$, par commutativité
de
% https://q.uiver.app/#q=WzAsNixbMCwxLCJcXEYnKFZfeFxcY2FwIFZfe3gnfSkiXSxbMSwxLCJcXEYnKFZfeFxcY2FwIFZfe3gnfSkiXSxbMCwwLCJcXEYnKFZfeCkiXSxbMSwwLCJcXEYoVl94KSJdLFswLDIsIlxcRicoVl97eCd9KSJdLFsxLDIsIlxcRihWX3t4J30pIl0sWzAsMV0sWzIsM10sWzIsMF0sWzMsMV0sWzQsNV0sWzQsMF0sWzUsMV1d
\[\begin{tikzcd}
	{\F'(V_x)} & {\F(V_x)} \\
	{\F'(V_x\cap V_{x'})} & {\F'(V_x\cap V_{x'})} \\
	{\F'(V_{x'})} & {\F(V_{x'})}
	\arrow[from=1-1, to=1-2]
	\arrow[from=1-1, to=2-1]
	\arrow[from=1-2, to=2-2]
	\arrow[from=2-1, to=2-2]
	\arrow[from=3-1, to=2-1]
	\arrow[from=3-1, to=3-2]
	\arrow[from=3-2, to=2-2]
\end{tikzcd}\]
on a 
\[f(V_x\cap V_{x'})(s'_x|_{V_x\cap V_{x'}})=s|_{V_x\cap V_{x'}}=
f(V_x\cap V_{x'})(s'_{x'}|_{V_x\cap V_{x'}})\]
d'où par injectivité de $f(V_x\cap V_{x'})$ on a 
\[s'_x|_{V_x\cap V_{x'}}=s'_{x'}|_{V_x\cap V_{x'}}\]
comme $\F'$ est un faisceau on peut relever les $s'_x$ en un $s'\in U$.
Par commutativité du carré du haut dans l'avant dernier diagramme, comme
$\F$ est un faisceau et $f(V_x)(s'|_{V_x})=s|_{V_x}$ on obtient que
$f(U)(s')=s$ d'où $\ker(g(U))\subset \im(f(U))$. 

Soit maintenant $s=f(U)(s')$ pour $s'\in \F'(U)$. Pour tout $x\in U$,
par commutativité de 
% https://q.uiver.app/#q=WzAsNixbMCwxLCJcXEYnX3giXSxbMSwxLCJcXEZfeCJdLFsyLDEsIlxcRicnX3giXSxbMCwwLCJcXEYnKFUpIl0sWzEsMCwiXFxGKFUpIl0sWzIsMCwiXFxGJycoVSkiXSxbMCwxLCJmX3giLDJdLFsxLDIsImdfeCIsMl0sWzMsNF0sWzQsNV0sWzMsMF0sWzQsMV0sWzUsMl1d
\[\begin{tikzcd}
	{\F'(U)} & {\F(U)} & {\F''(U)} \\
	{\F'_x} & {\F_x} & {\F''_x}
	\arrow[from=1-1, to=1-2]
	\arrow[from=1-1, to=2-1]
	\arrow[from=1-2, to=1-3]
	\arrow[from=1-2, to=2-2]
	\arrow[from=1-3, to=2-3]
	\arrow["{f_x}"', from=2-1, to=2-2]
	\arrow["{g_x}"', from=2-2, to=2-3]
\end{tikzcd}\]
et par exactitude de la ligne du bas on a 
\[(V_x,0)=g_x(f_x((U,s')))=g_x((U,s)).\]
On obtient un recouvrement 
$U=\cup_x V_x$ de $U$ par les $V_x$ tel que 
$g(U)(s)|_{V_x}=g(V_x)(s|_{V_x})=0$.
Comme $\F''$ est un faisceau, on obtient $g(U)(s)=0$ d'où 
$\ker(g(U))=\im(f(U))$.

\section*{Exercice 0.2}
\begin{center}
	\textbf{1)}
\end{center}
Avec les notations de l'exercice, on montre que $S$ est inductif. 
Par le lemme de Zorn, on obtient le résultat. Soit 
$((U_i, s_i))_{i\in I}$ une chaîne donnée par un ordre total $I$. Alors
pour $i,j\in I$ et $i\leq j$ on a 
\[s_j|_{U_i\cap U_j}=s_j|_{U_i}=s_i=s_i|_{U_i\cap U_j}\]
d'où la famille se relève en un $s\in \cup_{i\in I} U_i$ tel que 
$s|_{U_i}=s_i$ et $U_i\subset \cup_{j\in I}U_j$ alors 
\[(\cup_{i\in I} U_i, s)\]
est un majorant de $((U_i,s_i))_{i\in I}$ contenu dans $S$. D'où
$S$ est inductif puis le résultat.
\begin{center}
	\textbf{2)}
\end{center}
Soit $(s, V)$ un élément maximal de $S$. Comme 


\section*{Exercice 0.3}
On utilisera librement dans tout l'exercice que $Sh(X)$ est une
catégorie abélienne avec les résultats suivants du cours : 
\begin{enumerate}
	\item Le noyau d'une flèche $f\colon F\to F'$ dans $Sh(X)$
		coincide avec le faisceau défini par 
		\[\ker(U):=\ker(\F(U)\to\F'(U))\]
	\item L'image d'une flèche $f\colon F\to F'$ dans $Sh(X)$
		coincide avec le faisceautisé du préfaisceau défini
		par
		\[\im\colon U\mapsto \im(U):=\im(\F(U)\to\F'(U))\]
	\item Le conoyau d'une flèche $f\colon F\to F'$ dans $Sh(X)$
		coincide avec le faisceautisé du préfaisceau défini
		\[\coker\colon U\mapsto \coker(U):=\coker(\F(U)\to\F'(U))\]
	\item Le faisceau $0_{Sh(X)}$ défini par $0_{Sh(X)}(U):= 0_{Ab}$
		est un objet zéro dans $Sh(X)$.
\end{enumerate}
On note $(\_)^\sharp\colon PSh(X)\to Sh(X)$ le foncteur de 
faisceautisation. Enfin, si $\F$ est un faisceau sur un espace 
topologique on pourra décrire une section $\bar s\colon U\to Et(\F)$ 
de l'espace étalé de $\F$ au dessus de $U$ comme un tuple 
$(s_x)_{x\in U}=(\bar s(x))_{x\in U}$.

\begin{center}
	\textbf{1)}
\end{center}
Soit $\F$ un faisceau dans $Sh(X)$. Pour tout $x\in X$, on a 
$0=0_{F_x}\in \F_x$ un élément neutre car
$\F_x$ est un groupe abélien. En particulier, pour tout ouverts 
$V\subseteq U$ de $X$, si $s=(s_x)_{x\in V}\in C^0(\F)(V)$ alors
si on note $s':=(s'_x)_{x\in U}$ la section telle que $s'_x=s_x$ pour
$x\in V$ et $s'_x=0$ pour $x\in U-V$ on a $s'|_V=s$ d'où $C^0(\F)$
est flasque.

\begin{center}
	\textbf{2)}
\end{center}
Soit $\F\in Sh(X)$. On définit $i^0(\F)\colon \F\to C^0(\F)$ le
morphisme de faisceau défini par 
$i^0(\F)(U)\colon s\mapsto (s_x)_{x\in U}$ pour tout ouvert $U$ de $X$
et où $s_x\in \F_x$ est 
l'image de $s$ dans la fibre $\F_x$ induite par $\F$. On montre que
$i^0(\F)$ est injectif en montrant que $\ker(i^0(\F))=0_{Sh(X)}$. 
Soit $U$ un ouvert sur $X$ et $s\in \ker(i^0(\F))(U)$, alors 
$(s_x)_{x\in U}=((V_x, 0))_{x\in U}$, en particulier $s|_{V_x}=0$ 
et $\cup V_x = U$ d'où $s=0\in \F(U)$, en particulier on a bien 
$\ker(i^0(\F))=0_{Sh(X)}$.

\begin{center}
	\textbf{3)}
\end{center}
On définit en plus $Z^0(\F)=\F$, $d_0^0\colon C^0(\F)\to Z^1(\F)$
et $d^0=i^0(\F)\circ(d_0^0)$. On suppose maintenant défini
$d_0^{i-1}\colon C^{i-1}(\F) \to Z^i(\F)$ et 
\[d^{i-1}=i^0(Z^{i-1})\circ(d_0^{i-1})\colon C^{i-1}(\F)\to C^i(\F)\]
pour $n\geq i\geq 1$. On pose 
\[d_0^{n}=\coker(i^0(Z^{n-1}(\F))):=(C^{n-1}(\F)\to Z^n(\F))\]
la flèche canonique du conoyau et $d^n=i^0(Z^n(\F))\circ d_0^n$. 

\begin{center}
	\textbf{4)}
\end{center}

On commence par remarquer qu'un noyau est invariant par post-composition
par un monomorphisme, de même une image est invariante par 
pré-composition par un épimorphisme dans une catégorie abélienne.
On le montre en annexe.
\newline

On montre maintenant que $0\to \F\to C^\bullet(\F)$ est exacte. 
Par la question 2), et comme les morphismes injectifs de faisceaux sont
des monomorphismes, par un exercice du cours $0\to \F\to C^0(\F)$ est 
exacte. On montre donc que $\F\to C^\bullet(\F)$ est exacte. On utilise
ici $\im$ pour désigner l'image catégorique.
Par la première partie de la remarque précédente et par la question 2)
on a 
$\ker(d^n)=\ker(d_0^n)$. Maintenant par construction, et par le cours 
\begin{align*}
	\ker(d_0^n)&=\ker(\coker(i^0(Z^n)))\\
		   &=\ker(\coker(Z^n(\F)\to C^n(\F)))\\
		   &=\im(Z^n(\F)\to C^n(\F))\\
		   &=\im(C^{n-1}(\F)\to C^n(\F))\\
\end{align*}
où la dernière égalité est dûe à la deuxième partie de la remarque
précédente car 
$(C^{n-1}(\F)\to Z^n(\F))=\coker(Z^{n-1}(\F)\to C^{n-1}(\F))$
est un épimorphisme. En particulier, $0\to \F\to C^\bullet(\F)$ est
exacte.

\begin{center}
	\textbf{5)}
\end{center}



Dans cet exercice on désigne par $d(\F)^n$ (resp. $d(\F)_0^n$) la
flèche $C^n(\F)\to C^{n+1}(\F)$ (resp. $C^n(\F)\to Z^{n+1}(\F)$) 
construite en 3).
\newline

Étant donné $f\colon \F'\to \F$ dans $Sh(X)$ on commence par construire 
$C^0(f)$ et $C^\bullet(f)$ avant de montrer que $C^0(\_)$ et 
$C^\bullet(\_)$ sont des foncteurs exacts. Ça suffit à prouver le 
résultat, en effet si $C^\bullet(\_)$ est un foncteur exact alors pour
toute suite exacte \[0\to \F'\to \F\to \F''\to 0\]
dans $Sh(X)$ et tout ouvert $U\subset X$,
comme $C^n(\F')=C^0(Z^n(\F'))$ est flasque par la question 1),
\[0\to C^n(\F')(U)\to C^n(\F)(U)\to C^n(\F'')(U)\to 0\] est exacte
par l'exercice 2. D'où le résultat.

On commence par montrer que $C^0(\_)$ est un foncteur. On pose
\[C^0(f)(U)\colon (s_x)_{x\in U}\mapsto (f_x s_x)_{x\in U}.\]
Par définition, le diagramme
% https://q.uiver.app/#q=WzAsNCxbMCwwLCJcXEYnIl0sWzEsMCwiXFxGIl0sWzAsMSwiQ14wKFxcRicpIl0sWzEsMSwiQ14wKFxcRikiXSxbMSwzLCJpXjAoXFxGKSJdLFswLDIsImleMChcXEYnKSIsMl0sWzIsMywiQ14wKGYpIiwyXSxbMCwxLCJmIl1d
\[\begin{tikzcd}
	{\F'} & \F \\
	{C^0(\F')} & {C^0(\F)}
	\arrow["f", from=1-1, to=1-2]
	\arrow["{i^0(\F')}"', from=1-1, to=2-1]
	\arrow["{i^0(\F)}", from=1-2, to=2-2]
	\arrow["{C^0(f)}"', from=2-1, to=2-2]
\end{tikzcd}\]
commute et $C^0(f)$ est bien un morphisme de faisceau. Maintenant 
si $g\colon \F\to \F''$ est une autre flèche dans $Sh(X)$ alors 
$C^0(g\circ f)=C^0(g)\circ C^0(f)$ par définition car le foncteur
fibre $(\_)_x \colon Sh(X)\to Ab$ pour $x\in X$ est un foncteur. Enfin
$C^0(id_\F)=id_{C^0(\F)}$ pour la même raison. D'où
$C^0(\_)$ est un foncteur.

On construit maintenant 
$C^\bullet(f)=(f^n\colon C^n(\F')\to C^n(\F)_{n\geq 0}$ 
étant donné $f \colon \F'\to \F$ dans $Sh(X)$. On pose
$f^0=C^0(f)$, il suffit de construire $Z^\bullet(f):=
(f_0^n\colon Z^n(\F')\to Z^n(\F))$ telles que 
% https://q.uiver.app/#q=WzAsNCxbMCwwLCJDXntuLTF9KFxcRicpIl0sWzEsMCwiQ157bi0xfShcXEYpIl0sWzAsMSwiWl5uKFxcRicpIl0sWzEsMSwiWl5uKFxcRikiXSxbMSwzLCJkXzBee24tMX0iXSxbMCwyLCJkXzBee24tMX0iLDJdLFsyLDMsImZfMF57bn0iLDJdLFswLDEsImZfe24tMX0iXV0=
\[\begin{tikzcd}
	{C^{n-1}(\F')} & {C^{n-1}(\F)} \\
	{Z^n(\F')} & {Z^n(\F)}
	\arrow["{f_{n-1}}", from=1-1, to=1-2]
	\arrow["{d(\F')_0^{n-1}}"', from=1-1, to=2-1]
	\arrow["{d(\F)_0^{n-1}}", from=1-2, to=2-2]
	\arrow["{f_0^{n}}"', from=2-1, to=2-2]
\end{tikzcd}\]
commute pour tout $n\geq 0$ et de poser $(f^n)_{n\geq0}=(C^0(f_0^n))$.
On le fait par récurrence sur $n$. Pour $n=1$ on regarde
% https://q.uiver.app/#q=WzAsNixbMCwxLCJDXnswfShcXEYnKSJdLFsxLDEsIkNeezB9KFxcRikiXSxbMCwyLCJaXjEoXFxGJykiXSxbMSwyLCJaXjEoXFxGKSJdLFswLDAsIlxcRiciXSxbMSwwLCJcXEYiXSxbMSwzLCJkXzBeezB9Il0sWzAsMiwiZF8wXnswfSIsMl0sWzAsMSwiZl4wIl0sWzQsMF0sWzQsNSwiZiJdLFs1LDFdLFsyLDMsImZfMF4xIiwyXV0=
\[\begin{tikzcd}
	{\F'} & \F \\
	{C^{0}(\F')} & {C^{0}(\F)} \\
	{Z^1(\F')} & {Z^1(\F)}
	\arrow["f", from=1-1, to=1-2]
	\arrow[from=1-1, to=2-1]
	\arrow[from=1-2, to=2-2]
	\arrow["{f^0}", from=2-1, to=2-2]
	\arrow["{d_0^{0}}"', from=2-1, to=3-1]
	\arrow["{d_0^{0}}", from=2-2, to=3-2]
	\arrow["{f_0^1}"', from=3-1, to=3-2]
\end{tikzcd}\]
et on veut construire la flèche du bas. On remarque que
$(\F'\to C^0(\F'))$ se factorise par $\ker(d(\F)_0^0\circ f^0)$. En effet
par commutatitivité du carré du haut et exactitude de la colonne de
droite on a
\[0_{\F',Z^1(\F)}=0_{\F,Z^1(\F)}\circ f=(d(\F)_0^0\circ i^0(\F))\circ f=d(\F)_0^0\circ f^0\circ
i^0(\F')\] 
d'où $\F'\to C^0(\F')=\F'\to\ker(d_0^0\circ f^0)\to C^0$
se factorise par le noyau. On écrit maintenant
\begin{align*}
	Z^1(\F')&=\coker(\F'\to C^0(\F'))\\
		&=\coim(C^0(\F')\to Z^1(\F'))
\end{align*}
où la dernière égalité est par construction. Par le point précédent,
en notant 
\[\F'=\ker\left(C^0(\F)\to \coim(C^0(\F')\to Z^1(\F'))\right)\]
on obtient une flèche
\[\ker\left(C^0(\F')\to \coim(C^0(\F')\to Z^1(\F'))\right)\to \ker(d_0^0\circ f^0)\]
d'où une flèche induite 
\[\coim\left(C^0(\F')\to \coim(C^0(\F')\to Z^1(\F'))\right)\to \coim(d_0^0\circ f^0\]
en particulier, le membre de gauche est 
$\coim(C^0(\F')\to Z^1(\F'))=Z^1(\F')$. Finalement, on obtient
une flèche induite
\[f_0^1\colon Z^1(\F')\to \coim(d(\F)_0^0\circ f^0)\to\im(d(\F)_0^0\circ f^0) \to Z^1(\F)\] 
ce qui conclut le cas $n=1$. La commutativité étant par construction.
Le cas $n\geq 2$ se fait de manière identique en remplaçant $\F=Z^0(\F)$
par $Z^{n-1}(\F)$, $C^0(\F)$ par $C^n(\F)$ et $Z^1(\F)$ par $Z^n(\F)$.
\newline

On suppose maintenant que $C^\bullet$ est un foncteur, une tentative de
preuve est en annexe.
\newline

On montre maintenant que $C^\bullet$ est exact. Supposons d'abord que $C^0(\_)$ est exact et soit 
% https://q.uiver.app/#q=WzAsNSxbMSwwLCJcXEYnIl0sWzAsMCwiMCJdLFsyLDAsIlxcRiJdLFszLDAsIlxcRicnIl0sWzQsMCwiKCopIl0sWzEsMF0sWzAsMiwiZiIsMl0sWzIsMywiZyIsMl1d
\[\begin{tikzcd}
	0 & {\F'} & \F & {\F''} & 0 & {(*)}
	\arrow[from=1-1, to=1-2]
	\arrow["f"', from=1-2, to=1-3]
	\arrow["g"', from=1-3, to=1-4]
	\arrow[from=1-4, to=1-5]
\end{tikzcd}\]
une suite exacte dans $Sh(X)$. On montre que le morphisme de complexe
induit $0\to C^\bullet(\F')\to C^\bullet(\F)\to C^\bullet(\F'')\to 0$
est exact. Il suffit de montrer pour tout $n\geq 0$ la suite
\[0\to C^n(\F')\to C^n(\F)\to C^n(\F'')\to 0\]
est exacte dans $Sh(X)$. En plus on a supposé que $C^0(\_)$ était exact,
il suffit donc de montrer que la suite
\[0\to Z^n(\F')\to Z^n(\F)\to Z^n(\F'')\to 0\]
est exacte. On le montre par récurrence sur $n$. En posant $Z^0(\F):=\F$
on obtient le cas $n=0$ par l'hypothèse $(*)$. Supposons 
maintenant le résultat vrai pour $0\leq i\leq n-1$.
On considère le diagramme 

% https://q.uiver.app/#q=WzAsMTAsWzAsMSwiMCJdLFsxLDEsIkNee24tMX0oXFxGJykiXSxbMiwxLCJDXntuLTF9KFxcRikiXSxbMywxLCJDXntuLTF9KFxcRicnKSJdLFs0LDEsIjAiXSxbMSwwLCJaXntuLTF9KFxcRicpIl0sWzIsMCwiWl57bi0xfShcXEYpIl0sWzMsMCwiWl57bi0xfShcXEYnJykiXSxbNCwwLCIwIl0sWzAsMCwiMCJdLFsxLDIsIkNeMChmXzBee24tMX0pIl0sWzIsMywiQ14wKGdfMF57bi0xfSkiXSxbMyw0XSxbMCwxXSxbNSw2LCJmXzBee24tMX0iXSxbNiw3LCJmXzBee24tMX0iXSxbNSwxXSxbNiwyXSxbNywzXSxbNyw4XSxbOSw1XV0=
\[\begin{tikzcd}
	0 & {Z^{n-1}(\F')} & {Z^{n-1}(\F)} & {Z^{n-1}(\F'')} & 0 \\
	0 & {C^{n-1}(\F')} & {C^{n-1}(\F)} & {C^{n-1}(\F'')} & 0
	\arrow[from=1-1, to=1-2]
	\arrow["{f_0^{n-1}}", from=1-2, to=1-3]
	\arrow[from=1-2, to=2-2]
	\arrow["{f_0^{n-1}}", from=1-3, to=1-4]
	\arrow[from=1-3, to=2-3]
	\arrow[from=1-4, to=1-5]
	\arrow[from=1-4, to=2-4]
	\arrow[from=2-1, to=2-2]
	\arrow["{C^0(f_0^{n-1})}", from=2-2, to=2-3]
	\arrow["{C^0(g_0^{n-1})}", from=2-3, to=2-4]
	\arrow[from=2-4, to=2-5]
\end{tikzcd}\]
Par hypothèse de récurrence, les lignes sont exactes. En plus, par 
fonctorialité de $C^0(\_)$ et par construction, le diagramme commute.
Comme $Sh(X)$ est abélienne, on applique le lemme du serpent pour obtenir
la suite exacte
% https://q.uiver.app/#q=WzAsOCxbMSwwLCJcXGtlcihpXjAoWl57bi0xfShcXEYnKSkpIl0sWzAsMCwiMCJdLFsyLDAsIlxca2VyKGleMChaXntuLTF9KFxcRikpKSJdLFszLDAsIlxca2VyKGleMChaXntuLTF9KFxcRicnKSkpIl0sWzEsMSwiXFxjb2tlcihpXjAoWl57bi0xfShcXEYnKSkpIl0sWzIsMSwiXFxjb2tlcihpXjAoWl57bi0xfShcXEYpKSkiXSxbMywxLCJcXGNva2VyKGleMChaXntuLTF9KFxcRicnKSkpIl0sWzQsMSwiMCJdLFsxLDBdLFswLDIsIlxca2VyKGZfMF57bi0xfSkiXSxbMiwzLCJcXGtlcihnXzBee24tMX0pIl0sWzMsNF0sWzUsNl0sWzQsNV0sWzYsN11d
\[\begin{tikzcd}
	0 & {\ker(i^0(Z^{n-1}(\F')))} & {\ker(i^0(Z^{n-1}(\F)))} & {\ker(i^0(Z^{n-1}(\F'')))} \\
	& {\coker(i^0(Z^{n-1}(\F')))} & {\coker(i^0(Z^{n-1}(\F)))} & {\coker(i^0(Z^{n-1}(\F'')))} & 0
	\arrow[from=1-1, to=1-2]
	\arrow["{\ker(f_0^{n-1})}", from=1-2, to=1-3]
	\arrow["{\ker(g_0^{n-1})}", from=1-3, to=1-4]
	\arrow[from=1-4, to=2-2]
	\arrow[from=2-2, to=2-3]
	\arrow[from=2-3, to=2-4]
	\arrow[from=2-4, to=2-5]
\end{tikzcd}\]
or $i^0(\F)$ est un monomorphisme de faisceaux pour tout $\F\in Sh(X)$,
d'où on obtient une suite exacte
% https://q.uiver.app/#q=WzAsNSxbMSwwLCJcXGNva2VyKGleMChaXntuLTF9KFxcRicpKSkiXSxbMiwwLCJcXGNva2VyKGleMChaXntuLTF9KFxcRikpKSJdLFszLDAsIlxcY29rZXIoaV4wKFpee24tMX0oXFxGJycpKSkiXSxbNCwwLCIwIl0sWzAsMCwiMCJdLFsxLDJdLFswLDFdLFsyLDNdLFs0LDBdXQ==
\[\begin{tikzcd}
	0 & {\coker(i^0(Z^{n-1}(\F')))} & {\coker(i^0(Z^{n-1}(\F)))} & {\coker(i^0(Z^{n-1}(\F'')))} & 0
	\arrow[from=1-1, to=1-2]
	\arrow[from=1-2, to=1-3]
	\arrow[from=1-3, to=1-4]
	\arrow[from=1-4, to=1-5]
\end{tikzcd}\]
mais par définition $\coker(i^0(Z^{n-1}(\F)))=Z^n(\F)$ pour tout 
faisceau $\F$. D'où le résultat, enfin la preuve du lemme du serpent
permet de placer la suite du dessus dans un diagramme commutatif
% https://q.uiver.app/#q=WzAsMTAsWzAsMCwiMCJdLFsxLDAsIkNee24tMX0oXFxGJykiXSxbMiwwLCJDXntuLTF9KFxcRikiXSxbMywwLCJDXntuLTF9KFxcRicnKSJdLFs0LDAsIjAiXSxbMSwxLCJaXntufShcXEYnKSJdLFsyLDEsIlpee259KFxcRikiXSxbMywxLCJaXntufShcXEYnJykiXSxbNCwxLCIwIl0sWzAsMSwiMCJdLFsxLDIsIkNeMChmXzBee24tMX0pIl0sWzIsMywiQ14wKGdfMF57bi0xfSkiXSxbMyw0XSxbMCwxXSxbNSw2LCJmXzBee24tMX0iXSxbNiw3LCJmXzBee24tMX0iXSxbMSw1XSxbMiw2XSxbMyw3XSxbNyw4XSxbOSw1XV0=
\[\begin{tikzcd}
	0 & {C^{n-1}(\F')} & {C^{n-1}(\F)} & {C^{n-1}(\F'')} & 0 \\
	0 & {Z^{n}(\F')} & {Z^{n}(\F)} & {Z^{n}(\F'')} & 0
	\arrow[from=1-1, to=1-2]
	\arrow["{C^0(f_0^{n-1})}", from=1-2, to=1-3]
	\arrow["d(\F')_0^{n-1}",from=1-2, to=2-2]
	\arrow["{C^0(g_0^{n-1})}", from=1-3, to=1-4]
	\arrow["d(\F)_0^{n-1}",from=1-3, to=2-3]
	\arrow[from=1-4, to=1-5]
	\arrow["d(\F'')_0^{n-1}",from=1-4, to=2-4]
	\arrow[from=2-1, to=2-2]
	\arrow[from=2-2, to=2-3]
	\arrow[from=2-3, to=2-4]
	\arrow[from=2-4, to=2-5]
\end{tikzcd}\]
d'où par commutativité des carrés, les flèches du bas sont bien 
$f_0^n$ et $g_0^n$.

On montre enfin que $C^0(\_)\colon Sh(X)\to Sh(X)$ est
exact. Soit $f\colon \F'\to\F$ dans $Sh(X)$. La flèche $C^0(f)$ est 
donnée par
\[C^0(f)(U)\colon (s'_x)_{x\in U}\mapsto (f_x(s'_x))_{x\in U}\] 
où $f_x\colon \F'_x\to \F_x$ est la flèche induite par $f$ sur les
fibres. Il suffit de montrer que 
$C^0(\ker(\F'\to \F))=\ker(C^0(\F')\to C^0(\F))$
et $C^0(\im(\F'\to \F))=\im(C^0(\F')\to C^0(\F))$. Si c'est le cas,
alors comme $C^0(0_{Sh(X)})=0_{Sh(X)}$, car tout produit d'objets
terminaux est terminal, on obtient directement pour toute suite exacte
% https://q.uiver.app/#q=WzAsNSxbMSwwLCJcXEYnIl0sWzAsMCwiMCJdLFsyLDAsIlxcRiJdLFszLDAsIlxcRicnIl0sWzQsMCwiKCopIl0sWzEsMF0sWzAsMiwiZiIsMl0sWzIsMywiZyIsMl1d
\[\begin{tikzcd}
	0 & {\F'} & \F & {\F''} & 0 & {(*)}
	\arrow[from=1-1, to=1-2]
	\arrow["f"', from=1-2, to=1-3]
	\arrow["g"', from=1-3, to=1-4]
	\arrow[from=1-4, to=1-5]
\end{tikzcd}\]
dans $Sh(X)$ que
% https://q.uiver.app/#q=WzAsNSxbMSwwLCJcXEYnIl0sWzAsMCwiMCJdLFsyLDAsIlxcRiJdLFszLDAsIlxcRicnIl0sWzQsMCwiKCopIl0sWzEsMF0sWzAsMiwiZiIsMl0sWzIsMywiZyIsMl1d
\[\begin{tikzcd}
	0 & {C^0(\F')} & {C^0(\F)} & {C^0(\F'')} & 0 & {(*)}
	\arrow[from=1-1, to=1-2]
	\arrow["C^0(f)"', from=1-2, to=1-3]
	\arrow["C^0(g)"', from=1-3, to=1-4]
	\arrow[from=1-4, to=1-5]
\end{tikzcd}\]
est exacte par fonctorialité de $C^0(\_)$. 

On commence par montrer que
$C^0(\ker(\F'\to \F))=\ker(C^0(\F')\to C^0(\F))$. Par la description
donnée en $1.$ on a 
\[\ker(C^0(\F')\to C^0(\F))(U)=\ker(\prod_{x\in U} f_x\colon \prod_{x\in U} \F'_x\to \prod_{x\in U} \F_x)\]
et de plus $C^0(\ker(\F'\to \F))(U)=\prod_{x\in U}\ker(\F'\to\F)_x$, 
il suffit donc de montrer que si 
\[(s_x)_{x\in U}\in \ker(\prod_{x\in U} f_x\colon \prod_{x\in U} \F'_x\to \prod_{x\in U} \F_x)\]
alors pour tout $x\in U$ on a $s_x\in \ker(f_x)$ et inversement.
Mais c'est immédiat par définition du noyau dans $Ab$. 

On montre maintenant que 
$C^0(\im(\F'\to \F)^\sharp)=\im(C^0(\F')\to C^0(\F))^\sharp$.
Par la description donnée en 2. on peut écrire 
\[\im(C^0(\F')\to C^0(\F))(U)=\{(f_xs'_x)_{x\in U}| s'_x\in \F'_x\}\]
et on remarque que c'est déjà un faisceau par définition des 
restrictions. En plus $C^0(\im(\F'\to \F)^\sharp)=C^0(\im(\F'\to\F))$
car un préfaisceau et son faisceautisé ont les mêmes fibres, en
particulier $C^0(\im(\F'\to\F))$ et $\im(C^0(\F')\to C^0(\F))$ ont 
la même description donc coincident. On en déduit de $C^0(\_)$ est
exact.

\[\textrm{\textbf{6)}}\]
La preuve est exactement identique à celle du cours en remplaçant
la définition de $H^n(\F, U)$ du cours par la notre et si je l'écrivais
je la recopierai mot pour mot.

\[\textrm{\textbf{7)}}\]
On remarque que pour tout $n\geq 0$ $Z^n(\F)$ est flasque si 
$\F$ est flasque. En effet, c'est clair pour $n=0$, maintenant si 
$n\geq 1$ on a une suite exacte 
\[0\to Z^{n-1}(\F)\to C^n(\F)\to Z^n(\F)\to 0\]
et par récurrence $Z^{n-1}(\F)$ est flasque de même, 
$C^n(\F)=C^0(Z^{n-1}(\F))$ est flasque par la question, d'où par le
cours $Z^n(\F)$ est flasque. Maintenant, par l'exercice $2$ pour tout
ouvert $U\subseteq X$ 
\[0\to Z^{n-1}(\F)(U)\to C^n(\F)(U)\to Z^n(\F)(U)\to 0\qquad (**)\]
est exacte dans $Ab$ or par définition 
\[H^n(C^\bullet(\F), U):=\ker(d(\F)^n(U))/\im(d(\F)^{n-1}(U))\]
d'où par $(**)$ et par les annexes 1.1 et 1.2 on a 
\begin{enumerate}
	\item $\ker(d(\F)^n(U))=\ker(d(\F)_0^n(U))=Z^{n-1}(\F)(U)$.
	\item $Z^{n-1}(\F)(U)=\im(i^0(Z^{n-1})(U))=\im(d(\F)^{n-1}(U))$.
\end{enumerate}
Puis $H^n(C^\bullet(\F),U)=0$ pour tout $n\geq 1$.


\[\textrm{\textbf{8)}}\]
On commence par remarquer qu'un faisceau abélien $\G$ sur $X^\delta$
est de la forme $\G(U)=\prod_{x\in U} G_x$ pour tout ouvert
$U$ de $X^\delta$.

En effet pour tout ouvert $U\subseteq X^\delta$ et pour
toute section $s\in \G(U)$ on peut relever la collection 
$(s|_{\{x\}})_{x\in U}$ de manière unique car 
$\{x\}\cap \{x'\}=\emptyset$ d'où la condition de recollement sur
les intersections est vide. En particulier, 
$\G(U)=\prod_{x\in U}\G(\{x\})$.

Il reste à montrer que $\G(\{x\})=\G_x$
pour tout $x\in X^\delta$. Mais $\G(\{x\})$ est un majorant du diagramme
filtrant donné par $(\G(U))_{x\in U}$ muni des restrictions, qui 
appartient au diagramme. En particulier on obtient directement 
$\varprojlim_{x\in U} \G(U)=\G(\{x\})$ et le résultat.

Enfin, on pose
$\G=f^*\F$ pour $\F$ un faisceau abélien sur $X$. Alors pour tout
ouvert $U\subseteq X^\delta$ on a 
\[f^*\F(U)=\prod_{x\in U} (f^*\F)_x = \prod_{x\in U} \F_x\]
où la dernière égalité est par le cours et le fait que $f(x)=x$. On 
conclut en remarquant que $f_*f^*\F$ est le faisceau 
\[U\mapsto \prod_{x\in U}\F_x\]
de $\Ouv(X)$ dans $Ab$ et qu'il coincide exactement avec $C^0(\F)$ d'où
le résultat.

\section{Résultats annexes}
\subsection{Noyau et post-composition par un monomorphisme}
On doit montrer que si
$(i_K\colon K\to X)=\ker(f\colon X\to Y)$ et $j\colon Y\to Z$ est un
monomorphisme, alors $\ker(X\to Y\to Z)=(i_K\colon K\to X)$ pour 
$X,Y,Z,K$ dans une catégorie abélienne quelconque.

En effet si on note $(i_K'\colon K'\to X) =\ker(X\to Y\to Z)$ alors 
\[j\circ 0_{K',Y}=j\circ f\circ i_K'=0_{K',Z}\]
d'où $f\circ i_K'=0_{K',Y}$ car $j$ est un monomorphisme. On obtient
$k'\colon K'\to K$ une flèche telle que $K'\to K\to X=i_K'$.
À l'inverse on a $j\circ (f\circ i_K)=j\circ 0_{K, Y}=0_{K, Z}$ d'où 
on obtient $k'\colon K\to K'$ une flèche telle que $K\to K'\to X=i_K$,
en particulier, \[i_K\circ id_K=i_K'\circ k=i_K\circ k'\circ k\] et 
\[i_K'\circ id_K'=i_K\circ k'=i_K'\circ k\circ k'\]
or comme $i_K$ et $i_K'$ sont des monomorphismes
par le cours, $k$ est un isomorphisme d'inverse $k'$. 


\subsection{Image et pré-composition par un épimorphisme}
On doit montrer que dans une catégorie
abélienne quelconque, si $e\colon Z\to X$ est un épimorphisme et 
$f\colon X\to Y$ une flèche quelconque, alors $\im(f)=\im(f\circ e)$.
Par le cours, $\im(f)=(\im(f)\to Y)=\ker(Y\to \coker(f))$, 
\textbf{todo, peut-être passer par ModR}.



\subsection{Fonctorialité de $C^\bullet$}
Pour montrer la fonctorialité de $C^\bullet$ il faut montrer que
si $g\colon \F\to \F''$ est une autre flèche dans $Sh(X)$ alors 
dans la construction de 5) on a $(g\circ f)_0^n=g_0^n\circ f_0^n$
pour tout $n\geq 0$. On prouve par récurrence sur $n$ que
\[(g\circ f)_0^n=g_0^n\circ f_0^n\]
alors 
\[(g\circ f)^n=g^n\circ f^n\]
par fonctorialité de $C^0$. Le cas $n=0$ est immédiat car $Sh(X)$ est
une catégorie. On suppose $n\geq 1$.
On conclut à l'aide du lemme suivant en remplaçant 
\begin{itemize}
	\item $A$ par $C^{n-1}(\F')$,
	\item $C$ par $C^{n-1}(\F'')$,
	\item $D$ par $Z^n(\F')$,
	\item $F$ par $Z^n(\F'')$
	\item $f$ par $(g\circ f)^{n-1}=g^{n-1}\circ f^{n-1}$ par récurrence,
	\item $d_A$ par $d(\F')_0^{n-1}$ qui bien un épimorphisme
		car un conoyau.
	\item $d_C$ par $d(\F'')_0^{n-1}$
	\item $h_1$ par $g_0^n\circ f_0^n$ et $h_2$ par $(g\circ f)_0^n$.
\end{itemize}
\begin{lem}
	Soit deux diagrammes commutatifs 
% https://q.uiver.app/#q=WzAsNCxbMCwwLCJBIl0sWzIsMCwiQyJdLFswLDEsIkQiXSxbMiwxLCJGIl0sWzAsMiwiZF9BIl0sWzEsMywiZF9DIl0sWzIsMywiaF8yIiwxXSxbMCwxLCJmIiwxXV0=
\[\begin{tikzcd}
	A && C \\
	D && F
	\arrow["f"{description}, from=1-1, to=1-3]
	\arrow["{d_A}", from=1-1, to=2-1]
	\arrow["{d_C}", from=1-3, to=2-3]
	\arrow["{h_1}"{description}, from=2-1, to=2-3]
\end{tikzcd}\]
	et 
% https://q.uiver.app/#q=WzAsNCxbMCwwLCJBIl0sWzIsMCwiQyJdLFswLDEsIkQiXSxbMiwxLCJGIl0sWzAsMiwiZF9BIl0sWzEsMywiZF9DIl0sWzIsMywiaCIsMV0sWzAsMSwiZiIsMV1d
\[\begin{tikzcd}
	A && C \\
	D && F
	\arrow["f"{description}, from=1-1, to=1-3]
	\arrow["{d_A}", from=1-1, to=2-1]
	\arrow["{d_C}", from=1-3, to=2-3]
	\arrow["h_2"{description}, from=2-1, to=2-3]
\end{tikzcd}\]
	dans une catégorie abélienne $\Cat$ tels que $d_A$
	est un épimorphisme. Alors $h_1=h_2$.
\end{lem}
\begin{proof}
	La preuve consiste à dire que 
	$h_1\circ d_A=d_C\circ f=h_2\circ d_A$. D'où $h_1=h_2$ car
	$d_A$ est un épimorphisme.
\end{proof}
\textbf{à supprimer}


Par construction on a deux flèches canoniques 
\begin{align*}
(g\circ f)_0^n\colon Z^n(\F')&\to \coim(d(\F'')_0^{n-1}\circ (g\circ f)^{n-1})\\
			     &\to\im(d(\F'')_0^{n-1}\circ (g\circ f)^{n-1}) \\
			     &\to Z^n(\F'')
\end{align*}
et 
\begin{align*}
g_0^n\circ f_0^n\colon Z^n(\F')&\to \coim(d(\F)_0^{n-1}\circ f^{n-1}) \\
			       &\to \im(d(\F)_0^{n-1}\circ f^{n-1}) \\
			       &\to Z^n(\F)\\
			       &\to \coim(d(\F'')_0^{n-1}\circ g^{n-1}) \\
			       &\to\im(d(\F'')_0^{n-1}\circ g^{n-1})\\
			       &\to Z^n(\F'')
\end{align*}
telles que \[Z^n(\F')\to \coim(d_0^{n-1}\circ (g\circ f)^{n-1})=
Z^n(\F')\to \coim(d_0^{n-1}\circ g^{n-1}\circ f^{n-1})\]
par hypothèse de récurrence. En particulier, 
\[\coim(d_0^{n-1}\circ (g\circ f)^{n-1})\to \im(d_0^{n-1}\circ (g\circ f)^{n-1})\]
est égal à 
\[\coim(d_0^{n-1}\circ g^{n-1}\circ f^{n-1})\to \im(d_0^{n-1}\circ g^{n-1}\circ f^{n-1})\]
et il suffit de montrer que 
\begin{align*}
g_0^n\circ f_0^n\colon Z^n(\F')&\to \coim(d(\F)_0^{n-1}\circ f^{n-1}) \\
			       &\to \im(d(\F)_0^{n-1}\circ f^{n-1}) \\
			       &\to Z^n(\F)\\
			       &\to \coim(d(\F'')_0^{n-1}\circ g^{n-1}) \\
			       &\to\im(d(\F'')_0^{n-1}\circ g^{n-1})\\
			       &\to Z^n(\F'')
\end{align*}
coincide avec 
\begin{align*}
	Z^n(\F')&\to \coim(d(\F'')_0^{n-1}\circ g^{n-1}\circ f^{n-1})\\
		&\to\im(d(\F'')_0^{n-1}\circ g^{n-1}\circ f^{n-1})\\
			     &\to Z^n(\F'')
\end{align*}
mais 


\end{document}



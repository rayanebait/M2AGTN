\documentclass[a4paper,12pt]{article}
\usepackage{amsmath,  amsthm,enumerate}
\usepackage{csquotes}
\usepackage[provide=*,french]{babel}
\usepackage[dvipsnames]{xcolor}
\usepackage{quiver, tikz}

%symbole caligraphique
\usepackage{mathrsfs}

%hyperliens
\usepackage{hyperref}

%pseudo-code
\usepackage{algorithm}
\usepackage{algpseudocode}


%bibliographie
\usepackage[
backend=biber,
style=alphabetic,
sorting=ynt
]{biblatex}


\definecolor{wgrey}{RGB}{148, 38, 55}

\setlength\parindent{24pt}

\newcommand{\Z}{\mathbb{Z}}
\newcommand{\R}{\mathbb{R}}
\newcommand{\rel}{\omathcal{R}}
\newcommand{\Q}{\mathbb{Q}}
\newcommand{\C}{\mathbb{C}}
\newcommand{\Cat}{\mathcal{C}}
\newcommand{\Dat}{\mathcal{D}}
\newcommand{\N}{\mathbb{N}}
\newcommand{\K}{\mathbb{K}}
\newcommand{\A}{\mathbb{A}}
\newcommand{\B}{\mathcal{B}}
\newcommand{\Or}{\mathcal{O}}
\newcommand{\F}{\mathscr F}
\newcommand{\Hom}{\textrm{Hom}}
\newcommand{\disc}{\textrm{disc}}
\newcommand{\Pic}{\textrm{Pic}}
\newcommand{\End}{\textrm{End}}
\newcommand{\Spec}{\textrm{Spec}}
\newcommand{\Supp}{\textrm{Supp}}
\renewcommand{\Im}{\textrm{Im}}
\newcommand{\Ouv}{\textrm{Ouv}}
\newcommand{\im}{\textrm{im}}


\newcommand{\cL}{\mathscr{L}}
\newcommand{\G}{\mathscr{G}}
\newcommand{\D}{\mathscr{D}}
\newcommand{\E}{\mathscr{E}}
\renewcommand{\P}{\mathscr{P}}
\renewcommand{\H}{\mathscr{H}}

\makeatletter
\newcommand{\colim@}[2]{%
  \vtop{\m@th\ialign{##\cr
    \hfil$#1\operator@font colim$\hfil\cr
    \noalign{\nointerlineskip\kern1.5\ex@}#2\cr
    \noalign{\nointerlineskip\kern-\ex@}\cr}}%
}
\newcommand{\colim}{%
  \mathop{\mathpalette\colim@{\rightarrowfill@\scriptscriptstyle}}\nmlimits@
}
\renewcommand{\varprojlim}{%
  \mathop{\mathpalette\varlim@{\leftarrowfill@\scriptscriptstyle}}\nmlimits@
}
\renewcommand{\varinjlim}{%
  \mathop{\mathpalette\varlim@{\rightarrowfill@\scriptscriptstyle}}\nmlimits@
}
\makeatother

\theoremstyle{plain}
\newtheorem{thm}[subsection]{Théoreme}
\newtheorem{lem}[subsection]{Lemme}
\newtheorem{prop}[subsection]{Proposition}
\newtheorem{cor}[subsection]{Corollaire}
\newtheorem{heur}{Heuristique}
\newtheorem{rem}{Remarque}
\newtheorem{note}{Note}

\theoremstyle{definition}
\newtheorem{conj}{Conjecture}
\newtheorem{prob}{Problème}
\newtheorem{quest}{Question}
\newtheorem{prot}{Protocole}
\newtheorem{algo}{Algorithme}
\newtheorem{defn}[subsection]{Définition}
\newtheorem{exmp}[subsection]{Exemples}
\newtheorem{exo}[subsection]{Exercices}
\newtheorem{ex}[subsection]{Exemple}
\newtheorem{exs}[subsection]{Exemples}
\newtheorem{slog}{Slogan}

\theoremstyle{remark}

\definecolor{wgrey}{RGB}{148, 38, 55}
\definecolor{wgreen}{RGB}{100, 200,0} 
\hypersetup{
    colorlinks=true,
    linkcolor=wgreen,
    urlcolor=wgrey,
    filecolor=wgrey
}

\title{Exercices d'algèbre homologique}
\author{Rayane Bait}
\date{ }

\begin{document}
\maketitle

\section*{Exercice 0.1}
On utilise les notations de l'exercice. Soit 
% https://q.uiver.app/#q=WzAsNSxbMSwwLCJcXEYnIl0sWzAsMCwiMCJdLFsyLDAsIlxcRiJdLFszLDAsIlxcRicnIl0sWzQsMCwiKCopIl0sWzEsMF0sWzAsMiwiZiIsMl0sWzIsMywiZyIsMl1d
\[\begin{tikzcd}
	0 & {\F'} & \F & {\F''} & {(*)}
	\arrow[from=1-1, to=1-2]
	\arrow["f"', from=1-2, to=1-3]
	\arrow["g"', from=1-3, to=1-4]
\end{tikzcd}\]
une suite exacte dans $Sh(X)$ et $U\subseteq X$ un ouvert de $X$.
On doit montrer que 
% https://q.uiver.app/#q=WzAsNCxbMSwwLCJcXEYnKFUpIl0sWzAsMCwiMCJdLFsyLDAsIlxcRihVKSJdLFszLDAsIlxcRicnKFUpIl0sWzEsMF0sWzAsMiwiZihVKSIsMl0sWzIsMywiZyhVKSIsMl1d
\[\begin{tikzcd}
	0 & {\F'(U)} & {\F(U)} & {\F''(U)}
	\arrow[from=1-1, to=1-2]
	\arrow["{f(U)}"', from=1-2, to=1-3]
	\arrow["{g(U)}"', from=1-3, to=1-4]
\end{tikzcd}\]
est exacte dans $Ab$.
Autrement dit que $\ker(f(U))=0$ et $\im(f(U))=\ker(g(U))$ dans $Ab$.

On montre d'abord le premier point. Soit $s\in \ker(f(U))$ et
$x\in X$. Par commutativité de 
% https://q.uiver.app/#q=WzAsNSxbMSwwLCJcXEYnKFUpIl0sWzIsMCwiXFxGKFUpIl0sWzEsMSwiXFxGX3giXSxbMiwxLCJcXEYnX3giXSxbMCwxLCIwIl0sWzAsMSwiZihVKSIsMl0sWzAsMl0sWzIsM10sWzEsM10sWzQsMl1d
\[\begin{tikzcd}
	& {\F'(U)} & {\F(U)} \\
	0 & {\F_x} & {\F'_x}
	\arrow["{f(U)}"', from=1-2, to=1-3]
	\arrow["{f_x}"', from=2-2, to=2-3]
	\arrow[from=1-2, to=2-2]
	\arrow[from=1-3, to=2-3]
	\arrow[from=2-1, to=2-2]
	\arrow[from=2-2, to=2-3]
\end{tikzcd}\]
pour les flèches évidentes et par exactitude de la ligne du bas,
on a $(U,s)=(V_x,0)\in \F_x$ pour tout $x\in U$. Quitte à remplacer
$V_x$ par $U\cap V_x$, on peut supposer $V_x\subset U$. En particulier,
on obtient un recouvrement de $U$ par des ouverts $V_x$ tels que 
\[s|_{V_x}=0\]
pour tout $x\in U$. D'où $s=0\in \F(U)$ car $\F$ est un faisceau. On a 
montré que $f(U)$ est injective pour tout ouvert $U$ de $X$. 

On montre maintenant le second point par double inclusion, d'abord
$\ker(g(U))\subset \im(f(U))$ : Soit $s\in \ker(g(U))$ et $x\in U$, par
commutativité de 
% https://q.uiver.app/#q=WzAsNCxbMCwwLCJcXEYoVSkiXSxbMSwwLCJcXEYnJyhVKSJdLFswLDEsIlxcRl94Il0sWzEsMSwiXFxGJydfeCJdLFswLDEsImcoVSkiLDJdLFswLDJdLFsyLDMsImdfeCIsMl0sWzEsM11d
\[\begin{tikzcd}
	{\F(U)} & {\F''(U)} \\
	{\F_x} & {\F''_x}
	\arrow["{g(U)}"', from=1-1, to=1-2]
	\arrow[from=1-1, to=2-1]
	\arrow[from=1-2, to=2-2]
	\arrow["{g_x}"', from=2-1, to=2-2]
\end{tikzcd}\]
on a, $g_x((U,s))=0$. Maintenant par exactitude de
% https://q.uiver.app/#q=WzAsMyxbMCwwLCJcXEYnX3giXSxbMSwwLCJcXEZfeCJdLFsyLDAsIlxcRicnX3giXSxbMCwxLCJmX3giLDJdLFsxLDIsImdfeCIsMl1d
\[\begin{tikzcd}
	{\F'_x} & {\F_x} & {\F''_x}
	\arrow["{f_x}"', from=1-1, to=1-2]
	\arrow["{g_x}"', from=1-2, to=1-3]
\end{tikzcd}\]
il existe $(V_x, s'_x)\in \F'_x$ tel que $f_x((V_x,s'_x))=(U,s)$. Quitte
à prendre $V_x\cap U=V_x$ on peut supposer $V_x\subset U$. Maintenant
par commutativité de
% https://q.uiver.app/#q=WzAsOSxbMSwwLCJcXEYnKFUpIl0sWzIsMCwiXFxGKFUpIl0sWzEsMiwiXFxGJ194Il0sWzIsMiwiXFxGX3giXSxbMCwyLCIwIl0sWzAsMCwiMCJdLFswLDEsIjAgIl0sWzEsMSwiXFxGJyhWX3gpIl0sWzIsMSwiXFxGKFZfeCkiXSxbMCwxLCJmKFUpIiwyXSxbMiwzLCJmX3giLDJdLFs0LDJdLFs1LDBdLFs2LDddLFs3LDgsImYoVl94KSIsMl0sWzEsOF0sWzAsN10sWzcsMl0sWzgsM11d
\[\begin{tikzcd}
	0 & {\F'(U)} & {\F(U)} \\
	{0 } & {\F'(V_x)} & {\F(V_x)} \\
	0 & {\F'_x} & {\F_x}
	\arrow[from=1-1, to=1-2]
	\arrow["{f(U)}"', from=1-2, to=1-3]
	\arrow[from=1-2, to=2-2]
	\arrow[from=1-3, to=2-3]
	\arrow[from=2-1, to=2-2]
	\arrow["{f(V_x)}"', from=2-2, to=2-3]
	\arrow[from=2-2, to=3-2]
	\arrow[from=2-3, to=3-3]
	\arrow[from=3-1, to=3-2]
	\arrow["{f_x}"', from=3-2, to=3-3]
\end{tikzcd}\]
on obtient $f(V_x)(s'_x)=s|_{V_x}$ pour chaque $x\in U$ et 
$U=\cup_{x\in U} V_x$. Enfin, pour tout $x,x'\in U$, par commutativité
de
% https://q.uiver.app/#q=WzAsNixbMCwxLCJcXEYnKFZfeFxcY2FwIFZfe3gnfSkiXSxbMSwxLCJcXEYnKFZfeFxcY2FwIFZfe3gnfSkiXSxbMCwwLCJcXEYnKFZfeCkiXSxbMSwwLCJcXEYoVl94KSJdLFswLDIsIlxcRicoVl97eCd9KSJdLFsxLDIsIlxcRihWX3t4J30pIl0sWzAsMV0sWzIsM10sWzIsMF0sWzMsMV0sWzQsNV0sWzQsMF0sWzUsMV1d
\[\begin{tikzcd}
	{\F'(V_x)} & {\F(V_x)} \\
	{\F'(V_x\cap V_{x'})} & {\F'(V_x\cap V_{x'})} \\
	{\F'(V_{x'})} & {\F(V_{x'})}
	\arrow[from=1-1, to=1-2]
	\arrow[from=1-1, to=2-1]
	\arrow[from=1-2, to=2-2]
	\arrow[from=2-1, to=2-2]
	\arrow[from=3-1, to=2-1]
	\arrow[from=3-1, to=3-2]
	\arrow[from=3-2, to=2-2]
\end{tikzcd}\]
on a 
\[f(V_x\cap V_{x'})(s'_x|_{V_x\cap V_{x'}})=s|_{V_x\cap V_{x'}}=
f(V_x\cap V_{x'})(s'_{x'}|_{V_x\cap V_{x'}})\]
d'où par injectivité de $f(V_x\cap V_{x'})$ on a 
\[s'_x|_{V_x\cap V_{x'}}=s'_{x'}|_{V_x\cap V_{x'}}\]
comme $\F'$ est un faisceau on peut relever les $s'_x$ en un $s'\in U$.
Par commutativité du carré du haut dans l'avant dernier diagramme, comme
$\F$ est un faisceau et $f(V_x)(s'|_{V_x})=s|_{V_x}$ on obtient que
$f(U)(s')=s$ d'où $\ker(g(U))\subset \im(f(U))$. 

Soit maintenant $s=f(U)(s')$ pour $s'\in \F'(U)$. Pour tout $x\in U$,
par commutativité de 
% https://q.uiver.app/#q=WzAsNixbMCwxLCJcXEYnX3giXSxbMSwxLCJcXEZfeCJdLFsyLDEsIlxcRicnX3giXSxbMCwwLCJcXEYnKFUpIl0sWzEsMCwiXFxGKFUpIl0sWzIsMCwiXFxGJycoVSkiXSxbMCwxLCJmX3giLDJdLFsxLDIsImdfeCIsMl0sWzMsNF0sWzQsNV0sWzMsMF0sWzQsMV0sWzUsMl1d
\[\begin{tikzcd}
	{\F'(U)} & {\F(U)} & {\F''(U)} \\
	{\F'_x} & {\F_x} & {\F''_x}
	\arrow[from=1-1, to=1-2]
	\arrow[from=1-1, to=2-1]
	\arrow[from=1-2, to=1-3]
	\arrow[from=1-2, to=2-2]
	\arrow[from=1-3, to=2-3]
	\arrow["{f_x}"', from=2-1, to=2-2]
	\arrow["{g_x}"', from=2-2, to=2-3]
\end{tikzcd}\]
et par exactitude de la ligne du bas on a 
\[(V_x,0)=g_x(f_x((U,s')))=g_x((U,s)).\]
On obtient un recouvrement 
$U=\cup_x V_x$ de $U$ par les $V_x$ tel que 
$g(U)(s)|_{V_x}=g(V_x)(s|_{V_x})=0$.
Comme $\F''$ est un faisceau, on obtient $g(U)(s)=0$ d'où 
$\ker(g(U))=\im(f(U))$.

\section*{Exercice 0.2}

\section*{Exercice 0.3}
On utilisera librement dans tout l'exercice que $Sh(X)$ est une
catégorie abélienne avec les résultats suivants du cours : 
\begin{enumerate}
	\item Le noyau d'une flèche $f\colon F\to F'$ dans $Sh(X)$
		coincide avec le faisceau défini par 
		\[\ker(U):=\ker(\F(U)\to\F'(U))\]
	\item L'image d'une flèche $f\colon F\to F'$ dans $Sh(X)$
		coincide avec le faisceautisé du préfaisceau défini
		par
		\[\im\colon U\mapsto \im(U):=\im(\F(U)\to\F'(U))\]
	\item Le conoyau d'une flèche $f\colon F\to F'$ dans $Sh(X)$
		coincide avec le faisceautisé du préfaisceau défini
		\[\coker\colon U\mapsto \coker(U):=\coker(\F(U)\to\F'(U))\]
	\item Le faisceau $0_{Sh(X)}$ défini par $0_{Sh(X)}(U):= 0_{Ab}$
		est un objet zéro dans $Sh(X)$.
\end{enumerate}
On note $(\_)^\sharp\colon PSh(X)\to Sh(X)$ le foncteur de 
faisceautisation. Enfin, si $\F$ est un faisceau sur un espace 
topologique on pourra décrire une section $\bar s\colon U\to Et(\F)$ 
de l'espace étalé de $\F$ comme un tuple 
$(s_x)_{x\in U}=(\bar s(x))_{x\in U}$.

\begin{center}
	\textbf{1)}
\end{center}
Soit $\F$ un faisceau dans $Sh(X)$. Pour tout $x\in X$, on a 
$0=0_{F_x}\in \F_x$ un élément neutre car
$\F_x$ est un groupe abélien. En particulier, pour tout ouverts 
$V\subseteq U$ de $X$, si $s=(s_x)_{x\in V}\in C^0(\F(V))$ alors
si on note $s':=(s'_x)_{x\in U}$ la section telle que $s'_x=s_x$ pour
$x\in V$ et $s'_x=0$ pour $x\in U-V$ on a $s'|_V=s$ d'où $C^0(\F)$
est flasque.

\begin{center}
	\textbf{2)}
\end{center}
Soit $\F\in Sh(X)$. On définit $i^0(\F)\colon \F\to C^0(\F)$ le
morphisme de faisceau défini par 
$i^0(\F)(U)\colon s\mapsto (s_x)_{x\in U}$ pour tout ouvert $U$ de $X$
et où $s_x\in \F_x$ est 
l'image de $s$ dans la fibre $\F_x$ induite par $\F$. On montre que
$i^0(\F)$ est injectif en montrant que $\ker(i^0(\F))=0_{Sh(X)}$. 
Soit $U$ un ouvert sur $X$ et $s\in \ker(i^0(\F))(U)$, alors 
$(s_x)_{x\in U}=((V_x, 0))_{x\in U}$, en particulier $s|_{V_x}=0$ 
et $\cup V_x = U$ d'où $s=0\in \F(U)$.

\begin{center}
	\textbf{3)}
\end{center}
On définit en plus $Z^0(\F)=\F$, $d_0^0\colon C^0(\F)\to Z^1(\F)$
et $d^0=i^0(\F)\circ(d_0^0)$. On suppose maintenant défini
$d_0^{i-1}\colon C^{i-1}(\F) \to Z^i(\F)$ et 
\[d^{i-1}=i^0(Z^{i-1})\circ(d_0^{i-1})\colon C^{i-1}(\F)\to C^i(\F)\]
pour $n\geq i\geq 1$. 

\begin{center}
	\textbf{4)}
\end{center}

\begin{center}
	\textbf{5)}
\end{center}
On montre d'abord que $C^\bullet(\_)$ est un foncteur. Étant donné
$g\circ f \colon \F'\to \F\to \F'' $ dans $Sh(X)$. On pose d'abord
$f_0=f$ et $g_0=g$, on définit ensuite $Z^n(\F')\to Z^n(\F)$ pour tout
$n\geq 1$.

Supposons d'abord que $C^0(\_)$ est exact et soit 
% https://q.uiver.app/#q=WzAsNSxbMSwwLCJcXEYnIl0sWzAsMCwiMCJdLFsyLDAsIlxcRiJdLFszLDAsIlxcRicnIl0sWzQsMCwiKCopIl0sWzEsMF0sWzAsMiwiZiIsMl0sWzIsMywiZyIsMl1d
\[\begin{tikzcd}
	0 & {\F'} & \F & {\F''} & 0 & {(*)}
	\arrow[from=1-1, to=1-2]
	\arrow["f"', from=1-2, to=1-3]
	\arrow["g"', from=1-3, to=1-4]
	\arrow[from=1-4, to=1-5]
\end{tikzcd}\]
une suite exacte dans $Sh(X)$. On montre que le morphisme de complexe
induit $0\to C^\bullet(\F')\to C^\bullet(\F)\to C^\bullet(\F'')\to 0$
est exact. Il suffit de montrer qu'il l'est à chaque niveau, c'est à
dire que pour tout $n\geq 0$ 
\[0\to C^n(\F')\to C^n(\F)\to C^n(\F'')\to 0\]
est exact. En plus on a supposé que $C^0(\_)$ était exact, il suffit
donc de montrer que 
\[0\to Z^n(\F')\to Z^n(\F)\to Z^n(\F'')\to 0\]
est exact. On le montre par récurrence sur $n$. En posant $Z^0(\F):=\F$
on obtient le cas $n=0$ par l'hypothèse $(*)$. Supposons 
maintenant le résultat vrai pour $0\leq i\leq n-1$.
On considère le diagramme 

% https://q.uiver.app/#q=WzAsOCxbMSwwLCJBIl0sWzIsMCwiQiJdLFszLDAsIkMiXSxbNCwwLCIwIl0sWzMsMSwiQyciXSxbMiwxLCJCJyJdLFsxLDEsIkEnIl0sWzAsMSwiMCJdLFswLDEsIlxcYWxwaGEiXSxbMSwyLCJcXGJldGEiXSxbMiwzXSxbNSw0LCJcXGJldGEnIl0sWzYsNSwiXFxhbHBoYSciXSxbNyw2XSxbMCw2LCJ1Il0sWzEsNSwidiJdLFsyLDQsInciXV0=
\[\begin{tikzcd}
	0 & {C^{n-1}(\F')} & {C^{n-1}(\F)} & {C^{n-1}(\F'')} & 0 \\
	0 & {Z^n(\F')} & {Z^n(\F)} & {Z^n(\F'')} & 0
	\arrow["C^0(f_{n-1})", from=1-2, to=1-3]
	\arrow["d'^{n-1}", from=1-2, to=2-2]
	\arrow["C^0(g_{n-1})", from=1-3, to=1-4]
	\arrow["d^{n-1}", from=1-3, to=2-3]
	\arrow[from=1-4, to=1-5]
	\arrow[from=1-1, to=1-2]
	\arrow[from=2-4, to=2-5]
	\arrow["d''^{n-1}", from=1-4, to=2-4]
	\arrow[from=2-1, to=2-2]
	\arrow["{f_n}", from=2-2, to=2-3]
	\arrow["{g_n}", from=2-3, to=2-4]
\end{tikzcd}\]
où on note que $f_0=f$, $g_0=g$ et $f_n$ est la flèche induite de 
$d_0^{n-1}\circ C^0(f_{n-1})$ avec $C^0(d_0^{n-1})=d^{n-1}$ pour
$n\geq 1$ 
par passage au quotient dans la catégorie abélienne $Sh(X)$, de 
même pour $d_0^''$ et $g_n$  via $d_0''^{n-1}\circ C^0(g_{n-1})$.





On montre maintenant que $C^0(\_)\colon Sh(X)\to Sh(X)$ est
exact. Soit $f\colon \F'\to\F$ dans $Sh(X)$. La flèche $C^0(f)$ est 
donnée par
\[C^0(f)(U)\colon (s'_x)_{x\in U}\mapsto (f_x(s'_x))_{x\in U}\] 
où $f_x\colon \F'_x\to \F_x$ est la flèche induite par $f$ sur les
fibres. Il suffit de montrer que 
$C^0(\ker(\F'\to \F))=\ker(C^0(\F')\to C^0(\F))$
et $C^0(\im(\F'\to \F))=\im(C^0(\F')\to C^0(\F))$. Si c'est le cas,
alors comme $C^0(0_{Sh(X)})=0_{Sh(X)}$, car tout produit d'objets
terminaux est terminal, on obtient directement pour toute suite exacte
% https://q.uiver.app/#q=WzAsNSxbMSwwLCJcXEYnIl0sWzAsMCwiMCJdLFsyLDAsIlxcRiJdLFszLDAsIlxcRicnIl0sWzQsMCwiKCopIl0sWzEsMF0sWzAsMiwiZiIsMl0sWzIsMywiZyIsMl1d
\[\begin{tikzcd}
	0 & {\F'} & \F & {\F''} & 0 & {(*)}
	\arrow[from=1-1, to=1-2]
	\arrow["f"', from=1-2, to=1-3]
	\arrow["g"', from=1-3, to=1-4]
	\arrow[from=1-4, to=1-5]
\end{tikzcd}\]
dans $Sh(X)$ que
% https://q.uiver.app/#q=WzAsNSxbMSwwLCJcXEYnIl0sWzAsMCwiMCJdLFsyLDAsIlxcRiJdLFszLDAsIlxcRicnIl0sWzQsMCwiKCopIl0sWzEsMF0sWzAsMiwiZiIsMl0sWzIsMywiZyIsMl1d
\[\begin{tikzcd}
	0 & {C^0(\F')} & {C^0(\F)} & {C^0(\F'')} & 0 & {(*)}
	\arrow[from=1-1, to=1-2]
	\arrow["C^0(f)"', from=1-2, to=1-3]
	\arrow["C^0(g)"', from=1-3, to=1-4]
	\arrow[from=1-4, to=1-5]
\end{tikzcd}\]
est exacte par fonctorialité de $C^0(\_)$. 

On commence par montrer que
$C^0(\ker(\F'\to \F))=\ker(C^0(\F')\to C^0(\F))$. Par la description
donnée en $1.$ on a 
\[\ker(C^0(\F')\to C^0(\F))(U)=\ker(\prod_{x\in U} f_x\colon \prod_{x\in U} \F'_x\to \prod_{x\in U} \F_x)\]
et de plus $C^0(\ker(\F'\to \F))(U)=\prod_{x\in U}\ker(\F'\to\F)_x$, 
il suffit donc de montrer que si 
\[(s_x)_{x\in U}\in \ker(\prod_{x\in U} f_x\colon \prod_{x\in U} \F'_x\to \prod_{x\in U} \F_x)\]
alors pour tout $x\in U$ on a $s_x\in \ker(f_x)$ et inversement.
Mais c'est immédiat par définition du noyau dans $Ab$. 

On montre maintenant que 
$C^0(\im(\F'\to \F)^\sharp)=\im(C^0(\F')\to C^0(\F))^\sharp$.
Par la description donnée en 2. on peut écrire 
\[\im(C^0(\F')\to C^0(\F))(U)=\{(f_xs'_x)_{x\in U}| s'_x\in \F'_x\}\]
et on remarque que c'est déjà un faisceau par définition des 
restrictions. En plus $C^0(\im(\F'\to \F)^\sharp)=C^0(\im(\F'\to\F))$
car un préfaisceau et son faisceautisé ont les mêmes fibres, en
particulier $C^0(\im(\F'\to\F))$ et $\im(C^0(\F')\to C^0(\F))$ ont 
la même description donc coincident. On en déduit de $C^0(\_)$ est
exact.











\end{document}


\documentclass[a4paper,12pt]{book}
\usepackage{amsmath,  amsthm,enumerate}
\usepackage{csquotes}
\usepackage[provide=*,french]{babel}
\usepackage[dvipsnames]{xcolor}
\usepackage{quiver, tikz}

%symbole caligraphique
\usepackage{mathrsfs}

%hyperliens
\usepackage{hyperref}

%pseudo-code
\usepackage{algorithm}
\usepackage{algpseudocode}

\usepackage{fancyhdr}

\pagestyle{fancy}
\addtolength{\headwidth}{\marginparsep}
\addtolength{\headwidth}{\marginparwidth}
\renewcommand{\chaptermark}[1]{\markboth{#1}{}}
\renewcommand{\sectionmark}[1]{\markright{\thesection\ #1}}
\fancyhf{}
\fancyfoot[C]{\thepage}
\fancyhead[LO]{\textit \leftmark}
\fancyhead[RE]{\textit \rightmark}
\renewcommand{\headrulewidth}{0pt} % and the line
\fancypagestyle{plain}{%
    \fancyhead{} % get rid of headers
}

%bibliographie
\usepackage[
backend=biber,
style=alphabetic,
sorting=ynt
]{biblatex}

\addbibresource{bib.bib}

\usepackage{appendix}
\renewcommand{\appendixpagename}{Annexe}

\definecolor{wgrey}{RGB}{148, 38, 55}

\setlength\parindent{24pt}

\newcommand{\Z}{\mathbb{Z}}
\newcommand{\R}{\mathbb{R}}
\newcommand{\rel}{\omathcal{R}}
\newcommand{\Q}{\mathbb{Q}}
\newcommand{\C}{\mathbb{C}}
\newcommand{\Cat}{\mathcal{C}}
\newcommand{\Aat}{\mathcal{A}}
\newcommand{\N}{\mathbb{N}}
\newcommand{\K}{\mathbb{K}}
\newcommand{\A}{\mathbb{A}}
\newcommand{\B}{\mathcal{B}}
\newcommand{\Or}{\mathcal{O}}
\newcommand{\F}{\mathscr F}
\newcommand{\Hom}{\textrm{Hom}}
\newcommand{\disc}{\textrm{disc}}
\newcommand{\Pic}{\textrm{Pic}}
\newcommand{\End}{\textrm{End}}
\newcommand{\Spec}{\textrm{Spec}}
\newcommand{\Supp}{\textrm{Supp}}
\newcommand{\Ouv}{\textrm{Ouv}}
\renewcommand{\Im}{\textrm{Im}}


\newcommand{\cL}{\mathscr{L}}
\newcommand{\G}{\mathscr{G}}
\newcommand{\D}{\mathscr{D}}
\newcommand{\E}{\mathscr{E}}
\renewcommand{\P}{\mathscr{P}}
\renewcommand{\H}{\mathscr{H}}

\makeatletter
\newcommand{\colim@}[2]{%
  \vtop{\m@th\ialign{##\cr
    \hfil$#1\operator@font colim$\hfil\cr
    \noalign{\nointerlineskip\kern1.5\ex@}#2\cr
    \noalign{\nointerlineskip\kern-\ex@}\cr}}%
}
\newcommand{\colim}{%
  \mathop{\mathpalette\colim@{\rightarrowfill@\scriptscriptstyle}}\nmlimits@
}
\renewcommand{\varprojlim}{%
  \mathop{\mathpalette\varlim@{\leftarrowfill@\scriptscriptstyle}}\nmlimits@
}
\renewcommand{\varinjlim}{%
  \mathop{\mathpalette\varlim@{\rightarrowfill@\scriptscriptstyle}}\nmlimits@
}
\makeatother

\theoremstyle{plain}
\newtheorem{thm}{Théoreme}
\newtheorem{lem}{Lemme}
\newtheorem{prop}{Proposition}
\newtheorem{cor}{Corollaire}
\newtheorem{heur}{Heuristique}
\newtheorem{rem}{Remarque}
\newtheorem{note}{Note}

\theoremstyle{definition}
\newtheorem{conj}{Conjecture}
\newtheorem{prob}{Problème}
\newtheorem{quest}{Question}
\newtheorem{prot}{Protocole}
\newtheorem{algo}{Algorithme}
\newtheorem{defn}{Définition}
\newtheorem{exmp}{Exemples}
\newtheorem{exo}{Exercices}
\newtheorem{ex}{Exemple}
\newtheorem{exs}{Exemples}

\theoremstyle{remark}

\definecolor{wgrey}{RGB}{148, 38, 55}
\definecolor{wgreen}{RGB}{100, 200,0} 
\hypersetup{
    colorlinks=true,
    linkcolor=wgreen,
    urlcolor=wgrey,
    filecolor=wgrey
}

\title{Notes sur les faisceaux}
\date{2024-2025}

\begin{document}
\maketitle
\tableofcontents

\chapter{Faisceau}

\section{Caractérisation}

Étant donné un foncteur $\F\colon \Ouv(X)\to \Aat$, pour chaque 
recouvrement $\bigcup_i U_i = X$, on peut définir la suite
\[0\to\F(U)\to\prod_i \F(U_i)\to\prod_{(i,j)}\F(U_i\cap U_j)\]
notée $C((U_i)_i,\F)$. Maintenant 

\[\textrm{Être un faisceau c'est rendre
la suite exacte.}\]


\section{Préfaisceau séparé}
Un préfaisceau est séparé si il rend 
\[0\to \F(U)\to \prod_{x\in U} \F_x\]
injective.

\section{Faisceautisation et espace étalé}
Ducoup étant donné le faisceau
\[C^0(\F)\colon U\mapsto \prod_{x\in U} \F_x\]
on va trouver un sous-faisceau $\F^\sharp\subseteq C^0(\F)$.
\subsection{Faisceautisation I}
Y'a deux manières de le construire, la simple
\[\F^\sharp(U):=\{(s_x)_x\in \prod_{x\in U} \F_x|\forall x, \exists V_x,
t\in\F(V_x),\forall P\in V_x,t_P=s_P\}\]
mais technique. Essentiellement, une section c'est la donnée d'un
recouvrement $(f_U, U)$ à équivalence faible près.
\subsection{Espace étalé}
L'autre c'est avec l'espace étalé, on définit 
\[Et(\F):=\bigsqcup_{x\in X} \F_x\]
ensuite on définit la projection 
\[\pi \colon Et(\F)\to X\]
via $s_x\mapsto x$. Puis enfin, à toute $s\in \F(U)$, on peut associer
\[\bar s\colon U\to Et(\F)\]
qui a $x$ associe $s_x$.
\begin{rem}
  On remarque que c'est des sections
$\pi\circ\bar s=id_U$. 
\end{rem}
Pour la \textbf{topologie} donc via 
\[\F(U)\ni s\mapsto \bar s\in \Hom(U, Et(\F)\]
on prends la topologie finale associée à l'ensemble de $\bar s$.
\begin{rem}
  Pour rappel $U$ est ouvert sur $X$ ssi $\bar s^{-1}(U)$ est ouvert
  pour tout $s\in \F(U)$.
\end{rem}

\subsection{Faisceautisation II}
Maintenant on peut définir 
\[\F^+(U):= U\mapsto \{f\in\Hom_{Top}(U,Et(\F))| \pi\circ f = id_U\}\]
déjà on remarque que si $P\in U$ alors $f(P)\in \F_P$ parce que c'est
une section. D'où c'est \textbf{injectif} et
\[\F^+(U)\subset \prod_{x\in U} \F_x\]
maintenant la continuité dit que $f^{-1}(f(U)\cap S)=U$ est ouvert 
ssi localement il existe $s$ t.q $\bar s^{-1}(f(U)\cap S)= U$. Autrement
dit $\bar s(P)=f(P)$ pour tout $P\in U$. On voit en fait que par 
définition y'a un recouvrement de $U=\bigcup_i U_i$ où
\begin{enumerate}
  \item $f|_{U_i}=\bar s_i$.
  \item Sur $U_i\cap U_j$, $\bar s_i=\bar s_j$ ont les mêmes fibres.
\end{enumerate}
\begin{rem}
  Explorer un peu si y'a essentiellement un unique manière d'écrire
  $f$ à "équivalence faible".
\end{rem}
\section{Adjonctions}
\section{Image directe et inverse}



%\printbibliography

\end{document}


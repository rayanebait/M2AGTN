\documentclass[a4paper,12pt]{book}
\usepackage{amsmath,  amsthm,enumerate}
\usepackage{csquotes}
\usepackage[provide=*,french]{babel}
\usepackage[dvipsnames]{xcolor}
\usepackage{quiver, tikz}

%symbole caligraphique
\usepackage{mathrsfs}

%hyperliens
\usepackage{hyperref}

%pseudo-code
\usepackage{algorithm}
\usepackage{algpseudocode}

\usepackage{fancyhdr}

\pagestyle{fancy}
\addtolength{\headwidth}{\marginparsep}
\addtolength{\headwidth}{\marginparwidth}
\renewcommand{\chaptermark}[1]{\markboth{#1}{}}
\renewcommand{\sectionmark}[1]{\markright{\thesection\ #1}}
\fancyhf{}
\fancyfoot[C]{\thepage}
\fancyhead[LO]{\textit \leftmark}
\fancyhead[RE]{\textit \rightmark}
\renewcommand{\headrulewidth}{0pt} % and the line
\fancypagestyle{plain}{%
    \fancyhead{} % get rid of headers
}

%bibliographie
\usepackage[
backend=biber,
style=alphabetic,
sorting=ynt
]{biblatex}

\addbibresource{bib.bib}

\usepackage{appendix}
\renewcommand{\appendixpagename}{Annexe}

\definecolor{wgrey}{RGB}{148, 38, 55}

\setlength\parindent{24pt}

\newcommand{\Z}{\mathbb{Z}}
\newcommand{\R}{\mathbb{R}}
\newcommand{\rel}{\omathcal{R}}
\newcommand{\Q}{\mathbb{Q}}
\newcommand{\C}{\mathbb{C}}
\newcommand{\Cat}{\mathcal{C}}
\newcommand{\N}{\mathbb{N}}
\newcommand{\K}{\mathbb{K}}
\newcommand{\A}{\mathbb{A}}
\newcommand{\B}{\mathcal{B}}
\newcommand{\Or}{\mathcal{O}}
\newcommand{\F}{\mathscr F}
\newcommand{\Hom}{\textrm{Hom}}
\newcommand{\disc}{\textrm{disc}}
\newcommand{\Pic}{\textrm{Pic}}
\newcommand{\End}{\textrm{End}}
\newcommand{\Spec}{\textrm{Spec}}
\newcommand{\Supp}{\textrm{Supp}}
\renewcommand{\Im}{\textrm{Im}}


\newcommand{\cL}{\mathscr{L}}
\newcommand{\G}{\mathscr{G}}
\newcommand{\D}{\mathscr{D}}
\newcommand{\E}{\mathscr{E}}
\renewcommand{\P}{\mathscr{P}}
\renewcommand{\H}{\mathscr{H}}

\makeatletter
\newcommand{\colim@}[2]{%
  \vtop{\m@th\ialign{##\cr
    \hfil$#1\operator@font colim$\hfil\cr
    \noalign{\nointerlineskip\kern1.5\ex@}#2\cr
    \noalign{\nointerlineskip\kern-\ex@}\cr}}%
}
\newcommand{\colim}{%
  \mathop{\mathpalette\colim@{\rightarrowfill@\scriptscriptstyle}}\nmlimits@
}
\renewcommand{\varprojlim}{%
  \mathop{\mathpalette\varlim@{\leftarrowfill@\scriptscriptstyle}}\nmlimits@
}
\renewcommand{\varinjlim}{%
  \mathop{\mathpalette\varlim@{\rightarrowfill@\scriptscriptstyle}}\nmlimits@
}
\makeatother

\theoremstyle{plain}
\newtheorem{thm}{Théoreme}
\newtheorem{lem}{Lemme}
\newtheorem{prop}{Proposition}
\newtheorem{cor}{Corollaire}
\newtheorem{heur}{Heuristique}
\newtheorem{rem}{Remarque}
\newtheorem{note}{Note}

\theoremstyle{definition}
\newtheorem{conj}{Conjecture}
\newtheorem{prob}{Problème}
\newtheorem{quest}{Question}
\newtheorem{prot}{Protocole}
\newtheorem{algo}{Algorithme}
\newtheorem{defn}{Définition}
\newtheorem{exmp}{Exemples}
\newtheorem{exo}{Exercices}
\newtheorem{ex}{Exemple}
\newtheorem{exs}{Exemples}

\theoremstyle{remark}

\definecolor{wgrey}{RGB}{148, 38, 55}
\definecolor{wgreen}{RGB}{100, 200,0} 
\hypersetup{
    colorlinks=true,
    linkcolor=wgreen,
    urlcolor=wgrey,
    filecolor=wgrey
}

\title{Notes de (co)-Homologie (des faisceaux)}
\date{2024-2025}

\begin{document}
\maketitle
\tableofcontents
\chapter{Point sur le cours 09/10/2024, catégories abéliennes}

\section{Catégories abéliennes}
On rajoute tout les égalisateurs et coégalisateurs.
\section{Catégories additives}
La section du dessous motive la définition comme, catégories avec objet
zéro, produits et coproduits. Je sais pas si on suppose que le hom est
un groupe ducoup mais c'est un monoide abélien.

\section{Catégories préadditives}
D'abord y'a les catégories préadditives, concrètement c'est 
juste que les $\Hom$ sont des monoides abéliens. Mais on peut
voir en fait que c'est une conséquence de l'existence de morphismes
nuls. Où simplement d'objet zéro. On montre que le produit est 
un coproduit :
% https://q.uiver.app/#q=WzAsNSxbMCwxLCJBIl0sWzEsMCwiQSJdLFsxLDIsIkIiXSxbMSwxLCJBXFx0aW1lcyBCIl0sWzIsMSwiQiJdLFswLDEsImlkIl0sWzAsMiwiT197QSxCfSIsMl0sWzAsM10sWzMsMV0sWzMsMl0sWzQsMiwiaWQiXSxbNCwxLCJPX3tCLEF9IiwyXSxbNCwzXV0=
\[\begin{tikzcd}
	& A \\
	A & {A\times B} & B \\
	& B
	\arrow["id", from=2-1, to=1-2]
	\arrow["i_1", from=2-1, to=2-2]
	\arrow["{O_{A,B}}"', from=2-1, to=3-2]
	\arrow[from=2-2, to=1-2]
	\arrow[from=2-2, to=3-2]
	\arrow["{O_{B,A}}"', from=2-3, to=1-2]
	\arrow["i_2",from=2-3, to=2-2]
	\arrow["id", from=2-3, to=3-2]
\end{tikzcd}\]
puis si 
% https://q.uiver.app/#q=WzAsNCxbMCwwLCJBIl0sWzEsMSwiQVxcdGltZXMgQiJdLFswLDIsIkIiXSxbMiwxLCJYIl0sWzAsMSwiaV8xIiwyXSxbMiwxLCJpXzIiXSxbMCwzLCJmIiwwLHsiY3VydmUiOi0xfV0sWzIsMywiZyIsMix7ImN1cnZlIjoxfV0sWzEsM11d
\[\begin{tikzcd}
	A \\
	& {A\times B} & X \\
	B
	\arrow["{i_1}"', from=1-1, to=2-2]
	\arrow["f", curve={height=-6pt}, from=1-1, to=2-3]
	\arrow["h"{description}, from=2-2, to=2-3]
	\arrow["{i_2}", from=3-1, to=2-2]
	\arrow["g"', curve={height=6pt}, from=3-1, to=2-3]
\end{tikzcd}\]
Alors avec $id = i_1\circ p_1+i_2\circ p_2$ on obtient directement que
$h=f\circ p_1+g\circ p_2$. Maintenant la codiagonale $\delta_A\colon 
A\times A\to A$ peut être décrite via simplement 
% https://q.uiver.app/#q=WzAsNCxbMCwxLCJBXFx0aW1lcyBBIl0sWzAsMCwiQSJdLFsxLDEsIkEiXSxbMCwyLCJcXGJ1bGxldCJdLFsxLDIsImlkIl0sWzMsMiwiaWQiLDJdLFsxLDBdLFszLDBdLFswLDJdXQ==
\[\begin{tikzcd}
	A \\
	{A\times A} & A \\
	A
	\arrow[from=1-1, to=2-1]
	\arrow["id", from=1-1, to=2-2]
	\arrow[from=2-1, to=2-2]
	\arrow[from=3-1, to=2-1]
	\arrow["id"', from=3-1, to=2-2]
\end{tikzcd}\]
i.e. $\delta_A=id\circ p_1 + id\circ p_2$. Mais le point important 
c'est que on peut réecrire $f+g$ comme $\delta_B\circ (f,g)$ :
% https://q.uiver.app/#q=WzAsNSxbMCwxLCJBIl0sWzIsMCwiQiJdLFsyLDIsIkIiXSxbMiwxLCJCXFx0aW1lcyBCIl0sWzMsMSwiQiJdLFswLDEsImciXSxbMCwyLCJmIiwyXSxbMCwzLCIoZixnKSIsMV0sWzMsNCwiXFxkZWx0YV9CIl0sWzMsMl0sWzMsMV1d
\[\begin{tikzcd}
	&& B \\
	A && {B\times B} & B \\
	&& B
	\arrow["g", from=2-1, to=1-3]
	\arrow["{(f,g)}"{description}, from=2-1, to=2-3]
	\arrow["f"', from=2-1, to=3-3]
	\arrow[from=2-3, to=1-3]
	\arrow["{\delta_B}", from=2-3, to=2-4]
	\arrow[from=2-3, to=3-3]
\end{tikzcd}\]
et donc 
\begin{itemize}
    \item la structure de monoide commutatif vient de l'existence de
l'objet zéro.
\end{itemize}


%\printbibliography

\end{document}


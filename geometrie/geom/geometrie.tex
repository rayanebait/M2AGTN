\documentclass[a4paper,12pt]{book}
\usepackage{amsmath,  amsthm,enumerate}
\usepackage{csquotes}
\usepackage[provide=*,french]{babel}
\usepackage[dvipsnames]{xcolor}
\usepackage{quiver, tikz}

%symbole caligraphique
\usepackage{mathrsfs}

%hyperliens
\usepackage{hyperref}

%pseudo-code
\usepackage{algorithm}
\usepackage{algpseudocode}

\usepackage{fancyhdr}

\pagestyle{fancy}
\addtolength{\headwidth}{\marginparsep}
\addtolength{\headwidth}{\marginparwidth}
\renewcommand{\chaptermark}[1]{\markboth{#1}{}}
\renewcommand{\sectionmark}[1]{\markright{\thesection\ #1}}
\fancyhf{}
\fancyfoot[C]{\thepage}
\fancyhead[LO]{\textit \leftmark}
\fancyhead[RE]{\textit \rightmark}
\renewcommand{\headrulewidth}{0pt} % and the line
\fancypagestyle{plain}{%
    \fancyhead{} % get rid of headers
}

%bibliographie
\usepackage[
backend=biber,
style=alphabetic,
sorting=ynt
]{biblatex}

\addbibresource{bib.bib}

\usepackage{appendix}
\renewcommand{\appendixpagename}{Annexe}

\definecolor{wgrey}{RGB}{148, 38, 55}

\setlength\parindent{24pt}

\newcommand{\Z}{\mathbb{Z}}
\newcommand{\R}{\mathbb{R}}
\newcommand{\rel}{\omathcal{R}}
\newcommand{\Q}{\mathbb{Q}}
\newcommand{\C}{\mathbb{C}}
\newcommand{\N}{\mathbb{N}}
\newcommand{\K}{\mathbb{K}}
\newcommand{\A}{\mathbb{A}}
\newcommand{\B}{\mathcal{B}}
\newcommand{\Or}{\mathcal{O}}
\newcommand{\F}{\mathscr F}
\newcommand{\Hom}{\textrm{Hom}}
\newcommand{\disc}{\textrm{disc}}
\newcommand{\Pic}{\textrm{Pic}}
\newcommand{\End}{\textrm{End}}
\newcommand{\Spec}{\textrm{Spec}}
\newcommand{\Supp}{\textrm{Supp}}
\newcommand{\supp}{\textrm{supp}}
\renewcommand{\Im}{\textrm{Im}}


\newcommand{\cL}{\mathscr{L}}
\newcommand{\G}{\mathscr{G}}
\newcommand{\D}{\mathscr{D}}
\newcommand{\E}{\mathscr{E}}
\renewcommand{\P}{\mathscr{P}}
\renewcommand{\H}{\mathscr{H}}

\makeatletter
\newcommand{\colim@}[2]{%
  \vtop{\m@th\ialign{##\cr
    \hfil$#1\operator@font colim$\hfil\cr
    \noalign{\nointerlineskip\kern1.5\ex@}#2\cr
    \noalign{\nointerlineskip\kern-\ex@}\cr}}%
}
\newcommand{\colim}{%
  \mathop{\mathpalette\colim@{\rightarrowfill@\scriptscriptstyle}}\nmlimits@
}
\renewcommand{\varprojlim}{%
  \mathop{\mathpalette\varlim@{\leftarrowfill@\scriptscriptstyle}}\nmlimits@
}
\renewcommand{\varinjlim}{%
  \mathop{\mathpalette\varlim@{\rightarrowfill@\scriptscriptstyle}}\nmlimits@
}
\makeatother

\theoremstyle{plain}
\newtheorem{thm}[subsection]{Théoreme}
\newtheorem{lem}[subsection]{Lemme}
\newtheorem{prop}[subsection]{Proposition}
\newtheorem{cor}[subsection]{Corollaire}
\newtheorem{heur}{Heuristique}
\newtheorem{rem}{Remarque}
\newtheorem{note}{Note}

\theoremstyle{definition}
\newtheorem{conj}{Conjecture}
\newtheorem{prob}{Problème}
\newtheorem{quest}{Question}
\newtheorem{prot}{Protocole}
\newtheorem{algo}{Algorithme}
\newtheorem{defn}[subsection]{Définition}
\newtheorem{exmp}[subsection]{Exemples}
\newtheorem{exo}[subsection]{Exercices}
\newtheorem{ex}[subsection]{Exemple}
\newtheorem{exs}[subsection]{Exemples}

\theoremstyle{remark}

\definecolor{wgrey}{RGB}{148, 38, 55}
\definecolor{wgreen}{RGB}{100, 200,0} 
\hypersetup{
    colorlinks=true,
    linkcolor=wgreen,
    urlcolor=wgrey,
    filecolor=wgrey
}

\title{Géometrie}
\date{2023-2024}

\begin{document}
\maketitle
\tableofcontents
\chapter*{Introduction}
Je me rappelle plus le but du cours c'était quoi, je sais qu'il
va présenter la courbure de Ricci et montrer pourquoi c'est une 
bonne notion de courbure (une courbure suffisante montre un diamètre fini,
où le suffisant est la courbure de ricci de la n-sphère).

\chapter{Géometrie différentielle}
\section{Définitions}
\begin{defn}[\textbf{Variété topologique}]
    Espace topologique séparé localement homéomorphe à $R^n$ pour un $n$
    fixé à base dénombrable (second countable).
\end{defn}
\begin{defn}[\textbf{Atlas $C^k$}]
    C'est une collection de cartes $(U,\phi)$ tels que
    les changements de cartes sont $C^k$.
\end{defn}
\begin{defn}[\textbf{Structure différentielle}]
    Classe d'équivalence d'atlas, où l'équivalence est donnée par 
    \textit{l'union est un atlas}.
\end{defn}
\begin{defn}[\textbf{Manifold/Variété différentielle}]
    Variété topologique munie d'une structure différentielle $C^{\infty}$.
\end{defn}

\begin{exs}
    \textbf{(1)} Un ouvert $U$ de $R^n$ muni de la classe d'équivalence 
    de $(U,\iota)$ où $\iota$ est l'injection. En particulier, 
    les ouverts $(U\cap U_{\alpha}, \iota_{\alpha})$ sont dans la classe
    d'équivalence (la topologie induite).

    \textbf{(2)} Si on prends $GL_n(\R)\subset R^{n^2}$ on obtient 
    une variété différentielle en regardant celle induite sur l'ouvert
    $\det^{-1}(\R-0)$ qui est homéomorphe par déf à $GL_n(\R)$.
\end{exs}
\section{Morphismes}
\begin{note}
    Les équivalences de critères pour montrer qu'un sous ensemble de 
    $\R^n$ est un submanifold ça a l'air vraiment pratique. À faire
    à un moment donné mais, y'en a deux que je peux facilement 
    citer.
\end{note}
\begin{defn}[Immersion et submersion]
    Une $C^k$-flèche $f\colon U\cap \R^n\to \R^d$ est une immersion
    (resp. submersion) en $P\in U$ si sa différentielle 
    \[df\colon \R^n\to \R^d\]
    est surjective (resp. injective).
\end{defn}
En gros faut regarder le rang de la jacobienne sur $U$, si il est 
constant de rang $n-d$ pour la surjectivité, constant plein pour
l'injectivité.
\begin{prop}[Des critères pour avoir des structures de sous-variété.]
    Soit $M\subset \R^n$, $M$ est une $C^k$-sous-variété de $\R^n$
    de dimension $d$ ssi
    \begin{itemize}
        \item \textbf{Critère local :} Pour tout $P$ il existe
            $g\colon U\cap\R^n\to \R^{n-d}$ une submersion telle que 
            $M\cap U=g^{-1}(\{0\})$.
        \item \textbf{Sous forme de graphe :} Au changement d'axe près,
            pour tout $P\in U$ il existe 
            $F\colon \Omega\subset \R^d\to \R^{n-d}$ une 
            $C^k$-flèche telle que $M\cap U=Graph(F)$.
    \end{itemize}
\end{prop}
Y'a un troisième critère qui demande une immersion 
$h\colon \Omega\to M\cap U$ qui est un homéomorphisme et envoie $0$ sur 
$P$.

\section{Plan, fibré et espace tangent}
Dans une sous-variété $M$ de $\R^n$ c'est facile, on déf
\begin{defn}[Espace tangent]
    Le plan tangent en $P$ de $\R^n$ est l'ensemble 
    \[\{v\in\R^n : \exists c\in C^1(I,M),~v=c'(0),~P=c(0)\}\]
\end{defn}
Sur une variété différentielle abstraite c'est plus compliqué
\begin{defn}[Plan tangent]
    On définit une classe d'équivalence sur les $(I,c)$ où 
    $I$ est un intervalle ouvert de $\R$ et $c\in C^1(I,M)$ avec 
    $c(0)=P$ par $(I_1,c_1)~(I_2,c_2)$ si il existe une carte 
    autour de $P$ t.q 
    \[(\phi\circ c_1)'(0)=(\phi\circ c_2)'(0)\]

    On définit alors le plan tangent comme l'ensemble des vecteurs 
    tangents en $P$ où un vecteur tangent est une classe d'équivalence
    $[I,c]$.
\end{defn}
\begin{rem}
    La flèche $\phi\circ c$ va de $I$ dans $\R^n$.
\end{rem}

Pour la prochaine fois : Structure d'ev sur le plan tangent indép de la 
carte puis fibré.

\section{Partitions de l'unité, paracompacité et local-global.}
Il s'avère que ls manifolds sont T4 et métrisables. On peut montrer 
qu'y sont T4 en montrant qu'ils sont T3 et paracompacts. Et la 
paracompacité permet à son tour de déterminer l'existence de partitions
de l'unité, qui permettent un local-global.
\begin{defn}[\textbf{Raffinement de recouvrements}]
    Un recouvrement $(V_{\alpha})$ raffine $(U_{\beta})$ si pour tout
    $\beta$ on peut trouver $\alpha$ tel que 
    $V_{\alpha}\subset U_{\beta}$.
\end{defn}
\begin{defn}[\textbf{Locale finitude}]
    Une collection d'ensemble $(A_{\alpha})$ inclus dans un manifold $M$
    est localement fini si pour tout $m\in M$ on peut trouver $W_m$ un
    voisinage de $m$ tel que 
    $\#\{\alpha|A_{\alpha}\cap W_m \ne \emptyset\}$ est fini.
\end{defn}
\begin{defn}[\textbf{Partition de l'unité}]
    Une partition de l'unité sur $M$ est un ensemble de fonctions 
    $C^{\infty}$, $(\phi_i)_i$ telles que
    \begin{itemize}
        \item $\{\supp(\phi_i), i\in I\}$ est localement fini.
        \item Pour tout $p\in M$,$ \sum_{i\in I} \phi(p) = 1$ et les 
            $\phi_i$ sont positives.
    \end{itemize}
\end{defn}

\section{Variétés riemanniennes}
À rattraper, 
$g=\sum g_{ij}d_{x_i}\otimes d_{x_j}$ dans 
$(T_PM)_{0,2}=T_PM^*\otimes T_PM^*$.

\begin{defn}[Géodésique]
    Une courbe $C^1$ par morceau $\gamma\colon [a,b]\to (M,g)$ est
    une géodésique si $\forall [c,d]\subset [a,b]$ on a 
    \[d_g(\gamma(c), \gamma(d))=lg(\gamma|_{[c,d]}) \]
    On demande pour l'unicité plus tard que elle soit de vitesse 
    constante (la norme de la dérivée par la dérivée).
\end{defn}
\begin{quest}
    Étant donné deux points de $(M,g)$ peut-on trouver une géodésique 
    entre les deux.
\end{quest}

Non, par exemple dans le plan ou on enlève $(0,0)$ pas de géodésique
entre $P$ et $-P$.
\begin{thm}[Hopf-Ridow]
    Tout deux points de $(M,g)$ sont joints par une géodésique ($M$ est
    géodésiquement complète) ssi $(M, d_g)$ est complète.
\end{thm}
\begin{rem}
    Toujours vrai si $M$ est compacte. 
\end{rem}

Soit $(U,\phi)$ une carte de $M$ et $g$ une métrique Riemannienne sur 
$M$. Soit $D\subset U$ un domaine (ouvert connexe de cloture compacte).

\begin{defn}[Volume de $D$]
Le volume de $D$ est \[\mu_g(D)=\int_{\phi(D)} \sqrt{det(g_{ij})\circ\phi^{-1}} dx^1\ldots dx^{d}\]
    avec la mesure Lebesgue sur $\R^d$.
\end{defn}
La matrice $(g_{ij})$ est définie positive donc c'est bien défini. Reste
à voir pourquoi c'est indépendant de la carte et ca mesure bien le 
volume. 
\begin{proof}
    On fixe $u=(u_1,\ldots, u_d)$ une base orthogonale de $(T_PM,g_P)$.
    On écrit $\partial_i=\sum a_{ki}u_k$. On rappelle que $\partial_i=
    (d_P\phi)^{-1}e_i$ dans la base canonique $e=(e_1,\ldots,e_d)$. Alors
    $(a_{ij})=A=[(d_P\phi)^{-1}]_{e\to u}$. On veut m.q :
    \[\sqrt{det(g_{ij})\phi^{-1}}=det(A)!\]
    mais $g_{ij}=g_P(\partial_i,\partial_j)=g_P(\sum_ka_{ki}uk,\sum_l a_{lj}u_l)$
    puis $=\sum_{k,l} a_{ki}a_{lj}g_P(u_k,u_l)$ et 
    $g_P(u_k,u_l)=\delta_k^l$ d'où $=\sum a_{ki}a_{kh}=(tA.A)_{ij}$.
    Soit maintenant $\psi\colon U\to \R^d$ une autre carte avec $\psi=
    (y_1,\ldots,y_d)$. On a $h_{ij}=g(\frac{\partial}{\partial y_i},\frac{\partial}{\partial y_j})$ et 
    $\frac{\partial}{\partial y_i}=(d_P\psi)^{-1}e_i$. On a 
    \begin{align*}
        \mu_g(D)&=\int_{\phi(D)}|det[((d_{\phi(\ldots)}\phi)^{-1}]_{e\to u}\circ \phi^{-1} dx_1\ldots dx_d\\
                &=\int_{\phi(D)} |det[d_x(\phi^{-1})]_{e\to u}dx_1\ldots dx_d\\
    \end{align*}
    et on a $\phi^{-1}=\psi^{-1}\circ\psi\circ\phi^{-1}$ d'où
    \begin{align*}
        &=\int_{\phi(D)}|det[d_{\psi\circ\phi^{-1}}\psi^{-1}]_{e\to u}
                |.|det[d_x\psi\circ\phi^{-1}]_{e\to e}|dx_1\ldots dx_d\\
        &=\int_{\phi(D)}|det[d_y\psi^{-1}]|dy_1\ldots y_d\\
        (claim)&=\int_{\phi(D)}\sqrt{det(h_{ij})\circ\psi^{-1}}dy_1\ldots dy_d
    \end{align*}
    \begin{rem}
        $\mu_g$ est absolument continue par rapport à la mesure de 
        Lebesgue. (i.e. elle est nulle dès que la mesure de lebesgue est
        nulle) D'où $\mu_g(\bar D- D)=0$.
    \end{rem}
    On coupe alors $D$ en des $D_i$ contenus dans dans des cartes 
    de sorte à être des domaines. C'est là que la remarque est utile.
\end{proof}

On veut construire des dérivations de $\gamma$ ($\gamma\colon I\to (M,g)$
, $C^{\infty}$) de sorte que $\gamma$ est une géodésique ssi $(\gamma')'$
. Pour que ca fasse sens on demande que la norme de la vitesse de la
dérivée soit constante.
\chapter{Connexions de Levi-Cevita, dérivée covariantes}
\section{Connexion de Levi-Cevita}
\begin{defn}
    Une connexion sur une variété différentielle est une forme 
    $\R$-bilinéaire 
    \[D\colon \Gamma(M)\times \Gamma(M)\to \Gamma(M)\]
    qu'on note $D(X,Y):= D_XY$. Vérifiant pour tout $f\in C^{\infty}(M)$
    \begin{itemize}
        \item $D_{f(X)}Y=fD_XY$
        \item $D_X(fY)=X.fY+fD_XY$ où $X.f=L_X(f)$ la dérivée de Lie.
    \end{itemize}
    Une connexion est sans torsion si $D_XY-D_YX=[X,Y]$. 
\end{defn}
\begin{rem}
    On peut remarquer que $(D_XY)_x$ dépend seulement en $X_x$ et en 
    $Y$ sur un voisinage de $x$. Plus précisément, si $X,Z\in \Gamma(M)$
    tel que \[X_x=Z_x\] alors $(D_XY)_x=(D_ZY)x$. On le vérifie, 
    c'est équivalent à ce que $X_x=0\implies $ 
    Si $f\in C^{\infty}(M)$ telle que $f(x)=1$ alors 
    $(D_{fX}Y)_x=f(x)(D_XY)_x)=(D_XY)_x$. On considère maintenant une
    carte $(U,\phi)$ autour de $x$ et on suppose que $f=0$ sur $M-U$.
    Alors $fX=0$ sur $M-U$. On écrit $X=\sum_i X^i\partial_i$ sur $U$, 
    on a
    \[fX=\sum_ifX^i\partial_i\]
    avec $=0~ou~\partial$ sur $M-U$. On a 
    \begin{align*}
        (D_XY)_x&=(D_{fX}Y)_x\\
                &=(D_{\sum_i fX^i\partial_i}Y)_x\\
        (linearite)&=\sum_iX^i(x)(D_{f\partial_i}Y)_x\\
                   &=0
    \end{align*}
    On dit que $D$ est tensoriel par rapport à la premiere variable.
\end{rem}
\begin{ex}[Exemples de connexions]
    Si on prends $X,Y\in \Gamma(\R^n)$, un champ de vecteur sur 
    $\R^n$ correspond à une flèche $(X,Y)\colon\R^n\to\R^n$. Alors
    $D_XY=dY(X)$ définit une connexion sans torsion.
\end{ex}

\begin{defn}
    Une connexion $D$ sur une variété Riemannienne $(M,g)$ est 
    compatible avec la métrique $g$ si pour tout $X,Y,Z\in \Gamma(M)$
    on a 
    \[X.g(Y,Z)=g(D_XY,Z)+g(Y,D_XZ)\]
\end{defn}

\begin{heur}
    La connexion $D$ s'intérprète comme la dérivée $dY$ dans la 
    direction $X$. La donnée de $(D_XY)_x$ dépend de $X_x$. Sur $\R^n$
    on a choisi un champ de vecteur à chaque point. On peut choisir
    une courbe $c$ t.q. $c(0)=Y$ et $c'(0)=X_x$, autrement dit
    telle que $[c]=X_x$. On a 
    \begin{align*}
    \frac{d}{dt|_{t=0}}Y_{c(t)}&=d_{c(0)}Y(c'(0))\\
                               &=d_xY(X_x)\\
                               &=(D_XY)_x
    \end{align*}

    (Si on bouge le long de $X$ comment $Y$ change?)
\end{heur}
\begin{rem}[Exemple de compatibiité]
    $D_XY=dY(X)$ est compatible dans $\R^n$ avec la métrique euclidienne.
    En un point $x$, si $X_x=c'(0)$ alors
    \begin{align*}
        (X.g(Y,Z))^f_x&=(X.f)_x\\
                      &=(df(X))_x\\
                      &=d_xf(X_x)\\
                      &=\frac{d}{dt|_{t=0}}f(c(t))\\
                      &(t=0)\\
                      &(g_{c(t)} \textrm{constant en métrique euclidienne})\\
                      &=\frac{d}{dt}g_{c(t)}(Y_{c(t)},Z_{c(t)})\\
                      &=<\frac{d}{dt}Y_{c(t)},Z>+<Y,\frac{d}{dt}Z_{c(t)}>\\
                      &=<D_XY,Z>+<Y,D_XZ>
    \end{align*}
\end{rem}

\begin{ex}[Exemple de non compatibilité]
    Dans $H=\{(x,y)|y>0\}$ le demi-plan. Est-ce que $D_XY=dY(X)$ est 
    compatible avec $g_H=\frac{dx^2+dy^2}{y^2}$? Pour 
    $X=(0,1)=\partial_y$ et $Y=(1,0)=\partial_x$. On a 
    \[D_XY=dY(X)=0\]
    et 
    \begin{align*}
        (X.g_H(Y,Y))_{(x,y)}&=\partial_y1/y^2\\
                            &=-2/y^3\\
                            &=2g(D_XY,Y)=0
    \end{align*}
    D'où $D_XY=dY(X)$ n'est pas compatible avec $g_H$. On doit changer
    la "dérivée" par rapport à $Y$ pour contenir la variation de la 
    métrique par rapport à $y$.
\end{ex}

\begin{prop}
    Toute variété Riemannienne a une unique connexion sans torsion 
    compatible avec sa métrique. On l'appelle la connexion de 
    Levi-Cevita (une seule personne).
\end{prop}
\begin{proof}
    Soit $X,Y,Z\in \Gamma(M)$. Si $D$ est une connexion sans torsion
    compatible avec une métrique $g$. On a 
    \begin{align}
        X.g(Y,Z)&=g(D_XY,Z)+g(Y,D_XZ)\\
        Y.g(X,Z)&=g(D_YX,Z)+g(X,D_YZ)\\
        Z.g(X,Y)&=g(D_ZX,Y)+g(X,D_ZY)
    \end{align}
    En calculant $(2.1)+(2.2)-(2.3)$ on a par linéarité
    \begin{align*}
    X.g(Y,Z)+Y.g(X,Z)-Z.g(X,Y)&=g(D_XY+D_YX,Z)+g(Y,[X,Z])+g(X,[Y,Z])\\
                              &=2g(D_XY,Z)+\\
                              &g([Y,X],Z)+g(Y,[X,Z])+g(X,[Y,Z])
    \end{align*}
    Alors 
    \begin{align*}\label{eq:metriquecompatible}
        2g(D_XY,Z)&=X.g(Y,Z)+Y.g(X,Z)-Z.g(X,Y)+\\
                  &g([X,Y],Z)-g([X,Z],Y)-g(X,[Y,Z])
    \end{align*}
    Ca prouve l'unicité car $D$ apparaît pas à droite.
    Maintenant on montre que $g(D_XY, Z)$ est tensorielle en $Z$, i.e.
    $\forall f\in C^{\infty}(M)$ on a $g(D_XY,fZ)=fg(D_XY,Z)$ (c'est
    juste un calcul on le fait pas). Alors $g(D_XY,Z)_x$ ne dépend que de
    $Z_x$ et $X,Y$. Soit $(e_i$ une base orthonormale de $(T_xM, g_x)$,
    on a
    \begin{align*}
        (D_XY)_x&=\sum_ig_x((D_xY)_x,e_i)e_i\\
                &=\sum g(D_XY, Z_i)e_i\\
                &(\textrm{où }Z_{i,x}=e_i, Z_i\in \Gamma(M))\\
    \end{align*}
    alors le côté droit de \ref{eq:metriquecompatible} détermine
    $(D_XY)_x$. Un calcul montre que le côté droit de 
    \ref{eq:metriquecompatible} vérifie les axiomes.
\end{proof}
\begin{ex}
    $D_XY=dY(X)$ est la connexion de Levi-Cevita sur $\R^n$ pour la 
    métrique euclidienne.
\end{ex}
\textit{Dans une carte :} Sur $(M,g)$  et $D$ sa connexion de Levi-Cevita
, $(U,\phi)$ une carte, $\phi=(x_1,\ldots,x_d)$ des coordonnées, comme
$\partial_i,\partial_j\in \Gamma(U)$ on a $D_{\partial_i}\partial_j\in
\Gamma(U)$. Donc on peut écrire 
\[D_{\partial_i}\partial_j=\sum_{k=1}^d \Gamma_{ij}^k\partial_k\]
où $\Gamma_{ij}^k\in C^{\infty}(U)$ sont appelés les symboles de 
Christoff, ils déterminent $D$. 

\begin{ex}
    Tout le long on utilise que $\partial_x$ et $\partial_y$ sont
    perpendiculaire (pour la métrique hyperbolique, regarder la matrice).
    On calcule la connexion de Levi-Cevita de la métrique hyperbolique.
    On a \[g_H\colon (g_{H,(i,j)}=\begin{pmatrix}1/y^2&0\\0&1/y^2\end{pmatrix}\].
    Avec \[\partial_x.g(\partial_x,\partial_x)=0=2g_x(D_{\partial_x}\partial_x,\partial_x\]) 
    On a la perpendicularité d'où $\Gamma_{11}^1=0$ et $D_{\partial_x}\partial_x=\Gamma_{11}^2\partial_y$.
    Ensuite \[\partial_x.g(\partial_x,\partial_y)=0=g(D_{\partial_x}\partial_x,
    \partial_y)+g(\partial_x,D_{\partial_x}\partial_y)\]
    et \[\partial_y.g(\partial_x,\partial_x)=-2/y^3=2g(D_{\partial_y}\partial_x,\partial_x)=2g(D_{\partial_x}\partial_y,\partial_x)\]
    sachant que $D_{\partial_x}\partial_y-D_{\partial_y}\partial_x=
    [\partial_x,\partial_y]=0$.
    Bon c'est le bordel.
\end{ex}


\end{document}


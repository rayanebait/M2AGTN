\documentclass[a4paper,12pt]{book}
\usepackage{amsmath,  amsthm,enumerate}
\usepackage{csquotes}
\usepackage[provide=*,french]{babel}
\usepackage[dvipsnames]{xcolor}
\usepackage{quiver, tikz}

%symbole caligraphique
\usepackage{mathrsfs}

%hyperliens
\usepackage{hyperref}

%pseudo-code
\usepackage{algorithm}
\usepackage{algpseudocode}

\usepackage{fancyhdr}

\pagestyle{fancy}
\addtolength{\headwidth}{\marginparsep}
\addtolength{\headwidth}{\marginparwidth}
\renewcommand{\chaptermark}[1]{\markboth{#1}{}}
\renewcommand{\sectionmark}[1]{\markright{\thesection\ #1}}
\fancyhf{}
\fancyfoot[C]{\thepage}
\fancyhead[LO]{\textit \leftmark}
\fancyhead[RE]{\textit \rightmark}
\renewcommand{\headrulewidth}{0pt} % and the line
\fancypagestyle{plain}{%
    \fancyhead{} % get rid of headers
}

%bibliographie
\usepackage[
backend=biber,
style=alphabetic,
sorting=ynt
]{biblatex}

\addbibresource{bib.bib}

\usepackage{appendix}
\renewcommand{\appendixpagename}{Annexe}

\definecolor{wgrey}{RGB}{148, 38, 55}

\setlength\parindent{24pt}

\newcommand{\Z}{\mathbb{Z}}
\newcommand{\R}{\mathbb{R}}
\newcommand{\rel}{\omathcal{R}}
\newcommand{\Q}{\mathbb{Q}}
\newcommand{\C}{\mathbb{C}}
\newcommand{\N}{\mathbb{N}}
\newcommand{\K}{\mathbb{K}}
\newcommand{\A}{\mathbb{A}}
\newcommand{\B}{\mathcal{B}}
\newcommand{\Or}{\mathcal{O}}
\newcommand{\F}{\mathscr F}
\newcommand{\Hom}{\textrm{Hom}}
\newcommand{\disc}{\textrm{disc}}
\newcommand{\Pic}{\textrm{Pic}}
\newcommand{\End}{\textrm{End}}
\newcommand{\Spec}{\textrm{Spec}}
\newcommand{\Supp}{\textrm{Supp}}
\renewcommand{\Im}{\textrm{Im}}


\newcommand{\m}{\mathfrak{m}}
\newcommand{\p}{\mathfrak{p}}


\newcommand{\cL}{\mathscr{L}}
\newcommand{\G}{\mathscr{G}}
\newcommand{\D}{\mathscr{D}}
\newcommand{\E}{\mathscr{E}}
\renewcommand{\Pr}{\mathbb{P}}
\renewcommand{\P}{\mathscr{P}}
\renewcommand{\H}{\mathscr{H}}

\makeatletter
\newcommand{\colim@}[2]{%
  \vtop{\m@th\ialign{##\cr
    \hfil$#1\operator@font colim$\hfil\cr
    \noalign{\nointerlineskip\kern1.5\ex@}#2\cr
    \noalign{\nointerlineskip\kern-\ex@}\cr}}%
}
\newcommand{\colim}{%
  \mathop{\mathpalette\colim@{\rightarrowfill@\scriptscriptstyle}}\nmlimits@
}
\renewcommand{\varprojlim}{%
  \mathop{\mathpalette\varlim@{\leftarrowfill@\scriptscriptstyle}}\nmlimits@
}
\renewcommand{\varinjlim}{%
  \mathop{\mathpalette\varlim@{\rightarrowfill@\scriptscriptstyle}}\nmlimits@
}
\makeatother

\theoremstyle{plain}
\newtheorem{thm}[subsection]{Théoreme}
\newtheorem{lem}[subsection]{Lemme}
\newtheorem{prop}[subsection]{Proposition}
\newtheorem{cor}[subsection]{Corollaire}
\newtheorem{heur}{Heuristique}
\newtheorem{rem}{Remarque}
\newtheorem{note}{Note}

\theoremstyle{definition}
\newtheorem{conj}{Conjecture}
\newtheorem{prob}{Problème}
\newtheorem{quest}{Question}
\newtheorem{prot}{Protocole}
\newtheorem{algo}{Algorithme}
\newtheorem{defn}[subsection]{Définition}
\newtheorem{exmp}[subsection]{Exemples}
\newtheorem{exo}[subsection]{Exercices}
\newtheorem{ex}[subsection]{Exemple}
\newtheorem{exs}[subsection]{Exemples}

\theoremstyle{remark}

\definecolor{wgrey}{RGB}{148, 38, 55}
\definecolor{wgreen}{RGB}{100, 200,0} 
\hypersetup{
    colorlinks=true,
    linkcolor=wgreen,
    urlcolor=wgrey,
    filecolor=wgrey
}

\title{Recollements}
\date{}

\begin{document}
\maketitle
\tableofcontents

\chapter{Recollements}
\section{Union disjointe}
En gros ça va être le recollement selon
des ouverts vides :
\[X=\sqcup X_i\]

\subsection{Faisceau}
Le faisceau est clair via 
\[0\to \Or_X\to \Or_X|_{X_1}\times\ldots\times \Or_X|_{X_k}\to \prod \Or_X(X_i\cap X_j)\to 0\]
mais le dernier terme c'est le faisceau nul. 
\subsection{Topologie}
Ducoup là la topologie est assez claire, c'est des unions d'ouverts
des $X_i$. En particulier les $X_i$ sont ouverts.


\section{Cas général}
\begin{enumerate}
  \item Des variétés affines $(X_i)_{i}$,
  \item Des ouverts $U_ij\subset X_i$,
  \item Des isomorphismes 
    \[\varphi_{ij}\colon U_ij\simeq U_{ji}\]
    tels que $\varphi_{ij}=\varphi_{kj}\circ \varphi_{ik}$
    en se restreignant aux bons ouverts, i.e. 
% https://q.uiver.app/#q=WzAsNixbMCwwLCJYX2kiXSxbMiwzLCJYX2siXSxbNCwwLCJYX2oiXSxbMSwxLCJVX3tpan1cXGNhcCBVX3tpa30iXSxbMywxLCJVX3tqaX1cXGNhcCBVX3tqa30iXSxbMiwyLCJVX3traX1cXGNhcCBVX3tran0iXSxbMyw0XSxbMyw1XSxbNSw0XV0=
\[\begin{tikzcd}
	{X_i} &&&& {X_j} \\
	& {U_{ij}\cap U_{ik}} && {U_{ji}\cap U_{jk}} \\
	&& {U_{ki}\cap U_{kj}} \\
	&& {X_k}
	\arrow[from=2-2, to=2-4]
	\arrow[from=2-2, to=3-3]
	\arrow[from=3-3, to=2-4]
\end{tikzcd}\]
\end{enumerate}
Ça fait une relation d'équivalence $x_i\sim x_j$
si $\varphi_{ij}(x_i)=x_j$. En particulier, on peut
former 
\[\pi \sqcup_i X_i\to(\sqcup_i X_i)/\sim\]
maintenant faut décrire le faisceau et la topologie!
\section{Topologie}
Donc naturellement on met la topologie quotient. Et
on regarde le morphisme 
\[\pi \sqcup_i X_i\to(\sqcup_i X_i)/\sim.\]
Donc un sous-ensemble $E$ du quotient est ouvert
ssi $p^{-1}E$ est ouvert. Comme on part d'une union
disjointe, ça va être une unione disjointe d'ouverts
et on peut regarder essentiellement (réflechir
un peu plus à comment les décrire)
\[p^{-1}p(U)\]
pour $U\subset X_i$.
\begin{prop}
  Pour $U\subset X_i$ on a
  \[p^{-1}p(U)=U\cup \sqcup_j \varphi_{ij}U\cap U_{ij}.\]
\end{prop}
Et ces ouverts sont une base de la topologie. (Faut juste
intersecter avec les $\bar(X_i)$ qui sont ouverts)
\section{Faisceau}
À nouveau on va étudier 
\[\Or_X\to \pi_*\Or_{\sqcup_i X_i}\]
sur les intersections.


\section{Conditions de recollements explicits}
\subsection{Cas de deux ouverts $(X_1\sqcup X_2)/\sim_\rho=X$}
Essentiellement, la surjectivité de $X_1\sqcup X_2\to X$
induit une injection 
\[\Or_X\hookrightarrow \Or_{X_1\sqcup X_2}|_{X_1}\times \Or_{X_1\sqcup X_2}|_{X_2}\]
\begin{rem}
  En fait plus naturellement, des "fonctions" sur $U\subset X$
  c'est vraiment des fonctions sur $X_1\sqcup X_2$ qui 
  passent au quotient.
\end{rem}
Ducoup le morphisme on le décrit via 
\[\Or_{X}(\pi(X_i))\ni g_i\mapsto (g_i\circ \pi|_{X_i},g_i\circ \pi\circ \rho_{ji})\]
sur $\pi(X_i)$ parce que 
\[\pi^{-1}(\pi(X_i))=X_i\sqcup \rho_{ij}(X_i\cap U_{ij}=U_{ij}).\]
Et pour $U\subset X_i$ on écrit 
\[\Or_{X}(\pi(U))\ni g_i\mapsto (g_i\circ \pi|_{U},g_i\circ \pi\circ \rho_{ji}|_{\rho_{ij}(U\cap U_{ij})})\]
Bon maintenant les conditions de recollement peuvent s'écrire par
restriction:
\[(g_1\circ\pi|_{X_1})|_{U_{12}}=g_2\circ\pi\circ \rho_{12}\]
et 
\[g_1\circ\pi\circ \rho_{21}=(g_2\circ\pi|_{X_2})|_{U_{21}}\]
mais en fait avoir l'un ou l'autre c'est équivalent vu que 
$\rho_{21}\circ\rho_{12}=id_{U_{21}}$.
\subsection{Cas général : $X=(\sqcup X_i)/\sim$}
Cette fois faut noter à nouveau que 
\[\pi^{-1}\pi X_i=X_i\sqcup\bigsqcup_{j\ne i} U_{ji}\]
et donc si on a $g_i\in \pi(X_i)$ qu'on voir si on peut les 
relever on écrit via $\Or_X\hookrightarrow \pi_*\Or_{\sqcup X_i}$:
\[g_i\mapsto (g_i\circ \pi|_{X_i}, (g_i\circ\pi\circ\rho_{ji})_j)\]
alors pour $g_{j_1}$ et $g_{j_2}$ on a la condition sur 
$\pi(U_{j_1j_2})$ :
\[g_{j_1}\circ\pi|_{U_{j_1j_2}}=g_{j_2}\circ\pi\circ\rho_{j_1j_2}\]
et les autres conditions sur $\rho_{j_1i}(U_{j_1j_2}\cap U_{j_1i})=
\rho_{j_2i}(U_{j_2j_1}\cap U_{j_2i})=V$ (Je crois que c'est égal
mais ça change rien) données par 
\[g_1\circ\pi\circ\rho_{ij_1}|_V=g_2\circ\pi\circ \rho_{ij_2}|_V\]
mais $V\subset U_{ij_1}\cap U_{ij_2}$ et on a les deux identités
\[\rho_{ij_1}\circ\rho_{j_1i}=id\]
\[\rho_{ij_2}\circ\rho_{j_1i}=\rho_{j_1j_2}\]
en particulier suffit de vérifier que la première condition.

\section{Sections de $\Pr_k^n$}
\begin{quest}
  Étant donné une section sur $\A^n(k)$ non constante est-ce 
qu'on peut relever à $\Pr^n(k)$?
\end{quest}
Donc là $X_i=\A^n(k)$ pour tout $i=0,\ldots, n$ le changement de
carte 
$\rho_{ij}\colon U_{ij}\to U_{ji}$ est donnée par 
\[\rho_{ij}(t_0,\ldots,\hat t_i,\ldots, t_n)\mapsto (t_0/t_j,\ldots,\hat t_j, \ldots, t_{i-1}/t_j,1/t_j,\ldots, t_n/t_j)\]
d'où la flèche
\[(\rho_{ij})_*\colon k[T_0,\ldots, \hat T_j, \ldots, T_n]_{T_i}=k[U_{ji}]\to k[U_{ij}] k[T_0,\ldots,\hat T_i,\ldots, T_n]_{T_j}\]
est donnée \[(\rho_{ij})_*(T_k)=\begin{cases} 1/T_j\textrm{ si $k=i$}\\ T_k/T_j\textrm{ sinon.}\end{cases}\]
En particulier dès qu'on a un polynôme non trivial $P$ sur $X_i$ et
que $T_j$ apparaît disons bah via $(\rho_{ji})_*$ on obtient 
une section \[(\rho_{ji})_*(P)\in k[U_{ji}]\backslash k[X_j]\] donc
se relève pas (parce que $1/T_i$ apparait et est pas déf 
globalement).
\section{Cas de $\Pr_k^1$}
J'entends souvent Aphelli parler de $k[X,Y]/(XY-1)$ comme de 
l'intersection de $U_0$ et $U_1$. Faudrait explorer.




\chapter{Propriété explicite de faisceau}
Quand on a une variété $X$ on peut y penser via un recollement
d'affines. C'est à dire via (déjà deux ouverts) :
\[0\to \Or_X(U\cup V)\to \Or_X(U)\times
\Or_X(V)\to \Or_X(U\cap V)\to 0\]
donné par 




\printbibliography
\end{document}


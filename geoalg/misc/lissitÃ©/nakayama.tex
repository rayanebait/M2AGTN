\documentclass[a4paper,12pt]{book}
\usepackage{amsmath,  amsthm,enumerate}
\usepackage{csquotes}
\usepackage[provide=*,french]{babel}
\usepackage[dvipsnames]{xcolor}
\usepackage{quiver, tikz}

%symbole caligraphique
\usepackage{mathrsfs}

%hyperliens
\usepackage{hyperref}

%pseudo-code
\usepackage{algorithm}
\usepackage{algpseudocode}

\usepackage{fancyhdr}

\pagestyle{fancy}
\addtolength{\headwidth}{\marginparsep}
\addtolength{\headwidth}{\marginparwidth}
\renewcommand{\chaptermark}[1]{\markboth{#1}{}}
\renewcommand{\sectionmark}[1]{\markright{\thesection\ #1}}
\fancyhf{}
\fancyfoot[C]{\thepage}
\fancyhead[LO]{\textit \leftmark}
\fancyhead[RE]{\textit \rightmark}
\renewcommand{\headrulewidth}{0pt} % and the line
\fancypagestyle{plain}{%
    \fancyhead{} % get rid of headers
}

%bibliographie
\usepackage[
backend=biber,
style=alphabetic,
sorting=ynt
]{biblatex}

\addbibresource{bib.bib}

\usepackage{appendix}
\renewcommand{\appendixpagename}{Annexe}

\definecolor{wgrey}{RGB}{148, 38, 55}

\setlength\parindent{24pt}

\newcommand{\Z}{\mathbb{Z}}
\newcommand{\R}{\mathbb{R}}
\newcommand{\rel}{\omathcal{R}}
\newcommand{\Q}{\mathbb{Q}}
\newcommand{\C}{\mathbb{C}}
\newcommand{\N}{\mathbb{N}}
\newcommand{\K}{\mathbb{K}}
\newcommand{\A}{\mathbb{A}}
\newcommand{\B}{\mathcal{B}}
\newcommand{\Or}{\mathcal{O}}
\newcommand{\F}{\mathscr F}
\newcommand{\Hom}{\textrm{Hom}}
\newcommand{\disc}{\textrm{disc}}
\newcommand{\Pic}{\textrm{Pic}}
\newcommand{\End}{\textrm{End}}
\newcommand{\Spec}{\textrm{Spec}}
\newcommand{\Supp}{\textrm{Supp}}
\renewcommand{\Im}{\textrm{Im}}


\newcommand{\m}{\mathfrak{m}}
\newcommand{\n}{\mathfrak{n}}
\newcommand{\p}{\mathfrak{p}}


\newcommand{\cL}{\mathscr{L}}
\newcommand{\G}{\mathscr{G}}
\newcommand{\D}{\mathscr{D}}
\newcommand{\E}{\mathscr{E}}
\renewcommand{\Pr}{\mathbb{P}}
\renewcommand{\P}{\mathscr{P}}
\renewcommand{\H}{\mathscr{H}}

\makeatletter
\newcommand{\colim@}[2]{%
  \vtop{\m@th\ialign{##\cr
    \hfil$#1\operator@font colim$\hfil\cr
    \noalign{\nointerlineskip\kern1.5\ex@}#2\cr
    \noalign{\nointerlineskip\kern-\ex@}\cr}}%
}
\newcommand{\colim}{%
  \mathop{\mathpalette\colim@{\rightarrowfill@\scriptscriptstyle}}\nmlimits@
}
\renewcommand{\varprojlim}{%
  \mathop{\mathpalette\varlim@{\leftarrowfill@\scriptscriptstyle}}\nmlimits@
}
\renewcommand{\varinjlim}{%
  \mathop{\mathpalette\varlim@{\rightarrowfill@\scriptscriptstyle}}\nmlimits@
}
\makeatother

\theoremstyle{plain}
\newtheorem{thm}[subsection]{Théoreme}
\newtheorem{lem}[subsection]{Lemme}
\newtheorem{prop}[subsection]{Proposition}
\newtheorem{cor}[subsection]{Corollaire}
\newtheorem{heur}{Heuristique}
\newtheorem{rem}{Remarque}
\newtheorem{note}{Note}

\theoremstyle{definition}
\newtheorem{conj}{Conjecture}
\newtheorem{prob}{Problème}
\newtheorem{quest}{Question}
\newtheorem{prot}{Protocole}
\newtheorem{algo}{Algorithme}
\newtheorem{defn}[subsection]{Définition}
\newtheorem{exmp}[subsection]{Exemples}
\newtheorem{exo}[subsection]{Exercices}
\newtheorem{ex}[subsection]{Exemple}
\newtheorem{rep}{Réponse}
\newtheorem{exs}[subsection]{Exemples}

\theoremstyle{remark}

\definecolor{wgrey}{RGB}{148, 38, 55}
\definecolor{wgreen}{RGB}{100, 200,0} 
\hypersetup{
    colorlinks=true,
    linkcolor=wgreen,
    urlcolor=wgrey,
    filecolor=wgrey
}

\title{Lissité sur les schémas}
\date{}

\begin{document}
\maketitle
\tableofcontents
\[\ldots\]   

\chapter{Outils}
ATTENTION : Il faut tout faire avec $I$ radical.
\section{La définition de base}
Étant donné un affine $Z(I)=Z(F_1,\ldots, F_m)\subset\A^n$ 
on peut définir le plan tangent via
\[(D_PI)^\perp =\{t| D_P(F)(t)=0 \forall F\in I\}\]
\section{Plan tangent de Zariski}
\begin{quest}
    $T_{X,p}\simeq (\m/\m^2)$
\end{quest}
Pour rappel on a un diagramme 
% https://q.uiver.app/#q=WzAsMTAsWzAsMCwiMCJdLFsxLDAsIkkvSVxcY2FwIFxcbWF0aGZyYWsgbl4yIl0sWzIsMCwiXFxtYXRoZnJhayBuL1xcbWF0aGZyYWsgbl4yIl0sWzMsMCwiXFxtYXRoZnJhayBtL1xcbWF0aGZyYWsgbV4yIl0sWzQsMCwiMCJdLFszLDEsIlxcbWF0aGZyYWsgbS9cXG1hdGhmcmFrIG1eMiJdLFs0LDEsIjAiXSxbMiwxLCJFXlxcd2VkZ2UiXSxbMSwxLCJEX1BJIl0sWzAsMSwiMCJdLFswLDFdLFsxLDJdLFsyLDNdLFszLDRdLFszLDUsIiIsMCx7InN0eWxlIjp7ImhlYWQiOnsibmFtZSI6Im5vbmUifX19XSxbNSw2XSxbMiw3LCJEX1AiXSxbMSw4LCJEX1AiXSxbOCw3XSxbOSw4XSxbNyw1XV0=
\[\begin{tikzcd}
	0 & {I/I\cap \mathfrak n^2} & {\mathfrak n/\mathfrak n^2} & {\mathfrak m/\mathfrak m^2} & 0 \\
	0 & {D_PI} & {E^\vee} & {\mathfrak m/\mathfrak m^2} & 0
	\arrow[from=1-1, to=1-2]
	\arrow[from=1-2, to=1-3]
	\arrow["{D_P}", from=1-2, to=2-2]
	\arrow[from=1-3, to=1-4]
	\arrow["{D_P}", from=1-3, to=2-3]
	\arrow[from=1-4, to=1-5]
	\arrow[no head, from=1-4, to=2-4]
	\arrow[from=2-1, to=2-2]
	\arrow[from=2-2, to=2-3]
	\arrow[from=2-3, to=2-4]
	\arrow[from=2-4, to=2-5]
\end{tikzcd}\]
qui devient en passant au dual
% https://q.uiver.app/#q=WzAsNSxbMywwLCIoXFxtYXRoZnJhayBtL1xcbWF0aGZyYWsgbV4yKV5cXHdlZGdlIl0sWzQsMCwiMCJdLFsyLDAsIihFXlxcd2VkZ2UpXlxcd2VkZ2UiXSxbMSwwLCIoRF9QSSleXFx3ZWRnZSJdLFswLDAsIjAiXSxbMSwwXSxbMiwzXSxbMyw0XSxbMCwyXV0=
\[\begin{tikzcd}
	0 & {(D_PI)^\vee} & {(E^\vee)^\vee} & {(\mathfrak m/\mathfrak m^2)^\vee} & 0
	\arrow[from=1-2, to=1-1]
	\arrow[from=1-3, to=1-2]
	\arrow[from=1-4, to=1-3]
	\arrow[from=1-5, to=1-4]
\end{tikzcd}\]
mais en regardant $(E^\vee)^\vee$ comme l'ensemble des
morphismes d'évaluations $ev_Q\colon f\mapsto f(Q)$, le noyau 
a droite c'est les $ev_Q=g\in(E^\vee)^\vee$ tels que 
\[g|_{D_PI}=0\] autrement dit tels que $D_P(F)(Q)=0$ pour tout
$F\in I$.

\begin{rep}
    Pour conclure le noyau à droite bah c'est exactement 
    $T_{X,P}$ par l'identification.
\end{rep}

\section{Cadre de base avec la nouvelle définition}
\begin{defn}
    On définit $\dim_P X:=\inf\{\dim U|P\in U\subset X\}$. 
    En particulier si $X=\cup_i Z_i$,
    \[\dim_P X=\sup_{P\in Z_i} \dim Z_i\]
    vu que un ouvert qui croise $Z_i$ est dense dedans.
\end{defn}
Maintenant 
\begin{defn}
    On définit la lissité de $X$ en $P$ via 
    $\dim_P X=\dim_P T_{P,X}$.
\end{defn}
\begin{note}
Cette définition est une conséquence de la dernière déf psq 
on peut dire que 
\[\dim T_{X,P}\geq \dim T_{Z_X(f),P}-1\]
en prenant $f\in \m-\m^2$ (d'où une récurrence).
\end{note}
\section{Critère jacobien}
Si on identifie maintenant $E$ a $E^\vee$ par la base 
canonique, i.e. 
\[D_P(F_j)\sim \begin{pmatrix} \partial_1F_j\\\vdots\\\partial_n F_j\end{pmatrix}\]
alors 
\[J(X)_P=\begin{pmatrix} D_P(F_1)\vert&\ldots&\vert D_P(F_m)\end{pmatrix}.\]
D'où 
\[rk(J(X)_P)=\dim_k(D_PI)\]
et en passant à l'orthogonal
\[\dim_k(T_{X,P})=n-rk(J(X)_P).\]
\section{Ouvert des points non singuliers}
\subsection{Cas d'une hypersurface affine séparable}
Dans le cas d'une hypersurface $Z(H)$ où $H$ est séparable pour
l'une des variables :
\begin{enumerate}
    \item On peut montrer que $Z(H)$ a un point lisse. On note
        $H(T_1,\ldots, T_n)(S)$ séparable en $S$.
    \item $\Delta\in k[T_1,\ldots,T_n]$ le discriminant en $S$.
    \item $Z(H)\subset \A^n\times A^1$ et si $(q,s)\in D(\Delta)
        \times p_S(Z(H))$ alors 
        \[\frac{\partial H}{\partial S}(q,s)=(H(q)')(s)\ne 0\]
        d'où $H$ est lisse en $(q,s)$. $(D(\Delta)\cap p_T(Z(H))$
        est non vide.
\end{enumerate}
\subsection{Cas général}
On remarque que sur chaque affine :
\begin{enumerate}
    \item Être lisse est une condition ouverte via le jacobien.
        En particulier l'ensemble des points lisses est ouvert,
        faut montrer qu'il est non vide et dense partout.
\end{enumerate}
Ensuite, toute variété intègre de $\dim = r$ est birationnelle a
une hypersurface
de $\Pr^{r+1}$. I.e. \[k(X)\simeq k(T_1,\ldots, T_r, z)\] avec $z$
séparable sur $k(T_1,\ldots, T_n)$. Comme on doit juste montrer
que tout les ouverts contiennent un point lisse c'est fini 
par la section d'avant.

\subsection{Autre approche : via le critère jacobien}
On se ramène au cas affine irréductible et on montre que l'ensemble
des points ou le plan tangent est de dim $\geq k$ est fermé via le
jacobien. Et contient un ouvert, via la birationalité avec 
l'hypersurface!

\chapter{Utilisations et cadre}
\section{En résumé}
De $T_{X,P}\simeq (\m/\m^2)^\vee$ de manière fonctorielle on peut
faire de l'algèbre pour obtenir des injections/surjections de 
$T_{X,P}$ dans d'autres espaces tangents. Je pense qu'on a un
foncteur donc 
\[k\textrm{-Var}_*\to \textrm{Mod}_k\]
où à gauche c'est les variétés pointées.

\subsection{Si $\dim X=n$ alors $\dim T_{X,P}\leq n$}
on peut se ramener au cas affine. Alors $X\hookrightarrow \A^n$
donne $\n_P/\n_P^2\to \m_P/\m_P^2\to 0$ est exacte avec
$\m_P\subset A(X)$ et $\n_P\subset A(\A^m)$ ($m\leq n$). Ça donne
\[0\to T_{X,P}\hookrightarrow T_{\A^m,P}\]
avec celui de droite de dimension $m\leq n$.


\subsection{Singularités aux intersections des composantes}
Si on a $X=\cup_i Z_i$ alors un point $P$ est non singulier 
seulement si une seule composante passe par lui. Pour le prouver,
plusieurs approches :
\begin{enumerate}
    \item Via le critère jacobien : On peut supposer $X$ affine
        $dim_p X= dim X$ et garder que les composantes qui passent
        par $P$.
    \item Alors $I(X)=\cap I(Z_i)$. Idée : si on montre que 
        $X$ pas irred implique son idéal est engendré par des 
        produits. Alors on a fini car y'a des colonnes nulles en
        plus sur l'intersection. Problème : c'est faux.

    \item Directement : on se met à nouveau dans le cas affine
        $Z(I)=Z(\cap_i \p_i)$ :
        \[\m_{\cap_i\p_i}/\m_{\cap_i\p_i}^2\to \m_{\p_i}/\m_{\p_i}^2\to 0\]
        qui est exacte et un noyau non trivial : suffit de prendre
        $Q\in \p_i - \cup_{j\ne i} \p_j$ irréductible, alors 
        $Q\notin \m_{\cap_i\p_i}^2$ et $Q\in\m_{\cap \p_i}$. Donc
        dans le noyau. Enfin en passant au dual :
        \[0\to T_{Z(\p_i),P}\to T_{X,P}\]
        est une injection stricte ! 
\end{enumerate}
\begin{rem}
    Pas besoin de considérer $T_{P,\cap Z(\p_i)}$.
\end{rem}



\printbibliography
\end{document}


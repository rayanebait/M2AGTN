\documentclass[a4paper,12pt]{book}
\usepackage{amsmath,  amsthm,enumerate}
\usepackage{csquotes}
\usepackage[provide=*,french]{babel}
\usepackage[dvipsnames]{xcolor}
\usepackage{quiver, tikz}

%symbole caligraphique
\usepackage{mathrsfs}

%hyperliens
\usepackage{hyperref}

%pseudo-code
\usepackage{algorithm}
\usepackage{algpseudocode}

\usepackage{fancyhdr}

\pagestyle{fancy}
\addtolength{\headwidth}{\marginparsep}
\addtolength{\headwidth}{\marginparwidth}
\renewcommand{\chaptermark}[1]{\markboth{#1}{}}
\renewcommand{\sectionmark}[1]{\markright{\thesection\ #1}}
\fancyhf{}
\fancyfoot[C]{\thepage}
\fancyhead[LO]{\textit \leftmark}
\fancyhead[RE]{\textit \rightmark}
\renewcommand{\headrulewidth}{0pt} % and the line
\fancypagestyle{plain}{%
    \fancyhead{} % get rid of headers
}

%bibliographie
\usepackage[
backend=biber,
style=alphabetic,
sorting=ynt
]{biblatex}

\addbibresource{bib.bib}

\usepackage{appendix}
\renewcommand{\appendixpagename}{Annexe}

\definecolor{wgrey}{RGB}{148, 38, 55}

\setlength\parindent{24pt}

\newcommand{\Z}{\mathbb{Z}}
\newcommand{\R}{\mathbb{R}}
\newcommand{\rel}{\omathcal{R}}
\newcommand{\Q}{\mathbb{Q}}
\newcommand{\C}{\mathbb{C}}
\newcommand{\N}{\mathbb{N}}
\newcommand{\K}{\mathbb{K}}
\newcommand{\A}{\mathbb{A}}
\newcommand{\B}{\mathcal{B}}
\newcommand{\Or}{\mathcal{O}}
\newcommand{\F}{\mathscr F}
\newcommand{\Hom}{\textrm{Hom}}
\newcommand{\disc}{\textrm{disc}}
\newcommand{\Pic}{\textrm{Pic}}
\newcommand{\End}{\textrm{End}}
\newcommand{\Spec}{\textrm{Spec}}
\newcommand{\Spm}{\textrm{Spm}}
\newcommand{\Supp}{\textrm{Supp}}
\renewcommand{\Im}{\textrm{Im}}


\newcommand{\m}{\mathfrak{m}}
\newcommand{\p}{\mathfrak{p}}
\newcommand{\q}{\mathfrak{p}}


\newcommand{\cL}{\mathscr{L}}
\newcommand{\G}{\mathscr{G}}
\newcommand{\D}{\mathscr{D}}
\newcommand{\E}{\mathscr{E}}
\renewcommand{\Pr}{\mathbb{P}}
\renewcommand{\P}{\mathscr{P}}
\renewcommand{\H}{\mathscr{H}}

\makeatletter
\newcommand{\colim@}[2]{%
  \vtop{\m@th\ialign{##\cr
    \hfil$#1\operator@font colim$\hfil\cr
    \noalign{\nointerlineskip\kern1.5\ex@}#2\cr
    \noalign{\nointerlineskip\kern-\ex@}\cr}}%
}
\newcommand{\colim}{%
  \mathop{\mathpalette\colim@{\rightarrowfill@\scriptscriptstyle}}\nmlimits@
}
\renewcommand{\varprojlim}{%
  \mathop{\mathpalette\varlim@{\leftarrowfill@\scriptscriptstyle}}\nmlimits@
}
\renewcommand{\varinjlim}{%
  \mathop{\mathpalette\varlim@{\rightarrowfill@\scriptscriptstyle}}\nmlimits@
}
\makeatother

\theoremstyle{plain}
\newtheorem{thm}[subsection]{Théoreme}
\newtheorem{lem}[subsection]{Lemme}
\newtheorem{prop}[subsection]{Proposition}
\newtheorem{cor}[subsection]{Corollaire}
\newtheorem{heur}{Heuristique}
\newtheorem{rem}{Remarque}
\newtheorem{note}{Note}
\newtheorem{strat}{Stratégie}

\theoremstyle{definition}
\newtheorem{conj}{Conjecture}
\newtheorem{prob}{Problème}
\newtheorem{quest}{Question}
\newtheorem{prot}{Protocole}
\newtheorem{algo}{Algorithme}
\newtheorem{defn}[subsection]{Définition}
\newtheorem{exmp}[subsection]{Exemples}
\newtheorem{exo}[subsection]{Exercices}
\newtheorem{ex}[subsection]{Exemple}
\newtheorem{exs}[subsection]{Exemples}

\theoremstyle{remark}

\definecolor{wgrey}{RGB}{148, 38, 55}
\definecolor{wgreen}{RGB}{100, 200,0} 
\hypersetup{
    colorlinks=true,
    linkcolor=wgreen,
    urlcolor=wgrey,
    filecolor=wgrey
}

\title{Notes perso : Géométrie algébrique}
\date{}

\begin{document}
\maketitle
\tableofcontents


\chapter{Produits fibrés et foncteur de points}
\section{Remarques sur le lemme de Yoneda}
Le légendaire. Je l'énonce que pour ce qui m'intéresse.
En gros on a un foncteur $h_{\_}:=(X\mapsto \Hom_C(\_,X)$
qui est pleinement fidèle, autrement dit les diagrammes :
% https://q.uiver.app/#q=WzAsNixbMSwwLCJYIl0sWzEsMSwiWSJdLFsyLDAsImhfWDo9XFxIb21fQyhcXF8sWCkiXSxbMiwxLCJoX1k6PVxcSG9tX0MoXFxfLFkpIl0sWzAsMCwiXFxIb21fe0N9KFxcXyxcXF8pIl0sWzAsMSwiXFxIb21fe0N9KFxcXyxcXF8pIl0sWzAsMSwiZiIsMl0sWzIsM10sWzAsMl0sWzEsM10sWzQsMCwiXFxjb2xvbiIsMSx7InN0eWxlIjp7ImJvZHkiOnsibmFtZSI6Im5vbmUifSwiaGVhZCI6eyJuYW1lIjoibm9uZSJ9fX1dLFs1LDEsIlxcY29sb24iLDEseyJzdHlsZSI6eyJib2R5Ijp7Im5hbWUiOiJub25lIn0sImhlYWQiOnsibmFtZSI6Im5vbmUifX19XV0=
\[\begin{tikzcd}
	{\Hom_{C}(\_,\_)} & X & {h_X:=\Hom_C(\_,X)} \\
	{\Hom_{C}(\_,\_)} & Y & {h_Y:=\Hom_C(\_,Y)}
	\arrow["\colon"{description}, draw=none, from=1-1, to=1-2]
	\arrow[from=1-2, to=1-3]
	\arrow["f"', from=1-2, to=2-2]
	\arrow[from=1-3, to=2-3]
	\arrow["\colon"{description}, draw=none, from=2-1, to=2-2]
	\arrow[from=2-2, to=2-3]
\end{tikzcd}\]
commutent et 
\[\Hom_C(X,Y)\to \Hom_{\widehat C}(h_X,h_Y)\]
est une bijection donnée par 
\[f\mapsto (W\mapsto (\alpha(W)(g)=f\circ g)).\]
\begin{rem}
  L'énoncé général s'écrit plutôt :
  \[\Hom_{\widehat C}(h_X,h_Y)\ni \alpha\mapsto (\alpha(X)(id_X)\colon X\to Y)\]
\end{rem}
\begin{rem}
  Voir le carnet pour des choses plus deep. Pas oublier
  que c'est bien grâce à ça les propriétés universelles!
\end{rem}
\section{Point de vue du foncteur de point}
Essentiellement, avoir un morphisme de schémas
$X\to Y$ équivaut alors à avoir une flèche 
$h_X\to h_Y$. Si on traduit sur les schémas, il 
suffit d'avoir 
% https://q.uiver.app/#q=WzAsNCxbMCwwLCJYX1MoVCkiXSxbMSwwLCJZX1MoVCkiXSxbMCwxLCJYX1MoVCcpIl0sWzEsMSwiWV9TKFQnKSJdLFswLDJdLFsyLDNdLFswLDFdLFsxLDNdXQ==
\[\begin{tikzcd}
	{X_S(T)} & {Y_S(T)} \\
	{X_S(T')} & {Y_S(T')}
	\arrow[from=1-1, to=1-2]
	\arrow[from=1-1, to=2-1]
	\arrow[from=1-2, to=2-2]
	\arrow[from=2-1, to=2-2]
\end{tikzcd}\]
pour tout $T=Spec(B)$ affine. Plusieurs
choses à debunker : pas oublier comment retrouver le morphisme
de faisceaux. Pas oublier le morphisme topologique.


\section{Propriété universelle et cas $k$-schéma}
En gros c'est l'unique triplé $(X\times_S Y, p,q)$
tel que \[\Hom_{Sch/S}(W, X\times_S Y)\to \Hom_{Sch/S}(W,X)\times \Hom_{Sch/S}(W,Y)\]
donné par $h\mapsto (p\circ h,q\circ h)$ est une bijection.
\subsection{Étude de la définition}
À noter, la propriété de composition $W\to (X\to S)=W\to (Y\to S)$
est bien contenue dans la définition parce que c'est des 
$S$-morphismes. Donc y'a pas de $\times_S$ c'est juste $\times$.
Dans le cas des variétés, on a automatiquement des 
$\bar k$-schémas.
\begin{quest}
  Impact ?
\end{quest}

\begin{rem}
  Y'a pas encore de lemme de Yoneda explicit là, autre que pour
  dire que c'est l'unique triplé tel que la propriété universelle.
\end{rem}


\section{Remarques sur le foncteur de points}
En fait si on a un objet qui représente le produit de foncteurs
de points alors c'est le produit fibré. C'est pour ça que ça 
suffit! Ça l'air vachement utile. 
\begin{strat}
  \item Identifier l'ensemble $\to$ par exemple
    $|X\times_k Y|=|X|\times_k|Y|$ dans le cas $k=\bar k$.
  \item Identifier l'espace topologique. Ça ça a l'air d'être
    induit.
  \item Identifier le faisceau : voir comment faire de l'algèbre.
    (se rappeler de la preuve que les variétés projectives sont
    propres)
\end{strat}

\section{Foncteur de points}
Petite intuition du foncteur de points : en fait sur un $R$-schéma $T$
on a des $R$-flèches $T\to \Spec(R[T_1,\ldots,T_n])$, essentiellement 
c'est donné par 
\[\Hom_R(R[T_1,\ldots,T_n],\Gamma(T,\Or_T))\simeq \Gamma(T,\Or_T)^n\]
là où c'est marrant c'est que si on regarde le noyau donné par des 
polynômes (c'est une $R$-flèche) $f_1,\ldots, f_s$ on obtient une 
flèche dans 
\[\Hom_R(R[T_1,\ldots, T_n]/(f_1,\ldots, f_s), \Or_T(T))\]
en particulier des équations pour des sections globales de $T$.

\begin{rem}
    En toute généralité je sais pas si ça dit grand chose de $T$. Mais
    ça doit être intéressant de creuser.
\end{rem}


\section{Construction}
\subsection{Cas affine}
Étant donné $\Spec(A)$ et $\Spec(B)$ des $C$-schémas affines. Le 
produit fibré est $\Spec(A\otimes_C B)$ (par l'équivalence de 
catégorie). Le faisceau a pour fibres ? Via le morphisme 
topologique 
\[X\times_Z Y\to |X|\times_{|Z|}|Y|\]

\[(A_\p\otimes_{} B_\q)\]
\subsection{}



\chapter{Produits tensoriels}
Via \href{https://kconrad.math.uconn.edu/blurbs/linmultialg/tensorprod.pdf}{ça}, un petit cours du légendaire Keith Conrad.
Aussi les exercices de Vakil seraient cools.
\section{Quand est-ce que $m\otimes n= 0$?}
Étant donné une application bilinéaire 
\[f\colon M\times N\to K\]
on peut la factoriser via $M\times N\to M\otimes N\to K$. Alors
$f(m,n)=0$ dès que $m\otimes n=0$. L'inverse est aussi vrai parce
que $M\times N\to M\otimes N$ est bilinéaire!

\begin{rem}
  Suffit donc de trouver $M\times N\to K$ bilinéaire
  telle que $(m,n)\mapsto k\ne 0$ pour prouver que 
  $m\otimes n\ne 0$.
\end{rem}
\section{Cas des polynômes}
J'aimerai montrer que si $k[T_1,\ldots, T_n]\to A$ est injective
et $k[S_1,\ldots,S_m]\to B$ aussi alors 
\[k[T_1,\ldots,T_n,S_1,\ldots,S_m]\to A\otimes_k B\]
aussi. Avec ce morphisme étant celui qui étend
\[k[T_1,\ldots,T_n]\times k[S_1,\ldots, S_m]\to A\otimes_k B\]
donné par $(f,g)\mapsto f\otimes g$. 
\begin{rem}
  Si on prouve que $A$ et $B$ sont des quotients d'anneaux de 
  polynômes c'est immédiat.
\end{rem}
\subsection{Produits d'anneaux de polynômes}
On a 
$k[T_1,\ldots, T_n]\otimes_k k[S_1,\ldots,S_m]=k[T_1,\ldots, T_n,S_1,\ldots, S_m]$.
Pour le prouver on peut juste remarquer que $(f,g)\mapsto f.g$
est universelle et bilinéaire. 

\subsection{Injectivité}
Ducoup clairement $f\otimes g\ne0$  par ce que $f.g\ne 0$ dès
que $f\ne 0$ ou $g\ne 0$. Ensuite si $\sum_i a_i f_i\otimes g_i=0$

\chapter{Sur les $\bar k$-variétés}
\section{Construction}
Donc là 
\[A(X)\otimes_k A(Y)=k[T_1,\ldots, T_n, S_1,\ldots, S_m]/(I,J)\]
donc le produit fibré est 
\[Z(I,J)\subset \A_k^{n+m}\]
et le faisceau
\section{Adapter les définitions : la bijection ensembliste}
Donc en regardant même pas les schémas, les variétés sur $\bar k$
qu'on regarde correspondent à des $\bar k$-schémas. Mais en 
fait plus simplement, dans le cas affine on à $X\simeq Spm(A(X))$
avec la topologie de Zariski ($\{\m|I\subset \m\}=V(I)$), i.e.
$Spm(A(X))$ est une variété à la Weil! Donc ça fait sens d'écrire :
\[Y,X\to \Spm(k)\]
on déduit direct par le produit fibré dans une catégorie $C$ 
quelconque (donc ici $C=k-$Var) dans la catégorie variétés
abstraites que
\[\Hom_{k-Var}(\Spm(k), X\times_k Y)\approx \Hom_{k-Var}(\Spm(k),X)
\times \Hom_{k-Var}(\Spm(k),Y)\]
via Yoneda maintenant on obtient 
\[|X\times_k Y|=|X|\times_k|Y|\]
une bijection d'ensemble !

\begin{rem}
  C'est pas un homéomorphisme, Yoneda agit au niveau ensembliste.
\end{rem}
\section{Topologie : cas quasi-projectif}
Essentiellement, on regarde la topologie induite sur $\A^n\times
\A^m$, donc quand on a $\Pr^n\times \Pr^m$, $Z^+(I)$ donné par 
la graduation de $\Pr^n$ est la bonne topologie par restriction.

\section{Topologie : cas général}
On peut remarquer $X\times_k Y$ contient (strictement) la topologie
produit, sinon dans le cas p
\section{Faisceau}

\printbibliography
\end{document}


\documentclass[a4paper,12pt]{book}
\usepackage{amsmath,  amsthm,enumerate}
\usepackage{csquotes}
\usepackage[provide=*,french]{babel}
\usepackage[dvipsnames]{xcolor}
\usepackage{quiver, tikz}

%symbole caligraphique
\usepackage{mathrsfs}

%hyperliens
\usepackage{hyperref}

%pseudo-code
\usepackage{algorithm}
\usepackage{algpseudocode}

\usepackage{fancyhdr}

\pagestyle{fancy}
\addtolength{\headwidth}{\marginparsep}
\addtolength{\headwidth}{\marginparwidth}
\renewcommand{\chaptermark}[1]{\markboth{#1}{}}
\renewcommand{\sectionmark}[1]{\markright{\thesection\ #1}}
\fancyhf{}
\fancyfoot[C]{\thepage}
\fancyhead[LO]{\textit \leftmark}
\fancyhead[RE]{\textit \rightmark}
\renewcommand{\headrulewidth}{0pt} % and the line
\fancypagestyle{plain}{%
    \fancyhead{} % get rid of headers
}

%bibliographie
\usepackage[
backend=biber,
style=alphabetic,
sorting=ynt
]{biblatex}

\addbibresource{bib.bib}

\usepackage{appendix}
\renewcommand{\appendixpagename}{Annexe}

\definecolor{wgrey}{RGB}{148, 38, 55}

\setlength\parindent{24pt}

\newcommand{\Z}{\mathbb{Z}}
\newcommand{\R}{\mathbb{R}}
\newcommand{\rel}{\omathcal{R}}
\newcommand{\Q}{\mathbb{Q}}
\newcommand{\C}{\mathbb{C}}
\newcommand{\N}{\mathbb{N}}
\newcommand{\K}{\mathbb{K}}
\newcommand{\A}{\mathbb{A}}
\newcommand{\B}{\mathcal{B}}
\newcommand{\Or}{\mathcal{O}}
\newcommand{\F}{\mathscr F}
\newcommand{\Hom}{\textrm{Hom}}
\newcommand{\disc}{\textrm{disc}}
\newcommand{\Pic}{\textrm{Pic}}
\newcommand{\End}{\textrm{End}}
\newcommand{\Spec}{\textrm{Spec}}
\newcommand{\Supp}{\textrm{Supp}}
\renewcommand{\Im}{\textrm{Im}}


\newcommand{\m}{\mathfrak{m}}
\newcommand{\n}{\mathfrak{n}}
\newcommand{\p}{\mathfrak{p}}


\newcommand{\cL}{\mathscr{L}}
\newcommand{\G}{\mathscr{G}}
\newcommand{\D}{\mathscr{D}}
\newcommand{\E}{\mathscr{E}}
\renewcommand{\Pr}{\mathbb{P}}
\renewcommand{\P}{\mathscr{P}}
\renewcommand{\H}{\mathscr{H}}

\makeatletter
\newcommand{\colim@}[2]{%
  \vtop{\m@th\ialign{##\cr
    \hfil$#1\operator@font colim$\hfil\cr
    \noalign{\nointerlineskip\kern1.5\ex@}#2\cr
    \noalign{\nointerlineskip\kern-\ex@}\cr}}%
}
\newcommand{\colim}{%
  \mathop{\mathpalette\colim@{\rightarrowfill@\scriptscriptstyle}}\nmlimits@
}
\renewcommand{\varprojlim}{%
  \mathop{\mathpalette\varlim@{\leftarrowfill@\scriptscriptstyle}}\nmlimits@
}
\renewcommand{\varinjlim}{%
  \mathop{\mathpalette\varlim@{\rightarrowfill@\scriptscriptstyle}}\nmlimits@
}
\makeatother

\theoremstyle{plain}
\newtheorem{thm}[subsection]{Théoreme}
\newtheorem{lem}[subsection]{Lemme}
\newtheorem{prop}[subsection]{Proposition}
\newtheorem{cor}[subsection]{Corollaire}
\newtheorem{heur}{Heuristique}
\newtheorem{rem}{Remarque}
\newtheorem{note}{Note}

\theoremstyle{definition}
\newtheorem{conj}{Conjecture}
\newtheorem{prob}{Problème}
\newtheorem{quest}{Question}
\newtheorem{prot}{Protocole}
\newtheorem{algo}{Algorithme}
\newtheorem{defn}[subsection]{Définition}
\newtheorem{exmp}[subsection]{Exemples}
\newtheorem{exo}[subsection]{Exercices}
\newtheorem{ex}[subsection]{Exemple}
\newtheorem{rep}{Réponse}
\newtheorem{concl}{Conclusion}
\newtheorem{exs}[subsection]{Exemples}

\theoremstyle{remark}

\definecolor{wgrey}{RGB}{148, 38, 55}
\definecolor{wgreen}{RGB}{100, 200,0} 
\hypersetup{
    colorlinks=true,
    linkcolor=wgreen,
    urlcolor=wgrey,
    filecolor=wgrey
}

\title{Notes perso : Géométrie algébrique}
\date{}

\begin{document}
\maketitle
\tableofcontents
\[\ldots\]   

On prends la convention qu'une courbe est une variété
séparée de dimension pure $1$ (ses composantes sont de dimension
$1$).
\chapter{Courbes intègres lisses propres et corps de fonctions.}
Le but c'est de montrer l'équivalence de catégorie entre 
\begin{enumerate}
  \item Corps de fonctions de degré de transcendance $1$,
  \item Courbes intègres lisses propres (munies de morphismes non constants).
\end{enumerate}
\section{Courbe associée à un corps.}
Étant donné un corps $K$ :
\begin{enumerate}
  \item On prends $U$ intègre affine t.q $k(U)=K$.
  \item On prends une clôture projective $\bar U$.
  \item On normalise $\pi\colon X\to \bar U$.
\end{enumerate}
Alors $X$ est lisse car normale, birationnelle à
$U$ donc $K=k(U)=k(X)$ et propre car $\pi$ est fini
et $Y$ est propre.

\section{Étendre des morphismes}
Étant donné $C$ une courbe lisse. On peut étendre
tout morphisme $C\supset U \to Y$ avec $U$ dense 
en $C\to Y$ dès que $Y$ est propre!
C'est unique puisque $Y$ est séparée.
\subsection{Candidat pour l'extension}
On peut se ramener au cas $C$ affine irréductible 
et $U=C-\{p\}$. Faut trouver un candidat pour
le morphisme étendu :
\begin{enumerate}
  \item De $f\colon U\to Y$ on identifie 
    \[\Gamma_f\subset U\times Y\]
    et $U$. Ensuite on prends
    \[Z=\overline{\Gamma_f}\subset C\times Y\]
    l'image de $id_C\times f$.
  \item Le candidat est maintenant
    \[(p_1|_{\Gamma_f})^{-1}\circp_2 \colon C\to Z \to Y.\]
\end{enumerate} 
Ce qu'on montre c'est que $p_1\colon C\times Y \supset Z\to C$
est un isomorphisme. 

\subsection{Extension}
À noter $p_1|_Z=g$ est un morphisme fermé car $Y$ est propre et
l'image contient $U$ donc est surjective. C'est aussi 
birationnel car un isomorphisme sur $U$. Et là donc on
construit un morphisme autour de $z\in g^{-1} p$.  
\begin{enumerate}
  \item On prends $z\in W$ ouvert affine. On obtient
    $A(C)\hookrightarrow A(W)\hookrightarrow k(C)$.
  \item À noter $A(W)$ est de type fini sur $k$. 
  \item Maintenant, $C$ est lisse en $p$ donc on peut réduire
  $C$ en $V$ avec $p\in V=D(f)$ tel que $\m_p=(t)$. (Nakayama
  donne $f\in 1+\m_p$.)
  \item Les générateurs $b_i$ de $A(W)$ vérifient 
    $b_i(z)=t^{n_i}(z)a_i/u_i(z)$ avec $a_i,u_i\in A(C)$.
    Via 
    \[u_i(z)b_i(z)=t^{n_i}(z)a_i(z)\]
    ça force $n_i\geq 0$ car $v_p(u_i)=v_p(a_i)=0$
    et $b_i$ est régulière.
    (pas oublier l'identification de $A(C)$ a son 
    image, $u_i(z)=u_i(p)$.)
  \item Maintenant pas finitude on obtient
    \[A(C)\hookrightarrow A(W)\hookrightarrow A(C)_{u_0}\]
    d'où on obtient \[g^{-1}D(u_0)\cap W\to D(u_0)\]
    est un isomorphisme qui coincide avec $A(C)\to A(W)$
    sur les intersections.
  \item Enfin, il est définit en $p$!
\end{enumerate}

\begin{concl}
  On sait étendre les morphismes $C-\{p\} \to Y$ 
  si $C$ est lisse et $Y$ est propre.
\end{concl}
\section{L'équivalence de catégorie}
\subsection{Isomorphismes de courbes intègres propres lisses}
On sait qu'un morphisme birationnel $X\supset U\to V\subset Y$
s'étend en $X\to Y$ un isomorphisme.

\subsection{Morphismes de courbes intègres propres lisses}
Un tel morphisme est soit constant soit fini surjectif ! 
\begin{enumerate}
  \item La surjectivité est claire.
  \item La finitude c'est juste que si
    \[\pi \colon X'\to Y\]
    est la normalisaiton de $Y$ dans $k(X)$ alors
    $X\simeq X'$ par l'identité de $k(X)\simeq k(X')$.
    Alors $X\to Y$ coincide avec $X\to X'\to Y$ est fini.
\end{enumerate}

\section{Les courbes propres intègres propres lisses sont projectives.}
Comme le résultat de la section!




\printbibliography
\end{document}


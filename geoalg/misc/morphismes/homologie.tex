\documentclass[a4paper,12pt]{book}
\usepackage{amsmath,  amsthm,enumerate}
\usepackage{csquotes}
\usepackage[provide=*,french]{babel}
\usepackage[dvipsnames]{xcolor}
\usepackage{quiver, tikz}

%symbole caligraphique
\usepackage{mathrsfs}

%hyperliens
\usepackage{hyperref}

%pseudo-code
\usepackage{algorithm}
\usepackage{algpseudocode}

\usepackage{fancyhdr}

\pagestyle{fancy}
\addtolength{\headwidth}{\marginparsep}
\addtolength{\headwidth}{\marginparwidth}
\renewcommand{\chaptermark}[1]{\markboth{#1}{}}
\renewcommand{\sectionmark}[1]{\markright{\thesection\ #1}}
\fancyhf{}
\fancyfoot[C]{\thepage}
\fancyhead[LO]{\textit \leftmark}
\fancyhead[RE]{\textit \rightmark}
\renewcommand{\headrulewidth}{0pt} % and the line
\fancypagestyle{plain}{%
    \fancyhead{} % get rid of headers
}

%bibliographie
\usepackage[
backend=biber,
style=alphabetic,
sorting=ynt
]{biblatex}

\addbibresource{bib.bib}

\usepackage{appendix}
\renewcommand{\appendixpagename}{Annexe}

\definecolor{wgrey}{RGB}{148, 38, 55}

\setlength\parindent{24pt}

\newcommand{\Z}{\mathbb{Z}}
\newcommand{\R}{\mathbb{R}}
\newcommand{\rel}{\omathcal{R}}
\newcommand{\Q}{\mathbb{Q}}
\newcommand{\C}{\mathbb{C}}
\newcommand{\N}{\mathbb{N}}
\newcommand{\K}{\mathbb{K}}
\newcommand{\A}{\mathbb{A}}
\newcommand{\B}{\mathcal{B}}
\newcommand{\Or}{\mathcal{O}}
\newcommand{\F}{\mathscr F}
\newcommand{\Hom}{\textrm{Hom}}
\newcommand{\disc}{\textrm{disc}}
\newcommand{\Pic}{\textrm{Pic}}
\newcommand{\End}{\textrm{End}}
\newcommand{\Spec}{\textrm{Spec}}
\newcommand{\Supp}{\textrm{Supp}}
\renewcommand{\Im}{\textrm{Im}}


\newcommand{\m}{\mathfrak{m}}
\newcommand{\p}{\mathfrak{p}}


\newcommand{\cL}{\mathscr{L}}
\newcommand{\G}{\mathscr{G}}
\newcommand{\D}{\mathscr{D}}
\newcommand{\E}{\mathscr{E}}
\renewcommand{\Pr}{\mathbb{P}}
\renewcommand{\P}{\mathscr{P}}
\renewcommand{\H}{\mathscr{H}}

\makeatletter
\newcommand{\colim@}[2]{%
  \vtop{\m@th\ialign{##\cr
    \hfil$#1\operator@font colim$\hfil\cr
    \noalign{\nointerlineskip\kern1.5\ex@}#2\cr
    \noalign{\nointerlineskip\kern-\ex@}\cr}}%
}
\newcommand{\colim}{%
  \mathop{\mathpalette\colim@{\rightarrowfill@\scriptscriptstyle}}\nmlimits@
}
\renewcommand{\varprojlim}{%
  \mathop{\mathpalette\varlim@{\leftarrowfill@\scriptscriptstyle}}\nmlimits@
}
\renewcommand{\varinjlim}{%
  \mathop{\mathpalette\varlim@{\rightarrowfill@\scriptscriptstyle}}\nmlimits@
}
\makeatother

\theoremstyle{plain}
\newtheorem{thm}[subsection]{Théoreme}
\newtheorem{lem}[subsection]{Lemme}
\newtheorem{prop}[subsection]{Proposition}
\newtheorem{cor}[subsection]{Corollaire}
\newtheorem{heur}{Heuristique}
\newtheorem{rem}{Remarque}
\newtheorem{note}{Note}

\theoremstyle{definition}
\newtheorem{conj}{Conjecture}
\newtheorem{prob}{Problème}
\newtheorem{quest}{Question}
\newtheorem{prot}{Protocole}
\newtheorem{algo}{Algorithme}
\newtheorem{defn}[subsection]{Définition}
\newtheorem{exmp}[subsection]{Exemples}
\newtheorem{exo}[subsection]{Exercices}
\newtheorem{ex}[subsection]{Exemple}
\newtheorem{exs}[subsection]{Exemples}
\newtheorem{concl}{Conclusion}

\theoremstyle{remark}

\definecolor{wgrey}{RGB}{148, 38, 55}
\definecolor{wgreen}{RGB}{100, 200,0} 
\hypersetup{
    colorlinks=true,
    linkcolor=wgreen,
    urlcolor=wgrey,
    filecolor=wgrey
}

\title{Notes perso : Géométrie algébrique}
\date{}

\begin{document}
\maketitle
\tableofcontents

\chapter{Morphismes d'espaces localement annelés}
Le but là c'est de revoir pourquoi dans le cas localement annelé 
$(f,f^\sharp)\colon X\to Y$ la flèche $f^\sharp$ correspond bien au
pullback $g\mapsto g\circ f$ (c'est pas tjr défini sur un Schéma affine)
et l'intuition est exacte dans le cas des variétés. Ensuite de voir
pourquoi les schémas affines généralisent bien les variétés et ensembles
algébriques initiales.

\section{Le cas des $k$-algèbre de type fini}
Dans le cas des variétés abstraites, on prends deux variétés affines
$X\simeq (Z(I)\subset \A^n, \Or_X)$ et $(Y,\Or_Y)$. Étant donné un
morphisme d'espaces localement annelés $(f,f^\sharp)\colon X\to Y$ on 
a
% https://q.uiver.app/#q=WzAsMTAsWzEsMCwiZl57XFwjfVxcY29sb25cXG1hdGhjYWwgT19ZKFUpIl0sWzIsMCwiXFxtYXRoY2FsIE9fWChmXnstMX1VKSJdLFsxLDEsIlxcbWF0aGNhbCBPX3tZLGYoeCl9Il0sWzIsMSwiXFxtYXRoY2FsIE9fe1gseH0iXSxbMSwyLCJcXGthcHBhKGYoeCkpIl0sWzIsMiwiXFxrYXBwYSh4KSJdLFswLDAsInMiXSxbMCwyLCJzKGYoeCkpIl0sWzMsMCwiZl57XFwjfShzKSJdLFszLDIsImZee1xcI30ocykoeCkiXSxbMCwxXSxbMCwyXSxbMSwzXSxbMiwzXSxbMiw0XSxbMyw1XSxbNCw1LCJpZCJdLFs2LDcsIiIsMSx7InN0eWxlIjp7InRhaWwiOnsibmFtZSI6Im1hcHMgdG8ifX19XSxbOCw5LCIiLDAseyJzdHlsZSI6eyJ0YWlsIjp7Im5hbWUiOiJtYXBzIHRvIn19fV1d
\[\begin{tikzcd}
	s & {f^{\#}\colon\mathcal O_Y(U)} & {\mathcal O_X(f^{-1}U)} & {f^{\#}(s)} \\
	& {\mathcal O_{Y,f(x)}} & {\mathcal O_{X,x}} \\
	{s(f(x))} & {\kappa(f(x))=k} & {\kappa(x)=k} & {f^{\#}(s)(x)}
	\arrow[maps to, from=1-1, to=3-1]
	\arrow[from=1-2, to=1-3]
	\arrow[from=1-2, to=2-2]
	\arrow[from=1-3, to=2-3]
	\arrow[maps to, from=1-4, to=3-4]
	\arrow[from=2-2, to=2-3]
	\arrow[from=2-2, to=3-2]
	\arrow[from=2-3, to=3-3]
	\arrow["id", from=3-2, to=3-3]
\end{tikzcd}\]
Un truc qui fait tiquer c'est que ça semble pas utiliser le fait que
c'est localement annelé, déjà c'est pas clair que c'est un morphisme
de $k$-algèbres les morphismes de corps résiduels. En fait l'endroit
où ça l'utilise c'est que l'existence de ce morphisme vient du fait que
\[\Or_{Y,f(x)}\to \Or_{X,x}\to \kappa(x)=\Or_{X,x}/\p_x\]
passe au quotient parce que $f^\sharp_x(\p_{f(x)})\subseteq \p_x$.


\subsection{Le morphisme sur les fibres}
Retour sur l'adjonction entre $f^{-1}$ et $f_*$ pour un morphisme de 
faisceaux $f\colon (X,\F)\to (Y,\G)$. On remarque que des deux carrés 
% https://q.uiver.app/#q=WzAsOSxbMSwwLCJcXE9yX1goZl57LTF9VSkiXSxbMCwwLCJcXE9yX1koVSkiXSxbMCwxLCJcXE9yX3tZLGYoeCl9Il0sWzEsMSwiXFxPcl97WCx4fSJdLFs0LDAsIlxcT3JfWChWKSJdLFs0LDEsIlxcT3Jfe1gseH0iXSxbMywxLCIoZl57LTF9XFxPcl9ZKV97Zih4KX0iXSxbMywwLCIoZl57LTF9XFxPcl9ZKShWKSJdLFszLDIsIlxcT3Jfe1ksZih4KX0iXSxbMSwwXSxbMSwyXSxbMiwzXSxbMCwzXSxbNiw1XSxbNyw2XSxbNiw4LCJcXHNpbWVxIiwxLHsic3R5bGUiOnsiYm9keSI6eyJuYW1lIjoibm9uZSJ9LCJoZWFkIjp7Im5hbWUiOiJub25lIn19fV0sWzcsNF0sWzQsNV1d
\[\begin{tikzcd}
	{\Or_Y(U)} & {\Or_X(f^{-1}U)} && {(f^{-1}\Or_Y)(V)} & {\Or_X(V)} \\
	{\Or_{Y,f(x)}} & {\Or_{X,x}} && {(f^{-1}\Or_Y)_{f(x)}} & {\Or_{X,x}} \\
	&&& {\Or_{Y,f(x)}}
	\arrow[from=1-1, to=1-2]
	\arrow[from=1-1, to=2-1]
	\arrow[from=1-2, to=2-2]
	\arrow[from=1-4, to=1-5]
	\arrow[from=1-4, to=2-4]
	\arrow[from=1-5, to=2-5]
	\arrow[from=2-1, to=2-2]
	\arrow[from=2-4, to=2-5]
	\arrow["\simeq"{description}, draw=none, from=2-4, to=3-4]
\end{tikzcd}\]
C'est étonnamment celui de droite qui est intuitif à prouver. On
regarde la famille des $f^{-1}\Or_Y(V)\to \Or_X(V)\to \Or_{X,x}$ ça 
donne une flèche de la limite des $x\in V$ tels que $f(V)\subseteq U$
d'où la limite des $f(x)\in U$!



\section{Conséquences}
Le fait que $f\colon X\to Y$ est un morphisme d'espaces localement
annelé force $f^\sharp(U)(P)=P\circ f|_U$ ça a tout de suite des 
conséquences. Par exemple, si $f$ est dominant, à quelle conditions
\[P\circ f|_U=Q\circ f|_U\] force $(P=Q)|_U$ ? De ça on peut déduire
des critères pour que le morphisme de faisceau soit surjectif ou 
injectif.

\subsection{Cas affine}

\section{Utilisation de la caractérisation $f^\sharp(g)=g\circ f$}
\subsection{Variétés affines vers espaces localement annelés}
Pour montrer que $Z\mapsto (Z,\Or_Z)$ est un pleinement fidèle
faut montrer qu'étant donné $f\colon X\to Y$, et $g\in \Or_Y(V)$
alors $g\circ f\in \Or_X(f^{-1}V)$.
\begin{rem}
  C'est bien une \textbf{fonction}, mais c'est pas clair qu'elle
  est régulière.
\end{rem}
Dans notre cas avec la définition faut prouver que $g\circ f$ est
régulière sachant qu'on a 
\[f_*\colon A(Y)\to A(X).\]
Pour $x_0\in f^{-1}V$ on prends $f(x_0)\subset V_0$ tel que 
$g|_{V_0}=G/H$ avec $H(f(x_0))\ne 0$, alors
$g|_{V_0}\circ f=f_*G/f_*H$ (comme fonctions) est bien
une fraction avec $f_*H(x_0)\ne 0$.
\begin{rem}
  $f_*$ se définit via $\A^n\to \A^m$.
\end{rem}
À l'inverse faut montrer que de $f^\sharp$ on peut trouver une
fonction polynomiale $X\to Y\subset \A^m$. Typiquement via
$k[T_1,\ldots, T_m]\to A(X)$ et $f^\sharp(g)=g\circ f$ sur les
coordonnées.



\chapter{Immersions}
La page stacks sur les \href{https://stacks.math.columbia.edu/tag/01IM}{
immersions}.

\section{Immersions ouvertes}
C'est un homéomorphisme $f\colon X\to U\subset Y$ de $X$ sur $U$
d'espaces localement annelés tel que 
\[f^\flat \colon f^{-1}\Or_Y\to \Or_X\] est un isomorphisme.
\newline

Dans le cas des variétés c'est pareil. $X$ est isomorphe à un ouvert
de $Y$.

\section{Immersions fermées}
On demande à ce que $f(X)$ soit fermé et $X\to f(X)$ soit un 
isomorphisme. Pour rappel on peut seulement demander à ce que ce soit
un homéomorphisme et $f^\sharp$ est surjective.
\begin{rem}
  Dans le stacks ils disent même qu'on peut dire que c'est un morphisme 
  affine tel que $f^\sharp$ est surjective. C'est super intuitif ducoup.
  Et l'équivalence à l'air claire! 
\end{rem}

\subsection{Rappel sur $f^{-1}$}
À nouveau, on peut le voir comme un ensemble de classes d'équivalences
avant de faisceautiser.
\subsection{À faire}
Regarder cette co-unité et la préservation des épis/mono.
Les adjoints ça préserve les limites/colimites finies. Et adjoint
à gauche préserve toutes les colimites. 

\subsection{Étude de la déf}
En gros ce qu'on voudrait c'est l'équivalence entre :
\begin{enumerate}
  \item Structure donnée par $X$ sur $f(X)$, via $f_*\Or_X$.
  \item Structure donnée par $Y$ sur $f(X)$, via $i_{f(X)}^{-1}\Or_Y$.
\end{enumerate}
pour $i_{f(X)}\colon f(X)\hookrightarrow Y$ l'inclusion. Où la
deuxième structure est la structure naturelle.
\newline

Pour un fermé $Z\subset Y$. On peut d'abord regarder si 
\[\Or_Y\to (i_Z)_*i_Z^{-1}\Or_Y\]
est surjectif. C'est marrant ça provient de la co-unité.
À remarquer, 
\[((i_Z)_*i_Z^{-1}\Or_Y)_y=\begin{cases}(i_Z^{-1}\Or_Y)_{y}=\Or_{Y,y}\textrm{ Si }y\in Z\\
0\textrm{ sinon, car limite vide}\end{cases}\]
Maintenant la surjectivité est claire sur les fibres on a 
\[i_Z^\sharp\begin{cases}\Or_{Y,y}\to \Or_{Y,i_Z(y)}\textrm{ si }y\in Z\\
  \Or_{Y,y}\to 0\textrm{ sinon.}
\end{cases}\]
Maintenant tout se recolle, en posant $f(X)=Z$ alors 
\[\Or_Y\to (i_{f(X)})_*\Or_{f(X)}\simeq (i_{f(X)})_*i_{f(X)}^{-1}\Or_Y\to 0\]
est exacte. D'où la surjectivité qu'on voulait. 
\begin{concl}
  Pour conclure, la structure naturelle sur $Z\subset Y$ est donnée
  par $i_Z^{-1}\Or_Y$. Et naturellement, la définition par $X\simeq f(X)$
  est la bonne. En particulier, on obtient $i_Z^{-1}\Or_Y\simeq f_*\Or_X$
  pour $Z=f(X)$.
\end{concl}
Pour l'inverse, si $\Or_Y\to f_*\Or_X\to 0$ est exacte. On note
$I$ le noyau, et si $f$ est affine.
Inversement, si $\Or_Y\to f_*\Or_X\to 0$ est exacte on peut dire quelque
chose.

\subsection{Cas affine}
Pour le cas affine c'est plus simple si $f\colon X\to Y$ a une image
fermée, le morphisme induit
\[f^\sharp(Y)\colon A(Y)\to A(X)\]
est surjectif parce que $f(X)=\overline{f(X)}=Z(\ker(f^\sharp(Y)))$
d'où ça se factorise en
\[A(Y)\to A(Y)/I(Z(\ker(f^\sharp(Y))))\simeq A(X).\]
En fait en termes de faisceaux, si on regarde

\[0\to I\to \Or_Y\to f_*\Or_X\to 0\]
on peut poser $I\colon U\mapsto I(f(X))\Or_Y(U)$.

\subsection{Cas des variétés}


\subsection{Avec la définition affine et faisceau surjectif}
Eh ben c'est super cool parce que localement on obtient clairement
la tête de l'image via la sous-section d'avant.


\chapter{Morphismes affines}
Y'a un cas qui est évident que je veux traiter celui des 
variétés quasi-projectives. Et je crois que les variétés
séparées aussi. Déjà, suffit d'avoir ça

\begin{defn}
    Un morphisme $f\colon X\to Y$ est affine si $f^{-1}U$
    est affine dès que $U$ est affine ouvert de $Y$.
\end{defn}
\begin{prop}
    Un morphisme est affine ssi il existe un recouvrement
    $Y=\cup U_i$ t.q $f^{-1}U_i$ est affine pour chaque $i$.
\end{prop}
\subsection{Cas quasi-projectif}
Si on a $f\colon X\to Y$ entre variétés quasi-projectives, 
faut remarquer si 


\chapter{Morphismes séparés}
\section{Exos}
Séparé ssi il existe un recouvrement affine $\cup_i X_i$ t.q
\begin{enumerate}
  \item $X_i\cap X_j$ est affine.
  \item $X_i\cap X_j\to X_i\times X_j$ est une immersion fermée.
\end{enumerate}
Application : Produits de variétés séparées sont séparées.
Application 2 : Variétés projectives sont séparées.
\subsection{Ligne à double point}
On la construit via $T\mapsto T$ sur $D(T)\subset \A^1$.
J'appelle $L$ le recollement et ses ouverts c'est clair. Pour 
le faisceau ses sections globales c'est clair aussi via le critère.
Pour la séparation, si $O_1,O_2$ c'est
les deux points du recollement alors $(O_1,O_2)\notin \Delta_L$
mais tout ouvert qui contient $(O_1,O_2)$ contient un $D(f_1)\times
D(f_2)$ et $D(f_1)\cap D(f_2)\ne 0$ d'où $\Delta_L$ est pas fermé.

\section{Intersections d'affines est affine}
Si $X\to k$ est séparé, alors pour tout $U,V$ affines :
\[U\cap V\to U\cap V\times U\cap V\to U\times V\]
est une immersion fermée vers un affine! Ça donne une réciproque
au truc d'avant.

\section{Sur des $k$-schémas}
On veut que 
\[\Delta_S(X)\colon X\to X\times_S X\]
soit une immersion fermée. Alors $X\to S$ est
séparé.
\begin{rem}
  Pour rappel, on prends $X\times_S Y = (i_X,i_Y)^{-1}(\Delta_k(S))$
  où $i_X$ et $i_Y$ sont les morphismes structurants.
\end{rem}

\chapter{Morphismes propres}
On dit que $X\to S$ est propre si $X$ est
universellement fermé dans $Sch/S$ et séparé
(y'a une condition de finitude aussi).
\section{La ligne a point double est universellement fermée
mais pas séparée}
Petit exercice.

\section{Conséquence de la propreté}
Soit $X$ est propre : 
\begin{enumerate}
  \item Si $Y$ est séparée, tout morphisme $X\to Y$ est fermé.
\end{enumerate}




\chapter{Morphismes finis}
\section{Fini implique affine}
Je crois que c'est pas définition. Pas besoin d'aller chercher
pourquoi là je pense.
\section{Surjectivité des morphismes finis}
Ducoup si on regarde la condition $x\in f^{-1}(y)$, ça se traduit par
$f_*\m_y\subset \m_x$. Inversement, si $f^{-1}(y)$ est vide alors
$f_*\m_yk[X]=k[X]$ et on applique le lemme de Nakayama car 
$\m_y\ne k[Y]$. \textbf{Faudrait essayer de voir plus concrètement
comment trouver une racine.}


\section{Trouver un ouvert dans $f(X)$}
L'idée c'est de décomposer $f$ : $f\colon X\to Y\times \A^r\to Y$ 
défini par $k[Y]\subset k[Y][u_1,\ldots, u_r]\subset k[X]$ où 
$k(X)=k(Y)(u_1,\ldots, u_r, z)$. La première flèche faut juste 
choisir $u_i\in k[X]$, rendre l'extension entière avec la 
technique habituelle puis tout dérouler en utilisant le fait 
que les morphismes finis sont surjectifs. La deuxième c'est
juste que la première projection on peut trouver un ouvert dans
l'image d'un ouvert facilement.

\begin{rem}
  Y'a quand même un petit truc intéressant, le morphisme induit par 
  l'inclusion
  \[k[Y][u_1,\ldots, u_r]\subset k[X]\]
  on l'interprète en $X\to Y\times \A^r$, en disant que 
  $k[\A^r]=k[u_1,\ldots, u_r]$ sauf que y'a une subtilité pour que ça
  marche. Vaudrait mieux dire \[k[Y][T_1,\ldots, T_r]\hookrightarrow
  f_*k[Y][u_1,\ldots, u_r]\]
  pour que ce soit plus clair. Autrement dit, le statut de 
  $u_1,\ldots,u_r$ comme fonctions sur $\A^r$ est pas clair, c'est pas 
  le même que celui de $u_1,\ldots, u_r$ sur $X$ ? Aussi, ça marche 
  parce que être algébriquement indépendant sur $k(Y)$ implique l'être
  sur $k$ ! Et ça c'est peut-être non trivial, au sens où on a une
  flèche dominante $X\supset U\to \A^r$ ($U$ un ouvert de définition
  des $u_i$).
\end{rem}

\section{Projections}
Ça c'était cool, et plus deep que prévu. Ducoup, 
on considère des "e.v" de $\Pr^n$ (juste les regarder
dans $\Pr^{n+1}$). Donnés par $E=Z(L_1,\ldots,L_{n-d})$,
et on déf 
\[\pi\colon \Pr^n-E\to \Pr^{n-d-1};(\bar x)\mapsto (L_i(\bar x))_i\]
L'idée géometrique c'est que si on prend 
$\Pr^{n-d-1}\simeq H\subset \Pr^n-E$, alors $H\cap E=0$ et $H\oplus E=
\A^{n+1}$ ce qui définit un projecteur. En particulier, si
$x\in \Pr^{n}-E$ alors 
\[E\oplus <x>\cap H\textrm{ est de dimension }1.\]
Donc on a un point d'intersection dans $\Pr^{n}$, et il est donné par 
$\pi$. C'est pas entièrement clair, mais c'est clair que c'est une 
projection au sens où $H\oplus E=\A^{n+1}$ a des projecteurs associés.

Maintenant pour prouver que c'est une flèche finie, c'est plutôt cool,
déjà on regarde $\pi\colon D(L_j)\cap X\to \A^{n-d-1}\cap \pi(X)$ et 
on veut montrer que $\pi^*\colon k[\A^{n-d-1}]\to k[D(L_j)\cap X]$
est finie, et $g\in \Or_X(D(L_j))$ est de la forme $g=G/L_i^m$
($\Pr^n-E=\cup D(L_i)$). Maintenant $\pi^*(T_i)=L_i/L_j$ donc on doit 
trouver une relation algébrique unitaire entre $g$ est les $\pi^*(T_i)$.
C'est là que c'est fort. On remarque que l'image de
\[\pi_m\colon X\to \Pr^{n-d};(\bar x)\mapsto (L_1^m,\ldots,
L_{n-d}^m,G)\]
est fermée. On obtient $F_1,\ldots, F_s$ des polynômes qui annulent 
l'image ! Mais ça suffit pas encore, j'ai l'impression qu'on peut
montrer au plus qu'on a seulement une relation du type 
$X\circ(L_i)^{\alpha}G^k -\ldots$ pas unitaire et qu'on devra localiser 
à trop de $L_i$. Le deuxième point important c'est que
\[[0:\ldots:0:1]\notin \pi_m(X)\]
parce que les $L_i$ s'annulent pas simultanément, en particulier, 
\[V(F_1,\ldots,F_s, z_0,\ldots, z_{n-d-1})=\emptyset\]
D'où ... il existe $k$ t.q. 
$z_{n-d}^k\in (F_1(\bar z),\ldots, F_s(\bar z))+(z_i, i) \subset 
k[z_0,\ldots, z_{n-d}]$. Tu vois le truc venir ? En particulier, 
si on regarde dans $\pi_m(X)$ on obtient 
\[\psi(z_1,\ldots,z_{n-d})=z_{n-d}-\sum z_j H_j = 0\mod I(\pi_m(X))\]
En regardant dans $z_i\ne 0$ on obtient 
\[k[\A^{n-d}\cap D(z_i)]=k[z_0,\ldots, z_{n-d}]_{z_i}\to 
\Or_X(D(z_i\circ \pi_m)\cap X)\]
la flèche $z_j\mapsto L_j^m/L_i^m$ et $z_{n-d}\mapsto G/L_i^m$. De sorte
que dans $\Or_X(D(z_i\circ \pi_m)\cap X)$ on ait 
\[\psi(G_i/L_i^m)=0;
    ~\psi\in k[\pi(X)\cap D(T_{i})][T]\textrm{ unitaire.}
\]




\printbibliography
\end{document}


\documentclass[a4paper,12pt]{book}
\usepackage{amsmath,  amsthm,enumerate}
\usepackage{csquotes}
\usepackage[provide=*,french]{babel}
\usepackage[dvipsnames]{xcolor}
\usepackage{quiver, tikz}

%symbole caligraphique
\usepackage{mathrsfs}

%hyperliens
\usepackage{hyperref}

%pseudo-code
\usepackage{algorithm}
\usepackage{algpseudocode}

\usepackage{fancyhdr}

\pagestyle{fancy}
\addtolength{\headwidth}{\marginparsep}
\addtolength{\headwidth}{\marginparwidth}
\renewcommand{\chaptermark}[1]{\markboth{#1}{}}
\renewcommand{\sectionmark}[1]{\markright{\thesection\ #1}}
\fancyhf{}
\fancyfoot[C]{\thepage}
\fancyhead[LO]{\textit \leftmark}
\fancyhead[RE]{\textit \rightmark}
\renewcommand{\headrulewidth}{0pt} % and the line
\fancypagestyle{plain}{%
    \fancyhead{} % get rid of headers
}

%bibliographie
\usepackage[
backend=biber,
style=alphabetic,
sorting=ynt
]{biblatex}

\addbibresource{bib.bib}

\usepackage{appendix}
\renewcommand{\appendixpagename}{Annexe}

\definecolor{wgrey}{RGB}{148, 38, 55}

\setlength\parindent{24pt}

\newcommand{\Z}{\mathbb{Z}}
\newcommand{\R}{\mathbb{R}}
\newcommand{\rel}{\omathcal{R}}
\newcommand{\Q}{\mathbb{Q}}
\newcommand{\C}{\mathbb{C}}
\newcommand{\N}{\mathbb{N}}
\newcommand{\K}{\mathbb{K}}
\newcommand{\A}{\mathbb{A}}
\newcommand{\B}{\mathcal{B}}
\newcommand{\Or}{\mathcal{O}}
\newcommand{\F}{\mathscr F}
\newcommand{\Hom}{\textrm{Hom}}
\newcommand{\disc}{\textrm{disc}}
\newcommand{\Pic}{\textrm{Pic}}
\newcommand{\End}{\textrm{End}}
\newcommand{\Spec}{\textrm{Spec}}
\newcommand{\Supp}{\textrm{Supp}}
\renewcommand{\Im}{\textrm{Im}}


\newcommand{\m}{\mathfrak{m}}
\newcommand{\p}{\mathfrak{p}}


\newcommand{\cL}{\mathscr{L}}
\newcommand{\G}{\mathscr{G}}
\newcommand{\D}{\mathscr{D}}
\newcommand{\E}{\mathscr{E}}
\renewcommand{\Pr}{\mathbb{P}}
\renewcommand{\P}{\mathscr{P}}
\renewcommand{\H}{\mathscr{H}}

\makeatletter
\newcommand{\colim@}[2]{%
  \vtop{\m@th\ialign{##\cr
    \hfil$#1\operator@font colim$\hfil\cr
    \noalign{\nointerlineskip\kern1.5\ex@}#2\cr
    \noalign{\nointerlineskip\kern-\ex@}\cr}}%
}
\newcommand{\colim}{%
  \mathop{\mathpalette\colim@{\rightarrowfill@\scriptscriptstyle}}\nmlimits@
}
\renewcommand{\varprojlim}{%
  \mathop{\mathpalette\varlim@{\leftarrowfill@\scriptscriptstyle}}\nmlimits@
}
\renewcommand{\varinjlim}{%
  \mathop{\mathpalette\varlim@{\rightarrowfill@\scriptscriptstyle}}\nmlimits@
}
\makeatother

\theoremstyle{plain}
\newtheorem{thm}[subsection]{Théoreme}
\newtheorem{lem}[subsection]{Lemme}
\newtheorem{prop}[subsection]{Proposition}
\newtheorem{cor}[subsection]{Corollaire}
\newtheorem{heur}{Heuristique}
\newtheorem{rem}{Remarque}
\newtheorem{note}{Note}

\theoremstyle{definition}
\newtheorem{conj}{Conjecture}
\newtheorem{prob}{Problème}
\newtheorem{quest}{Question}
\newtheorem{prot}{Protocole}
\newtheorem{algo}{Algorithme}
\newtheorem{defn}[subsection]{Définition}
\newtheorem{exmp}[subsection]{Exemples}
\newtheorem{exo}[subsection]{Exercices}
\newtheorem{ex}[subsection]{Exemple}
\newtheorem{exs}[subsection]{Exemples}

\theoremstyle{remark}

\definecolor{wgrey}{RGB}{148, 38, 55}
\definecolor{wgreen}{RGB}{100, 200,0} 
\hypersetup{
    colorlinks=true,
    linkcolor=wgreen,
    urlcolor=wgrey,
    filecolor=wgrey
}

\title{Notes perso : Géométrie algébrique}
\date{}

\begin{document}
\maketitle
\tableofcontents
\[\ldots\]   

Bon encore et toujours, le Nakayama, j'ai vu une preuve un peu en détail.
En gros, si $M$ est un $A$-module de type fini. Et si $IM=M$, on peut 
dire plusieurs choses : soit $(m_0,\ldots, m_n)$ une famille génératrice
de $M$, on a 
\[m_j = \sum_{i=1}^n \alpha_{ij}m_i\]
avec $\alpha_{ij}\in I$. D'où la matrice $A=(\alpha_{ij})$ vérifie,
$\ker(A-I_n)=M$ en tant qu'endomorphisme du $A$-module $M$. Maintenant
on peut faire plusieurs choses.

\section{Avec la formule de Cramer}
Par la formule de Cramer, on a $\det(A-I_n).M=d.M=0$. Avec $d\in 1+I$, 
ça ca se vérifie en développant la diagonale. 

\section{Directement mdr}
En fait la matrice de $M\mapsto IM$ c'est vraiment l'identité ptdr,
en particulier si on écrit la matrice de l'application, on l'appelle
$C$, alors $Ce_i=e_i$ pour tout vecteur de la famille génératrice. 
Juste on le réecrit à coefficients dans $I$. Et on a $C\ne I$ parce que
$1\notin I$. Bon maintenant avec Cayley-Hamilton :
\[C^n+a_{n-1}C^{n-1}+\ldots+a_1C+a_0=0\]
avec $a_i\in I$. Bon on a $\phi_C.v=v$ pour tout $v$, en particulier,
$\phi_C(\sum a_i + 1)=0$. En particulier, l'endomorphisme 
$m_x=m_{\sum a_i +1}$ a pour noyau tout $M$ et $\phi_C$ est l'identité.
D'où $xM=0$ et $\sum a_i \ne -1$ sinon $1\in I$.

\section{Conséquences}
On a des objets maintenant il reste juste à trouver des critères sur
$1+I$ et $d$.

\subsection{$M=B$ est un anneau}
Si $A\subset B=M$ est un sur-anneau de $A$, on a $1_A\in M=B$,
d'où $aB=0$ seulement si $a=0$. On peut en déduire via 
$0\in1+I$ ssi $I=(1)$ que 
\[I\subset A\implies IB\subset B\]
où les inclusions sont strictes.

\subsection{Radical de Jacobson}

Maintenant pareil, si $1+I$ ne contient que des inversibles. 
Par exemple si $I$ est le radical de Jacobson, on obtient le critère 
avec \[I=\bigcap_{\m\in Specm(A)} \m\]
que $I.M=M$ implique $M=0$, ici car $d$ est inversible.
\subsection{Cas général : $1+I$ contient que des inversibles}
Plus généralement ducoup si $1+I$
ne contient que des inversible, si $M'+IM=M$ pour un $M'$ quelconque 
alors $M=M'$, y suffit d'appliquer la méthode à $M/M'$.

\subsection{Dimension $d$ en tant que $k$-ev implique généré par
$d$ éléments}
L'espace $\m/\m^2$ un $A/\m$ espace-vectoriel. En particulier, si
\[\m/\m^2=\oplus_{i=1}^d ke_i\]
alors $\m=(e_1,\ldots,e_d)+\m^2$ dans $A$. En posant $M=\m$ et 
$N=(e_1,\ldots, e_d)$ on peut regarder
\[\m/(e_1,\ldots,e_d)=\m.(\m/(e_1,\ldots,e_d))\]

avec $M/N$ un $A$-module et $I=\m$. On obtient $f\in 1+\m$ tel que
$f\m=(e_1,\ldots,e_d)$. En particulier si on localise en $f$ on a fini.
Et si $A$ est local y'a égalité.

\section{Morphisme fini implique surjectif}
En fait la fibre $f^{-1}(y)$ vérifie des équations $f^*\m_y$ et 
$f(x)=y$ veut dire que $f^*\m_y\subset \m_x$. À l'inverse, 
$f^{-1}(y)=\emptyset$ implique $f_*\m_y=k[X]$.
On remarque alors que en identifiant 
\[k[Y]=f_*k[Y]\]
\begin{rem}
    Les flèches finies sont dominantes donc cette identification est pas
    bizarre, la flèche est injective.
\end{rem}
si $X\to Y$ est fini, alors $k[X]$ est un $k[X]$-module de type fini
et \[\m_yk[X]=k[X]\] via l'identification. En particulier, via le 
critère où $M$ est un anneau ça peut pas arriver. D'où la surjectivité.

\begin{exo}
    Preuve constructive ?
\end{exo}
\section{Propreté des variétés projectives}
\begin{note}
    On utilise seulement l'existence du $f\in 1+I$.
\end{note}

La preuve de séparation est dans mes notes. Sinon on doit prouver que
$X\times Z\to Z$ est fermée. On peut se ramener à $\P^n\times \A^m$.
La condition $y\in \A^m-p_2(Z)$ avec $Z=Z(I)\subset \A^m$ se réecrit
\[(U_0,\ldots,U_n)^N\subset I+\m_yk[U_0,\ldots, U_n,T_1,\dots T_m]\]
parce que un fermé du produit est donné par des 
\[G_i(U_0,\ldots, U_n,T_1,\ldots,T_m)=0\]
avec les $G_i$ homogènes en $\bar U$. Et en évaluant sur $\A^m$ en $y$,
on doit pas avoir de solution pour les $G_i(\bar U,y)=0$ dans $\Pr^n$.
D'où en quotientant par $\m_y$ le résultat. 
\begin{rem}
    Penser $A=k[U_0,\ldots, U_n,T_1, \ldots, T_m]\to A/\m_y$ qui 
    correspond à $\A^m\to \A^{n+1}\times \A^m$.
\end{rem}
On peut réecrire ducoup en considérant uniquement les éléments homogènes
de degré $N$ :
\[(U_0,\ldots, U_n)^N=B_N\subset I_N+\m_yB_N\]
avec $I_N=I\cap B_N$. D'où on obtient $P\in 1+\m_y\subset k[T_1,\ldots,
T_m]$ homogène tel que $PB_n=I_N$ dans 
$k[\bar U, T_1,\ldots, T_m]$.
\begin{rem}
    Là c'est 
    \[P(\bar T)(U_0,\ldots, U_n)^N\subset I_N\subset k[\bar U,\bar T]\]
\end{rem}
En particulier, $y\in D^+(P)$ implique en localisant en $P$ via
$p_2^{-1}(D(P))$ dans $\Pr^n\times \A^m$ et via $D(P)\cap Z$ dans 
$\A^m$ on obtient le résultat.
\begin{exo}
     Bien écrire les diagrammes et les déductions.
\end{exo}



\printbibliography
\end{document}


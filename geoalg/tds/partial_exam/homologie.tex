\documentclass[a4paper,12pt]{book}
\usepackage{amsmath,  amsthm,enumerate}
\usepackage{csquotes}
\usepackage[provide=*,french]{babel}
\usepackage[dvipsnames]{xcolor}
\usepackage{quiver, tikz}

%symbole caligraphique
\usepackage{mathrsfs}

%hyperliens
\usepackage{hyperref}

%pseudo-code
\usepackage{algorithm}
\usepackage{algpseudocode}

\usepackage{fancyhdr}

\pagestyle{fancy}
\addtolength{\headwidth}{\marginparsep}
\addtolength{\headwidth}{\marginparwidth}
\renewcommand{\chaptermark}[1]{\markboth{#1}{}}
\renewcommand{\sectionmark}[1]{\markright{\thesection\ #1}}
\fancyhf{}
\fancyfoot[C]{\thepage}
\fancyhead[LO]{\textit \leftmark}
\fancyhead[RE]{\textit \rightmark}
\renewcommand{\headrulewidth}{0pt} % and the line
\fancypagestyle{plain}{%
    \fancyhead{} % get rid of headers
}

%bibliographie
\usepackage[
backend=biber,
style=alphabetic,
sorting=ynt
]{biblatex}

\addbibresource{bib.bib}

\usepackage{appendix}
\renewcommand{\appendixpagename}{Annexe}

\definecolor{wgrey}{RGB}{148, 38, 55}

\setlength\parindent{24pt}

\newcommand{\Z}{\mathbb{Z}}
\newcommand{\R}{\mathbb{R}}
\newcommand{\rel}{\omathcal{R}}
\newcommand{\Q}{\mathbb{Q}}
\newcommand{\C}{\mathbb{C}}
\newcommand{\N}{\mathbb{N}}
\newcommand{\K}{\mathbb{K}}
\newcommand{\A}{\mathbb{A}}
\newcommand{\B}{\mathcal{B}}
\newcommand{\Or}{\mathcal{O}}
\newcommand{\F}{\mathscr F}
\newcommand{\Hom}{\textrm{Hom}}
\newcommand{\disc}{\textrm{disc}}
\newcommand{\Pic}{\textrm{Pic}}
\newcommand{\End}{\textrm{End}}
\newcommand{\Spec}{\textrm{Spec}}
\newcommand{\Supp}{\textrm{Supp}}
\renewcommand{\Im}{\textrm{Im}}
\newcommand{\m}{\mathfrak{m}}
\renewcommand{\P}{\mathbb{P}}


\newcommand{\cL}{\mathscr{L}}
\newcommand{\G}{\mathscr{G}}
\newcommand{\D}{\mathscr{D}}
\newcommand{\E}{\mathscr{E}}
\newcommand{\Po}{\mathscr{P}}
\renewcommand{\H}{\mathscr{H}}

\makeatletter
\newcommand{\colim@}[2]{%
  \vtop{\m@th\ialign{##\cr
    \hfil$#1\operator@font colim$\hfil\cr
    \noalign{\nointerlineskip\kern1.5\ex@}#2\cr
    \noalign{\nointerlineskip\kern-\ex@}\cr}}%
}
\newcommand{\colim}{%
  \mathop{\mathpalette\colim@{\rightarrowfill@\scriptscriptstyle}}\nmlimits@
}
\renewcommand{\varprojlim}{%
  \mathop{\mathpalette\varlim@{\leftarrowfill@\scriptscriptstyle}}\nmlimits@
}
\renewcommand{\varinjlim}{%
  \mathop{\mathpalette\varlim@{\rightarrowfill@\scriptscriptstyle}}\nmlimits@
}
\makeatother

\theoremstyle{plain}
\newtheorem{thm}[subsection]{Théoreme}
\newtheorem{lem}[subsection]{Lemme}
\newtheorem{prop}[subsection]{Proposition}
\newtheorem{cor}[subsection]{Corollaire}
\newtheorem{heur}{Heuristique}
\newtheorem{rem}{Remarque}
\newtheorem{note}{Note}

\theoremstyle{definition}
\newtheorem{conj}{Conjecture}
\newtheorem{prob}{Problème}
\newtheorem{quest}{Question}
\newtheorem{prot}{Protocole}
\newtheorem{algo}{Algorithme}
\newtheorem{defn}[subsection]{Définition}
\newtheorem{exmp}[subsection]{Exemples}
\newtheorem{exo}[subsection]{Exercices}
\newtheorem{ex}[subsection]{Exemple}
\newtheorem{exs}[subsection]{Exemples}

\theoremstyle{remark}

\definecolor{wgrey}{RGB}{148, 38, 55}
\definecolor{wgreen}{RGB}{100, 200,0} 
\hypersetup{
    colorlinks=true,
    linkcolor=wgreen,
    urlcolor=wgrey,
    filecolor=wgrey
}

\title{Examen partiel géométrie algébrique}
\date{}

\begin{document}
\maketitle
\tableofcontents

\section{Exercice 3)}
Pour l'exo $3)$ sur les codimensions :
\begin{enumerate}
    \item Cas irreductible :
\[codim(Z,X)=\sup\{\#\{Z = Z_0\subsetneq\ldots\subsetneq Z_n\}\}\]
    \item En général :
	\[\inf codim(W,X)\]
	pour les composantes $W$ de $Z$.
\end{enumerate}
\subsection{$\dim(X)\geq codim(Z,X)+dim(Z)$}
Clair.
\subsection{$codim(Z,X)=0$ ssi $Z$ contient une composante de $X$}
On peut supposer $X$ irréductible ? Si $Z$ contient une composante
c'est clair la codimension. Si $codim(Z,X)=0$, et si $Z$ ne contient
pas de composante, étant donné une chaine de longueur maximale 
$Z_0\subsetneq\ldots\subsetneq Z_n=Z$ on peut pas conclure direct
parce que la chaine peut-être de longueur maximale pour $Z$ ET maximale
pour $X$ sans être de longueur maximale. Maintenant $Z$ irreductible
implique $Z\subset W$ une composante de $X$. Et $codim(Z,W)=0$ implique
$Z\subset W$ n'est pas stricte, d'où $Z=W$.
\newline
Suffit de supposer $Z$ irréductible (on prends une de ses composantes
de dim maximale).

\subsection{$X$ affine, $Z$ fermé irreductible de $X$, 
$codim(Z,X)=1\leftrightarrow dim A(X)_{I(Z)}=1$}
On a une bijection entre $Spec(A(X)_{I(Z)}$ et $Spec(A(X))\cap D(I(Z))$
(Faut pas croiser $A-I(Z)$) préservant l'inclusion.
Le résultat tombe direct.
\subsection{$X$ affine intégre et $Z$ irred, $codim(Z,X)=1$}
Faut montrer que pour tout $f\in I(Z)-0$ il existe $h\in A(X)$ t.q
\[Z\cap D(h)=Z(f)\cap D(h)\ne \emptyset.\]
Pour un tel $f$ on a $Z\subset Z(f)$, si $Z(f)=Z$ on a $h=0$ suffit.
Sinon, on a $Z\subsetneq Z(f)$ et $codim(Z,Z(f))=0$ d'où
$Z$ est une composante de $Z(f)\cup \bigcup_i W_i$. On prends des
générateurs des $I(W_i)=(g_{ij},j)$ puis on pose $h=\prod g_{ij}$ et
le résultat via la dimension ?

\subsection{$X$ intègre, $Z$ irred fermé, $codim(Z,X)=dim(X)-dim(Z)$}
Par récurrence apparemment, pour codimension $0$ c'est clair, 
Pour la codimension $1$ c'est question $4)$? On peut prouver maintenant
que pour tout $Z\subsetneq Z_1$ irréd avec $codim(Z,Z_1)=1$ on a
\[codim(Z_1,X)\leq codim(Z,X)-1\]
alors $codim(Z,X)\geq dim(X)-dim(Z_1)+1=dim(X)-dim(Z)$. D'où le
résultat par double inégalité.
\newline
Le claim c'est que si 
\[Z\subsetneq Z_1\subsetneq Z_2\subsetneq\ldots\subsetneq Z_{codim(Z,X)}\]











\printbibliography
\end{document}


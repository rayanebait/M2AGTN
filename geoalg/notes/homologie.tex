\documentclass[a4paper,12pt]{book}
\usepackage{amsmath,  amsthm,enumerate}
\usepackage{csquotes}
\usepackage[provide=*,french]{babel}
\usepackage[dvipsnames]{xcolor}
\usepackage{quiver, tikz}

%symbole caligraphique
\usepackage{mathrsfs}

%hyperliens
\usepackage{hyperref}

%pseudo-code
\usepackage{algorithm}
\usepackage{algpseudocode}

\usepackage{fancyhdr}

\pagestyle{fancy}
\addtolength{\headwidth}{\marginparsep}
\addtolength{\headwidth}{\marginparwidth}
\renewcommand{\chaptermark}[1]{\markboth{#1}{}}
\renewcommand{\sectionmark}[1]{\markright{\thesection\ #1}}
\fancyhf{}
\fancyfoot[C]{\thepage}
\fancyhead[LO]{\textit \leftmark}
\fancyhead[RE]{\textit \rightmark}
\renewcommand{\headrulewidth}{0pt} % and the line
\fancypagestyle{plain}{%
    \fancyhead{} % get rid of headers
}

%bibliographie
\usepackage[
backend=biber,
style=alphabetic,
sorting=ynt
]{biblatex}

\addbibresource{bib.bib}

\usepackage{appendix}
\renewcommand{\appendixpagename}{Annexe}

\definecolor{wgrey}{RGB}{148, 38, 55}

\setlength\parindent{24pt}

\newcommand{\Z}{\mathbb{Z}}
\newcommand{\R}{\mathbb{R}}
\newcommand{\rel}{\omathcal{R}}
\newcommand{\Q}{\mathbb{Q}}
\newcommand{\C}{\mathbb{C}}
\newcommand{\N}{\mathbb{N}}
\newcommand{\K}{\mathbb{K}}
\newcommand{\A}{\mathbb{A}}
\newcommand{\B}{\mathcal{B}}
\newcommand{\Or}{\mathcal{O}}
\newcommand{\F}{\mathscr F}
\newcommand{\Hom}{\textrm{Hom}}
\newcommand{\disc}{\textrm{disc}}
\newcommand{\Pic}{\textrm{Pic}}
\newcommand{\End}{\textrm{End}}
\newcommand{\Spec}{\textrm{Spec}}
\newcommand{\Supp}{\textrm{Supp}}
\renewcommand{\Im}{\textrm{Im}}
\newcommand{\m}{\mathfrak{m}}
\renewcommand{\P}{\mathbb{P}}
\newcommand{\p}{\mathfrak{p}}


\newcommand{\cL}{\mathscr{L}}
\newcommand{\G}{\mathscr{G}}
\newcommand{\D}{\mathscr{D}}
\newcommand{\E}{\mathscr{E}}
\newcommand{\Po}{\mathscr{P}}
\renewcommand{\H}{\mathscr{H}}

\makeatletter
\newcommand{\colim@}[2]{%
  \vtop{\m@th\ialign{##\cr
    \hfil$#1\operator@font colim$\hfil\cr
    \noalign{\nointerlineskip\kern1.5\ex@}#2\cr
    \noalign{\nointerlineskip\kern-\ex@}\cr}}%
}
\newcommand{\colim}{%
  \mathop{\mathpalette\colim@{\rightarrowfill@\scriptscriptstyle}}\nmlimits@
}
\renewcommand{\varprojlim}{%
  \mathop{\mathpalette\varlim@{\leftarrowfill@\scriptscriptstyle}}\nmlimits@
}
\renewcommand{\varinjlim}{%
  \mathop{\mathpalette\varlim@{\rightarrowfill@\scriptscriptstyle}}\nmlimits@
}
\makeatother

\theoremstyle{plain}
\newtheorem{thm}[subsection]{Théoreme}
\newtheorem{lem}[subsection]{Lemme}
\newtheorem{prop}[subsection]{Proposition}
\newtheorem{cor}[subsection]{Corollaire}
\newtheorem{heur}{Heuristique}
\newtheorem{rem}{Remarque}
\newtheorem{note}{Note}

\theoremstyle{definition}
\newtheorem{conj}{Conjecture}
\newtheorem{prob}{Problème}
\newtheorem{quest}{Question}
\newtheorem{prot}{Protocole}
\newtheorem{algo}{Algorithme}
\newtheorem{defn}[subsection]{Définition}
\newtheorem{exmp}[subsection]{Exemples}
\newtheorem{exo}[subsection]{Exercices}
\newtheorem{ex}[subsection]{Exemple}
\newtheorem{exs}[subsection]{Exemples}

\theoremstyle{remark}

\definecolor{wgrey}{RGB}{148, 38, 55}
\definecolor{wgreen}{RGB}{100, 200,0} 
\hypersetup{
    colorlinks=true,
    linkcolor=wgreen,
    urlcolor=wgrey,
    filecolor=wgrey
}

\title{Géométrie algébrique}
\date{}

\begin{document}
\maketitle
\tableofcontents

J'vais juste prendre des notes. J'ai du mal à
noter en détail.

Je me disais que ce serait bien de noter les définitions
en détail pour voir exactement ce qu'il faut prouver à 
chaque fois.
\chapter{Spectre maximal}
En général, si $f\colon A \to B$ est un morphisme
$f^{-1}\m$ est pas forcément maximal. Par exemple
$\Z\to \Q$ et l'inverse de $(0)$. 
\begin{thm}
    Si $f\colon A\to B$ est un morphisme de $k$-algèbres
    de type fini alors $f^{-1}Spm(B)\subset Spm(A)$.
\end{thm}
\begin{proof}
    L'idée c'est que $k\to A/f^{-1}\m\to B/\m$ est finie
    par Noether. d'où $B/\m$ est entier sur $A/f^{-1}\m$
    donc on obtient un corps.
\end{proof}
\begin{rem}
    Attention au fait que $B/A$ entier et $A$ corps équivaut $B$ corps
    c'est vrai quand les deux sont \textbf{intègre}. Dans le thm
    $f^{-1}\m$ est premier donc c'est bon.
\end{rem}
\begin{prop}
    Si $f\colon A\to B$ est entier :
    \begin{enumerate}
	\item dim$(B)\leq$dim$(A)$.
	\item On a à nouveau $f_*\colon Spm(B)\to Spm(A)$!
	\item Si $f$ est injective, $f_*$ est surjective.
	\item Si $f$ est injective $dim(A)=dim(B)$.
    \end{enumerate}
\end{prop}
\begin{proof}
    Pour $1.$ faut montrer que $f_*$ d'une chaine est une chaine.
    Ça a l'air de suggérer que deux idéaux envoyés sur le même idéal
    doivent être de même dimension.


    On prends $\p_0\subset \p_1$ et on veut m.q 
    $f_*\p_0\nsubseteq f_*\p_1$. On quotiente par $\p_0$ et on doit
    juste montrer que $f_* \p_1\ne 0$ avec $A$, $B$ intègres. Si 
    on prends $b\in \p_1-\{0\}$. On a une relation $P(b)=0$ minimale.
    Et on peut conclure.(on aurait potentiellement probablement pu faire
    plus simple?)

    Pour $2.$ c'est l'argument habituel, les quotients sont intègres.

    Pour $3.$, $B\otimes_A A_{\p}\simeq B_{\p}:=(f(A-\p))^{-1}B$ en tant
    que $A$-module. Reste à montrer que l'idéal maximal de $B_{\p}$
    s'envoie sur $\p$, c'est clair car il est maximal par $2.$. En gros
% https://q.uiver.app/#q=WzAsOCxbMSwwLCJBX3tcXHB9Il0sWzIsMCwiU157LTF9QiJdLFsyLDEsIkIiXSxbMSwxLCJBIl0sWzMsMSwiXFxwIl0sWzMsMCwiXFxtIl0sWzAsMSwiXFxwIl0sWzAsMCwiXFxwIEFfe1xccH0iXSxbMCwxXSxbMywwXSxbMywyXSxbMiwxXSxbNCw1XV0=
\[\begin{tikzcd}
	{\p A_{\p}} & {A_{\p}} & {S^{-1}B} & \m \\
	\p & A & B & \p
	\arrow[from=1-2, to=1-3]
	\arrow[from=2-1, to=1-1]
	\arrow[from=2-2, to=1-2]
	\arrow[from=2-2, to=2-3]
	\arrow[from=2-3, to=1-3]
	\arrow[from=2-4, to=1-4]
\end{tikzcd}\]
    faut juste prouver que $(f(A-\p))^{-1}B$ est non nul mdr. On utilise
    le produit tensoriel pour montrer que c'est entier c'est tout.

    La $4.$ est claire.
\end{proof}

Petite preuve que la fibre est finie maintenant : Faut utiliser
que $(f_*)^{-1}(\p)=(f_*)^{-1}(\cap_{\p\subset\m}\m)=\cap (f_*)^{-1}\m =
\cap \m_y=\m_y$ pour $Z(\p)$ une composante de $f^{-1}y$. Là on a 
\[(f_*)^{-1}\colon Spm(\Or_X(X))\to Spm(\Or_Y(Y))\]

\begin{cor}[Dimension d'hypersurface]
    Si $f\in k[T_1,\ldots,T_n]-k$ alors
    \[dim k[T_1,\ldots,T_n]/(f)=dim k[T_1,\ldots, T_{n-1}\]
\end{cor}
\begin{proof}
    Il dit que dans la preuve de Noether, $f=T_n^r+G$ à isomorphisme
    près avec $deg_{T_n}G\leq r-1$. D'où la flèche 

    \[k[T_1,\ldots,T_{n-1}]\to k[T_1,\ldots, T_n]/(f)\]
    est finie et y ont la même dimension. Pour prouver que 
    $dim(A^n)=n$, on peut le faire par induction, quotienter par $F\in$
    le premier idéal. 
\end{proof}

On peut à nouveau redéfinir les fermés de $X$ via $f^{-1}(0)$ pour 
$f\in\Or_X(X)$.
\section{k-algèbres de types finis}
Il a prouvé que le degré de transcendance est bien défini. Et que
\begin{prop}
    Une variété est irréductible ssi les $\Or_X(U)$ sont intègres.
\end{prop}
\begin{proof}
    L'idée c'est que irred équivaut à il existe de ouverts qui se 
    croisent pas (à voir). D'où 
    $\Or_X(U\cup V)\simeq\Or_X(U)\oplus \Or_X(V)$.
\end{proof}
\begin{exo}
    Faire toutes les preuves de ce cours.
\end{exo}

Chapitre du Gortz sur les schémas intègres.


\printbibliography
\end{document}


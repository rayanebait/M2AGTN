\documentclass[a4paper,12pt]{book}
\usepackage{amsmath,  amsthm,enumerate}
\usepackage{csquotes}
\usepackage[provide=*,french]{babel}
\usepackage[dvipsnames]{xcolor}
\usepackage{quiver, tikz}

%symbole caligraphique
\usepackage{mathrsfs}

%hyperliens
\usepackage{hyperref}

%pseudo-code
\usepackage{algorithm}
\usepackage{algpseudocode}

\usepackage{fancyhdr}

\pagestyle{fancy}
\addtolength{\headwidth}{\marginparsep}
\addtolength{\headwidth}{\marginparwidth}
\renewcommand{\chaptermark}[1]{\markboth{#1}{}}
\renewcommand{\sectionmark}[1]{\markright{\thesection\ #1}}
\fancyhf{}
\fancyfoot[C]{\thepage}
\fancyhead[LO]{\textit \leftmark}
\fancyhead[RE]{\textit \rightmark}
\renewcommand{\headrulewidth}{0pt} % and the line
\fancypagestyle{plain}{%
    \fancyhead{} % get rid of headers
}

%bibliographie
\usepackage[
backend=biber,
style=alphabetic,
sorting=ynt
]{biblatex}

\addbibresource{bib.bib}

\usepackage{appendix}
\renewcommand{\appendixpagename}{Annexe}

\definecolor{wgrey}{RGB}{148, 38, 55}

\setlength\parindent{24pt}

\newcommand{\Z}{\mathbb{Z}}
\newcommand{\R}{\mathbb{R}}
\newcommand{\rel}{\omathcal{R}}
\newcommand{\Q}{\mathbb{Q}}
\newcommand{\C}{\mathbb{C}}
\newcommand{\N}{\mathbb{N}}
\newcommand{\K}{\mathbb{K}}
\newcommand{\A}{\mathbb{A}}
\newcommand{\B}{\mathcal{B}}
\newcommand{\Or}{\mathcal{O}}
\newcommand{\F}{\mathscr F}
\newcommand{\Hom}{\textrm{Hom}}
\newcommand{\disc}{\textrm{disc}}
\newcommand{\Pic}{\textrm{Pic}}
\newcommand{\End}{\textrm{End}}
\newcommand{\Spec}{\textrm{Spec}}
\newcommand{\Supp}{\textrm{Supp}}
\renewcommand{\Im}{\textrm{Im}}
\newcommand{\m}{\mathfrak{m}}
\renewcommand{\P}{\mathbb{P}}


\newcommand{\cL}{\mathscr{L}}
\newcommand{\G}{\mathscr{G}}
\newcommand{\D}{\mathscr{D}}
\newcommand{\E}{\mathscr{E}}
\newcommand{\Po}{\mathscr{P}}
\renewcommand{\H}{\mathscr{H}}

\makeatletter
\newcommand{\colim@}[2]{%
  \vtop{\m@th\ialign{##\cr
    \hfil$#1\operator@font colim$\hfil\cr
    \noalign{\nointerlineskip\kern1.5\ex@}#2\cr
    \noalign{\nointerlineskip\kern-\ex@}\cr}}%
}
\newcommand{\colim}{%
  \mathop{\mathpalette\colim@{\rightarrowfill@\scriptscriptstyle}}\nmlimits@
}
\renewcommand{\varprojlim}{%
  \mathop{\mathpalette\varlim@{\leftarrowfill@\scriptscriptstyle}}\nmlimits@
}
\renewcommand{\varinjlim}{%
  \mathop{\mathpalette\varlim@{\rightarrowfill@\scriptscriptstyle}}\nmlimits@
}
\makeatother

\theoremstyle{plain}
\newtheorem{thm}[subsection]{Théoreme}
\newtheorem{lem}[subsection]{Lemme}
\newtheorem{prop}[subsection]{Proposition}
\newtheorem{cor}[subsection]{Corollaire}
\newtheorem{heur}{Heuristique}
\newtheorem{rem}{Remarque}
\newtheorem{note}{Note}

\theoremstyle{definition}
\newtheorem{conj}{Conjecture}
\newtheorem{prob}{Problème}
\newtheorem{quest}{Question}
\newtheorem{prot}{Protocole}
\newtheorem{algo}{Algorithme}
\newtheorem{defn}[subsection]{Définition}
\newtheorem{exmp}[subsection]{Exemples}
\newtheorem{exo}[subsection]{Exercices}
\newtheorem{ex}[subsection]{Exemple}
\newtheorem{exs}[subsection]{Exemples}

\theoremstyle{remark}

\definecolor{wgrey}{RGB}{148, 38, 55}
\definecolor{wgreen}{RGB}{100, 200,0} 
\hypersetup{
    colorlinks=true,
    linkcolor=wgreen,
    urlcolor=wgrey,
    filecolor=wgrey
}

\title{Géométrie algébrique}
\date{}

\begin{document}
\maketitle
\tableofcontents
\chapter*{Introduction}
On est censés prouver Riemann-Roch.
\chapter{Variétés algébriques}
\section{Nullstellensatz}
Pas oublier de rechopper mon carnet. Y'a les preuves complètes.

\begin{thm}
    Y'a une correspondance entre points fermés de
    $\A^n(k)$ et idéaux maximaux dans $Spm(k[T_1,\ldots,T_n])$.
\end{thm}
\begin{cor}
    Si $A$ est une $k$-algèbre de t.f. et $\m$ un idéal maximal
    alors $A/\m$ est une extension finie de $k$.
\end{cor}

\begin{lem}
    Si $A$ est une $k$-algèbre de t.f. alors 
    $\sqrt I = \cap_{\m \in Spm(A), I\subset \m}\m$
\end{lem}
\begin{lem}
    Si $k$ est algébriquement clos, c'est un homéomorphisme (entre
    $\A^n(k)$ et $Spm(k[T_1,\ldots,T_n])$.
\end{lem}
\begin{proof}
    On prends le morphisme quotient, c'est l'évaluation et le noyau est
    de la forme $(T_i-t_i)_i$.
\end{proof}
\begin{thm}[Nullstellensatz]
    Si $k=\bar k$ alors $I(Z(J))=\sqrt J$.
\end{thm}
\begin{proof}
    On a 
    \begin{align*}
	I(Z(J))&=I(\bigcup_{x\in Z(J)}{x})\\
	       &=\bigcap_{x\in Z(J)}I(\{x\})\\
	       &=\bigcap_{x\in Z(J)} \m_x\\
	       &=\bigcap_{\m\in Spm(A), J\subset M} \m
    \end{align*}
    et la dernière est $\sqrt J$ par le lemme. (omg, revoir la preuve
    dans Atiyaah)
\end{proof}
\begin{rem}[!]
    L'endroit où on utilise le weak nullstellensatz on a besoin de $k$
    algébriquement clos. La dernière qui vient du lemme y'a pas besoin.
    Autrement dit, on peut utiliser Spm pour faire de la géométrie 
    algébrique sur un corps non algébriquement clos.
\end{rem}


\begin{defn}
    $A(Z)=k[T_1,\ldots,T_n]/I(Z)$
\end{defn}

Pour $f\in A(z)$ et $\tilde f$ t.q $p(\tilde f)= f$ pour 
$p\colon k[T_1, \ldots, T_n]\to A(Z)$. Pour $z\in k^n$ on peut toujours 
déf $f(z):=\tilde f(z)$. En particulier, on peut déf
\begin{defn}
    $D(f)=\{s\in Z : f(z)\ne 0\}=D(\tilde f)\cap Z$. Avec 
    $D(\tilde f)=\A^n(k)-Z(\tilde f)$.
\end{defn}
\begin{rem}
    Comme d'hab juste il définit pour des fonctions a priori par déf
    sur $\A^n(k)$.
\end{rem}
\begin{rem}[C'est super chiant]
    Faut faire gaffe ducoup en fonction de la fonction que j'utilise
    ou de son lift pour les inclusions.
\end{rem}
\begin{cor}
    Si $f,g\in A(Z)$ et $Z\subset \A^n(k)$. On a 
    \begin{itemize}
	\item Pour $J_1,J_2\leq A(Z)$ : 
	    $Z(J_1)\subset Z(J_2)\leftrightarrow J_2\subset \sqrt J_1$.
	\item $D(f)\subset D(g)\leftrightarrow \exists h\in \A(Z)$ t.q.
	    $f^n=gh$.
	\item Les ouverts principaux forment une base de la topologie.
    \end{itemize}
\end{cor}
\begin{proof}
    Pour le premier point si $Z(J_1)\subset Z(J_2)$ alors faut lift 
    dans $k[T_1, \ldots, T_n]$ avant d'appliquer le nullstellensatz.
    Pour le deuxième, c'est clair.
    Pour le troisième, sur $\A^n(k)$ on prend $f\in I(Z)$ ,où 
    $U=\A^n(k)-Z$, t.q $f(x)\ne 0$ (possible car $x\notin Z$.
\end{proof}
\begin{prop}
    Soit $Z$ un ensemble algébrique affine. Alors $Z$ est irréductible
    ssi $I(Z)$ est premier. Si $k=\bar k$, $I\leq K[T_1,\ldots,T_n]$
    alors $Z(J)$ est irréductible ssi $\sqrt J$ est premier.
\end{prop}
\begin{proof}
    Avec les nouvelles notations c'est direct, avec les anciennes 
    si $Z(J)$ est irreductible $Z(f)\cup Z(g)=Z(J)$ implique 
    $Z(J)\subset Z(f)$ ou $Z(J)\subset Z(g)$. 
\end{proof}

\begin{lem}
    Soit $A$ un anneau noetherien, alors les idéaux radicaux sont 
    des intersections finies d'idéaux premiers.
\end{lem}
\begin{proof}
    On regarde l'ensemble des idéaux qui sont pas des intersections
    d'idéaux premiers. Comme $A$ est noethérien y'a un élement maximal
    $I$ qui n'est pas premier. Soit $a,b\in A-I$ t.q. $ab\in I$. On 
    considère $I_a\sqrt{I+aA}$ et $I_b=\sqrt{I+bA}$. Ils sont plus gros 
    que $I$ donc intersections d'idéaux premiers. En particulier
    on prouve facilement que $I=I_a\cap I_b$ ($I$ est radical).
\end{proof}
\begin{prop}
    Si $k=\bar k$, on a une décomposition unique des ensembles 
    algébriques en variétés irréductibles non contenues les unes dans 
    les autres.
\end{prop}
\begin{proof}
    $I(Z)=\bigcap_{i=1}^m \mathfrak b_i$. On retire les $\mathfrak b_i$
    contenus dans les autres.
\end{proof}
\section{Espace projectif}
On considère $k[T_0,\ldots, T_n]=\bigoplus_{d\geq 0} S_d$. 
\begin{lem}
    Sur les corps infinis, $f\in S_d$ ssi 
    $\lambda^df(x_i,i)=f(\lambda x_i,i)$.
\end{lem}
\begin{defn}
    Un idéal est homogène ssi dès que $f=f_1+\ldots+f_n\in I$ alors
    $f_i\in I$. C'est équivalent à être généré par des éléments 
    homogènes, i.e. $I=\bigoplus S_d\cap I$.
\end{defn}
\begin{rem}
    Comme en géo diff regarder ce qu'il se passe quand on regarde des 
    polynômes homogènes dans $\A^{n+1}$ et qu'on les pousse (homéo?).
\end{rem}
\begin{defn}
    Pour $I$ un idéal homogène de $k[T_0,\ldots, T_n]$, on définit 
    $Z_+(I)=\{P\in \P^n(k) : f(P)=0~\forall f\in I~\textrm{f homogène}\}$
    où autrement on lift $P$ et on prends $f$ quelconque. Si $k$ est
    infini et $Z\subset \P^n(k)$, on définit $I_+(Z)=I(\pi^{-1}(Z))$.
\end{defn}

\begin{thm}[Nullsellensatz projectif]
    On suppose $k=\bar k$ et $J$ homogène. On a 
    \begin{itemize}
	\item $Z_+(J)=\emptyset $ ssi $(T_0,\ldots,T_n)\subset J$.
	\item Si $Z_+(J)\ne \emptyset$ alors $I_+(J_+(J))=\sqrt J$.
    \end{itemize}
\end{thm}
\begin{proof}
    Si $Z_+(J)=\emptyset$ on lift à $\A^{n+1}-0$ pour voir que 
    $Z(J)\subset \{0\}=(T_0,\ldots, T_n)$. Sinon $I_+(Z_+(J))=I(\pi^{-1}
    (Z_+(J)))=I(\pi^{-1}(Z_+(J))\cup\{0\})=\sqrt J$.
\end{proof}
\section{foncions régulières}
Revoir que la topologie de Zariski c'est la plus petite topologie que
rend continue les polynômes. 
\begin{defn}[Fonction régulière]
    On décrit pour $Z\subset \A^n(k)$ l'anneau $\Or_Z(U)$. On prend
    les fonctions qui sont localement des fractions rationnelles.
\end{defn}
\begin{note}
    Trouver exactement où on peut écrire des polynômes, les ouverts sont
    quasi-compacts(!).
\end{note}
\begin{lem}
    $\Or_Z$ est un faisceau pour les restrictions naturelles.
\end{lem}
\begin{proof} C'est évident avec la déf mdr. \end{proof}

\begin{prop}
    Soit $Z\subset \A^n(k)$. 
    \begin{itemize}
	\item Les fonctions régulières sont continues.
	\item Pour tout $f\in \A(Z)$, la flèche $\A(Z)\to \Or_Z(D(f))$
	    passe au quotient en un isom $\A(Z)_f \simeq \Or_Z(D(f))$.
	\item $\A(Z)\simeq \Or_Z(Z)$.
    \end{itemize}
\end{prop}
\begin{proof}
    Pour le premier point l'idée c'est que localement on peut se
    mettre sur un ouvert tel que $f|_U(U)=\{pt\}$. Le deuxième point
    c'est la surjectivité qu'y faut voir. Le troisième point c'est
    le plus cool, c'est l'idée que on commence par décomposer 
    $Z$ en une union finie $\bigcup_i D(f_i)$ où on est une fraction
    rationnelle. Ensuite, on obtient $(gf_i-h_i)|_{D(f_i)}=0$, faut 
    relever puis dérouler avec le fait que $1\in(f_i,i)$ quelque part.
\end{proof}
\begin{rem}
    Si on prend $\A^2\backslash (0,0)$, il a les mêmes sections globales
    que $\A^2$. Ça prouve que cet ouvert est pas affine.
\end{rem}
\section{Morphismes d'ensembles algébriques}
Dans les ensembles algébriques on peut directement prendre des 
fonctions polynomiales ! C'est la preuve d'avant. 

\begin{thm}
    On a une équivalence de catégories entre les $k$-algèbres
    de type finies réduites et la $k$-variétés.
\end{thm}
\begin{note}
    Revoir vite fait la construction.
\end{note}
\section{Espaces annelés}
\begin{defn}
    Un espace annelé est un espace topologique $X$ muni d'un faisceau
    de $k$-algèbre pour nous.
\end{defn}

\begin{defn}
    Un morphisme d'espaces annelés 
    \[(|X|,\Or_X)\to (|Y|, \Or_Y)\] 
    est un couple $(|f|, f^{\#})$. Où 
    $|f|$ est un morphisme d'espaces topologiques et
    $f^{\#}\colon O_Y\to |f|_*\Or_X$
    un morphisme tels que les flèches induites sur 
    les fibres sont des morphismes d'anneaux 
    locaux.
\end{defn}
\begin{note}
    Le faisceau $|f|_*\Or_X$ est le pullback classique. Si
    $y=|f|(x)$, comme d'habitude on a 
    \[f^\#\colon O_{Y,y}\to (|f|_*\Or_X)_y\to \Or_{X,x}\]
    Enfin en fait comme c'est localement annelé apparemment on 
    peut montrer que $f^\#$ c'est automatiquement le pullback de 
    fonctions.
\end{note}
\begin{thm}
    Le couple $(Z,\Or_Z)$ est un espace annelé.
\end{thm}
\begin{proof}
    Les fibres $\Or_{Z,z}$ sont les $\A(Z)_{\m_z}$.
\end{proof}

de notre équivalence de base 
\[\{\textrm{ensembles algébriques affines}\}\to \{k\textrm{-algèbre réduite de type fini}\}\]

on plonge les ensembles algébriques dans les espaces localements
annelés. En fait c'est pleinement fidèle, on a pas un nouvel objet.
\begin{prop}
    On a une bijection 
    \[\Hom_{algSets}(X,Y)\to \Hom_{LocRingedSpace}((X,\Or_X),(Y,\Or_Y))\]
\end{prop}
\begin{proof}
    Étant donné $f\colon X\to Y$, on définit 
    $f^{\#}(U)\colon \Or_Y(U)\to \Or_X(f^{-1}U)$ par $s\mapsto s\circ f$.
    Et à $f$ on associe $(f,f^{\#}$. Maintenant si on a un $(f,f^{\#})$
    un morphisme d'espaces localement annelés quelconque, faut montrer
    que $f$ est un morphisme d'ensemble 
    algébriques et que $f^{\#}$ est bien le pullback habituel. Faut
    se rappeler que $f^{\#}(\m_y)\subset \m_x$ tel que $f(x)=y$. En 
    particulier, le grand carré de 
    % https://q.uiver.app/#q=WzAsOCxbMCwxLCJmXntcXCN9XFxjb2xvblxcbWF0aGNhbCBPX1koVSkiXSxbMiwxLCJcXG1hdGhjYWwgT19YKGZeey0xfVUpIl0sWzAsMiwiT197WSxmKHgpfSJdLFsyLDIsIlxcbWF0aGNhbCBPX3tYLHh9Il0sWzAsMywiaz1PX3tZLGYoeCl9L1xcbWF0aGZyYWsgbV97Zih4KX0iXSxbMiwzLCJPX3tYLHh9L1xcbWF0aGZyYWsgbV97eH09ayJdLFswLDAsInMiXSxbMiwwLCJmXntcXCN9cyJdLFswLDFdLFswLDJdLFsyLDNdLFsxLDNdLFsyLDRdLFs0LDVdLFszLDVdLFs2LDcsIiIsMCx7InN0eWxlIjp7InRhaWwiOnsibmFtZSI6Im1hcHMgdG8ifX19XV0=
\[\begin{tikzcd}[ampersand replacement=\&,cramped]
	s \&\& {f^{\#}s} \\
	{f^{\#}\colon\mathcal O_Y(U)} \&\& {\mathcal O_X(f^{-1}U)} \\
	{O_{Y,f(x)}} \&\& {\mathcal O_{X,x}} \\
	{k=O_{Y,f(x)}/\mathfrak m_{f(x)}} \&\& {O_{X,x}/\mathfrak m_{x}=k}
	\arrow[maps to, from=1-1, to=1-3]
	\arrow[from=2-1, to=2-3]
	\arrow[from=2-1, to=3-1]
	\arrow[from=2-3, to=3-3]
	\arrow[from=3-1, to=3-3]
	\arrow[from=3-1, to=4-1]
	\arrow[from=3-3, to=4-3]
	\arrow[from=4-1, to=4-3]
\end{tikzcd}\]
    commute, et la flèche $\Or_Y(U)\to k$ est l'évaluation, la flèche
    $k\to k$ est l'identité ($1\mapsto 1$), i.e. $f^{\#}(s)(x)=s(f(x))$.
    On a montré que $f^{\#}$ est le pullback habituel. Pour montrer
    que c'est un morphisme, on peut regarder 
    $\tilde f\colon X\to Y\to \A^n_k$. On obtient 
    \[\tilde f(\A^n(k))\colon k[T_1,\colon, T_n]\to \Or_X(X)\]
    qui à $T_1$ associe $f_1$. En particulier, c'est $T_1\circ f$ par
    le point d'avant. De sorte que $\tilde f$ est défini par
    des polynômes et donc un morphisme.
\end{proof}

\section{Recollements}
\section{Sous-variétés}
On appelle variété des unions finies de variétés affines avec le 
faisceau structurant.

\begin{defn}[Sous-variété]
    Pour un fermé dans $Z\subset X$ une variété algébrique. On définit,
    \[\Or_Z(V)=\{f\colon V\to k|\forall z\in V \exists g\in \Or_X(U),
    U\cap Z\subset V,g|_{U\cap Z}=f|_{U\cap Z}\}\]
    autrement dit c'est juste $i^{-1}\Or_X$ pour $i\colon Z\subset X$.
\end{defn}

\begin{lem}
    Soit $X$ une variété algébrique et $Z\subset X$ fermé. Alors
    $Z$ est une variété algébrique. Si $X$ est affine alors $Z$ aussi.
\end{lem}
\begin{proof}
    On doit montrer que $Z$ est un recouvers d'affine. Suffit de 
    montrer que $Z\cap U$ est affine si $U$ est affine. Suffit de
    le montrer dans $X=\A^n$ et là $\Or_Z$ c'est littéralement
    le faisceau de fonction régulière donc on a fini. (Il définit une
    variété affine comme un fermé topologique de $\A^n$ plus faisceau)
\end{proof}


\section{Variétés projectives}
Dans $\P^n(k)$ on définit sur $X=Z_+(I)$, \[\Or_X(U)=\{f\colon U\to k,
\textrm{Localement une fraction homogène}\}\]
Si on définit la localisation homogène en $P$ homogène, avec $B=k[T_1,
\ldots,T_n]/I$, via $B_{(P)}$ les fractions de degrés $0$. Plus
rigoureusement, via $k[T_1,\ldots, T_n]_{(P)}/I_{(P)}$ avec $I_{(P)}=
\{Q/P^n|Q\in I\}$.
\begin{prop}
    Si $X=Z_+(I)\subset \P^n(k)$ et $U=X\cap D_+(p)$ ($\bar P=p$). On a
    $B_{(p)}\simeq \Or_X(U)$.
\end{prop}

\begin{prop}
    Si $P$ est non constant homogène alors $D_+(P)$ est une variété 
    affine. Si $P\in \Or_Z(Z)=B$ de degré $\geq 1$ pareil.
\end{prop}

\begin{defn}
    Les variétés quasi-projectives sont les localement fermées dans
    $\P^n(k)$.
\end{defn}
\chapter{Dimension}
\begin{defn}
    On prend la dimension de Krull avec le sup des chaines de fermés
    irréductibles.
\end{defn}
\chapter{Produits fibrés}
Y'a des choses à rattraper, comme l'immersion fermée.
\section{Séparation et propreté}
\begin{defn}
    Une variété algébrique sur $k$ est séparée si l'image de 
    \[\delta=(id,id)\colon X\to X\times_k X\]
    est fermée.
\end{defn}
\begin{ex}
    Les variétés algébriques affine sont séparée, parce que 
    $A(X)\otimes A(X)$ est surjective, en effet $im(\delta)=Z(ker \delta_*)$.
\end{ex}

Comme la séparation imite le fait d'être Hausdorff, en ajoutant la
quasi-compacité on obtient la "compacité"? 

\begin{defn}
    Une variété algébrique $X$ sur $k$ est propre/complète si elle est
    séparée et pour toute variété $Z$ :
    \[X\times Z\to Z\]
    est fermée (on dit qu'elle est universellement fermée).
\end{defn}
\begin{ex}
    $\A^1_k$ n'est pas propre mais pourtant est séparée et 
    quasi-compacte. On regarde $\A^2\simeq \A^1\times \A^1\to \A^1$,
    et dans $\A^2$ on regarde $Z(XY-1)$, d'image $\A^1-0$! Le problème
    vient du point à l'infini qu'il manque.
\end{ex}
\begin{rem}
    Les fermées de variétés séparée (resp. propres) sont séparées (resp. 
    propres). Et les ouverts de variétés séparées sont séparées.
\end{rem}
\begin{prop}
    Les variétés projectives sont propres.
\end{prop}
\begin{rem}
    Voir la preuve dans les notes sur Nakayama, sinon dans les notes sur
    Shafarevich.
\end{rem}
\begin{proof}
    Il suffit de le prouver pour $\P^n_k$. On montre que c'est séparé,
    on peut restreindre $\P^n_k\to \P^n_k\times \P^n_k$ à 
    $D(T_i)\cap D(T_j)\to D(T_i)\times D(T_j)$ qui a une diagonale
    fermée car affine (le fait d'être une immersion fermée est
    local, check). Soit $Y$ une variété algébrique, on doit prouver
    que $\P^n_k\times Y\to Y$ est fermée, on peut le montrer localement
    puis supposer $Y$ affine et même $=\A^n$. En fait c'est la preuve
    de Shafarevich. Y'a un twist, il dit pour $y\in \A^n-p_2(Z)$
    on a 
    \[(T_0,\ldots,T_n)^N\subset I + \m_{y}k[T_0,\ldots,T_n,S_1,\ldots,
    S_m]\]
    puis $B_N\subset I_N+\m_y B_N$ avec 
    $B_N=k[T_0,\ldots,T_n]_N$ le module des polynomes de degré $N$.
    D'où par Nakayama, il existe $P\notin \m_y$ homogène dans $I$ t.q.
    $PB_N\subset I_N$ et $y\in D^+(P)$ (on veut les monomes dans $I$
    et $P$ est en $Y$ d'où dans $D^+(P)$, $P(y)\ne 0$ et les monomes
    sont dans $I_y$, (penser au quotient pas $\m_y$).
\end{proof}
\begin{defn}
    On définit le graphe comme l'image inverse de la diagonale $\delta_Y$
    par $(f,id_Y)\colon X\times Y\to Y\times Y$.
\end{defn}

\begin{rem}
    Si $Y$ est séparée, $\Gamma_f$ est fermé dans $X\times Y$.
    En effet, $(id_X, f)\colon X\to X\times Y$ est  un isomorphisme
    vers $\Gamma_f$.
\end{rem}

\begin{lem}
    Soit $f\colon X\to Y$ un morphisme de variétés algébriques avec
    $X$ propre et $Y$ séparée. Alors $f$ est fermée.
\end{lem}
\begin{proof}
    On regarde $X\to X\times Y\to Y$.
\end{proof}

\begin{prop}
    Soit $X$ une variété algébrique connexe et propre. Alors
    $\Or_X(X)=k$. 
\end{prop}
\begin{proof}
    Étant donné $f\colon X\to \A^1$, l'image est connexe et fermée
    dans $\P^1$ et incluse dans $\A^1$ donc est un point.
\end{proof}
\begin{cor}
    Une variété affine connexe est propre ssi $X=\A^0_k$.
\end{cor}

\chapter{Espace tangent}
La première définition à partir d'équations/d'un plongement affine.
\begin{defn}
    Soit $X=Z(I)$ une variété algébrique et $P=(a_1,\ldots, a_n)\in X$.
    On définit \[T_{X,P}:=\{(x_1,\ldots, x_n)\in \A^n_k| \sum_i T_i
    \partial_i F(P) =0,~\forall F\in I\}\]
    c'est un espace vectoriel.
\end{defn}
\begin{rem}
    Penser $T_i=X_i-a_i$, pour que ce soit un espace vectoriel.
\end{rem}
\begin{ex}
    Pour $Z(T_1-T_2^2)=Z$ et $P=(1,1)$ on a $T_PZ=\{(t_1,t_2)|
    t_1-2t_2=0\}$.
\end{ex}
Soit maintenant $E= k^n$ et $P\in \A^n(k)$. On regarde
\[D_p\colon k[T_1,\ldots, T_n]\to E^\wedge\]
donné par $F\mapsto D_p(F)=\sum T_i\delta_i F(P)$. On remarque que
$D_p(\m_P^2)=0$ car si $f,g\in \m_P$ alors $\partial_i(fg)(P)=f(P)
\partial_i g(P)+g(P)\partial_i f(P)=0$. D'où ça passe au quotient
en \[\m_P/\m_P^2\to E^\wedge\]
maintenant c'est des $k$-espaces vectoriels de dimensions $n$ d'où 
il suffit de montrer la surjectivité ce qui est clair.

\begin{prop}
    Soit $X$ une variété affine et $P\in X$. Alors 
    \[(\m_P)^\wedge\simeq T_{X,P}\]
\end{prop}
\begin{proof}
    Par définition $T_{X,P}$ est l'orthogonal de $D_P(I)$ ou l'ensemble
    des $v\in E$ tels que pour tout $\phi\in D_P(I)$, $\phi(v)=0$.
    On note $\m$ l'idéal de $k[T_1,\ldots, T_n]$ tel que $\m_P=\m/I$.
    Alors $\m_P/\m_P^2$ rentre dans un diagramme 

% https://q.uiver.app/#q=WzAsMTAsWzEsMCwiSS9JXFxjYXBcXG1eMiJdLFsyLDAsIlxcbS9cXG1eMiJdLFszLDAsIlxcbV9QL1xcbV9QXjIiXSxbNCwwLCIwIl0sWzAsMCwiMCJdLFsxLDEsIkRfUEkiXSxbMywxLCJcXG1fUC9cXG1fUF4yIl0sWzQsMSwiMCJdLFswLDEsIjAiXSxbMiwxLCJFXlxcd2VkZ2UiXSxbMCwxXSxbMSwyXSxbMiwzXSxbNCwwXSxbMCw1LCIiLDEseyJzdHlsZSI6eyJ0YWlsIjp7Im5hbWUiOiJob29rIiwic2lkZSI6InRvcCJ9fX1dLFsyLDYsIj0iLDEseyJzdHlsZSI6eyJib2R5Ijp7Im5hbWUiOiJub25lIn0sImhlYWQiOnsibmFtZSI6Im5vbmUifX19XSxbNiw3XSxbOCw1XSxbNSw5XSxbOSw2XSxbMSw5LCJEX1AiLDEseyJzdHlsZSI6eyJ0YWlsIjp7Im5hbWUiOiJob29rIiwic2lkZSI6InRvcCJ9fX1dXQ==
\[\begin{tikzcd}
	0 & {I/I\cap\m^2} & {\m/\m^2} & {\m_P/\m_P^2} & 0 \\
	0 & {D_PI} & {E^\wedge} & {\m_P/\m_P^2} & 0
	\arrow[from=1-1, to=1-2]
	\arrow[from=1-2, to=1-3]
	\arrow[hook, from=1-2, to=2-2]
	\arrow[from=1-3, to=1-4]
	\arrow["{D_P}"{description}, hook, from=1-3, to=2-3]
	\arrow[from=1-4, to=1-5]
	\arrow["{=}"{description}, draw=none, from=1-4, to=2-4]
	\arrow[from=2-1, to=2-2]
	\arrow[from=2-2, to=2-3]
	\arrow[from=2-3, to=2-4]
	\arrow[from=2-4, to=2-5]
\end{tikzcd}\]
    la flèche du bas montre bien le résultat en passant aux duaux
    (ça inverse les flèches).
\end{proof}

\begin{lem}
    Soit $A$ un anneau et $\m\leq A$ maximal. Y'a un isomorphisme
    $\m/\m^2\simeq \m A_\m/\m^2A_\m$.
\end{lem}
\begin{proof}
    Soit $x\in \m$ t.q $x\in \m^2 A_\m$. Alors il existe 
    $s\in A-\m$ tel que $s.x\in \m^2A$. En plus il existe $t\in A$
    t.q $st-1\in \m$ car $A/\m$ est un corps. Ensuite
    \[(st-1)x\in \m^2\]
    mais $stx\in \m^2$ car $sx\in \m^2$ d'où $x\in \m^2$ et la flèche
    est injective.

    Si on prends $x\in \m A_\m$ il existe $s\in A-\m$ tq $sx\in \m$.
    On prends $t$ comme avant. Alors $tsx=x \mod \m^2A_\m$. D'où
    $tsx\mapsto x$.
\end{proof}
\begin{rem}
    D'où l'espace tangent provient du stalk du faisceau.
\end{rem}
\begin{defn}
    Soit $X$ une variété algébrique. L'espace tangent de Zariski en 
    $P$ est le $k$-espace vectoriel 
    \[(\m\Or_{X,P}/\m^2\Or_{X,P})\]
\end{defn}
\section{Variétés lisses}
\begin{defn}
    On note $dim_P X=min\{dim U| P\in U\}$ pour les ouverts de $X$.
    (Oui la déf marche)
\end{defn}

\begin{rem}
    C'est aussi le maximum des dimensions des composantes irréductibles
    qui passent par $P$.
\end{rem}
\begin{prop}
    Soit $X$ une variété algébrique, alors pour tout $P\in X$ on a 
    \[dim_P X\leq dim T_{X,P}\]
\end{prop}
\begin{defn}
    On dit que $X$ est lisse/non singulière en $P$ si 
    $dim_P X=dim T_{X,P}$.
\end{defn}
\begin{proof}
    Par récurrence sur $dim T_{X,P}$. Si $dim T_{X,P}=0$, en passant
    au cas affine, $\m_P/\m_P^2=0$ d'où 
    \[\m_P=\m_P^2.\]
    Par le lemme
    de Nakayama il existe $f\in A-\m_P$ tel que $f\m_P=0$ avec $f\in
    D(f)$. En se plaçant sur $D(f)$ on remplace $\Or_X(X)$ par 
    $A_f$ d'où $f$ inversible et $\m_P=0$ puis $A_f=k$. En particulier,
    $dim_P X= 0$.

    Maintenant si $dim T_{X,P}\geq 1$. Soit $Z$ une
    composante irréductible de $X$ passant par $P$ et de dimension
    $dim_P X$. On a $\Or(Z)=\Or(X)/I$ puis $\m_{P,Z}=\m_{P,X}/I$
    puis $\m_{P,X}/\m_{P,X}^2$ se surjecte dans $\m_{P,Z}/\m_{P,Z}^2$.
    D'où $dim T_{P,X}\geq dim T_{P,Z}$. On peut supposer $X$ 
    irréductible. Supp $dim_P X\ne 0$. Alors il existe $f\in \m -\m^2$
et $Y=Z(f)\subset X$. En plus \[dim T_{X,P}\geq dim T_{Y,P}+1\]
car $T_{Y,P}\simeq (\m/f)/(\m/f)^2$. Avec $dim Y=dim X-1$, par 
récurrence $dim T_{P,Y}\geq dim_P Y$ et on conclut.
\end{proof}

Étant donnés des générateurs de $I$, $(f_1,\ldots, f_r)$, on déf
la jacobienne en $P$ par 
\[J(P)=(\partial_j f_i(P))_{i,j}\]
alors on a 
\[\ker(J(P))= T_pX\]
on a alors le critère jacobien :
\begin{prop}
    On a : 
    \[\textrm{$X$ est lisse en $P$}\]
    ssi
    \[\textrm{$J(P)$ est de rang $n-\dim_P(X)$}\]
    avec $f_i\in k[T_1,\ldots, T_n]$.
\end{prop}
\begin{proof}
Directement on a $\dim T_PX=\dim_k \ker(J(X)_P)=n-rk(J(X)_P)$.
\end{proof}
\begin{cor}
    Si $X=Z(F)\subset \A^n(k)$ alors $P$ est lisse si 
    au moins un des $\partial_i F(P)$ est non nulle.
\end{cor}
\begin{proof}
    On a $\dim_P(X)=n-1$ et la jacobienne est une ligne. Donc de 
    rang $1$ ssi y'en a un qu'est non nul.
\end{proof}
\begin{exo}
    L'ensemble des points lisse est ouvert.
\end{exo}

\begin{prop}
    Soit $k(X)$ un corps de fonctions. Alors $k(X)$ est finie séparable
    d'un $k(T_1,\ldots, T_n)$.
\end{prop}
\begin{proof}
    Par la normalisation de Noether on trouve 
    \[K=k(T_1,\ldots,T_n)\subset k(X)\]
    supposons que l'extension n'est pas séparable. On note $L$
    la clôture séparable de $K$ dans $k(X)$. On a $car(k)=p>0$. On prends
    $\theta\in k(X)-L$ t.q. $\theta^p\in L$. On montre que $L(\theta)$
    est séparable finie sur un $k(S_1,\ldots, S_n)$. 

    Soit $H(S)=S^r+f_{r-1}S^{r-1}+\ldots+f_0\in K[S]$ le pol min de
    $\theta^p$ sur $K$. Supposons que $f_i\in k(T_1^p,\ldots, T_n^p)$
    alors $H(S^p)=G(S)^p$ et $G$ est un polynôme minimal séparable de 
    $\theta$. D'où l'un des $f_i\notin k(T_1^p,\ldots, T_n^p)$. 
    \begin{rem}
	On peut pas dire directement que $H$ est séparable donc $H(S^p)$
	est une puissance $p$-ème. Parce que $S^p-S$ par exemple.
    \end{rem}
    Si une puissance d'un $T_i$ non divisible par $p$ apparaît dans
    un $f_i$, alors $T_i$ est algébrique sur 
    $K_1=k(\theta,T_2,\ldots, T_n)$. (en prenant $T_i=T_1$)

    Et en plus $L(\theta)$ est séparable sur $K_1(T_1)$ qui est séparable
    sur $K_1$. Mais $K_1$ a le degré de transcendance de $L(\theta)$ 
    qui est $n$. D'où $\theta,T_2,\ldots, T_n$ sont algébriquement
    indépendants.
\end{proof}
\begin{rem}
    Comparer avec la preuve de Sha. 
\end{rem}
Soit $R$ un anneau intègre et $K=Frac(R)$. On pose $H\in R[S]$ et
$\Delta(H)=\prod(s_i-s_j)^2\in R$ son discriminant. Alors $H$ est
séparable ssi $\Delta(H)\ne 0$.

Maintenant si on note $\varphi\colon 
R\to F$ (un morphisme de corps quelconque) et
\[\tilde\varphi\colon R[S]\to F(S)\]
alors $\Delta(\tilde\varphi(H))=\varphi(\Delta)$.

\begin{lem}
    Étant donné $H\in k[T_1,\ldots,T_n][S]$. Soit $X\in \A^n\times\A^1$
    la variété associée. Si $\Delta=\disc(H)\in k[T_1,\ldots, T_n]$
    alors \[X\cap D(\Delta)\]
    est non vide et est l'ensemble des points lisses.
\end{lem}
\begin{rem}
    Il suppose $H$ unitaire.
\end{rem}
\begin{proof}
    Soit $q\in (\emptyset\ne) D(\Delta)\subseteq \A^n$. On note 
    $\phi_q\colon k[T_1,\ldots, T_n]\to k$ t.q. $\phi_q(F)=F(q)$. Alors
    si $s\in k$ est une racine de \[\tilde\phi_q(H)(S)\] on a 
    $P=(q,s)\in X\cap D(\Delta)\ne \emptyset$. On regarde la jacobienne
    de $H$ en $P$. On a 
    \[\tilde\phi_q(H)(S)=\phi_q(\Delta)=\Delta(q)\ne 0\]
    d'où $\tilde\phi_q(S)$ est séparable puis 
    \[\partial_S H(P))=\tilde\phi(H)'(S)(s)\ne 0\]
\end{proof}
\begin{prop}
    Le smooth locus est dense dans $X$.
\end{prop}
\begin{proof}
    On prouve que tout les ouverts non vides contiennent un point lisse. 
    On peut supposer $X$ affine intègre. On peut donc prendre le
    corps de fonctions $k(X)$ et une extension séparable 
    \[k(T_1,\ldots, T_n)\subset k(X).\]
    On a donc $k(T_1,\ldots, T_n)(S)=k(T_1,\ldots, T_n)[S]/H(\bar T,S)$.
    avec $H$ séparable en $S$. On se place sur un ouvert ou y'a pas
    de dénominateurs. On a 
    $\tilde H = g^{\deg(H)}H\in k[T_1,\ldots, T_n][gS]$ 
    unitaire puis $k(\tilde H)=k(X)$ d'où on peut appliquer le résultat
    à $\tilde H$.
\end{proof}
\section{Variétés normales}
\begin{thm}
    Soit $X$ une variété algébrique lisse en $P$. Alors
    \begin{enumerate}
	\item Il n'y a qu'une seule composante de $X$ passant par $P$.
	\item Soit $U$ est ouvert affine connexe (implique intègre) 
	    de $X$ alors $\Or_X(U)$ est intégralement clos. (dans 
	    $Frac(\Or_X(U))$ qui est pas forcément $Frac(\Or_X(X))$.)
	\item $\Or_{X,P}$ est factoriel.
    \end{enumerate}
\end{thm}
\begin{rem}
    On a 
    \begin{enumerate}
	\item Lisse + connexe implique intègre.
	\item Si $\dim U=1$ alors $\Or_X(U)$ est de Dedekind.
    \end{enumerate}
\end{rem}
\begin{prop}
    Soit $P\in X$ et $d=\dim_P X$. Alors $X$ est lisse en $P$ ssi
    pour tout voisinage ouvert petit affine de $P$, l'idéal $\m_P\subset
    \Or_X(U)$ est généré par $d$ éléments. (Plus précisément 
    $\m_P\Or_{X,P}$)
\end{prop}
\begin{proof}
    Supposons que $\m$ est généré par $d$ éléments. Alors 
    \[\m\to\m/\m^2\]
    est généré par $d$ éléments en tant que $\Or_X(X)$-module d'où
    en tant que $\Or_X(X)/\m$-module. Puis $\dim T_pX = d$.
    À l'inverse si $X$ est lisse en $P$. On suppose $X$ affine. On
    note $\m\subset A(X)$ l'idéal de $P$. On sait que $\m/\m^2$
    est de dimension $d$ sur $A(X)/\m$, disons généré par $(\bar e_i)_i$.
    D'où $\m=(e_1,\ldots, e_d)+\m^2$ puis 
    $\m/(e_1,\ldots, e_d)+\m(\m/(e_1,\ldots,e_d)$. Par Nakayama il 
    existe $f\notin \m$ tel que $f\m=(e_1,\ldots,e_d)$. D'où 
    $\m A_f=(e_1,\ldots, e_d)A_f$. En particulier, 
    $\m A_f$ est l'idéal maximal qui correspond à $P\in D(f)$.
    D'où dans tout ouvert $U\subseteq D(f)$.
\end{proof}
\begin{cor}
    Si $X$ est une courbe affine intègre. ALors $X$ est lisse ssi
    $A(X)$ est de Dedekind.
\end{cor}
\begin{proof}
    C'est juste que tout les localisés sont des DVR et on applique le
    corollaire.
\end{proof}
\begin{defn}
    $X$ est dite normale si pour tout les $U$ affine ouverts 
    \[\textrm{$\Or_X(U)$ est intégralement clos}\]
\end{defn}
\begin{prop}
    Si $X$ est affine, $A(X)$ intégralement clos implique $X$ normale.
\end{prop}
\begin{proof}
    Étant donné $s\in k(X)$, on a $\Or_X(U)\hookrightarrow
    A(X)_f$. D'où $f^rs$ est entier sur $A(X)$ donc $s\in A(X)_f$. 
    Maintenant en prenant un recouvrement on a $s\in A(X)_{f_i}$. 
    D'où par la propriété du faisceau $s\in \Or_X(U)$.
\end{proof}
\begin{rem}
    La preuve rigoureuse doit utiliser la flèche canonique $\Or_X(U)
    \to k(X)$.
\end{rem}
\begin{defn}
    Une morphisme $X\to Y$ est affine si pour tout affine ouvert
    $V$ de $Y$, $f^{-1}(V)$ est affine. Le morphisme est fini si
    il est affine et pour tout ouvert affine $V\subseteq Y$ on a 
    \[\textrm{$\Or_X(f^{-1}V)$ est fini sur $\Or_Y(V)$}\]
\end{defn}
\begin{rem}
    On peut montrer que si il existe un recouvrement affine 
    $Y=\cup_i V_i$ tq pour tout $i$, $f^{-1} V_i$ est affine (resp.
    $\Or_Y(V)\to \Or_X(f^{-1}V)$ est fini) alors $f$ est affine
    (resp. fini).
\end{rem}

\begin{defn}
    Soit $X$ une variété intègre. La normalisation de $X$ est
    une variété normale $X'$ muni d'un morphisme fini birationnel
    \[X'\to X\]
\end{defn}
Soit $\phi\colon k(X)\to L$ une extension finie, la normalisation de 
$X$ dans $L$ est un morphisme fini dominant
\[\pi\colon X'\to X\]
t.q $k(X')=L$, $X'$ est normal et $k(X)\to L$ est induit par $\pi$.
\begin{prop}
    Soit $A$ une $k$-algèbre intègre de type fini et $K=Frac(A)$.
    Soit $B$ la clôture intégrale de $A$ dans $L$, $L$ une extension
    finie de $K$. Alors $Frac(B)=L$ et $B$ est finie sur $A$.
\end{prop}
\begin{rem}
    Si $L=K$, la normalisation est $Spm(\tilde A)\to Spm(A)$.
\end{rem}
\begin{ex}
    Si $X=Z(T_1^2-T_2^3)\subseteq \A^2_k$. Alors
    \[\A^1\to X\]
    donné par $t\mapsto (t^3,t^2)$ est la normalisation.
\end{ex}
\chapter{Courbes algébriques}
Une courbe algébrique est une variété séparée de dimension pure $1$.
\begin{rem}
    Un morphisme fini est fermé. Y'a la preuve du td, ou y'a la 
    preuve générale de Spec.
\end{rem}
\begin{defn}
    Pour rappel, $f$ est fini si $\Or_Y(V_i)\to \Or_Y(f^{-1}V_i)$
    est fini pour un recouvrement affine de $Y$.
\end{defn}

\begin{proof}
    Le point c'est que $f$ est surjective et $f(Z(f^\sharp(I(Z))))$
    est dense dans $Z$ là ou ca fait sens.
\end{proof}

\begin{lem}
    Si $f\colon X\to Y$ est fini et $Y$ est séparée (resp. propre) alors
    $X$ est séparée (resp. propre).
\end{lem}
\begin{proof}
    On doit checker que $X$ est séparée et universellement close.
    On regarde $X\times_Y \hookrightarrow X\times_k X$, on a 
    \[X\times_Y X=(f\times f)^{-1}(\Delta(Y))\]
    est fermée car $Y$ est séparée. On a clairement
    \[\Delta(X)\subset X\times_Y X\]
    il suffit de montrer que $\Delta(X)$ est fermée dedans. On 
    prend un recouvrement affine $\cup V_i= Y$. Maintenant
    $X\times_Y X$ est recouverte par les ouverts (check) $U_i\times_{V_i} U_i$
    avec $U_i=f^{-1}V_i$. Il suffit de montrer que $\Delta(U_i)$ est fermée 
    dans $U_i\times_{V_i} U_i$. Maintenant, 
\[U_i\to \Delta(U_i)\hookrightarrow U_i\times_{V_i} U_i\hookrightarrow U_i\times_k U_i\]
    est fermée. D'où $X$ est séparée. (on a pas utilisé la finitude)

    Soit $Z$ une variété algébrique. On regarde
% https://q.uiver.app/#q=WzAsMyxbMCwwLCJaXFx0aW1lc19rIFgiXSxbMSwxLCJaXFx0aW1lc19rIFkiXSxbMiwwLCJaIl0sWzAsMV0sWzEsMl0sWzAsMl1d
\[\begin{tikzcd}
	{Z\times_k X} && Z \\
	& {Z\times_k Y}
	\arrow[from=1-1, to=1-3]
	\arrow[from=1-1, to=2-2]
	\arrow[from=2-2, to=1-3]
\end{tikzcd}\]
    il suffit de montrer que $Z\times X\to Z\times Y$ est fermée, c'est
    clairement un morphisme fini car localement, les morphismes finis
    sont affines d'où on peut supposer $Z,X,Y$ affines et en regardant
% https://q.uiver.app/#q=WzAsNCxbMCwwLCJBKFkpIl0sWzIsMCwiQShYKSJdLFswLDEsIkEoWSlcXG90aW1lcyBBKFopIl0sWzIsMSwiQShYKVxcb3RpbWVzIEEoWikiXSxbMCwxXSxbMCwyXSxbMiwzXSxbMSwzXV0=
\[\begin{tikzcd}
	{A(Y)} && {A(X)} \\
	{A(Y)\otimes A(Z)} && {A(X)\otimes A(Z)}
	\arrow[from=1-1, to=1-3]
	\arrow[from=1-1, to=2-1]
	\arrow[from=1-3, to=2-3]
	\arrow[from=2-1, to=2-3]
\end{tikzcd}\]
    on obtient le résultat.
\end{proof}

\begin{lem}
    Soit $K$ un corps de fonction de degré de transcendance $1$.
    Il existe une courbe intègre lisse et propre telle que $K=k(X)$.
\end{lem}
\begin{proof}
    C'est un gros sketch de preuve. Soit $K$ une extension finie de 
    $k(X)$, alors $K=k(X)[T]/(P)= Frac(k[X][T]/(P))$ est le corps de 
    fonction d'une $k$-algèbre réduite de type fini. On prends une
    variété affine associée $Z$ et on obtient $\bar Z\subseteq \Pr^n$.
    Maintenant $\bar Z$ est intègre propre. Il reste à normaliser.
    On prends $X\to \bar Z$ sa normalisation. Alors la flèche est finie
    d'où $X$ est intègre propre et normale.
\end{proof}
\begin{note}
    Pour rappel, si $Y$ est séparée et $X$ intègre alors si 
    $f,g\colon X\to Y$ et $g|_U=f|_U$ alors $f=g$.
\end{note}
\begin{note}
    La propreté permet d'étendre uniquement de $U$ à $X$
    dans le cas des courbes.
\end{note}
\begin{note}
    Dans le cas lisse, les composantes connexes coincident avec les
    irréductibles.
\end{note}
\begin{thm}
    Si $U$ est une courbe algébrique lisse. Soit $Y$ une
    variété algébrique propre et $V\subseteq U$ ouvert. Alors toute
    flèche $f\colon V\to Y$ il existe un unique $g\colon U\to Y$
    qui étend $f$.
\end{thm}
\begin{proof}
    Il suffit de prouver l'existence, $U-V$ est un nombre fini de point.
    On montre qu'on peut l'étendre en chaque point. On prends $P\in W
    \subset U$ affine. On étend comme suit
% https://q.uiver.app/#q=WzAsMyxbMCwwLCJXXFxjYXAgViJdLFsyLDAsIlciXSxbMCwxLCJZIl0sWzAsMSwiIiwwLHsic3R5bGUiOnsidGFpbCI6eyJuYW1lIjoiaG9vayIsInNpZGUiOiJ0b3AifX19XSxbMSwyLCIiLDAseyJzdHlsZSI6eyJib2R5Ijp7Im5hbWUiOiJkYXNoZWQifX19XSxbMCwyXV0=
\[\begin{tikzcd}
	{W\cap V} && W \\
	Y
	\arrow[hook, from=1-1, to=1-3]
	\arrow[from=1-1, to=2-1]
	\arrow[dashed, from=1-3, to=2-1]
\end{tikzcd}\]
    On peut supposer $U$ affine et $V=U-\{P\}$. Soit $A=A(U)$, et
    soit $\m\leq A$ maximal. Comme $U$ est lisse on peut supposer
    \[\m=(t).\]
    On considère 
    \[f\colon V\to Y\]
    et $V\simeq\Gamma_f\subseteq V\times Y\subseteq U\times Y$ fermé car
    $Y$ est séparée. Soit $Z=\bar\Gamma_f\subseteq U\times Y$, alors
    $Z$ est birationnel à $U$. On va prover que $Z\to U$ est un
    isomorphisme. Si c'est le cas, $U\to Z\to Y$ étend $f$ donc on 
    est bon.

    On a $U\times Y\to U$ est fermée car $Y$ est propre. D'où 
    $h\colon Z\to U$ est fermée donc surjective. On note 
    $z\in h^{-1}(P)$. Soit $B=\Or_Z(W)$ un voisinage ouvert 
    $W$ de $z$. Il y a une flèche $A(U)\hookrightarrow B=A(W)
    \hookrightarrow A(W) \hookrightarrow K=Frac(A)$. Soit $b\in B$
    et $b=t^na/u$ avec $n\in \Z$ et $a,u\notin \m$. On obtient 
    $bu=t^na$ d'où $0=(bu)(z)=t^n(z)a(z)$ d'où $t^n(z)=0$ car 
    $a(z)\ne 0$. D'où $n\geq 1$ et $b\in A[1/u]$. Comme $B$ est
    finiment générée, $A\subset B\subset A[1/u_0]$ pour un 
    $u_0\notin \m$. D'où $B[1/u_0]=A[1/u_0]$. On prouve que 
    \[h\colon h^{-1}(D(u_0))\cap W\simeq D(u_0)\]
    On prends $\{z_1,\ldots,z_r\}=h^{-1}(P)$ t.q pour tout $i=1,\ldots,r$
    il existe $P\in U_0$ et $g_i\colon W_i\simeq U_0$ contenu dans
    $h\colon Z\to U$. Ca implique que $r=1$ sinon $g_1^{-1}$ et 
    $g_2^{-1}$ sont deux morphismes différents qui coincident avec
    l'inverse de $\Gamma_f \to V$.
% https://q.uiver.app/#q=WzAsNixbMCwwLCJcXEdhbW1hX2YiXSxbMSwwLCJWIl0sWzEsMSwiVSJdLFswLDEsIloiXSxbMCwyLCJXXzAiXSxbMSwyLCJVXzAiXSxbMywyLCJoIl0sWzAsMywiIiwxLHsic3R5bGUiOnsidGFpbCI6eyJuYW1lIjoiaG9vayIsInNpZGUiOiJ0b3AifX19XSxbMSwyLCIiLDEseyJzdHlsZSI6eyJ0YWlsIjp7Im5hbWUiOiJob29rIiwic2lkZSI6InRvcCJ9fX1dLFs1LDIsIiIsMSx7InN0eWxlIjp7InRhaWwiOnsibmFtZSI6Imhvb2siLCJzaWRlIjoidG9wIn19fV0sWzQsMywiIiwxLHsic3R5bGUiOnsidGFpbCI6eyJuYW1lIjoiaG9vayIsInNpZGUiOiJ0b3AifX19XSxbNCw1LCJcXHNpbWVxIiwxLHsic3R5bGUiOnsiYm9keSI6eyJuYW1lIjoibm9uZSJ9LCJoZWFkIjp7Im5hbWUiOiJub25lIn19fV0sWzAsMSwiXFxzaW1lcSIsMSx7InN0eWxlIjp7ImJvZHkiOnsibmFtZSI6Im5vbmUifSwiaGVhZCI6eyJuYW1lIjoibm9uZSJ9fX1dXQ==
\[\begin{tikzcd}
	{\Gamma_f} & V \\
	Z & U \\
	{W_0} & {U_0}
	\arrow["\simeq"{description}, draw=none, from=1-1, to=1-2]
	\arrow[hook, from=1-1, to=2-1]
	\arrow[hook, from=1-2, to=2-2]
	\arrow["h", from=2-1, to=2-2]
	\arrow[hook, from=3-1, to=2-1]
	\arrow["\simeq"{description}, draw=none, from=3-1, to=3-2]
	\arrow[hook, from=3-2, to=2-2]
\end{tikzcd}\]

\end{proof}

\begin{cor}
    Soit $X,Y$ deux courbes entières lisses propres et birationnelles.
    Alors elles sont isomorphes.
\end{cor}
\begin{proof}
    Simplement, la flèche entre les ouverts, $f|_U$, et son inverse, 
    $g|_U$ s'étendent à $X$, $Y$. Puis $g\circ f|_U=id|_U$. Pareil
    de l'autre côté.
\end{proof}
\begin{ex}
    En général c'est faux, $\Pr^2\simeq_{bir} \Pr^1\times \Pr^1$
    sont birationnels mmais pas isomorphes.
\end{ex}
\begin{lem}
    Un morphisme de courbe propres lisses intègres est constant
    ou fini.
\end{lem}
\begin{proof}
    COmme $X$ est propre, $f$ est fermé. Alors $f(X)$ est $Y$ ou un
    point. On suppose que c'est $Y$, alors on a $k(Y)\to k(X)$ est
    finie. On normalise $Y$ en $\pi\colon X'\to Y$ dans $k(X)$.
    On a que $\pi$ est finie et $X'$ est normale avec $k(X')=k(X)$.
    On a 
% https://q.uiver.app/#q=WzAsMyxbMCwwLCJYIl0sWzEsMCwiWSJdLFswLDEsIlgnIl0sWzAsMSwiZiIsMV0sWzAsMiwiZ1xcc2ltZXEiLDEseyJzdHlsZSI6eyJib2R5Ijp7Im5hbWUiOiJub25lIn0sImhlYWQiOnsibmFtZSI6Im5vbmUifX19XSxbMiwxLCJcXHBpIiwxXV0=
\[\begin{tikzcd}
	X & Y \\
	{X'}
	\arrow["f"{description}, from=1-1, to=1-2]
	\arrow["{g\simeq}"{description}, draw=none, from=1-1, to=2-1]
	\arrow["\pi"{description}, from=2-1, to=1-2]
\end{tikzcd}\]
    et $X',X$ induisent la même $k(Y)\hookrightarrow k(X)=k(X')$.
    D'où $f=\pi\circ g^{-1}$. Comme $\pi$ est finie $f$ est finie.
\end{proof}
\begin{thm}
    Il existe une équivalence de catégorie entre
    \[\textrm{les corps de fonction de degré de transcendance $1$ }\]
    et
    \[\textrm{Les courbes algébriques intègre propre lisse}\]
\end{thm}
\begin{cor}
    Toute courbe intègre lisse propre admet un morphisme fini
    \[X\to \Pr^1\]
\end{cor}
\begin{thm}
    Toute courbe intègre lisse propre est projective.
\end{thm}
\begin{proof}
    On prends un recouvrement affine $(V_i)_i$ fini. On étend un 
    des $V_i$ à $\Pr^n$ en $X_i$. Maintenant, $X_i$ est pas forcément
    lisse, et on prend $P=\prod X_i$ le produit, qui est projective.
    On regarde l'extension 
% https://q.uiver.app/#q=WzAsNCxbMCwxLCJcXGNhcCBVX2kiXSxbMSwxLCJcXHByb2QgXFxvdmVybGluZXtVX2l9Il0sWzAsMiwiWCJdLFsxLDAsIlxccHJvZCBVX2kiXSxbMCwxLCJmIiwxXSxbMCwyLCIiLDEseyJzdHlsZSI6eyJ0YWlsIjp7Im5hbWUiOiJob29rIiwic2lkZSI6InRvcCJ9fX1dLFsyLDEsImciLDEseyJzdHlsZSI6eyJib2R5Ijp7Im5hbWUiOiJkYXNoZWQifX19XSxbMCwzXSxbMywxXV0=
\[\begin{tikzcd}
	& {\prod U_i} \\
	{\cap U_i} & {\prod \overline{U_i}} \\
	X
	\arrow[from=1-2, to=2-2]
	\arrow[from=2-1, to=1-2]
	\arrow["f"{description}, from=2-1, to=2-2]
	\arrow[hook, from=2-1, to=3-1]
	\arrow["g"{description}, dashed, from=3-1, to=2-2]
\end{tikzcd}\]
    comme $f$ s'étend car $P$ est propre. Et on pose $g(X)=Z$.
    Alors $Z$ est projective et intègre. On montre qu'elle est
    lisse et propre. On montre que tout point $z\in Z$ admet un
    voisinage affine normal. Alors $z=g(x)$ pour un $x\in U_i$
% https://q.uiver.app/#q=WzAsMyxbMCwwLCJVX2kiXSxbMSwwLCJaIl0sWzEsMSwiWF9pIl0sWzAsMV0sWzEsMl0sWzAsMiwiIiwxLHsic3R5bGUiOnsidGFpbCI6eyJuYW1lIjoiaG9vayIsInNpZGUiOiJ0b3AifX19XV0=
\[\begin{tikzcd}
	{U_i} & Z \\
	& {X_i}
	\arrow[from=1-1, to=1-2]
	\arrow[hook, from=1-1, to=2-2]
	\arrow[from=1-2, to=2-2]
\end{tikzcd}\]
    $g$ est une inclusion car $\cap U_i\hookrightarrow X_i$
    est une inclusion et $X_i$ est séparée. On prends un voisinage
    affine de $z$ contenu dans $f(U)$ (Les $Z-f(U)\subseteq f(X-U_i)$
    sont finis). On a 
% https://q.uiver.app/#q=WzAsNCxbMCwwLCJmXnstMX1XIl0sWzEsMCwiVyJdLFsxLDEsImZeey0xfVciXSxbMiwxLCJYX2kiXSxbMCwxLCIiLDEseyJzdHlsZSI6eyJoZWFkIjp7Im5hbWUiOiJlcGkifX19XSxbMSwyXSxbMCwyLCJpZCIsMV0sWzIsMywiXFxzdWJzZXRlcSIsMSx7InN0eWxlIjp7ImJvZHkiOnsibmFtZSI6Im5vbmUifSwiaGVhZCI6eyJuYW1lIjoibm9uZSJ9fX1dXQ==
\[\begin{tikzcd}
	{f^{-1}W} & W \\
	& {f^{-1}W} & {X_i}
	\arrow[two heads, from=1-1, to=1-2]
	\arrow["id"{description}, from=1-1, to=2-2]
	\arrow[from=1-2, to=2-2]
	\arrow["\subseteq"{description}, draw=none, from=2-2, to=2-3]
\end{tikzcd}\]
    ce qui implique que $\Or_Z(W)\hookrightarrow \Or_X(f^{-1}W)$ d'où
    $f^{-1}W\simeq W$. Maintenant $f^{-1}W$ est lisse connectée. Donc
    normale. D'où $Z$ est lisse et birationnelle à $X$ donc isomorphe.
\end{proof}

\section{Diviseurs}
Les diviseurs sur une courbe intègre lisse sont des sommes formelles
\[D=\sum n_P(P)\in Z[|X|]=Z^1(X)\]
pour $P\in X$, on a $v_P\colon k(X)^\times\to \Z$ une valuation 
discrète, via $k(X)=Frac(\Or_{X,P})$ comme $X$ est lisse.

\begin{prop}
    Soit $f\in k(X)-0$, alors $v_x(f)=0$ sauf en un nombre fini de 
    $x$.
\end{prop}
\begin{proof}
    On a $f=a/b\in \Or_X(U)$ $U$ affine. Alors $v_x(f)=0$
    si $x\notin Z(a)\cup Z(b)$, mais c'est un nombre fini de points
    et $X-U$ est fini.
\end{proof}
\begin{ex}
    Sur $X=\Pr^1$, $t\in k(t)$ alors, $div(t)=(0)-(\infty)$. Y'a
    plusieurs remarques, $k(t)=k(\Pr^1)$ c'est via une écriture dans
    une carte. Le cgt de carte c'est $t\mapsto 1/t$, i.e. 
    $(U_1, 1/t)\sim (U_0,t)$.
\end{ex}
\begin{defn}
    On définit $Pic(X)=Z^1(X)/div(k(X)^\times)$.
\end{defn}
On a $div(f)=div_0(f)-div_\infty(f)$.

\begin{defn}
    On définit aussi le degré $\deg\colon Z^1(C)\to \Z$
\end{defn}
On veut montrer que ça passe au quotient.
\begin{defn}
    On déf
    \[f_*\colon Z^1(X)\to Z^1(Y)\]
    par 
    $f_*(\sum n_P(P)=\sum n_P(f(P))=\sum_{Q\in Y} (\sum_{P\in f^{-1}P}n_P)(Q)$.
\end{defn}
Si $f$ est constante, alors $f_*=0$. Pour tout $y\in Y$ et 
$x\in f^{-1}(y)$, on a une extension de DVR
\[f_x\colon \Or_{Y,y}\to \Or_{X,x}\]
\begin{defn}
    On déf la ramification 
    $e_{x,y}:=e_{\m_x,\m_y}=v_{\m_x}(f_x(\pi_y))$.
\end{defn}
\begin{defn}
    On définit ensuite $f^*\colon Z^1(Y)\to Z^1(X)$ par 
    \[\sum n_y(y)\mapsto \sum_x e_{x,f(x)}n_{f(x)}(x)\]
\end{defn}
\begin{ex}
    Pour $f\in K(X)$, on a $f\colon U\to \A^1$ régulière. Elle s'étend
    en $f\colon U\to \P^1$ puis $f\colon X\to \P^1$ car $X$ est
    projective. On a alors $div(f)=f^*((0)-(\infty))$
\end{ex}
\begin{defn}
    Si $f\colon X\to Y$ est un morphisme fini de courbes.
    On déf $\deg(f)=[K(Y):K(X)]$.
\end{defn}
\begin{cor}
    Comme $k=\bar k$, pour tout $y\in Y$ on a 
    \[\deg(f)=\sum_{f(x)=y} e_{x,y}\]
\end{cor}
La preuve c'est juste que le degré résiduel est tjr $1$!
\begin{cor}
    Soit $f\colon X\to Y$ fini de courbes, alors pour tout $D\in Z^1(Y)$
    \[f_*f^*D=\deg(f)D\]
\end{cor}
\begin{proof}
    Suffit de le montrer pour les diviseurs $(y)\in Z^1(Y)$.
    On a 
    \begin{align*}
	f_*f^*(y)&=f_*(\sum_{x\in f^{-1}(y)} e_{x,y}n_{y}(x))\\
		 &=f_*(\sum_{x\in f^{-1}(y)} e_{x,y}(x))\\
		 &=(\sum_{x\in f^{-1}(y)}e_{x,y})(y)\\
		 &=\deg(f)(y)
    \end{align*}
\end{proof}
\begin{lem}
    Si $X=\P^1(k)$ alors $\deg(div(K(X)^\times))=0$.
\end{lem}
\begin{proof}
    Suffit de le montrer pour $t-a$ si $f\in k(t)$. Mais
    \[div(t-a)=(a)-(\infty)\]
    d'où le résultat.
\end{proof}
\begin{prop}
    Soit $f\colon X\to Y$ un morphisme fini de courbes lisses propre.
    Si $D\in Z^1(Y)$, alors 
    \[\deg(f^*(D))=\deg(f)\deg(D)\]
\end{prop}
\begin{proof}
    $\deg(f^*D)=\deg(f_*f^* D)=\deg(\deg(f)D)=\deg(f)\deg(D)$
\end{proof}
\begin{cor}
    Pour toute courbe lisse propre intègre :
    \[\deg(div(K(X)^\times))=0\]
\end{cor}
On obtient un morphisme
\[\deg\colon Pic(X)\to \Z\]
\begin{prop}
    Si $X=\P^1(k)$, $\deg$ est un isomorphisme.
\end{prop}
\begin{prop}
    Soit $X$ une courbe intègre lisse projective. Si il existe
    $x_0,x_1\in X$ t.q $(x_0)\sim (x_1)$, alors 
    \[X\simeq \P^1(k)\]
\end{prop}
\begin{proof}
    Il existe $f\in K(X)^*$ t.q. $div(f)=(x_0)-(x_1)=f^*((0)-(\infty))$
    d'où $f^*(0)=(x_1)$ puis $\deg(f)=1$ et $f$ est un isomorphisme
    \[X\simeq \P^1(k)\]
\end{proof}
\section{Espaces de Riemann-Roch}
Soit $D$ un diviseur sur $X$. On pose
\[L(D):=\{f\in K(X)^*|div(f)+D\geq 0\}\cup \{0\}\]
avec $D\geq 0$ si $n_x\geq 0$ pour tout $x$. En plus on pose aussi
\[|D|:=\P_-(L(D))=\{D'\in Z^1(X)|D'\sim D,D'\geq 0\}\]
\begin{ex}
    $L(0)=\Or_X(X)=k$.
\end{ex}
\begin{rem}
    $L(D)$ est un espace vectoriel.
\end{rem}
\begin{prop}
    On note $l(D)=dim_k L(D)$.
    \begin{enumerate}
	\item Si $D\sim E$, $l(D)=l(E)$.
	\item $\deg(D)<0\implies l(D)=0$.
	\item Si $D'\leq D$ alors $l(D')\leq l(D)\leq l(D')+\deg(D)
	    -\deg(D')$.
	\item Si $\deg(D)\geq 0$ alors $l(D)\leq \deg(D)+1$.
	\item Si $\deg(D)=0$, alors $l(D)>0$ ssi $D\sim 0$.
    \end{enumerate}
\end{prop}
\begin{proof}
    Pour $1.$ si $D\sim E$ alors $|D|=|E|$, d'où $l(D)-1=l(E)-1$.
    Pour $2.$, $|D|=\emptyset$, psq si $D'\sim D$ alors 
    $\deg(D')=\deg(D)$ et si $D'\geq 0$, $\deg(D')\geq 0$.

    Pour $3.$, si $D'<D$, alors $l(D')\leq l(D)$ car on a 
    \[|D'|\hookrightarrow |D|\]
    via $E\mapsto E+D-D'$ (car $D-D'\geq 0$).

    Y suffit de montrer pour $D=D'+(x)$ que 
    $l(D)\leq l(D')+1$ par récurrence. Soit $t_x$ une coordonnée 
    locale en $x$. Si $f\in L(D)-0$ alors $v_x(f_*t_x^{n_x})\geq 0$
    où $n_x=v_x(D)$. On a
    \[v_x(ft_x^{n_x})=v_x(f)+n_x\geq 0\]
    et on pose 
    \[\rho\colon L(D)\to k\]
    donnée par $f\mapsto (ft_x^{n_x})(x)$.
    \newline
    Et là on fait de l'algèbre
    linéaire, omg d'où ça sort. 
    \newline
    On a $\ker(\rho)=\{f\in K(X)^*|div(ft_x^{n_x-1})\geq 0\}=L(D)$.
    D'où $l(D')\leq l(D)-1$.

    Pour $4.$ si $D\geq 0=:D'$, alors 
    \[l(D)\leq l(0)+\deg(D)\leq \deg(D)+1\]

    Pour $5.$, si $\deg(D)=0$, alors $l(D)>0$ d'où $|D|\ne \emptyset$
    puis il existe $E\geq 0$ t.q $E\sim D$ mais si $E\geq 0$ et 
    $\deg(E)=0$ alors $E=0$ d'où $D\sim 0$.
\end{proof}
\begin{cor}
    $L(D)$ est un $k$-espace vectoriel de dimension finie !!
\end{cor}
Par exemple, sur $\P^1(k)$, $l([0])=2$. Et 
$L(D)=<t^{-i},i=0,\ldots,\deg(D)>$.

\section{Théorème de Riemann-Roch}
Soit $X$ une courbe intègre lisse propre. Il existe $g\geq 0$
et un diviseur $K_X$ t.q. pour tout $D\in Z^1(X)$
\[l(D)-l(K_X-D)=\deg(D)+1-g\]
\begin{defn}
    On appelle $g$ le genre de $X$.
\end{defn}
\begin{lem}
    On a $\deg(K_X)=2g-2$ et $l(K_X)=g$.
\end{lem}
\begin{lem}
    $K_X$ est unique à équivalence linéaire.
\end{lem}
\begin{proof}
    Si il existe un autre $K_X$ qui marche, $K_X'$, alors
    \[\deg(K_X-K_X')=0\]
    puis \[l(K_X')-l(K_X-K_X')=g-1\]
    implique que $l(K_X-K_X')=1$, d'où $K_X\sim K_X'$.
\end{proof}
\begin{cor}
    Si $\deg(D)>2g-2$, alors 
    \[l(D)=\deg(D)+1-g\]
\end{cor}

\section{Différentielles de Kahler}
Il dit que le livre de Qing Liu est une bonne source.
\begin{defn}
    Soit $B$ une $k$-algèbre et $F$ la $B$-algèbre libre générée
    par les symboles $db$ pour $b\in B$. On note $E$ le sous-module
    de $F$ généré par $d\lambda$ pour $\lambda\in k$, $d(b_1+b_2)-
    d(b_1)-d(b_2)$ et $d(b_1b_2)-b_1d(b_2)-b_2d(b_1)$. On pose
    \[\Omega^1_{B/k}:=F/E\]
\end{defn}
\begin{rem}
    C'est un module pas un anneau là.
\end{rem}
\begin{ex}
    On remarque que si $B=k[T]$ alors $d(P(T))=P'(T)dT$ et 
    \[\Omega_{k[T]/k}=k[T]dT\]
\end{ex}
\begin{exo}
    $S^{-1}\Omega^1_{B/k}\simeq \Omega^1_{S^{-1}B/k}$.
\end{exo}
\begin{defn}
    Soit $X$ une variété algébrique, on définit un $\Or_X$-module
    $\Omega^1_{X/k}$ par 
    \[\Omega^1_{X/k}(U):=\Omega^1_{\Or_X(U),k}\]
    pour tout ouvert affine.
\end{defn}
Par l'exercice, $(\Omega^1_{X/k})_x=\Omega^1_{\Or_{X,x},k}$. 
\begin{prop}
    Si $X$ est une courbe algébrique lisse, alors 
    $(\Omega^{1}_{X/k})_x$ est un $\Or_{X,x}-module$ libre de rang $1$.
\end{prop}
\begin{proof}
    On peut supposer $X\hookrightarrow \A^n(k)$ affine donnée
    par $F_1,\ldots,F_m$ dans $k[T_1,\ldots, T_n]$. Le rang
    de $J(X)_x$ est $n-1$ d'où on peut garder que $F_1,\ldots,
    F_{n-1}$ dans la jacobienne. Puis $X\subset Z(F_1,\ldots, F_{n-1})$
    d'où égalité. Maintenant la jacobienne est de taille $n-1\times n$,
    on la complète en ajoutant la ligne $(1,0,\ldots, 0)$ la matrice
    devient inversible en $x$. On multiplie la nouvelle matrice par
    la colonne $(\partial T_1,\ldots, \partial T_n)$. On obtient
    $(d(T_1),dF_1,\ldots, dF_n)$. Maintenant, 
    \[A(X)=k[T_1,\ldots, T_n]/(F_1,\ldots, F_n)\]
    d'où dans $\Omega^{1}_{A(X),k}$, $dF_i=0$ pour tout $i$. Puis
    $(dT_1,dF_1,\ldots, dF_n)=(dT_1,0,\ldots,0)$ dans les 
    différentielles. En particulier,
    \[\Omega^1_{\Or_{X,x}/k}=\Or_{X,x}dT_1\]
\end{proof}

\begin{ex}
    Soit $K$ un corps de fonctions. Alors $\Omega^1_{K,k}$ est de
    dimension $1$ sur $K$. En effet, $K$ est une extension finie
    séparable de $k(t)$. D'où $K=k(t)[X]/P(X)$. Puis 
    \[\Omega^1_{K,k}=<dt,dX>\]
    Mais $0=dP(X)=P'(X)dX$ et $P'(X)\ne 0$. D'où $dX=0$ puis
    \[\Omega^1_{K,k}=Kdt=\Omega^1_{k(t),k}\otimes_k K\] 
\end{ex}
Soit $X$ une courbe lisse propre intègre. Et soit 
$\omega\in \Omega^1_{K(X),k}$. Soit $x\in X$, alors 
\[\omega=fdt_x\]
pour $f\in K(X)$. On définit alors
\[ord_x\omega:=v_x(f)\]
\begin{ex}
    Pour $\P^1(k)$, on a $ord_a dt=ord_a d(t-a)=0$ pour $a\in \A^1(k)$.
    À l'infini, la coordonnée locale est $s=1/t$. D'où
    $dt=d(1/s)=-(1/s^2)ds$ d'où $ord_\infty dt=-2$.
\end{ex}

Si $\omega\in \Omega^1_{K(X),k}$, on définit
\[div(w)=\sum_{x\in X} (ord_x\omega)(x)\]
faut montrer que le support est fini, il le fait pas, il dit que 
ça se fait bien. 

Comme $\Omega^1_{K(X),k}$ est de dimension $1$, le diviseur est
unique.
\begin{defn}
    On définit $K_X=div(\omega)$ pour toute forme différentielle
    $\Omega^1_{K(X),k}$.
\end{defn}

\begin{ex}
    On a $\deg(K_{\P^1(k)})=-2=-2+2.g$ d'où $g=0$ pour $\P^1(k)$.
\end{ex}

\begin{thm}[Riemann-Hurwitz]
    Soit $f\colon X\to Y$ un morphisme fini séparable de courbes
    lisses propres intègres. Alors
    \[2g(X)-2\geq \deg(f)(2g(Y)-2)+\sum_{x\in X}(e_{x,f(x)}-1)\]
    avec égalité si 
    \begin{enumerate}
	\item $char(k)=0$
	\item $char(k)=p>0$ et $p\nmid e_{x,f(x)}$ pour tout $x$
	    t.q $e_{x,f(x)}\ne 0$.
    \end{enumerate}
\end{thm}
\begin{rem}
    Si $f\colon X\to Y$ et $\omega=gdt$ sur $Y$, alors
    \[f^*\omega=f^*gd(f^*t)\]
\end{rem}
\begin{cor}
    Soit $f\colon \P^1(k)\to Y$ avec $Y$ lisse propre connectée. 
    Alors $Y$ est de genre $0$. Puis on peut montrer que $Y$ est 
    isomorphe à $\P^1(k)$.
\end{cor}



\printbibliography
\end{document}


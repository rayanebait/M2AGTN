\documentclass[a4paper,12pt]{book}
\usepackage{amsmath,  amsthm,enumerate}
\usepackage{csquotes}
\usepackage[provide=*,french]{babel}
\usepackage[dvipsnames]{xcolor}
\usepackage{quiver, tikz}

%symbole caligraphique
\usepackage{mathrsfs}

%hyperliens
\usepackage{hyperref}

%pseudo-code
\usepackage{algorithm}
\usepackage{algpseudocode}

\usepackage{fancyhdr}

\pagestyle{fancy}
\addtolength{\headwidth}{\marginparsep}
\addtolength{\headwidth}{\marginparwidth}
\renewcommand{\chaptermark}[1]{\markboth{#1}{}}
\renewcommand{\sectionmark}[1]{\markright{\thesection\ #1}}
\fancyhf{}
\fancyfoot[C]{\thepage}
\fancyhead[LO]{\textit \leftmark}
\fancyhead[RE]{\textit \rightmark}
\renewcommand{\headrulewidth}{0pt} % and the line
\fancypagestyle{plain}{%
    \fancyhead{} % get rid of headers
}

%bibliographie
\usepackage[
backend=biber,
style=alphabetic,
sorting=ynt
]{biblatex}

\addbibresource{bib.bib}

\usepackage{appendix}
\renewcommand{\appendixpagename}{Annexe}

\definecolor{wgrey}{RGB}{148, 38, 55}

\setlength\parindent{24pt}

\newcommand{\Z}{\mathbb{Z}}
\newcommand{\R}{\mathbb{R}}
\newcommand{\rel}{\omathcal{R}}
\newcommand{\Q}{\mathbb{Q}}
\newcommand{\C}{\mathbb{C}}
\newcommand{\N}{\mathbb{N}}
\newcommand{\K}{\mathbb{K}}
\newcommand{\A}{\mathbb{A}}
\newcommand{\B}{\mathcal{B}}
\newcommand{\Or}{\mathcal{O}}
\newcommand{\F}{\mathscr F}
\newcommand{\Hom}{\textrm{Hom}}
\newcommand{\disc}{\textrm{disc}}
\newcommand{\Pic}{\textrm{Pic}}
\newcommand{\End}{\textrm{End}}
\newcommand{\Spec}{\textrm{Spec}}
\newcommand{\Supp}{\textrm{Supp}}
\renewcommand{\Im}{\textrm{Im}}
\newcommand{\m}{\mathfrak{m}}
\renewcommand{\P}{\mathbb{P}}
\newcommand{\p}{\mathfrak{p}}


\newcommand{\cL}{\mathscr{L}}
\newcommand{\G}{\mathscr{G}}
\newcommand{\D}{\mathscr{D}}
\newcommand{\E}{\mathscr{E}}
\newcommand{\Po}{\mathscr{P}}
\renewcommand{\H}{\mathscr{H}}

\makeatletter
\newcommand{\colim@}[2]{%
  \vtop{\m@th\ialign{##\cr
    \hfil$#1\operator@font colim$\hfil\cr
    \noalign{\nointerlineskip\kern1.5\ex@}#2\cr
    \noalign{\nointerlineskip\kern-\ex@}\cr}}%
}
\newcommand{\colim}{%
  \mathop{\mathpalette\colim@{\rightarrowfill@\scriptscriptstyle}}\nmlimits@
}
\renewcommand{\varprojlim}{%
  \mathop{\mathpalette\varlim@{\leftarrowfill@\scriptscriptstyle}}\nmlimits@
}
\renewcommand{\varinjlim}{%
  \mathop{\mathpalette\varlim@{\rightarrowfill@\scriptscriptstyle}}\nmlimits@
}
\makeatother

\theoremstyle{plain}
\newtheorem{thm}[subsection]{Théoreme}
\newtheorem{lem}[subsection]{Lemme}
\newtheorem{prop}[subsection]{Proposition}
\newtheorem{cor}[subsection]{Corollaire}
\newtheorem{heur}{Heuristique}
\newtheorem{rem}{Remarque}
\newtheorem{note}{Note}

\theoremstyle{definition}
\newtheorem{conj}{Conjecture}
\newtheorem{prob}{Problème}
\newtheorem{quest}{Question}
\newtheorem{prot}{Protocole}
\newtheorem{algo}{Algorithme}
\newtheorem{defn}[subsection]{Définition}
\newtheorem{exmp}[subsection]{Exemples}
\newtheorem{exo}[subsection]{Exercices}
\newtheorem{ex}[subsection]{Exemple}
\newtheorem{exs}[subsection]{Exemples}

\theoremstyle{remark}

\definecolor{wgrey}{RGB}{148, 38, 55}
\definecolor{wgreen}{RGB}{100, 200,0} 
\hypersetup{
    colorlinks=true,
    linkcolor=wgreen,
    urlcolor=wgrey,
    filecolor=wgrey
}

\title{Géométrie algébrique}
\date{}

\begin{document}
\maketitle
\tableofcontents

On s'en fout si c'est pas dans le bon ordre.
\chapter{Espaces annelés et localement annelés}
Je viens de me rendre compte que les morphismes d'espaces localement
annelés c'est clair mais pas tant que ça. Déjà 
\begin{enumerate}
    \item \textbf{Annelés} veut juste dire $\Or_X$ est un faisceau
	d'anneau.
    \item \textbf{Localement annelés} faut rajouter que les 
	$\Or_{X,x}$ sont des anneaux locaux.
    \item Le morphisme $f^{\sharp}$ est entre $\Or_Y\to f_*\Or_X$, sinon 
	$f^{\flat}\colon f^{-1}\Or_Y\to \Or_X$.
\end{enumerate}
Donc si 
\begin{enumerate}
    \item $(X,\Or_X),(Y,\Or_Y)$ sont des espaces annelés. 
    \item $(f,f^{\sharp})\colon X\to Y$ un morphisme. Alors le
	$f^{\sharp}$ est juste un morphisme de faisceaux.
    \item Pour avoir l'intuition habituelle, on regarde localement
	annelé. C'est à dire si $f(x)=y$ alors $f^{\sharp}\colon 
	\Or_{Y,y}\to (f_*\Or_X)_y\to \Or_{X,x}$ est local :
	\[f^{\sharp}\m_y\subset\m_x.\]
\end{enumerate}
\begin{rem}
    \textbf{Et la !} En fait $(f_*\Or_X)_y$ c'est la limite des 
    $\Or_{X,x}$ pour $f(x)=y$! J'avais jamais tilt mdr trop bizarre.
\end{rem}
\newpage
\begin{rem}
    Ducoup on a ces comparaisons,
% https://q.uiver.app/#q=WzAsMyxbMCwwLCJcXHZhcmluamxpbV97eVxcaW4gVn1mXypcXE9yX1goVikiXSxbMiwwLCJcXE9yX3tYLHh9Il0sWzEsMSwiXFx2YXJpbmpsaW1fe2Zeey0xfXlcXHN1YnNldCBVfVxcT3JfWChVKSJdLFswLDEsIiIsMCx7InN0eWxlIjp7ImJvZHkiOnsibmFtZSI6ImRhc2hlZCJ9fX1dLFswLDIsIiIsMix7InN0eWxlIjp7ImJvZHkiOnsibmFtZSI6ImRhc2hlZCJ9fX1dLFsyLDEsIiIsMix7InN0eWxlIjp7ImJvZHkiOnsibmFtZSI6ImRhc2hlZCJ9fX1dXQ==
\[\begin{tikzcd}
	{\varinjlim_{y\in V}f_*\Or_X(V)} && {\Or_{X,x}} \\
	& {\varinjlim_{f^{-1}y\subset U}\Or_X(U)}
	\arrow[dashed, from=1-1, to=1-3]
	\arrow[dashed, from=1-1, to=2-2]
	\arrow[dashed, from=2-2, to=1-3]
\end{tikzcd}\]
     et les colimites sont filtrantes donc
    existent et on peut relever. Est-ce que y'a des égalités ?
\end{rem}
\chapter{Variétés abstraites}
Ducoup 
\begin{enumerate}
    \item Une variété abstraite affine c'est juste un ensemble algébrique
	affine $X$ muni du faisceau 
\chapter{Se remémorer les schémas}
\section{Topologie}
 Pour la topologie aller voir 2.1 du Gortz. Ensuite déf en 3.1/3.2
 de la topologie des schémas. 
\section{Cas $k$-schéma de type fini}

\chapter{Recoller}


\chapter{Schémas}
Quelques questions sur les définitions de schémas.
\begin{enumerate}
    \item Sous-schéma réduit est un sous-schéma ? (ça ca à l'air)
    \item Pourquoi cette déf d'immersions ouvertes/fermés ?
	Simplement pour les conditions d'avoir un faisceau
	sur l'image compatible ?
\end{enumerate}

\chapter{Le spectre maximal et le nullstellensatz}
\section{Nullstellensatz 1}
Comme dans les notes, le spectre maximal suffit pour 
les variétés parce qu'on regarde des $k$-algèbres et
qu'on a la normalisation de Noether. Ou plutôt on a le 
\textbf{nullstellensatz!} Ce qui permet de voir que 
\begin{itemize}
    \item $A\to B$ définit $Spm(A)\to Spm(B)$
\end{itemize}
Dans algebraic groups de Milne il a l'air d'en parler. Faut voir
avec la sobrification aussi?
\section{Nullstellensatz 2}
En fait je dis $2$ parce que 
\[\bigcap_{I\subset \p} \p =\sqrt{I}\]
c'est le nullstellensatz fort qui donne 
\[I(V(J))=\sqrt(J)\]

\printbibliography
\end{document}


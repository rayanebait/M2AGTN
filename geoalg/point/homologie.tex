\documentclass[a4paper,12pt]{book}
\usepackage{amsmath,  amsthm,enumerate}
\usepackage{csquotes}
\usepackage[provide=*,french]{babel}
\usepackage[dvipsnames]{xcolor}
\usepackage{quiver, tikz}

%symbole caligraphique
\usepackage{mathrsfs}

%hyperliens
\usepackage{hyperref}

%pseudo-code
\usepackage{algorithm}
\usepackage{algpseudocode}

\usepackage{fancyhdr}

\pagestyle{fancy}
\addtolength{\headwidth}{\marginparsep}
\addtolength{\headwidth}{\marginparwidth}
\renewcommand{\chaptermark}[1]{\markboth{#1}{}}
\renewcommand{\sectionmark}[1]{\markright{\thesection\ #1}}
\fancyhf{}
\fancyfoot[C]{\thepage}
\fancyhead[LO]{\textit \leftmark}
\fancyhead[RE]{\textit \rightmark}
\renewcommand{\headrulewidth}{0pt} % and the line
\fancypagestyle{plain}{%
    \fancyhead{} % get rid of headers
}

%bibliographie
\usepackage[
backend=biber,
style=alphabetic,
sorting=ynt
]{biblatex}

\addbibresource{bib.bib}

\usepackage{appendix}
\renewcommand{\appendixpagename}{Annexe}

\definecolor{wgrey}{RGB}{148, 38, 55}

\setlength\parindent{24pt}

\newcommand{\Z}{\mathbb{Z}}
\newcommand{\R}{\mathbb{R}}
\newcommand{\rel}{\omathcal{R}}
\newcommand{\Q}{\mathbb{Q}}
\newcommand{\C}{\mathbb{C}}
\newcommand{\N}{\mathbb{N}}
\newcommand{\K}{\mathbb{K}}
\newcommand{\A}{\mathbb{A}}
\newcommand{\B}{\mathcal{B}}
\newcommand{\Or}{\mathcal{O}}
\newcommand{\F}{\mathscr F}
\newcommand{\Hom}{\textrm{Hom}}
\newcommand{\disc}{\textrm{disc}}
\newcommand{\Pic}{\textrm{Pic}}
\newcommand{\End}{\textrm{End}}
\newcommand{\Spec}{\textrm{Spec}}
\newcommand{\Supp}{\textrm{Supp}}
\renewcommand{\Im}{\textrm{Im}}
\newcommand{\m}{\mathfrak{m}}
\renewcommand{\P}{\mathbb{P}}
\newcommand{\p}{\mathfrak{p}}


\newcommand{\cL}{\mathscr{L}}
\newcommand{\G}{\mathscr{G}}
\newcommand{\D}{\mathscr{D}}
\newcommand{\E}{\mathscr{E}}
\newcommand{\Po}{\mathscr{P}}
\renewcommand{\H}{\mathscr{H}}

\makeatletter
\newcommand{\colim@}[2]{%
  \vtop{\m@th\ialign{##\cr
    \hfil$#1\operator@font colim$\hfil\cr
    \noalign{\nointerlineskip\kern1.5\ex@}#2\cr
    \noalign{\nointerlineskip\kern-\ex@}\cr}}%
}
\newcommand{\colim}{%
  \mathop{\mathpalette\colim@{\rightarrowfill@\scriptscriptstyle}}\nmlimits@
}
\renewcommand{\varprojlim}{%
  \mathop{\mathpalette\varlim@{\leftarrowfill@\scriptscriptstyle}}\nmlimits@
}
\renewcommand{\varinjlim}{%
  \mathop{\mathpalette\varlim@{\rightarrowfill@\scriptscriptstyle}}\nmlimits@
}
\makeatother

\theoremstyle{plain}
\newtheorem{thm}[subsection]{Théoreme}
\newtheorem{lem}[subsection]{Lemme}
\newtheorem{prop}[subsection]{Proposition}
\newtheorem{cor}[subsection]{Corollaire}
\newtheorem{heur}{Heuristique}
\newtheorem{rem}{Remarque}
\newtheorem{note}{Note}

\theoremstyle{definition}
\newtheorem{conj}{Conjecture}
\newtheorem{prob}{Problème}
\newtheorem{quest}{Question}
\newtheorem{prot}{Protocole}
\newtheorem{algo}{Algorithme}
\newtheorem{defn}[subsection]{Définition}
\newtheorem{exmp}[subsection]{Exemples}
\newtheorem{exo}[subsection]{Exercices}
\newtheorem{ex}[subsection]{Exemple}
\newtheorem{exs}[subsection]{Exemples}
\newtheorem{res}{Résumé}
\newtheorem{rep}{Réponse}
\newtheorem{cons}{Conséquence}

\theoremstyle{remark}

\definecolor{wgrey}{RGB}{148, 38, 55}
\definecolor{wgreen}{RGB}{100, 200,0} 
\hypersetup{
    colorlinks=true,
    linkcolor=wgreen,
    urlcolor=wgrey,
    filecolor=wgrey
}

\title{Géométrie algébrique}
\date{}

\begin{document}
\maketitle
\tableofcontents



\chapter{Remarques générales}
Dans tous les autres chapitres. Dans le cas algébriquement clos, 
comme une variété algébrique abstraite est seulement isomorphe à une
variété algébrique abstraite. Ça fait sens de parler de schémas !
Enfin 
\[(\A^n(k),\Or_{\A^n(k)})\simeq (Spm(k[T_1,\ldots, T_n]),\Or_{Spm(...)})
\]
avec la topologie de Zariski. En particulier, 
\[(\A^0(k), \Or_{\A^0(k)})\simeq (Spm(k), \Or_{Spm(k)})\]

\section{Variétés vers schémas}
\subsection{Fonctions vs sections}
La construction de la correspondance schémas
affines et anneaux, avec des fonctions sur un corps algébriquement clos
c'est clair, avec des sections c'est moins clair.

Essentiellement, ça fait plus sens de dire, je prends la flèche
topologique $|f|$ et le morphisme de faisceaux s'écrit :
\[s\mapsto s\circ |f|\]
On peut juste dire, $f^\sharp(X)^{-1}(\p)=|f|(\p)$ modulo ce que ça veut
dire. En conséquence de la flèche locale 
\[\Or_{Y,|f|(x)}\to \Or_{X,x}\]

\subsection{Ambiguité du support}
Y'a un soucis avec le support de $f\in \Or_X(X)$ c'est pas analogue
vraiment à $f^{-1}(D(0))$. Quand on est nul dans une fibre en un premier
$\p_x$ ça a l'air plus fort parce que on est nul sur un ouvert autour
de $x$. Ç'est quoi ducoup $f^{-1}(0)$? Déjà 
\begin{enumerate}
    \item On a un truc qui semble similaire 
	\[\Or_X(U)=\Hom(U,\A^1_\Z)\]
	mais qui l'est pas du tout, $(0)$ est générique mais ni ouvert
	ni fermé!
    \item Sur un $k$-schéma $X$ par contre, 
	\[\Or_X(U)=\Hom_k(U,\A^1_k)\]
	d'où si $k=\bar k$, $f^{-1}((0))$ est fermé.
\end{enumerate}



\chapter{Cinquième point : Séparation}
\begin{defn}
    Une variété algébrique $X$ est séparée si 
    \[\Delta_X\colon X\to X\times_k X\]
    est une immersion fermée.
\end{defn}
\section{Petit lemme}
Quand la target est séparée, on a le critère usuel :
\[f,g\colon Z\to X\textrm{ tels que }f|_U=g|_U\textrm{ pour un ouvert
dense $U\subset Z$ alors $f=g$}\]
Essentiellement, on a deux égalités à prouver $|f|=|g|$ qui est clair
parce que $X$ est séparée. Et $f^\sharp=g^\sharp$.


\chapter{Quatrième point : produits fibrés, cas des variétés abstraites}
\section{Espace topologique sous jacent}
On cherche un pullback de
% https://q.uiver.app/#q=WzAsMyxbMCwxLCJYIl0sWzEsMCwiWSJdLFsxLDEsIlNwbShrKSJdLFswLDJdLFsxLDJdXQ==
\[\begin{tikzcd}
	& Y \\
	X & {Spm(k)}
	\arrow[from=1-2, to=2-2]
	\arrow[from=2-1, to=2-2]
\end{tikzcd}\]
en tant qu'espace topologiques on peut remarquer qu'un tel produit
\[X\times_k Y\]
vérifie $|X\times Y|=|X|\times |Y|$ par le foncteur de points. On a 
par propriété universelle
\[\Hom(W, X\times_k Y)\to \Hom(W,X)\times_k\Hom(W,Y)\]
est une bijection, en particulier si $W=Spm(k)$ on sait que 
$|X|\approx \Hom(Spm(k),X)$.

\begin{cons}
    L'espace topologique sous jaçent est bien le produit des espaces
    topologiques. À voir en fonction de la base, ici $S=Spm(k)$.
\end{cons}
\section{Cas affine}
Dans le cas affine si $X=Z(I)\subset \A^n(k)$ et $Y=Z(J)\subset \A^m(k)$
on peut par le point de vue foncteur de point regarder 
\[Z(I,J)\subset \A^{n+m}\]
Étant donné un couple \[(f,g)\in \Hom(k[T_1,\ldots, T_n]/I, \Or_W(W))
\times\Hom(k[T_{n+1},\ldots,T_{n+m}]/J, \Or_W(W))\]
on définit \[(f,g)\in \Hom(k[T_1,\ldots, T_n, T_{n+1},\ldots,T_{n+m}]
/(I,J), \Or_W(W))\]
par le morphisme évident $T_i\mapsto f(T_i)$ si $i\leq n$ et $g(T_i)$
sinon. Essentiellement par la flèche induite sur le produit tensoriel 
(adjonction).

\section{Cas général : points techniques}
Étend donné l'existence de $X\times_k Y$ c'est intéressant de 
montrer que si $W\subset X$ alors 
$W\times_k Y\hookrightarrow X\times_kY$. Et surtout on a les identités
\[p_X^{-1}(U)\cap p_Y^{-1}(V)=U\times_k V\]
et 
\[X_1\times Y\cap X_2\times Y=p_X^{-1}(X_1)\cap p_X^{-1}(X_2)=
(X_1\cap X_2)\times_k Y\]
On peut le dire parce que 
% https://q.uiver.app/#q=WzAsNixbMCwwLCJwX1heey0xfShYX2kpIl0sWzAsMSwiWF9pIl0sWzAsMiwiWCJdLFsyLDAsIlkiXSxbMSwxLCJYX2lcXHRpbWVzX2sgWSJdLFsyLDIsIlhcXHRpbWVzX2tZIl0sWzAsMV0sWzEsMiwiIiwxLHsic3R5bGUiOnsidGFpbCI6eyJuYW1lIjoiaG9vayIsInNpZGUiOiJ0b3AifX19XSxbMCwzXSxbMCw0LCJoXzEiLDEseyJzdHlsZSI6eyJib2R5Ijp7Im5hbWUiOiJkYXNoZWQifX19XSxbNCwxXSxbNCwzXSxbNSwyXSxbNSwzXSxbNCw1LCJoIiwxLHsic3R5bGUiOnsiYm9keSI6eyJuYW1lIjoiZGFzaGVkIn19fV1d
\[\begin{tikzcd}
	{p_X^{-1}(X_i)} && Y \\
	{X_i} & {X_i\times_k Y} \\
	X && {X\times_kY}
	\arrow[from=1-1, to=1-3]
	\arrow[from=1-1, to=2-1]
	\arrow["{h_1}"{description}, dashed, from=1-1, to=2-2]
	\arrow[hook, from=2-1, to=3-1]
	\arrow[from=2-2, to=1-3]
	\arrow[from=2-2, to=2-1]
	\arrow["h"{description}, dashed, from=2-2, to=3-3]
	\arrow[from=3-3, to=1-3]
	\arrow[from=3-3, to=3-1]
\end{tikzcd}\]
$h\circ h_1$ et l'inclusion sont deux flèches
$p_X^{-1}(X_i)\to X\times_kY$ qui font commuter le diagramme donc sont
égales. Pareil $h_1\circ h\circ h_1$  et $h_1$ c'est deux flèches 
\[p_X^{-1}(X_i) \to X_i\times_kY\].
donc sont égales par unicité, mais c'est pas clair que $h_1$ est un mono.

\begin{rem}
    J'ai pas montré l'existence de tout les autres produits fibrés là.
    Juste qu'ils coincident avec ce que j'ai dit.
\end{rem}
Pour montrer leur existence faut juste utiliser la commutativité des
flèches. Si on a 
% https://q.uiver.app/#q=WzAsNixbMCwwLCJXIl0sWzAsMSwiVSJdLFswLDIsIlgiXSxbMiwwLCJZIl0sWzEsMCwiViJdLFsyLDIsIlhcXHRpbWVzX2sgWSJdLFsxLDIsIiIsMSx7InN0eWxlIjp7InRhaWwiOnsibmFtZSI6Imhvb2siLCJzaWRlIjoidG9wIn19fV0sWzAsNCwiZiIsMV0sWzAsMSwiZyIsMV0sWzQsMywiIiwxLHsic3R5bGUiOnsidGFpbCI6eyJuYW1lIjoiaG9vayIsInNpZGUiOiJ0b3AifX19XSxbNSwzXSxbNSwyXSxbMCw1LCJoIiwxLHsic3R5bGUiOnsiYm9keSI6eyJuYW1lIjoiZGFzaGVkIn19fV1d
\[\begin{tikzcd}
	W & V & Y \\
	U \\
	X && {X\times_k Y}
	\arrow["f"{description}, from=1-1, to=1-2]
	\arrow["g"{description}, from=1-1, to=2-1]
	\arrow["h"{description}, dashed, from=1-1, to=3-3]
	\arrow[hook, from=1-2, to=1-3]
	\arrow[hook, from=2-1, to=3-1]
	\arrow[from=3-3, to=1-3]
	\arrow[from=3-3, to=3-1]
\end{tikzcd}\]
suffit de voir que $h(W)\subseteq p_X^{-1}(U)\cap p_X^{-1}(V)$ via
$p_X\circ h=g$ et $p_Y\circ h=f$.

C'est aussi intéressant de voir que si $U\hookrightarrow X$ est une
immersion ouverte alors
\[U\times_k Y\to X\times_k Y\]
aussi (directement via les points d'avant).
\section{Construction générale}
Étant donné le fait que $X_i\times Y\cap X_j\cap Y=(X_i\cap X_j)\times Y$
on peut recoller ! Et c'est ça la construction mdr. 

\begin{prop}
    La dimension de $X\times_k Y$ c'est $\dim(X)+\dim(Y)$.
\end{prop}
\begin{proof}
    On se ramène facilement au cas affine puis par Noether c'est clair.
\end{proof}


\chapter{Troisième point : dimension}

\section{Les définitions}
\subsection{Définition topologique}
Concrètement pour $X=\cup_{i=1}^n X_i$ une variété algébrique décomposée
en composantes irréductibles :
\[\dim(X)=\max(\dim(X_i))=\sup_{U\subset X~ouvert} \dim(U)\]
et pour une variété irréductible c'est le sup des longueurs de chaines
\[Y_0\subsetneq Y_1\subsetneq \ldots \subsetneq Y_d=X\]

\subsection{Définition par les corps de fonctions}
Ducoup dans le cas intègre, les restrictions du faisceau sont 
injectives et le faisceau est approximable par les ouverts principaux
affines ! En particulier 
\[k(X)\simeq k(U)\]
et \[k(X)\simeq Frac(\Or_X(U_0))\]
pour un affine ouvert $U_0\subset X$ quelconque. On peut montrer que
\[dim(X)= degtr_k k(X)\]
et c'est bien défini :
\begin{enumerate}
    \item Si on prends deux familles algébriquement 
indépendantes et $K$ algébriques sur les deux, on peut montrer qu'elles
ont la même cardinalité. 
    \item On peut se réduire au cas affine.
    \item On conclut par l'injection de Noether dans $A(X)$ qui fixe la
	dimension en passant au corps de fractions!
\end{enumerate}

\section{Hypersurfaces}
\subsection{Cas affine}
Essentiellement, y'a cette suite d'arguments :
\begin{enumerate}
    \item La dimension est invariante par extension d'anneaux entiers.
	(Y'a pas mal d'algèbre là dedans, j'en parlerai ailleurs)
    \item Par Noether si $F\in k[T_1,\ldots, T_n]-k$ alors 
	\[\dim k[T_1,\ldots, T_n]/(F)=\dim k[T_1,\ldots, T_{n-1}]\]
    \item Ensuite $\dim k[T_1,\ldots,T_n]=n$ par récurrence et 
	l'argument d'avant (faut faire un tout petit peu attention).
    \item Automatiquement, si $F\in k[T_1,\ldots, T_n]$ alors
	$\dim(Z(F))=n-1$.
\end{enumerate}
\subsection{Cas intègre}
Ça c'était le cas affine, maintenant le cas intègre : Étant donné
$f\in \Or_X(X)$ on a
\[\dim(Z(f))=\dim(X)-1\]
La preuve consiste à dire 
\begin{enumerate}
    \item $\dim(U)=\dim(X)$ en utilisant $k(U)\simeq k(X)$ d'où
	on se ramène au cas affine.
    \item On à une injection entière finie 
	\[k[T_1,\ldots, T_n]/fA\cap k[T_1,\ldots, T_n]\hookrightarrow
	A(X)/fA(X)\]
	où $fA\cap k[T_1,\ldots, T_n]$ c'est juste en identifiant avec
	l'image.
    \item Puis on a 
	\[fA\cap k[T_1,\ldots, T_n]\subset \sqrt{N_{k(X)/k(\A^n)}(f)}
	\subset \sqrt{fA\cap k[T_1,\ldots, T_n]}\]
    et on conclut par Noether.
\end{enumerate}
\begin{rem}
    Les anneaux de polynômes sont factoriels donc intégralement clos.
    D'où la norme fonctionne bien là.
\end{rem}
\begin{rem}
    Je sais vraiment pas si on est obligés d'utiliser $k(U)\simeq k(X)$
    mdr. À méditer. Si $F_1\cap U \subsetneq F_2\cap U$ et $U$ dense
    dans $X$, alors $\bar F_1 = \bar F_2$ implique $F_1$ dense dans 
    $F_2$ d'où y sont égaux dans $U$ car fermés ? 
\end{rem}
\subsection{Nombre d'équations d'un fermé}
Tout irréductible affine $Z$ de dimension $s$ dans $\A^n$ est une 
composante d'un
\[Z\subseteq Z(f_1,\ldots, f_{n-s})\]
dont toutes les composantes ont dimension $s$. 

À l'inverse
\[Z(f_1,\ldots, f_{s})\subset \A^n(k)\]
est de dimension $\geq n-s$.
\subsection{Dimension des fibres}
On en déduit que si $f\colon X\to Y$ est dominant alors
\[\dim(f^{-1}(y))\geq \dim(X)-\dim(Y)\]
parce que $f^{-1}(y)\subseteq Z(f_*\m_y)$ et $\m_y$ est défini
par $\dim(Y)$ équations !
\chapter{Deuxième point sur les cours : Variétés abstraites à la Weil}
\section{Premières définitions}
Ducoup essentiellement maintenant on travaille avec des 
\[(X,\Or_X)\]
ou $X$ est un ensemble algébrique affine et $\Or_X$ le faisceau de
fonctions régulières. Plus généralement
\[X=\cup X_i\]
une union finie d'affines et $\Or_X$ est le faisceau qui étend les 
$\Or_{X_i}$.
\begin{res}
 Pour tout définir on procède comme suit :
    \begin{enumerate}
	\item Équivalence de catégories entres variétés algébriques affines
	    et variétés algébriques abstraites affines.
	\item Les $D(f)$ sont des variétés abstraites.
	\item Définition d'une variété abstraite par réunion d'ouverts 
	    affines.
	\item Les ouverts d'affines sont des variétés abstraites.
	\item Les ouverts sont des variétés abstraites.
	\item la bijection :
    \end{enumerate}
\end{res}
\begin{res}
Le détail des preuves discute :
    \begin{enumerate}
	\item Comment décrire des morphismes de variétés abstraites.
	\item Bien décrire algébriquement les $D(r)$.
	\item La bijection
	    \[\Hom(X,Y)\to \Hom(\Or_Y(Y),\Or_X(X))\]
	    où $Y$ est affine.
    \end{enumerate}
\end{res}

À nouveau, on a 
\[Z(I)\cap D(F(\bar T))=X\cap D(F)\simeq Z(I,FS-1)\subset \A^{n+1}\]
en tant que variétés algébriques abstraites. La preuve du cours 
consiste à dire 
\[ i\colon(X\cap D(F),\Or_X|_{D(F)})\to Z(I,FS-1)\]
a pour inverse la projection $p\colon \A^{n+1}\to \A^n$ et il montre
que c'est un homéomorphisme à la main en disant que c'est ouvert par :
\begin{enumerate}
    \item Si on regarde $D(h)\cap Z(I,FS-1)$, alors 
	$h(\bar T, S)\mod FS-1$ s'écrit $h(\bar T, 1/F)$.
    \item On en déduit $F^N h(\bar T, S) \mod FS-1$ a un représentant
	$R$ dépendant pas de $S$. 
    \item Par déf $p(D(h)\cap Z(I, FS-1)=D(R)\cap X$.
\end{enumerate}
L'isomorphisme de faisceaux, on peut vouloir conclure directement en 
utilisant l'équivalence de catégorie $2$ puis $1$. Mais ça donne juste 
les morphismes de sections globales et je sais pas pq il veut pas
utiliser le fait que $p$ et $i$ sont régulières.

\begin{rem}
    J'ai compris l'intérêt de faire la preuve comme ça.. C'est juste
    que pour l'iso de faisceaux, si on veut se restreindre aux carrés
% https://q.uiver.app/#q=WzAsNCxbMSwwLCJBKFooSSxGUy0xKSkiXSxbMCwwLCJBKFgpX2YiXSxbMCwxLCJcXE9yX1goRChyKSkiXSxbMSwxLCJcXE9yX3taKC4uLil9KEQoUikpIl0sWzEsMF0sWzEsMl0sWzIsM10sWzAsM11d
\[\begin{tikzcd}
	{A(X)_f} & {A(Z(I,FS-1))} \\
	{\Or_X(D(r))} & {\Or_{Z(...)}(D(R))}
	\arrow[from=1-1, to=1-2]
	\arrow[from=1-1, to=2-1]
	\arrow[from=1-2, to=2-2]
	\arrow[from=2-1, to=2-2]
\end{tikzcd}\]
    faut bien montrer que l'image d'un principal c'est un principal.
    Le reste de la preuve paraît tellement compliqué pour rien mdr, 
    mais ça a l'air nécessaire : essentiellement, on peut tjr écrire
    $\Or_X(D(r))=A(X)[W]/(rW-1)$ et $D(r)\subset D(f)$ implique
    $f$ inversible dans $\Or_X(D(r))$ puis $A_f\simeq A$ si $f$
    est inversible dans $A$. Je mettrai pas tjr tout les détails.
\end{rem}
On peut définir les variétés abstraites par union finie:
\[(X,\Or_X)=\cup (X_i,\Or_X|_{X_i})\]
On en déduit que les ouverts sont bien des variétés. Enfin on a 
\[\Hom(X,Y)\to \Hom(\Or_Y(Y),\Or_X(X))\]
pour $Y$ affine et $X$ quelconque par recollement des flèches affines.
La continuité c'est le point flou mais qui est en fait le plus facile,
le recollement des faisceaux est clair.

\section{Recollement de variétés}
Étant donné des $(X_i)_i$ est des ouverts $U_{ij}\subset X_i$ tels que
\[\phi_{ij}\colon U_{ij}\simeq U_{ji}\]
et que les $\phi_{ij}$ engendrent une relation d'équivalence sur 
\[\sqcup X_i\]
alors le quotient est une variété définit par les ouverts $X_i$.
\begin{quest}
    Pourquoi les $X_i$ sont ouverts?
\end{quest}

\begin{rep}
    C'est \textbf{PAS} juste que la projection est ouverte pour la
    topologie quotient, puisque c'est pas tjr vrai. C'est plutôt que
    la classe d'équivalence de $X_i$ c'est $X_i \sqcup_j U_ji$ qui est
    ouvert.
\end{rep}
\begin{rem}
    La topologie finale pour la projection, par déf
    la topologie la plus fine sur le quotient qui rend la projection
    continue, en particulier la projection devient ouverte quand
    \[p^{-1}p(U)\]
    est ouvert autrement dit la classe d'équivalence d'un ouvert
    est ouvert. C'est le cas ici, ce serait le cas avec des 
    groupes/anneaux topologiques j'imagine et pour toute topologie
    "équilibrée".
\end{rem}
\section{Sous-variétés}
Concrètement, les sous variétés c'est immersions ouvertes ou fermées,
le cas des ouverts est clair. Pour les fermés, la version faiscautique :
\[\textrm{On prends un fermé $Z$ de $X$ et le pullback }i^*\Or_X\]
La traduction c'est que les fonctions sur $Z$ c'est localement des
restrictions de fonctions sur $X$.
\begin{rem}
    Sur les affines c'est même pas localement ça se recolle bien.
    On peut regarder les fibres, si $Z$ et $X$ sont affines on a 
    \[i^{-1}\Or_X(Z)_x=\Or_{X,i(x)}=A(X)_{\m_x}\]
    et en tensorisant par $\Or_Z(Z)=A(Z)$ on obtient que $i^*\Or_X$
    a les mêmes fibres que $\Or_Z$ avec un isomorphisme naturel.
\end{rem}
\begin{quest}
    Cas ou $X$ est un recollement de deux affines ?
\end{quest}
\section{Variétés quasi-projectives abstraites}
On définit sur $Z\subset \P^n(k)$ le faisceau donné par le faisceau
de fonctions régulières muni des recollements donnés par les cartes 
affines. En gros, avec le Proj : à faire,
\begin{enumerate}
    \item En gros le faisceau du Proj est bien un faisceau.
    \item Ducoup revoir un peu le Proj.
    \item C'est bien le faisceau de fonctions régulières.
    \item Les $Z^+(I)$ avec le faisceau $\Or_X$ de fonctions réguières
	est une variété.
\end{enumerate}
Ducoup il se passe un truc marrant
\[\textrm{Les ouverts du type $D^+(P)$ sont affines !}\]
L'idée c'est simplement d'utiliser le Veronese puis de montrer que 
$D^+(H)\simeq D^+(T_i)$ est affine. Pour ça suffit de montrer que
les morphismes de pullbacks 
\[(\phi_i)_*(U)\colon \Or_{\A^n_k}(U)\to \Or_{D^+(T_i)}(\phi_i^{-1}(U))\]
sont bien définis, c'est clair via la définition locale.  
\subsection{Le faisceau de fonctions régulières sur une variété
projective}
J'ai eu une idée ducoup, si on prends une variété projective $X=Z^+(I)$
et $P=p\mod I$ alors si $U=D^+(p)$ :
\[\Or_X(U)=(k[T_0,\ldots, T_n]/I)_{(p)}\]
Mon idée pour justifier la preuve c'est : si $g\in \Or_X(U)$ et 
\[g|_{U_i}=P_i/Q_i\]
 alors où est-ce qu'on peut regarder $gP_i=Q_i$? Je me disais on peut 
 regarder dans 
 \[S(X)=A(X')\]
 avec $X'$ le cone de $X$ dans $\A^{n+1}$. Mon guess c'est :
\[\Or_X(U)=A(X')_{(p)}\]
et dans ce cas, l'identité $gQ_i^2=P_iQ_i$ fait sens dans $A(X')$
parce que $D(p)$ est dense $Z(gQ_i^2-P_iQ_i)$.


\chapter{Point sur le chapitre I : Variétés algébriques classiques}
\section{Résumé}
\section{Framework}
Quand y s'agit de trouver des fermés projectifs ou des relations :
\begin{enumerate}
    \item Ajouter des relations adaptées à la situation jusqu'à obtenir
	le vide.
    \item Utiliser le nullstellensatz projectif.
    \item Faire de l'algèbre linéaire.
\end{enumerate}
pour le premier point une utilisation cool c'est de prouver que des
morphismes sont finis. Typiquement les projections linéaires.

Quand y s'agit de montrer des isomorphismes. La partie 
homéomorphisme est souvent claire :
\begin{enumerate}
    \item Tel point est dans tel ouvert ssi il vérifie telle ou telle
	relations.
\end{enumerate}
La partie isomorphisme est moins claire, faut montrer que la flèche
inverse est régulière et là c'est assez ad hoc.


\section{$k$-algèbres et Noether}
J'ai essayé de prouver la normalisation de Noether qui dit que
\begin{thm}[Normalisation de Noether]
    Une $k$-algèbre de type fini $A$ est entière sur un $k[T_1,\ldots,
    T_d]$.
\end{thm}
Et en fait y'a un
truc intéressant. Ma stratégie c'était:
\begin{enumerate}
    \item On suppose $k[T_1,\ldots,T_n]/I$
avec $n$ minimal alors $I$ petit.
    \item On peut supp $I$ premier, on se ramene au cas réduit 
	trivialement et au cas intègre avec le lemme chinois.
\end{enumerate}
Maintenant on peut regarder dans le corps de fractions une famille
alg indép maximale, enfin c'est ce que j'aurai aimé. Mais c'est pas 
clair que elle est de taille $n-1$! C'est là tout le pb. La raison 
c'est que :
\begin{center}
    Les variétés, par exemple les courbes ont pas nécessairement de
    modèles dans $\P^2$ non singuliers. D'où on peut rien dire sur 
    le nombre de générateurs de $I$ facilement (si $n$ minimal implique
    $I$ minimal et $dim(C)=1$ alors $I$ est engendré par un élément 
    et on est dans $\P^2$).
\end{center}

Bon la vraie preuve mtn :
\begin{itemize}
    \item Essentiellement, on veut juste montrer que pour chaque
	générateur $P$ on a un automorphisme qui le rend unitaire en
	une des variables. D'où le résultat par récurrence.
\end{itemize}
Deux questions maintenant :
\begin{enumerate}
    \item Pq à automorphisme près ça suffit ?
    \item Quel automorphisme ?
\end{enumerate}
Pour la première question ducoup y'a un ou deux petit trick mentaux
qu'il mentionne pas.
\begin{enumerate}
    \item Quand on obtient une relation unitaire pour $P$, on fait une 
	récurrence pour trouver l'injection entière $k[T_1,\ldots,T_d]$.
	Et surtout être entier c'est transitif.
    \item L'injection ça peut être $T_1\mapsto L_1(T_1,\ldots, T_d)$
	avec les $L_i$ de degré $1$ pas nécessairement $T_i$!
\end{enumerate}
Pour la deuxième question : on commence par regarder comment rendre
$T_1\ldots T_n$ unitaire en $T_n$ par exemple. On peut regarder
l'automorphisme 
\[T_i\mapsto T_i+T_1;T_n\mapsto T_n\]
C'est clair que le résultat est unitaire en $T_n$ ! Post-automorphisme
l'injection de $k[T_1,\ldots,T_n]$ est \textbf{l'inclusion}, 
pré-automorphisme faut composer avec l'automorphisme inverse. Bon
maintenant en général ça marche pas forcément ça faut tricker un peu,
parce qu'on a plusieurs monomes.
On peut prendre
\[T_i\mapsto T_i+T_n^{m_i}\]
et jouer sur le tuple $m=(m_1,\ldots,m_{n-1},1)$, mais c'est juste
de l'écriture.

Maintenant le nullstellensatz ! 
\section{Nullstellensatz}
\begin{cor}[Nullstellensatz faible]
    Si $A$ est une $k$-algèbre tf et $\m$ est maximal. Alors 
    $A/\m$ est une extension finie de $k$.
\end{cor}
La preuve consiste à simplement se rendre compte que 
\[k[T_1,\ldots,T_d]\hookrightarrow A\]
entier implique $k[T_1,\ldots, T_d]$ est un corps d'où $d=0$.
\begin{thm}[Nullstellensatz fort]
    Soit $A$ une $k$-algèbre tf, alors pour tout $I$
    \[\sqrt I =\cap_{I\subset \m}\m\]
\end{thm}
Une idée de preuve c'est de remarquer que si $\cap \m$ contient des
non nilpotents, disons $f$. Alors pour $\m'\subset A_f$ avec 
$\delta\colon A\to A_f$ non nul :
\[k\mapsto A/\delta^{-1}\m'\to A_f/\m'\]
est algébrique d'où la deuxième flèche est entière d'où $\delta^{-1}\m'$
est maximal contenant pas $f$ ce qui est contradictoire. À noter
qu'on utilise $A_f\simeq A[T]/(fT-1)$ pour montrer que c'est tf.
\begin{quest}
    De manière constructive? Déjà, les localisés $A_\m$ sont pas t.f sur
    $k$ mdr.
\end{quest}

En particulier on obtient la correspondance avec le spectre maximal
muni de sa topologie de Zariski. On en déduit direct :
\begin{thm}[Nullstellensatz fort 2]
    Sur un corps algébriquement clos, $I(Z(J))=\sqrt J$ pour tout idéal
    $J$ de $k[T_1,\ldots, T_n]$.
\end{thm}

On a maintenant une correspondance entre ensembles algébriques et
idéaux radicaux quand $k$ est \textbf{algébriquement clos}.
\section{Topologie et irréductibles}
Essentiellement, dans un anneau noethérien, on a 
\[\textrm{Tout idéal radical } I \textrm{ est intersection finie 
d'idéaux premiers.}\]
En particulier, on déduit la décomposition finie unique en irréductible
avec la correspondance de la section d'avant. La preuve se fait
bien par contradication et maximalité vu qu'on est dans le cas
Noethérien.
\begin{rem}
    Un critère d'irréducibilité c'est que les ouverts s'intersectent
    tout court.
\end{rem}
\begin{rem}
    La topologie induite sur un \textbf{sous-ensemble} existe toujours.
\end{rem}

\section{Espaces projectifs}
Ducoup les définitions à retenir :
\begin{enumerate}
    \item Un idéal homogène c'est un truc de la forme $\oplus A_d\cap I$.
	C'est à dire que toute les composantes homogènes d'un $f\in I$ 
	sont dans $I$.
    \item Les $Z^+(I)$ sont défs par les zéros de polynômes homogènes.
	De manière équivalentes par la projection de
	$Z(I)\subset \A^{n+1}$.
    \item À l'inverse $I^+(Z)=I(\pi^{-1}Z)$. 
\end{enumerate}
Maintenant le nullstellensatz devient :
\begin{center}
    Les idéaux homogènes différents de $(T_0,\ldots,T_n)$ correspondent
    aux variétés projectives.
\end{center}
un idéal définit le vide si et seulement si il contient
\[(T_0,\ldots,T_n)^s\] pour un $s$.

On obtient le générateur de relation (ce nullstellensatz)! À détailler un
peu plus avec la preuve que les projections sont entières !

\section{Fonctions régulières}
On prends toujours la définition locale sur les variétés 
quasi-projectives et même algébriques. Y'a toujours un élément de preuve
intéressant dans 
\[A(Z)_f\simeq \Or_Z(D(f))\]
sur les variétés affines pour $f\in A(Z)$. On recouvre $D(f)$ par des
$D(f_i)=D(f_i^2)$ avec $g\in \Or_Z(D(f))$ et $g|_{U_i}=g_i/f_i$.
En particulier, 
\[f^r=\sum a_if_i^2\]
puis $gf^r=\sum a_ig_if_i$ d'où la surjectivité de 
$A(Z)_f\to \Or_Z(D(f))$! 

\section{Morphismes de variétés affines, première déf}
Dans le cas des variétés affines : on définit littéralement via 
\begin{center}
    Des restrictions de morphismes $\A^n\to \A^m$.
\end{center}
On obtient directement les flèches 
% https://q.uiver.app/#q=WzAsNixbMCwxLCJrW1NfMSxcXGxkb3RzLFNfbV0iXSxbMSwxLCJrW1RfMSxcXGxkb3RzLFRfbl0iXSxbMCwyLCJBKFkpIl0sWzEsMiwiQShYKSJdLFswLDAsIlNfaiJdLFsxLDAsIlxccGhpX2koVF9qLGopIl0sWzAsMV0sWzIsM10sWzAsMl0sWzEsM10sWzQsNSwiIiwwLHsic3R5bGUiOnsidGFpbCI6eyJuYW1lIjoibWFwcyB0byJ9fX1dXQ==
\[\begin{tikzcd}
	{S_j} & {\phi_i(T_j,j)} \\
	{k[S_1,\ldots,S_m]} & {k[T_1,\ldots,T_n]} \\
	{A(Y)} & {A(X)}
	\arrow[maps to, from=1-1, to=1-2]
	\arrow[from=2-1, to=2-2]
	\arrow[from=2-1, to=3-1]
	\arrow[from=2-2, to=3-2]
	\arrow[from=3-1, to=3-2]
\end{tikzcd}\]
de l'équivalence de catégorie. Il faut juste prouver que de 
$A(Y)\to A(X)$ on peut relever un morphismes de variétés.


\chapter{Plan du cours}
Ici le but c'est de faire un sommaire du cours vu qu'il est pas très 
clair. Essentiellement faut maitriser de l'algèbre pour l'instant.


\section{Variétés affines}
\subsection{Premières définitions}

\begin{enumerate}
    \item Ensembles algébriques affines.
    \item Correspondance avec les idéaux.
    \item Lemme 1.1.10 : indice au weak nullstellensatz.
    \item 1.1.11 : Énoncé du nullstellensatz et corollaires.
\end{enumerate}
\subsection{Nullstellensatz}
\begin{enumerate}
    \item Normalisation de noether.
    \item Nullstellensatz faible.
    \item Nullstellensatz fort.
    \item Preuve de la correspondance idéaux radicaux et ensembles
	algébriques.
\end{enumerate}

\subsection{Topologie et irréductibles}
\begin{enumerate}
    \item Caractérisation des irréductibles via la correspondance.
    \item Via les ouverts.
    \item Décomposition unique des idéaux radicaux en idéaux premiers.
    \item Décomposition unique des ensembles algébriques en 
	irréductibles.
\end{enumerate}
Y'a pleins d'exos.

\subsection{Espaces projectifs}
\begin{enumerate}
    \item Définitions de la topologie.
    \item L'idéal "irrelevant".
\end{enumerate}
\subsection{Fonctions régulières}
\begin{enumerate}
    \item Intuitions sur comment ça doit marcher sur les affines.
    \item Définition locale.
    \item Continuité.
    \item Sur $D(f)$.
\end{enumerate}
\subsection{Morphismes d'ensembles algébriques}
\begin{enumerate}
    \item Intuitions.
    \item Définition intuitive.
    \item Équivalence de catégorie avec les $k$-algèbres réduites.
    \item Le corollaire : deux ensembles algébriques affines sont 
	isomorphes ssi leurs algèbres de fonctions le sont.
\end{enumerate}
Exercices.

\section{Variétés affines abstraites}
\subsection{Espaces annelés}
Ça ressemble au 1.1 au début.
\begin{enumerate}
    \item Définition et espaces localement annelés.
    \item Variétés algébriques affines abstraites.
    \item Équivalence de catégorie entres variétés affines et variétés
	affines abstraites.
\end{enumerate}



\printbibliography
\end{document}


\documentclass[a4paper,12pt]{book}
\usepackage{amsmath,  amsthm,enumerate}
\usepackage{csquotes}
\usepackage[provide=*,french]{babel}
\usepackage[dvipsnames]{xcolor}
\usepackage{quiver, tikz}

%symbole caligraphique
\usepackage{mathrsfs}

%hyperliens
\usepackage{hyperref}

%pseudo-code
\usepackage{algorithm}
\usepackage{algpseudocode}

\usepackage{fancyhdr}

\pagestyle{fancy}
\addtolength{\headwidth}{\marginparsep}
\addtolength{\headwidth}{\marginparwidth}
\renewcommand{\chaptermark}[1]{\markboth{#1}{}}
\renewcommand{\sectionmark}[1]{\markright{\thesection\ #1}}
\fancyhf{}
\fancyfoot[C]{\thepage}
\fancyhead[LO]{\textit \leftmark}
\fancyhead[RE]{\textit \rightmark}
\renewcommand{\headrulewidth}{0pt} % and the line
\fancypagestyle{plain}{%
    \fancyhead{} % get rid of headers
}

%bibliographie
\usepackage[
backend=biber,
style=alphabetic,
sorting=ynt
]{biblatex}

\addbibresource{bib.bib}

\usepackage{appendix}
\renewcommand{\appendixpagename}{Annexe}

\definecolor{wgrey}{RGB}{148, 38, 55}

\setlength\parindent{24pt}

\newcommand{\Z}{\mathbb{Z}}
\newcommand{\R}{\mathbb{R}}
\newcommand{\rel}{\omathcal{R}}
\newcommand{\Q}{\mathbb{Q}}
\newcommand{\C}{\mathbb{C}}
\newcommand{\N}{\mathbb{N}}
\newcommand{\K}{\mathbb{K}}
\newcommand{\A}{\mathbb{A}}
\newcommand{\B}{\mathcal{B}}
\newcommand{\Or}{\mathcal{O}}
\newcommand{\F}{\mathscr F}
\newcommand{\Hom}{\textrm{Hom}}
\newcommand{\disc}{\textrm{disc}}
\newcommand{\Pic}{\textrm{Pic}}
\newcommand{\End}{\textrm{End}}
\newcommand{\Spec}{\textrm{Spec}}
\newcommand{\Supp}{\textrm{Supp}}
\renewcommand{\Im}{\textrm{Im}}
\newcommand{\m}{\mathfrak{m}}
\renewcommand{\P}{\mathbb{P}}
\newcommand{\p}{\mathfrak{p}}


\newcommand{\cL}{\mathscr{L}}
\newcommand{\G}{\mathscr{G}}
\newcommand{\D}{\mathscr{D}}
\newcommand{\E}{\mathscr{E}}
\newcommand{\Po}{\mathscr{P}}
\renewcommand{\H}{\mathscr{H}}

\makeatletter
\newcommand{\colim@}[2]{%
  \vtop{\m@th\ialign{##\cr
    \hfil$#1\operator@font colim$\hfil\cr
    \noalign{\nointerlineskip\kern1.5\ex@}#2\cr
    \noalign{\nointerlineskip\kern-\ex@}\cr}}%
}
\newcommand{\colim}{%
  \mathop{\mathpalette\colim@{\rightarrowfill@\scriptscriptstyle}}\nmlimits@
}
\renewcommand{\varprojlim}{%
  \mathop{\mathpalette\varlim@{\leftarrowfill@\scriptscriptstyle}}\nmlimits@
}
\renewcommand{\varinjlim}{%
  \mathop{\mathpalette\varlim@{\rightarrowfill@\scriptscriptstyle}}\nmlimits@
}
\makeatother

\theoremstyle{plain}
\newtheorem{thm}[subsection]{Théoreme}
\newtheorem{lem}[subsection]{Lemme}
\newtheorem{prop}[subsection]{Proposition}
\newtheorem{cor}[subsection]{Corollaire}
\newtheorem{heur}{Heuristique}
\newtheorem{rem}{Remarque}
\newtheorem{note}{Note}

\theoremstyle{definition}
\newtheorem{conj}{Conjecture}
\newtheorem{prob}{Problème}
\newtheorem{quest}{Question}
\newtheorem{prot}{Protocole}
\newtheorem{algo}{Algorithme}
\newtheorem{defn}[subsection]{Définition}
\newtheorem{exmp}[subsection]{Exemples}
\newtheorem{exo}[subsection]{Exercices}
\newtheorem{ex}[subsection]{Exemple}
\newtheorem{exs}[subsection]{Exemples}
\newtheorem{res}{Résumé}
\newtheorem{rep}{Réponse}

\theoremstyle{remark}

\definecolor{wgrey}{RGB}{148, 38, 55}
\definecolor{wgreen}{RGB}{100, 200,0} 
\hypersetup{
    colorlinks=true,
    linkcolor=wgreen,
    urlcolor=wgrey,
    filecolor=wgrey
}

\title{Géométrie algébrique}
\date{}

\begin{document}
\maketitle
\tableofcontents


\chapter{Plan du cours}
Ici le but c'est de faire un sommaire du cours vu qu'il est pas très 
clair. Essentiellement faut maitriser de l'algèbre pour l'instant.


\section{Variétés affines}
\subsection{Premières définitions}

\begin{enumerate}
    \item Ensembles algébriques affines.
    \item Correspondance avec les idéaux.
    \item Lemme 1.1.10 : indice au weak nullstellensatz.
    \item 1.1.11 : Énoncé du nullstellensatz et corollaires.
\end{enumerate}
\subsection{Nullstellensatz}
\begin{enumerate}
    \item Normalisation de noether.
    \item Nullstellensatz faible.
    \item Nullstellensatz fort.
    \item Preuve de la correspondance idéaux radicaux et ensembles
	algébriques.
\end{enumerate}

\subsection{Topologie et irréductibles}
\begin{enumerate}
    \item Caractérisation des irréductibles via la correspondance.
    \item Via les ouverts.
    \item Décomposition unique des idéaux radicaux en idéaux premiers.
    \item Décomposition unique des ensembles algébriques en 
	irréductibles.
\end{enumerate}
Y'a pleins d'exos.

\subsection{Espaces projectifs}
\begin{enumerate}
    \item Définitions de la topologie.
    \item L'idéal "irrelevant".
\end{enumerate}
\subsection{Fonctions régulières}
\begin{enumerate}
    \item Intuitions sur comment ça doit marcher sur les affines.
    \item Définition locale.
    \item Continuité.
    \item Sur $D(f)$.
\end{enumerate}
\subsection{Morphismes d'ensembles algébriques}
\begin{enumerate}
    \item Intuitions.
    \item Définition intuitive.
    \item Équivalence de catégorie avec les $k$-algèbres réduites.
    \item Le corollaire : deux ensembles algébriques affines sont 
	isomorphes ssi leurs algèbres de fonctions le sont.
\end{enumerate}
Exercices.

\section{Variétés affines abstraites}
\subsection{Espaces annelés}
Ça ressemble au 1.1 au début.
\begin{enumerate}
    \item Définition et espaces localement annelés.
    \item Variétés algébriques affines abstraites.
    \item Équivalence de catégorie entres variétés affines et variétés
	affines abstraites.
\end{enumerate}

\chapter{Partie III}
On commence à defs des invariants comme la dimension.

\chapter{Partie II}
\section{Premières définitions}
Ducoup essentiellement maintenant on travaille avec des 
\[(X,\Or_X)\]
ou $X$ est un ensemble algébrique affine et $\Or_X$ le faisceau de
fonctions régulières. Plus généralement
\[X=\cup X_i\]
une union finie d'affines et $\Or_X$ est le faisceau qui étend les 
$\Or_{X_i}$.
\begin{res}
 Pour tout définir on procède comme suit :
    \begin{enumerate}
	\item Équivalence de catégories entres variétés algébriques affines
	    et variétés algébriques abstraites affines.
	\item Les $D(f)$ sont des variétés abstraites.
	\item Définition d'une variété abstraite par réunion d'ouverts 
	    affines.
	\item Les ouverts d'affines sont des variétés abstraites.
	\item Les ouverts sont des variétés abstraites.
	\item la bijection :
    \end{enumerate}
\end{res}
\begin{res}
Le détail des preuves discute :
    \begin{enumerate}
	\item Comment décrire des morphismes de variétés abstraites.
	\item Bien décrire algébriquement les $D(r)$.
	\item La bijection
	    \[\Hom(X,Y)\to \Hom(\Or_Y(Y),\Or_X(X))\]
	    où $Y$ est affine.
    \end{enumerate}
\end{res}

À nouveau, on a 
\[Z(I)\cap D(F(\bar T))=X\cap D(F)\simeq Z(I,FS-1)\subset \A^{n+1}\]
en tant que variétés algébriques abstraites. La preuve du cours 
consiste à dire 
\[ i\colon(X\cap D(F),\Or_X|_{D(F)})\to Z(I,FS-1)\]
a pour inverse la projection $p\colon \A^{n+1}\to \A^n$ et il montre
que c'est un homéomorphisme à la main en disant que c'est ouvert par :
\begin{enumerate}
    \item Si on regarde $D(h)\cap Z(I,FS-1)$, alors 
	$h(\bar T, S)\mod FS-1$ s'écrit $h(\bar T, 1/F)$.
    \item On en déduit $F^N h(\bar T, S) \mod FS-1$ a un représentant
	$R$ dépendant pas de $S$. 
    \item Par déf $p(D(h)\cap Z(I, FS-1)=D(R)\cap X$.
\end{enumerate}
L'isomorphisme de faisceaux, on peut vouloir conclure directement en 
utilisant l'équivalence de catégorie $2$ puis $1$. Mais ça donne juste 
les morphismes de sections globales et je sais pas pq il veut pas
utiliser le fait que $p$ est $i$ sont régulières.

\begin{rem}
    J'ai compris l'intérêt de faire la preuve comme ça.. C'est juste
    que pour l'iso de faisceaux, si on veut se restreindre aux carrés
% https://q.uiver.app/#q=WzAsNCxbMSwwLCJBKFooSSxGUy0xKSkiXSxbMCwwLCJBKFgpX2YiXSxbMCwxLCJcXE9yX1goRChyKSkiXSxbMSwxLCJcXE9yX3taKC4uLil9KEQoUikpIl0sWzEsMF0sWzEsMl0sWzIsM10sWzAsM11d
\[\begin{tikzcd}
	{A(X)_f} & {A(Z(I,FS-1))} \\
	{\Or_X(D(r))} & {\Or_{Z(...)}(D(R))}
	\arrow[from=1-1, to=1-2]
	\arrow[from=1-1, to=2-1]
	\arrow[from=1-2, to=2-2]
	\arrow[from=2-1, to=2-2]
\end{tikzcd}\]
    faut bien montrer que l'image d'un principal c'est un principal.
    Le reste de la preuve paraît tellement compliqué pour rien mdr, 
    mais ça a l'air nécessaire : essentiellement, on peut tjr écrire
    $\Or_X(D(r))=A(X)[W]/(rW-1)$ et $D(r)\subset D(f)$ implique
    $f$ inversible dans $\Or_X(D(r))$ puis $A_f\simeq A$ si $f$
    est inversible dans $A$. Je mettrai pas tjr tout les détails.
\end{rem}
On peut définir les variétés abstraites par union finie:
\[(X,\Or_X)=\cup (X_i,\Or_X|_{X_i})\]
On en déduit que les ouverts sont bien des variétés. Enfin on a 
\[\Hom(X,Y)\to \Hom(\Or_Y(Y),\Or_X(X))\]
pour $Y$ affine et $X$ quelconque par recollement des flèches affines.
La continuité c'est le point flou mais qui est en fait le plus facile,
le recollement des faisceaux est clair.

\section{Recollement de variétés}
Étant donné des $(X_i)_i$ est des ouverts $U_{ij}\subset X_i$ tels que
\[\phi_{ij}\colon U_{ij}\simeq U_{ji}\]
et que les $\phi_{ij}$ engendrent une relation d'équivalence sur 
\[\sqcup X_i\]
alors le quotient est une variété définit par les ouverts $X_i$.
\begin{quest}
    Pourquoi les $X_i$ sont ouverts?
\end{quest}

\begin{rep}
    C'est \textbf{PAS} juste que la projection est ouverte pour la
    topologie quotient, puisque c'est pas tjr vrai. C'est plutôt que
    la classe d'équivalence de $X_i$ c'est $X_i \sqcup_j U_ji$ qui est
    ouvert.
\end{rep}
\begin{rem}
    La topologie finale pour la projection, par déf
    la topologie la plus fine sur le quotient qui rend la projection
    continue, en particulier la projection devient ouverte quand
    \[p^{-1}p(U)\]
    est ouvert autrement dit la classe d'équivalence d'un ouvert
    est ouvert. C'est le cas ici, ce serait le cas avec des 
    groupes/anneaux topologiques j'imagine et pour toute topologie
    "équilibrée".
\end{rem}
\section{Sous-variétés}
Concrètement, les sous variétés c'est immersions ouvertes ou fermées,
le cas des ouverts est clair. Pour les fermés, la version faiscautique :
\[\textrm{On prends un fermé $Z$ de $X$ et le pullback }i^*\Or_X\]
La traduction c'est que les fonctions sur $Z$ c'est localement des
restrictions de fonctions sur $X$.
\begin{rem}
    Sur les affines c'est même pas localement ça se recolle bien.
    On peut regarder les fibres, si $Z$ et $X$ sont affines on a 
    \[i^{-1}\Or_X(Z)_x=\Or_{X,i(x)}=A(X)_{\m_x}\]
    et en tensorisant par $\Or_Z(Z)=A(Z)$ on obtient que $i^*\Or_X$
    a les mêmes fibres que $\Or_Z$ avec un isomorphisme naturel.
\end{rem}
\begin{quest}
    Cas ou $X$ est un recollement de deux affines ?
\end{quest}
\section{Variétés quasi-projectives abstraites}
On définit sur $Z\subset \P^n(k)$ le faisceau donné par le faisceau
de fonctions régulières muni des recollements donnés par les cartes 
affines. En gros, avec le Proj : à faire,
\begin{enumerate}
    \item En gros le faisceau du Proj est bien un faisceau.
    \item Ducoup revoir un peu le Proj.
    \item C'est bien le faisceau de fonctions régulières.
    \item Les $Z^+(I)$ avec le faisceau $\Or_X$ de fonctions réguières
	est une variété.
\end{enumerate}
Ducoup il se passe un truc marrant
\[\textrm{Les ouverts du type $D^+(P)$ sont affines !}\]
L'idée c'est simplement d'utiliser le Veronese puis de montrer que 
$D^+(H)\simeq D^+(T_i)$ est affine. Pour ça suffit de montrer que
les morphismes de pullbacks 
\[(\phi_i)_*(U)\colon \Or_{\A^n_k}(U)\to \Or_{D^+(T_i)}(\phi_i^{-1}(U))\]
sont bien définis, c'est clair via la définition locale.  
\subsection{Le faisceau de fonctions régulières sur une variété
projective}
J'ai eu une idée ducoup, si on prends une variété projective $X=Z^+(I)$
et $P=p\mod I$ alors si $U=D^+(p)$ :
\[\Or_X(U)=(k[T_0,\ldots, T_n]/I)_{(p)}\]
Mon idée pour justifier la preuve c'est : si $g\in \Or_X(U)$ et 
\[g|_{U_i}=P_i/Q_i\]
 alors où est-ce qu'on peut regarder $gP_i=Q_i$? Je me disais on peut 
 regarder dans 
 \[S(X)=A(X')\]
 avec $X'$ le cone de $X$ dans $\A^{n+1}$. Mon guess c'est :
\[\Or_X(U)=A(X')_{(p)}\]
et dans ce cas, l'identité $gQ_i^2=P_iQ_i$ fait sens dans $A(X')$
parce que $D(p)$ est dense $Z(gQ_i^2-P_iQ_i)$.


\chapter{Partie I}
\section{Résumé}
\section{Framework}
Quand y s'agit de trouver des fermés projectifs ou des relations :
\begin{enumerate}
    \item Ajouter des relations adaptées à la situation jusqu'à obtenir
	le vide.
    \item Utiliser le nullstellensatz projectif.
    \item Faire de l'algèbre linéaire.
\end{enumerate}
pour le premier point une utilisation cool c'est de prouver que des
morphismes sont finis. Typiquement les projections linéaires.

Quand y s'agit de montrer des isomorphismes. La partie 
homéomorphisme est souvent claire :
\begin{enumerate}
    \item Tel point est dans tel ouvert ssi il vérifie telle ou telle
	relations.
\end{enumerate}
La partie isomorphisme est moins claire, faut montrer que la flèche
inverse est régulière et là c'est assez ad hoc.


\section{$k$-algèbres et Noether}
J'ai essayé de prouver la normalisation de Noether qui dit que
\begin{thm}[Normalisation de Noether]
    Une $k$-algèbre de type fini $A$ est entière sur un $k[T_1,\ldots,
    T_d]$.
\end{thm}
Et en fait y'a un
truc intéressant. Ma stratégie c'était:
\begin{enumerate}
    \item On suppose $k[T_1,\ldots,T_n]/I$
avec $n$ minimal alors $I$ petit.
    \item On peut supp $I$ premier, on se ramene au cas réduit 
	trivialement et au cas intègre avec le lemme chinois.
\end{enumerate}
Maintenant on peut regarder dans le corps de fractions une famille
alg indép maximale, enfin c'est ce que j'aurai aimé. Mais c'est pas 
clair que elle est de taille $n-1$! C'est là tout le pb. La raison 
c'est que :
\begin{center}
    Les variétés, par exemple les courbes ont pas nécessairement de
    modèles dans $\P^2$ non singuliers. D'où on peut rien dire sur 
    le nombre de générateurs de $I$ facilement (si $n$ minimal implique
    $I$ minimal et $dim(C)=1$ alors $I$ est engendré par un élément 
    et on est dans $\P^2$).
\end{center}

Bon la vraie preuve mtn :
\begin{itemize}
    \item Essentiellement, on veut juste montrer que pour chaque
	générateur $P$ on a un automorphisme qui le rend unitaire en
	une des variables. D'où le résultat par récurrence.
\end{itemize}
Deux questions maintenant :
\begin{enumerate}
    \item Pq à automorphisme près ça suffit ?
    \item Quel automorphisme ?
\end{enumerate}
Pour la première question ducoup y'a un ou deux petit trick mentaux
qu'il mentionne pas.
\begin{enumerate}
    \item Quand on obtient une relation unitaire pour $P$, on fait une 
	récurrence pour trouver l'injection entière $k[T_1,\ldots,T_d]$.
    \item L'injection ça peut être $T_1\mapsto L_1(T_1,\ldots, T_d)$
	avec les $L_i$ de degré $1$ pas nécessairement $T_i$!
\end{enumerate}
Pour la deuxième question : on commence par regarder comment rendre
$T_1\ldots T_n$ unitaire en $T_n$ par exemple. On peut regarder
l'automorphisme 
\[T_i\mapsto T_i+T_1;T_n\mapsto T_n\]
C'est clair que le résultat est unitaire en $T_n$ ! Post-automorphisme
l'injection de $k[T_1,\ldots,T_n]$ est \textbf{l'inclusion}, 
pré-automorphisme faut composer avec l'automorphisme inverse. Bon
maintenant en général ça marche pas forcément ça faut tricker un peu,
parce qu'on a plusieurs monomes.
On peut prendre
\[T_i\mapsto T_i+T_n^{m_i}\]
et jouer sur le tuple $m=(m_1,\ldots,m_{n-1},1)$, mais c'est juste
de l'écriture.

Maintenant le nullstellensatz ! 
\section{Nullstellensatz}
\begin{cor}[Nullstellensatz faible]
    Si $A$ est une $k$-algèbre tf et $\m$ est maximal. Alors 
    $A/\m$ est une extension finie de $k$.
\end{cor}
La preuve consiste à simplement se rendre compte que 
\[k[T_1,\ldots,T_d]\hookrightarrow A\]
entier implique $k[T_1,\ldots, T_d]$ est un corps d'où $d=0$.
\begin{thm}[Nullstellensatz fort]
    Soit $A$ une $k$-algèbre tf, alors pour tout $I$
    \[\sqrt I =\cap_{I\subset \m}\m\]
\end{thm}
Une idée de preuve c'est de remarquer que si $\cap \m$ contient des
non nilpotents, disons $f$. Alors pour $\m'\subset A_f$ avec 
$\delta\colon A\to A_f$ non nul :
\[k\mapsto A/\delta^{-1}\m'\to A_f/\m'\]
est algébrique d'où la deuxième flèche est entière d'où $\delta^{-1}\m'$
est maximal contenant pas $f$ ce qui est contradictoire. À noter
qu'on utilise $A_f\simeq A[T]/(fT-1)$ pour montrer que c'est tf.
\begin{quest}
    De manière constructive? Déjà, les localisés $A_\m$ sont pas t.f sur
    $k$ mdr.
\end{quest}

En particulier on obtient la correspondance avec le spectre maximal
muni de sa topologie de Zariski. On en déduit direct :
\begin{thm}[Nullstellensatz fort 2]
    Sur un corps algébriquement clos, $I(Z(J))=\sqrt J$ pour tout idéal
    $J$ de $k[T_1,\ldots, T_n]$.
\end{thm}

On a maintenant une correspondance entre ensembles algébriques et
idéaux radicaux quand $k$ est \textbf{algébriquement clos}.
\section{Topologie et irréductibles}
Essentiellement, dans un anneau noethérien, on a 
\[\textrm{Tout idéal radical } I \textrm{ est intersection finie 
d'idéaux premiers.}\]
En particulier, on déduit la décomposition finie unique en irréductible
avec la correspondance de la section d'avant.
\begin{rem}
    Un critère d'irréducibilité c'est que les ouverts s'intersectent
    tout court.
\end{rem}

\section{Espaces projectifs}
Ducoup les définitions à retenir :
\begin{enumerate}
    \item Un idéal homogène c'est un truc de la forme $\oplus A_d\cap I$.
	C'est à dire que toute les composantes homogènes d'un $f\in I$ 
	sont dans $I$.
    \item Les $Z^+(I)$ sont défs par les zéros de polynômes homogènes.
	De manière équivalentes par la projection de
	$Z(I)\subset \A^{n+1}$.
    \item À l'inverse $I^+(Z)=I(\pi^{-1}Z)$. 
\end{enumerate}
Maintenant le nullstellensatz devient :
\begin{center}
    Les idéaux homogènes différents de $(T_0,\ldots,T_n)$ correspondent
    aux variétés projectives.
\end{center}
un idéal définit le vide si et seulement si il contient
\[(T_0,\ldots,T_n)^s\] pour un $s$.

On obtient le générateur de relation (ce nullstellensatz)! À détailler un
peu plus avec la preuve que les projections sont entières !

\section{Fonctions régulières}
On prends toujours la définition locale sur les variétés 
quasi-projectives et même algébriques. Y'a toujours un élément de preuve
intéressant dans 
\[A(Z)_f\simeq \Or_Z(D(f))\]
sur les variétés affines pour $f\in A(Z)$. On recouvre $D(f)$ par des
$D(f_i)=D(f_i^2)$ avec $g\in \Or_Z(D(f))$ et $g|_{U_i}=g_i/f_i$.
En particulier, 
\[f^r=\sum a_if_i^2\]
puis $gf^r=\sum a_ig_if_i$ d'où la surjectivité de 
$A(Z)_f\to \Or_Z(D(f))$! 

\section{Morphismes de variétés affines, première déf}
Dans le cas des variétés affines : on définit littéralement via 
\begin{center}
    Des restrictions de morphismes $\A^n\to \A^m$.
\end{center}
On obtient directement les flèches 
% https://q.uiver.app/#q=WzAsNixbMCwxLCJrW1NfMSxcXGxkb3RzLFNfbV0iXSxbMSwxLCJrW1RfMSxcXGxkb3RzLFRfbl0iXSxbMCwyLCJBKFkpIl0sWzEsMiwiQShYKSJdLFswLDAsIlNfaiJdLFsxLDAsIlxccGhpX2koVF9qLGopIl0sWzAsMV0sWzIsM10sWzAsMl0sWzEsM10sWzQsNSwiIiwwLHsic3R5bGUiOnsidGFpbCI6eyJuYW1lIjoibWFwcyB0byJ9fX1dXQ==
\[\begin{tikzcd}
	{S_j} & {\phi_i(T_j,j)} \\
	{k[S_1,\ldots,S_m]} & {k[T_1,\ldots,T_n]} \\
	{A(Y)} & {A(X)}
	\arrow[maps to, from=1-1, to=1-2]
	\arrow[from=2-1, to=2-2]
	\arrow[from=2-1, to=3-1]
	\arrow[from=2-2, to=3-2]
	\arrow[from=3-1, to=3-2]
\end{tikzcd}\]
de l'équivalence de catégorie. Il faut juste prouver que de 
$A(Y)\to A(X)$ on peut relever un morphismes de variétés.




\printbibliography
\end{document}


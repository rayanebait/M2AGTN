\documentclass[a4paper,12pt]{book}
\usepackage{amsmath,  amsthm,enumerate}
\usepackage{csquotes}
\usepackage[provide=*,french]{babel}
\usepackage[dvipsnames]{xcolor}
\usepackage{quiver, tikz}

%symbole caligraphique
\usepackage{mathrsfs}

%hyperliens
\usepackage{hyperref}

%pseudo-code
\usepackage{algorithm}
\usepackage{algpseudocode}

\usepackage{fancyhdr}

\pagestyle{fancy}
\addtolength{\headwidth}{\marginparsep}
\addtolength{\headwidth}{\marginparwidth}
\renewcommand{\chaptermark}[1]{\markboth{#1}{}}
\renewcommand{\sectionmark}[1]{\markright{\thesection\ #1}}
\fancyhf{}
\fancyfoot[C]{\thepage}
\fancyhead[LO]{\textit \leftmark}
\fancyhead[RE]{\textit \rightmark}
\renewcommand{\headrulewidth}{0pt} % and the line
\fancypagestyle{plain}{%
    \fancyhead{} % get rid of headers
}

%bibliographie
\usepackage[
backend=biber,
style=alphabetic,
sorting=ynt
]{biblatex}

\addbibresource{bib.bib}

\usepackage{appendix}
\renewcommand{\appendixpagename}{Annexe}

\definecolor{wgrey}{RGB}{148, 38, 55}

\setlength\parindent{24pt}

\newcommand{\Z}{\mathbb{Z}}
\newcommand{\R}{\mathbb{R}}
\newcommand{\rel}{\omathcal{R}}
\newcommand{\Q}{\mathbb{Q}}
\newcommand{\C}{\mathbb{C}}
\newcommand{\N}{\mathbb{N}}
\newcommand{\K}{\mathbb{K}}
\newcommand{\A}{\mathbb{A}}
\newcommand{\B}{\mathcal{B}}
\newcommand{\Or}{\mathcal{O}}
\newcommand{\F}{\mathscr F}
\newcommand{\Hom}{\textrm{Hom}}
\newcommand{\disc}{\textrm{disc}}
\newcommand{\Pic}{\textrm{Pic}}
\newcommand{\End}{\textrm{End}}
\newcommand{\Spec}{\textrm{Spec}}
\newcommand{\Supp}{\textrm{Supp}}
\renewcommand{\Im}{\textrm{Im}}
\newcommand{\m}{\mathfrak{m}}
\renewcommand{\P}{\mathbb{P}}
\newcommand{\p}{\mathfrak{p}}


\newcommand{\cL}{\mathscr{L}}
\newcommand{\G}{\mathscr{G}}
\newcommand{\D}{\mathscr{D}}
\newcommand{\E}{\mathscr{E}}
\newcommand{\Po}{\mathscr{P}}
\renewcommand{\H}{\mathscr{H}}

\makeatletter
\newcommand{\colim@}[2]{%
  \vtop{\m@th\ialign{##\cr
    \hfil$#1\operator@font colim$\hfil\cr
    \noalign{\nointerlineskip\kern1.5\ex@}#2\cr
    \noalign{\nointerlineskip\kern-\ex@}\cr}}%
}
\newcommand{\colim}{%
  \mathop{\mathpalette\colim@{\rightarrowfill@\scriptscriptstyle}}\nmlimits@
}
\renewcommand{\varprojlim}{%
  \mathop{\mathpalette\varlim@{\leftarrowfill@\scriptscriptstyle}}\nmlimits@
}
\renewcommand{\varinjlim}{%
  \mathop{\mathpalette\varlim@{\rightarrowfill@\scriptscriptstyle}}\nmlimits@
}
\makeatother

\theoremstyle{plain}
\newtheorem{thm}[subsection]{Théoreme}
\newtheorem{lem}[subsection]{Lemme}
\newtheorem{prop}[subsection]{Proposition}
\newtheorem{cor}[subsection]{Corollaire}
\newtheorem{heur}{Heuristique}
\newtheorem{rem}{Remarque}
\newtheorem{note}{Note}

\theoremstyle{definition}
\newtheorem{conj}{Conjecture}
\newtheorem{prob}{Problème}
\newtheorem{quest}{Question}
\newtheorem{prot}{Protocole}
\newtheorem{algo}{Algorithme}
\newtheorem{defn}[subsection]{Définition}
\newtheorem{exmp}[subsection]{Exemples}
\newtheorem{exo}[subsection]{Exercices}
\newtheorem{ex}[subsection]{Exemple}
\newtheorem{exs}[subsection]{Exemples}

\theoremstyle{remark}

\definecolor{wgrey}{RGB}{148, 38, 55}
\definecolor{wgreen}{RGB}{100, 200,0} 
\hypersetup{
    colorlinks=true,
    linkcolor=wgreen,
    urlcolor=wgrey,
    filecolor=wgrey
}

\title{Géométrie algébrique}
\date{}

\begin{document}
\maketitle
\tableofcontents


\chapter{Plan du cours}
Ici le but c'est de faire un sommaire du cours vu qu'il est pas très 
clair. Essentiellement faut maitriser de l'algèbre pour l'instant.


\section{Variétés affines}
\subsection{Premières définitions}

\begin{enumerate}
    \item Ensembles algébriques affines.
    \item Correspondance avec les idéaux.
    \item Lemme 1.1.10 : indice au weak nullstellensatz.
    \item 1.1.11 : Énoncé du nullstellensatz et corollaires.
\end{enumerate}
\subsection{Nullstellensatz}
\begin{enumerate}
    \item Normalisation de noether.
    \item Nullstellensatz faible.
    \item Nullstellensatz fort.
    \item Preuve de la correspondance idéaux radicaux et ensembles
	algébriques.
\end{enumerate}

\subsection{Topologie et irréductibles}
\begin{enumerate}
    \item Caractérisation des irréductibles via la correspondance.
    \item Via les ouverts.
    \item Décomposition unique des idéaux radicaux en idéaux premiers.
    \item Décomposition unique des ensembles algébriques en 
	irréductibles.
\end{enumerate}
Y'a pleins d'exos.

\subsection{Espaces projectifs}
\begin{enumerate}
    \item Définitions de la topologie.
    \item L'idéal "irrelevant".
\end{enumerate}
\subsection{Fonctions régulières}
\begin{enumerate}
    \item Intuitions sur comment ça doit marcher sur les affines.
    \item Définition locale.
    \item Continuité.
    \item Sur $D(f)$.
\end{enumerate}
\subsection{Morphismes d'ensembles algébriques}
\begin{enumerate}
    \item Intuitions.
    \item Définition intuitive.
    \item Équivalence de catégorie avec les $k$-algèbres réduites.
    \item Le corollaire : deux ensembles algébriques affines sont 
	isomorphes ssi leurs algèbres de fonctions le sont.
\end{enumerate}
Exercices.

\section{Variétés affines abstraites}
\subsection{Espaces annelés}
Ça ressemble au 1.1 au début.
\begin{enumerate}
    \item Définition et espaces localement annelés.
    \item Variétés algébriques affines abstraites.
    \item Équivalence de catégorie entres variétés affines et variétés
	affines abstraites.
\end{enumerate}


\chapter{Partie I}
\section{$k$-algèbres et Noether}
J'ai essayé de prouver la normalisation de Noether. Et en fait y'a un
truc intéressant. Ma stratégie c'était:
\begin{enumerate}
    \item On suppose $k[T_1,\ldots,T_n]/I$
avec $n$ minimal alors $I$ petit.
    \item On peut supp $I$ premier, on se ramene au cas réduit 
	trivialement et au cas intègre avec le lemme chinois.
\end{enumerate}


\printbibliography
\end{document}


\documentclass[a4paper,12pt]{book}
\usepackage{amsmath,  amsthm,enumerate}
\usepackage{csquotes}
\usepackage[provide=*,french]{babel}
\usepackage[dvipsnames]{xcolor}
\usepackage{quiver, tikz}

%symbole caligraphique
\usepackage{mathrsfs}

%hyperliens
\usepackage{hyperref}

%pseudo-code
\usepackage{algpseudocode}
\usepackage{algorithm}
\makeatletter
  \renewcommand{\ALG@name}{Algorithme}
  \makeatother
\usepackage{fancyhdr}

\pagestyle{fancy}
\addtolength{\headwidth}{\marginparsep}
\addtolength{\headwidth}{\marginparwidth}
\renewcommand{\chaptermark}[1]{\markboth{#1}{}}
\renewcommand{\sectionmark}[1]{\markright{\thesection\ #1}}
\fancyhf{}
\fancyfoot[C]{\thepage}
\fancyhead[LO]{\textit \leftmark}
\fancyhead[RE]{\textit \rightmark}
\renewcommand{\headrulewidth}{0pt} % and the line
\fancypagestyle{plain}{%
    \fancyhead{} % get rid of headers
}

%bibliographie
\usepackage[
backend=biber,
style=alphabetic,
sorting=ynt
]{biblatex}

\addbibresource{bib.bib}

\usepackage{appendix}
\renewcommand{\appendixpagename}{Annexe}

\definecolor{wgrey}{RGB}{148, 38, 55}

\setlength\parindent{24pt}

\newcommand{\Z}{\mathbb{Z}}
\newcommand{\R}{\mathbb{R}}
\newcommand{\rel}{\omathcal{R}}
\newcommand{\Q}{\mathbb{Q}}
\newcommand{\C}{\mathbb{C}}
\newcommand{\N}{\mathbb{N}}
\newcommand{\K}{\mathbb{K}}
\newcommand{\A}{\mathbb{A}}
\newcommand{\B}{\mathcal{B}}
\newcommand{\Or}{\mathcal{O}}
\newcommand{\F}{\mathbb F}
\newcommand{\m}{\mathfrak m}
\newcommand{\p}{\mathfrak p}
\renewcommand{\a}{\mathfrak a}
\newcommand{\I}{\mathfrak I}
\newcommand{\Hom}{\textrm{Hom}}
\newcommand{\disc}{\textrm{disc}}
\newcommand{\Pic}{\textrm{Pic}}
\newcommand{\End}{\textrm{End}}
\newcommand{\Spec}{\textrm{Spec}}

\newcommand{\cL}{\mathscr{L}}
\newcommand{\G}{\mathscr{G}}
\newcommand{\D}{\mathscr{D}}
\newcommand{\E}{\mathscr{E}}

\theoremstyle{plain}
\newtheorem{thm}[subsection]{Théoreme}
\newtheorem{lem}[subsection]{Lemme}
\newtheorem{prop}[subsection]{Proposition}
\newtheorem{cor}[subsection]{Corollaire}
\newtheorem{heur}{Heuristique}
\newtheorem{rem}{Remarque}
\newtheorem{rembis}{Remarque}
\newtheorem{note}{Note}

\theoremstyle{definition}
\newtheorem{conj}{Conjecture}
\newtheorem{prob}{Problème}
\newtheorem{quest}{Question}
\newtheorem{prot}{Protocole}
\newtheorem{algo}{Algorithme}
\newtheorem{defn}[subsection]{Définition}
\newtheorem{defnbis}{Définition}
\newtheorem{ex}{Exemple}
\newtheorem{exo}[subsection]{Exercices}

\theoremstyle{remark}

\definecolor{wgrey}{RGB}{148, 38, 55}
\definecolor{wgreen}{RGB}{100, 200,0} 
\hypersetup{
    colorlinks=true,
    linkcolor=wgreen,
    urlcolor=wgrey,
    filecolor=wgrey
}

\title{Corps locaux \\ \small{(Par Qing Liu)}}
\date{}

\begin{document}
\maketitle
\chapter*{Introduction}
Le but c'est d'abord de décrire les extensions finies de corps locaux
comment dans \cite{cassels}.
\chapter{Corps ultramétriques complets}
\section{Valuations et valeurs absolues}
Le but là c'est de classifier les valuations, d'abord sur des corps
premiers comme $\Q$. D'abord l'équivalence entre valuations et 
valeurs absolues. 
\begin{defn}[Valuation]
    Une valuation $v$ (de rang $1$) est un morphisme de groupe 
    (multiplicatif vers additif) \[K\to \R\cup\{\infty\}\]
    tel que $v(x+y)\geq min\{v(x), v(y)\}$ quand $x+y\ne 0$ avec
    $v(x+y)=\infty$.
\end{defn}
\begin{defn}[Valeur absolue]
    Une valeur absolue $|.|$ est un morphisme de groupes multiplicatifs
    $K\to \R$ étendu avec $0\mapsto 0$ et qui vérifie une inégalité 
    triangulaire.
\end{defn}
Le passage aux valeurs absolues : étant donné un réel $0<t<1$ on peut
déf une valeur absolue 
\[x\mapsto t^{v(x)}\]

Bon ensuite la caractérisation archimédienne :
\begin{defn}[Valeur absolue archimédienne]
    Un corps valué $(K,|.|)$ est archimédien si pour tout 
    $(x,c)\in K\times \R$ il existe $n$ t.q \[c\leq |n.x|\]
    et pas $n|x|$ attention.
\end{defn}
Ensuite la première partie de la caractérisation :
\begin{prop}
    Être archimédien c'est équivalent à 
    \begin{itemize}
        \item $(|n|)_{n\in \Z}$  est bornée.
        \item $(|n|)_{n\in \Z}$  est $\leq 1$.
        \item $|.|$ est ultramétrique!
    \end{itemize}
\end{prop}
\begin{proof}[Preuve]
    La première équivalence est directe. La deuxième aussi via 
    $x\mapsto x^k$ et la troisième se ramène à la deuxieme en 
    regardant $|1+x|\leq 1$ et la flèche $x\mapsto x^k$ à nouveau !
    (Ca caractérise les élements inferieurs à $1$)
\end{proof}
\begin{prop}[Valuations et valeurs absolues]
    Il y a une bijection entre valeurs absolues ultramétriques
    et valuations. L'inverse de la flèche du dessus est juste 
    \[|.|\mapsto \frac{1}{\log(1/2)}\log(|.|)\]
\end{prop}

On s'approche du théorème d'Ostrowski. Avant on déf les équivalences
de valeurs absolues.
\begin{prop}[La topologie induite]
    On obtient une distance à partir d'une valeur absolue puis la 
    topologie de la distance.
\end{prop}
Dans le cas ultramétrique c'est très bizarre.
\begin{prop}
    On a les propriétés suivantes.
    \begin{itemize}
        \item Si $|x|\ne |y|$ alors $|x+y|=max(|x|,|y|)$.
        \item Avec l'inégalité ultramétrique, les boules ont un seul
            centre.
        \item Puis deux disques qui s'intersectent forment en fait une
            chaine.
    \end{itemize}
\end{prop}
\begin{prop}[Anneau de valuation]
    On peut déf l'anneau de valuation de $(K,|.|)$ par 
    $\Or:=\bar D(0,1)$ les éléments de valuations plus petites que $1$.
    Et son idéal maximal la boule ouverte $D(0,1)$. Les inversibles
    sont dans la frontière.
\end{prop}
\begin{defn}[Équivalence de valeurs absolues]
    Deux valeurs absolues sont équivalentes si elles définissent la 
    même topologie.
\end{defn}
\begin{lem}[Caractérisation]
    Deux valeur sabsolues sont équivalentes ssi il existe 
    $s\in \R^*_+$ t.q $|.|_1=|.|_2^s$.
\end{lem}
\begin{proof}[Preuve]
    La bonne idée c'est la caractérisation de $D(0,1)$ par les trucs
    qui tendent vers $0$ par $x\mapsto x^k$. Faut comprendre que 
    prendre des limites dans $\R$ donne la même chose pour 
    $|.|_1$ et $|.|_2$.
\end{proof}

Ça suffit à prouver le théorème d'Ostrowski :
\begin{thm}[Ostrowski] 
    Une valeur absolue non triviale sur $\Q$ est équivalente soit
    à $|.|_{\infty}$ ou à une valeur absolue $p$-adique $|.|_p$.
\end{thm}
\begin{proof}[Preuve]
    Pour la partie non archimédienne, on regarde $\mathfrak m\cap \Z$
    pour obtenir un $p$. Ensuite, on montre via $|uk|=|1-vp|$ que
    $|k|=1$ dés que $k\wedge p=1$, $|vp|<1$ parce que 
    $vp\in \mathfrak m$. Ensuite y reste à comparer $|p|$ et $|p|_{p}$.
    Vu que $|x|=|p|^r|k|=|p|^r$. 

    Pour le cas archimédien l'idée c'est d'écrire $a$ en base $b$ pour
    comparer $a^n$ et $b^l$ où $l\sim n$. En particulier montrer que
    $|a|\geq 1$ implique $b\geq1$ dès que $a\geq 2$ ET 
    $|a|=|b|^{\log(a)/\log(b)}$, via la comparaison et par symétrie.
    Puis remplacer $b$ par $2$. On a montré que seul $|2|$ détermine
    le lien entre $|.|$ et $|.|_{\infty}$.
\end{proof}
\begin{defn}[Places]
    On définit une place comme une classe d'équivalence de valeurs
    absolues. Une place est finie/infinie si elle est 
    ultramétrique/archimédienne.
\end{defn}

\section{Complétions et corps complets}
Dès qu'on a un corps valué $(K,|.|)$, on peut construire une distance
\[d_{|.|}\colon(x,y)\mapsto |x-y|\]
puis définir des suites de Cauchy. 
\begin{defn}
    Un corps valué $(K,|.|)$ est complet si les suites de Cauchy pour 
    $d_{|.|}$ convergent dans $K$.
\end{defn}
\begin{defn}[Complétion]
    Pour tout corps valué $(K,|.|)$ il existe un corps valué complet 
    $(\hat K,|.|)$ tel que $|.|$ s'étend à $\hat K$.
\end{defn}
\begin{proof}[Preuve]
    L'idée c'est de regarder l'ensemble des suites de Cauchy 
    $\mathscr C(K)\subset K^{\N}$, de voir que c'est un anneau 
    puis de quotienter par l'idéal maximal des suites qui convergent
    vers $0$.
\end{proof}

\begin{defn}[Morphismes dans la catégorie des corps valués]
    Un morphisme de corps valués est un morphisme d'anneau et une
    isométrie.
\end{defn}

On peut étendre les morphismes vers des corps complets en morphismes
de corps valués unique à unique isomorphisme près.

\begin{rem}
    Si $(K,|.|)$ est ultramétrique et non triviale, $\hat K$ peut-être
    construit algébriquement. On fixe $t\in K$ t.q $0<|t|<1$. Dans
    l'anneau de valuation, on regarde 
    \[\widehat O_K=\varprojlim_n(O_K/t^nO_K)\]
    C'est un anneau intègre muni d'une valuation t.q
    \[\widehat v((x_n)_n)=v(y)\]
    où $(x_n)=\pi(y)$ et 
    $\pi\colon O_K \to \widehat O_K;~x\mapsto (x)_n$. C'est
    une valuation par densité. On l'étend au corps de fraction de la 
    manière évidente. On peut montrer que c'est complet \textbf{à faire}.
    Pareil en général \textbf{$I\subset A$}.
\end{rem}
\begin{exo}
    Rayon de convergence de $exp(z)=\sum_n \frac{z^n}{n!}$ dans 
    $\Q_p$? Sachant que en métrique $p$-adique 
    $1/n!\to_{n\to\infty}+\infty$. Faut calculer la valuation de 
    $n!$.
\end{exo}
Plus généralement, si $A$ est intègre et $\m a=tA$ est maximal
on peut définir la valuation $t$-adique associée et l'étendre au 
corps de fractions et c'est une valuation discrète. Je crois que l'idée 
c'est que 
\[I=\cap t^n A=\{0\}\]
parce que $tI=I$ puis Nakayama dans un anneau noethérien.

Apparemment il fait une construction de $\R$ sans complétion parce que
c'est tautologique? Ah bah oui les valuations c'est dans $\R$, mais
en fait on peut d'abord juste les prendre dans $\Q$.


Il regarde un corps $K$ avec un ordre total 
compatible archimédien et la topologie de l'ordre.

Alors on demande
enfin que $\iota\colon \Q\to K$ soit dense, on a nécessairemment que
$\iota$ est croissante. Par densité de $\Q$ dans $K$, on peut remplacer
toute les suites de Cauchy dans $K$ par des suites de Cauchy dans $\Q$.


En particulier, y'a équivalence entre les suites de Cauchy de $K$ et 
$\Q$. On regarde maintenant $C(\Q)$ les suites de Cauchy dans $\Q$ et
$I(\Q)$ les suites qui tendent vers $0$. Dans
\[\pi\colon C(\Q)\to C(\Q)/I(\Q)\]
on définit un ordre à droite via $\pi(x_n)_n\geq \pi(y_n)_n$ ssi
on a égalité où il existe $r\in \Q^*_+$ t.q $x_n\geq y_n+r$. On obtient
un corps totalement ordonné $C(\Q)/I(\Q)$ (totalement car 
$\pm\pi(x_n)_n\geq 0$). Bon maintenant faut juste conclure en montrant
que le quotient est complet. \textbf{à faire?}

\begin{rem}
    On a bel et bien utilisé que $\Q$.
\end{rem}
\begin{prop}
    $\R$ est unique à unique isomorphisme près en tant que
    corps totalement ordonné complet archimédien où $\Q$ est dense.
\end{prop}

\section{Espaces de Banach}
Une norme $K$ sur un espace vectoriel $V$ est une flèche $x\mapsto ||x||$
qui est nulle qu'en $0$, a une inégalité triangulaire et qui transforme
l'action de $K$ en action de $\R$ via $|.|$.

\begin{defn}
    Un e.v.n est un Banach si il est complet pour sa norme.
\end{defn}
\begin{rem}
    Un $k$-espace vectoriel de dimension finie sur un corps complet
    est un Banach. L'inverse est faux!
\end{rem}
\begin{defn}
    Deux normes sont équivalentes si elles définissent la même topologie.
    Ou immédiatemment si il existe $c_1,c_2$ t.q 
    \[c_1||x||_1\leq ||x||_2\leq c_2||x||_1\]
\end{defn}
L'intérêt c'est maintenant qu'étant donné une extension de corps $L/k$
et une valeur absolue sur $k$, qu'est-ce qu'il se passe quand on l'étend
à $L$? En fait c'est des normes sur $L$ en tant que $k$-ev, et elles
sont équivalentes en tant que norme ssi elles le sont en tant que v.a 
par définition!
\begin{thm}
    Si $k$ est complet, et $V$ est un $k$-ev de dimension finie, alors
    toutes les normes sur $V$ sont équivalentes et $V$ est un Banach.
\end{thm}
\begin{proof}
    La norme du max donne une structure de Banach grâce à la convergence
    normale. Il reste à montrer que toutes les normes sont équivalentes
    à la norme du max $||.||_{\infty}$. Un côté est simple, l'autre 
    par induction \textbf{Faire? Ah la preuve est non triviale mdr}.
\end{proof}

\section{Lemme de Hensel}
Soit $(K,|.|)$ un corps complet ultramétrique. On note $k=\Or_K/\m$
le corps résiduel. 
\begin{lem}[Lemme de Hensel]
    Soit $P(X)\in \Or_K[X]$, on suppose que $P\equiv f.g\mod \m$ tels que
    $gcd(f,g)=1$. Alors il existe $F,G\in \Or_K[X]$ tels que 
    \[P=F.G\]
    et $F\equiv f\mod \m$, $G\equiv g\mod\m$, avec $\deg(F)=\deg(f)$.
    La décomposition est unique à inversible près.
\end{lem}
\begin{proof}
    On commence par prendre un lift $F_0$ de $f$ et $G_0$ de $g$ de 
    degrés minimaux. On a $(f,g )=k[X]$ d'où $1\in (F_0,G_0)+\m[X]$,
    il existe donc $t\in K$ t.q $0<|t|<1$ et
    $
    \begin{cases}
        P-F_0G_0\in t\Or_K[X]\\
        1\in (F_0,G_0)+t\Or_K[X]\\
    \end{cases}
    $. On a déjà augmenté la précision. Soit $F_1=F_0+tV_1$ et 
    $G_1=G_0+tU_1$. On veut $P-F_1G_1\in t^2 \Or_K[X]$ t.q $\deg(F_1)=m$
    et $\deg(G_1)\leq d-m$. On regarde 
    \begin{align*}
        P-(F_0+tV_1)(G_0+tU_1)&=P-(F_0G_0+t(F_0U_1+G_0V_1)+t^2U_1V_1)\\
                              &=(P-F_0G_0)+t(F_0U_1+G_0V_1)+t^2*\\
                              &=tE_0+t(\_)+t^2*\\
                              &=t(E_0+F_0U_1+G_0V_1)+t^2*
    \end{align*}
    On prends $E_0=H_0F_0+R_0G_0+t*$.
    \textbf{Revoir la preuve ailleurs.}
\end{proof}

\begin{cor}
    On garde les hypothèses sur $K$. Étant donné un polynôme irréductible
    $P\in K[X]$, on a 
    \[max_i|a_i|=max\{|a_0|, |a_d|\}.\]
    En particulier si $a_d=1$ et $a_0\in \Or_K$ alors $P\in \Or_K[X]$.
\end{cor}
\begin{proof}
    Supp $|a_{i_0}|>|a_0|,|a_d|$. Alors $P'=P/a_{i_0}\in \Or_K[X]$.
    On a 
    \[P'=X^rg(X)\mod \m_K\]
    avec $r>0$ et $g(0)\ne0$. On a $r\leq \deg(P'\mod \m_K)< d$ et par le
    lemme de Hensel $P(X)=FG$ avec $0<\deg(F)=r<d$ par déf, c'est
    contradictoire.
\end{proof}
\begin{cor}[Lemme de Hensel]
    Si $P\in \Or_K[X]$ a une racine simple modulo $\m_K$. Alors
    $P$ a une unique racine congrue à elle modulo $\m_K$.
\end{cor}
\begin{proof}
    Clair par l'unicité du relèvement.
\end{proof}
\begin{cor}[Lemme de Newton]
    Pour $P\in \Or_K[X]$, on suppose qu'il existe $\alpha\in \Or_K$
    tel que \[|P(\alpha)|<|P'(\alpha)|^2\]
    alors il existe un unique $\tilde\alpha\in\Or_K$ tel que
    \[\begin{cases} P(\tilde\alpha)=0,\\ |\tilde\alpha-\alpha|<|P'(\alpha)\end{cases}\]
\end{cor}
\begin{proof}
    Soit $\lambda= P'(\alpha)$, l'expansion de taylor de $P$ en $\alpha$
    est
    \[P(X+\alpha)=P(\alpha)+P'(\alpha)X+X^2R(X)\]
    avec $R(X)\in \Or_K[X]$. Soit $H(X)=\lambda^{-2}P(\lambda X+\alpha)$,
    on a 
    \[P(\lambda X+\alpha)/\lambda^2=P(\alpha)/\lambda^2+X+X^2R_1(X)\]
    et $|P(\alpha)/\lambda^2|<1$ par hypothèse. Donc on a 
    $H(X)\in\Or_K[X]$ et $\bar H(X)=X(1+XG(X))$ pour un $G$. Et 
    $0$ est une racine simple de $\bar H$. Par le corollaire précédent
    on a une unique racine $t$ de $H$ telle que $|t|<1$. D'où
    $\lambda t+\alpha$ est une racine de $P$ et c'est la seule dans
    $D(\alpha, |\lambda|)$.
\end{proof}
\begin{ex}
    Dans $\Q_p$, si $\alpha\in\Q_p$ et $|\alpha|_p<1$ alors $1+\alpha$
    a une racine $n$-ème dans $\Z_p$ si $n$ est premier
    à $p$. On peut considérer $X^n-(1+\alpha)\in \Z_p[X]$, dans 
    $\F_p\colon X^n-(1+\alpha)=X^n-1$ a une racine simple $X=1$ puis
    lemme de Hensel.
\end{ex}
\begin{ex}
    Si $n=p-1$, $P=X^{p-1}-1$ est décomposé dans $\F_p$ d'où on 
    a les racines $p-1$-èmes de l'unité dans $\Z_p$.
\end{ex}
\begin{ex}
    Si $n=p$, $(1+p\alpha)^p=1+p^2\alpha*$ et 
    $(1+X)^p=1+pX+p(p-1)/2X^2\_+pX^{p-1}+X^p$
    et $p\mid \begin{pmatrix} p\\k\end{pmatrix}$ pour $1\leq k\leq p-1$.
    Enfin \[(1+p\alpha)^p=1+p^2\alpha *\]
    et $(1+p)\neq (1+\alpha)^p,\alpha\in \Z_p$. 
\end{ex}
\begin{ex}
    Enfin par exemple $X^2-p$ peut pas avoir de racines dans $\Z_p$
    parce que $\equiv X^2\mod p$ et 
    $v_p(\alpha^2)=2v_p(\alpha)\ne 1=v_p(p)$.
\end{ex}
\begin{thm}[Lemme de Hensel multivarié]
    Soit $F(X,Y)\in \Or_K[X,Y]$. On suppose que $\bar F$ a un zéro
    $(a,b)$. On suppose que $\bar F_X'(a,b)\ne 0$. Alors 
    \[\{(x,y)\in \Or_K^2\mid F(x,y)=0,\bar x=a,\bar y=b\}\]
    est en bijection avec 
    \[\{t\in\Or_K\mid |t|<1\}.\]
\end{thm}
\begin{proof}
    Soit $t\in \m_K$. On considère $P=F(X,\tilde b+t)\in \Or_K[X]$
    où $\tilde b \equiv b$ et $\bar P(X)=\bar F(X,b)$, $\bar P'(X)=
    \bar F_X'(X,b)$. Par le lemme de Hensel, $P$ a une unique racine
    $\alpha(t)\in \Or_K$, telle que $\bar\alpha(t)\equiv a$. Alors
    \[\{|t|<1\}\to\{(x,y)\in \Or_K^2| F(x,y)=0,\bar x=a, \bar y=b\}\]
    est injective car $t\mapsto b+t$ est injective. 

    Inversement, si $(x,y)$ est dans l'ensemble de droite. On pose
    $t=y-b$ d'où $P_t(x)=0$, $\bar x= a$ implique $x=\alpha(t)$. En fait,
    \[\{|t|<1\}\to \{F^{-1}(0)\ldots\}\] est continue et même analytique.
\end{proof}
\begin{ex}
    Soit $E\colon y^2=x^3+1$ sur $\Q_p$, $p>3$. On regarde 
    $F=Y^2-(X+1)^3$.
    On a $F_X=3X^2$ et $F_Y=2Y$. Pour tout $(a,b)\in\F_p^2$ tel que 
    $b^2=a^3+1$, au moins l'un des $\bar F_X(a,b)$ et $\bar F_Y(a,b)$
    est non nul. Alors $E(\Z_p)$ est une union disjointe de disques
    ouvert de $\Z_p$ indexés par $E(\F_p)$.
\end{ex}

\section{Extension de valeurs absolues}
On considère $L/K$ une extension finie de corps quelconques. Pour rappel
la norme \[N_{L/K}(\alpha)\colon L\to K\]
est définie comme le déterminant de $x\mapsto \alpha x$ dans $L$.

\begin{prop}
    Pour $\alpha\in L$,
    \begin{itemize}
        \item $N(\alpha)=\alpha^[L:K]$ si $\alpha\in K$.
        \item La norme est multiplicative.
        \item Étant donné $L/E/K$, on a 
            $N_{L/K}(\alpha)=N_{E/K}(N_{L/E}(\alpha))$. (dur)
        \item Si $P(X)=X^d+\ldots+a_0$est le polynôme minimal de 
            $\alpha$ sur $K$, alors, 
            $N_{K[\alpha]/K}(\alpha)=(-1)^d a_0$.
    \end{itemize}
\end{prop}
 En particulier, le cas $E=K[\alpha]$ est intéressant. 
\begin{thm}
    Si $K$ est complet ultramétrique. Soit $L/K$ une extension
    finie. Alors $|.|$ s'étend uniquement en une valeur absolue sur $L$
    via \[|\alpha|=|N_{L/K}(\alpha)|^{1/[L:K]}.\]
\end{thm}
\begin{proof}
    L'existence consiste à vérifier que c'est bien une valeur absolue.
    On supp $|\alpha|_L\leq 1$. On doit vérifier que 
    \[|1+\alpha|_L\leq 1\]
    mais $|\alpha|_L\leq 1$ dit que $|N_{L/K}(\alpha)|\leq 1$. Si
    $P$ est le polynôme minimal de $\alpha$ alors $|a_0|\leq 1$ d'où
    par le corollaire de Hensel $P\in \Or_K[X]$. En plus,
    $P(X-1)$ est le pol minimal de $1+\alpha$. D'où 
    $|N_{K[\alpha]/K}(1+\alpha)|=|1+\alpha|_L\leq 1$. Enfin l'unicité
    est dûe à l'unicité des normes sur $L$ en tant que $K$-ev. (\textbf{
    regarder})

    Si $|.|$ est triviale les extensions définissent des topologies
    discrètes donc sont triviales.
\end{proof}
Ça c'est pour les extensions finies.

\begin{cor}
    Une valeur absolue sur un corps ultramétrique complet $K$ s'étend
    uniquement en une valeur absolue sur $\bar K$!
\end{cor}
\begin{proof}
    L'idée c'est de définir 
    \[|\alpha|:=|N_{K[\alpha]/K}(\alpha)^{1/[K[\alpha]:K]}\]
    et tout se vérifie localement. En fait dès que $\alpha\in L$,
    on a $|N_{K[\alpha]/K}(\alpha)|^{1/[K[\alpha]:K]}=|\alpha|_L$.
\end{proof}

\begin{prop}
    Soit $(K,|.|)$ un corps ultramétrique et $L/K$ une extension finie.
    Alors il existe au plus $[L:K]$ extensions de $|.|$ à $L$.
\end{prop}
Sans les valeurs absolues, on a tjr un diagramme
% https://q.uiver.app/#q=WzAsNyxbMCwwLCJMIl0sWzAsMSwiSyJdLFsxLDEsIlxcd2lkZWhhdCBLIl0sWzIsMSwiXFx3aWRlaGF0IEteYyJdLFswLDIsInwufCJdLFsxLDIsInwufCJdLFsyLDIsInwufF9jIl0sWzAsMSwiIiwwLHsic3R5bGUiOnsiaGVhZCI6eyJuYW1lIjoibm9uZSJ9fX1dLFsxLDIsIiIsMCx7InN0eWxlIjp7ImhlYWQiOnsibmFtZSI6Im5vbmUifX19XSxbMiwzLCIiLDAseyJzdHlsZSI6eyJoZWFkIjp7Im5hbWUiOiJub25lIn19fV0sWzAsMywiIiwyLHsic3R5bGUiOnsiYm9keSI6eyJuYW1lIjoiZGFzaGVkIn19fV1d
\[\begin{tikzcd}
	L \\
	K & {\widehat K} & {\widehat K^c} \\
	{|.|} & {|.|} & {|.|_c}
	\arrow[no head, from=1-1, to=2-1]
	\arrow[dashed, from=1-1, to=2-3]
	\arrow[no head, from=2-1, to=2-2]
	\arrow[no head, from=2-2, to=2-3]
\end{tikzcd}\]
\begin{proof}
    On commence par remarquer qu'on en a une manière d'en avoir 
    $[L:K]$ via $x\mapsto |\sigma(x)|_c$ et qu'on a $|Aut(L/K)|\leq 
    [L:K]$. On montre que toute extension est donnée par 
    un plongement $\sigma\colon L\to \widehat K^c$. Étant donné 
    le diagramme 
% https://q.uiver.app/#q=WzAsNSxbMCwwLCIoTCx8LnxfTCkiXSxbMCwxLCIoSyx8LnwpIl0sWzEsMSwiSyciXSxbMSwwLCJcXHdpZGVoYXQgTCJdLFsyLDEsIlxcd2lkZWhhdCBLIl0sWzEsMiwiIiwwLHsic3R5bGUiOnsiaGVhZCI6eyJuYW1lIjoibm9uZSJ9fX1dLFswLDEsIiIsMCx7InN0eWxlIjp7ImhlYWQiOnsibmFtZSI6Im5vbmUifX19XSxbMCwzLCIiLDAseyJzdHlsZSI6eyJoZWFkIjp7Im5hbWUiOiJub25lIn19fV0sWzIsMywiXFxiaWdjdXAiLDEseyJzdHlsZSI6eyJib2R5Ijp7Im5hbWUiOiJub25lIn0sImhlYWQiOnsibmFtZSI6Im5vbmUifX19XSxbMiw0LCJcXHNpbSJdXQ==
\[\begin{tikzcd}
	{(L,|.|_L)} & {\widehat L} \\
	{(K,|.|)} & {K'} & {\widehat K}
	\arrow[no head, from=1-1, to=1-2]
	\arrow[no head, from=1-1, to=2-1]
	\arrow[no head, from=2-1, to=2-2]
	\arrow["\bigcup"{description}, draw=none, from=2-2, to=1-2]
	\arrow["\sim", from=2-2, to=2-3]
\end{tikzcd}\]
    où $K'$ est la cloture topologique de $K$ dans $\widehat L$, on
    obtient un isomorphisme $K'\to \widehat K$. On veut maintenant
    montrer que $\widehat L$ est finie sur $K'$. Si c'est le cas alors
    on a un $\tau$ tel que $|\tau(x)|':=|x|_{\widehat L}$ pour tout
    $x\in \widehat L$. C'est une valeur absolue sur 
    $\tau(\widehat L)\subset \widehat K^c$ qui étend celle de 
    $\widehat K$! L'unicité sur $\widehat K^c$ implique que 
    $|.|'=|.|_c$ d'où $|.|_{\widehat L}=|.|_c\circ\tau$.


    On prouve maintenant que $\widehat L$ est finie sur $K'$.
% https://q.uiver.app/#q=WzAsNSxbMCwwLCJMIl0sWzAsMiwiSyJdLFsyLDIsIksnIl0sWzIsMCwiXFx3aWRlaGF0IEwiXSxbMSwxLCJMLksnIl0sWzEsMiwiIiwwLHsic3R5bGUiOnsiaGVhZCI6eyJuYW1lIjoibm9uZSJ9fX1dLFswLDEsImZpbmllIiwyLHsic3R5bGUiOnsiaGVhZCI6eyJuYW1lIjoibm9uZSJ9fX1dLFswLDMsIiIsMCx7InN0eWxlIjp7ImhlYWQiOnsibmFtZSI6Im5vbmUifX19XSxbMiwzLCJcXGJpZ2N1cCIsMSx7InN0eWxlIjp7ImJvZHkiOnsibmFtZSI6Im5vbmUifSwiaGVhZCI6eyJuYW1lIjoibm9uZSJ9fX1dLFs0LDIsImZpbmllIiwwLHsic3R5bGUiOnsiaGVhZCI6eyJuYW1lIjoibm9uZSJ9fX1dLFs0LDAsIiIsMCx7InN0eWxlIjp7ImhlYWQiOnsibmFtZSI6Im5vbmUifX19XSxbNCwxLCIiLDEseyJzdHlsZSI6eyJoZWFkIjp7Im5hbWUiOiJub25lIn19fV0sWzQsMywiIiwxLHsic3R5bGUiOnsiaGVhZCI6eyJuYW1lIjoibm9uZSJ9fX1dXQ==
\[\begin{tikzcd}
	L && {\widehat L} \\
	& {L.K'} \\
	K && {K'}
	\arrow[no head, from=1-1, to=1-3]
	\arrow["finie"', no head, from=1-1, to=3-1]
	\arrow[no head, from=2-2, to=1-1]
	\arrow[no head, from=2-2, to=1-3]
	\arrow[no head, from=2-2, to=3-1]
	\arrow["finie", no head, from=2-2, to=3-3]
	\arrow[no head, from=3-1, to=3-3]
	\arrow["\bigcup"{description}, draw=none, from=3-3, to=1-3]
\end{tikzcd}\]
    On remarque que $L.K'$ est complet donc fermé dans $\widehat L$. 
    On conclut en remarquant que c'est dense parce que $L$ est dense
    dans $\widehat L$.
\end{proof}


\section{Extension de $|.|$ à une extension quelconque}
Il suffit de savoir étendre à une extension purement transcendante.
On regarde dans
\[K(x_i)_{i\in I}=\bigcup_{J\subset I;\#J<\infty} K(x_j)_{j\in J}\]
si on a une manière de définir des normes de manières canoniques sur les
extensions finiment générées. On a sur $K[X_1,\ldots,X_n]$ la norme
de Gauss 
\[||\sum \lambda_{k_1\ldots k_n} X_1^{k_1}\ldots X_n^{k_n}||=max|\lambda_{k_1\ldots k_n}|\in \R\]
est multiplicative! Et s'étend donc en une valeur absolue de 
$K(X_1,\ldots,X_n)$.
\begin{rem}
    En analyse $p$-adique on a besoin de corps algébriquement clos.
    Mais $\Q_p^c$ n'est pas complet mdr, donc on complète à nouveau 
    en $\widehat \Q_p^c$, et cette fois c'est complet et clos.
\end{rem}
\begin{thm}
    Soit $(K,|.|)$ un corps ultramétrique. Alors 
    \[\widehat{\widehat K^c}\]
    est complet et algébriquement clos.
\end{thm}
 On prouve d'abord une proposition, 
comme dans le cas réel ou complexe, soit $(K,|.|)$ un corps
valué. On ajoute la norme $1$ (au cas où c'est archimédien) sur
$K[X]$. On étend $|.|$ à $K^c$ ($\mathbb C$ si archimédien).
\begin{prop}[Continuité des racines]
    
    Soit $P\in K[X]$ un polynôme unitaire de degré $n$, alors 
    pour tout $\epsilon >0$, $\exists \delta>0$ dépendant uniquement
    de $P$ est $\epsilon$ tel que pour tout $Q\in K[X]$ unitaire
    de degré $n$ avec $||P-Q||_1<\delta$ on a, pour toute racine
    $\alpha\in K^c$ de $P$ il existe $|\alpha-\beta|<\epsilon$.
\end{prop}
\begin{proof}[Démonstration du théorème]
    On peut supposer $K$ complet. Soit $P\in \widehat{K^c}[X]$ et
    $\alpha$ une racine dans une extension finie. Pour tout $m\geq 1$
    et $\epsilon = 1/m$, on a $Q_m\in K^c[X]$ tel que 
    \[||P-Q_m||<\delta_m\]
    implique qu'il existe une racine $\beta_m$ de $Q_m(X)$ telle que
    $|\alpha-\beta_m|<1/m$. Clairement, $(\beta_m)_m$  est de 
    Cauchy et $\beta_m\in K^c$ d'où 
    $\alpha=\lim_m \beta_m\in \widehat{K^c}$ et on a fini. 
\end{proof}
\begin{proof}[Démonstration de la proposition]
    On peut noter que $||P||\geq 1$ implique $|\alpha|\leq ||P||$ si
    $|\alpha|\leq 1$. On suppose donc $||\alpha||\geq 1$. On a 
    \[|\alpha^n|=|-a_{n-1}\alpha^{n-1}-\ldots-a_0|\leq\left( \sum_{0\leq i\leq n-1}|a_i|\right)|\alpha|^{n-1}\]
    puis \[\leq ||P||.|\alpha|^{n-1}\implies |\alpha|\leq ||P||.\]
    Maintenant $|Q(\alpha)|\leq ||Q-P||max\{1,|\alpha|^n\}$ et 
    $Q(X)=\prod(X-\beta_i)\implies \prod |\alpha-\beta_i|\leq ||Q-P||max\{1,|\alpha|^n\}$
    c'est équivalent à ce qu'il existe $\beta-\beta_{i_0}$ t.q
    \[|\alpha-\beta|^n\leq ||Q-P||.||P||^n\implies |\alpha-\beta|\leq
    ||Q-P||^{1/n}.||P||.\]
\end{proof}

\section{À l'aide d'anneaux artiniens}
On peut commencer à discuter les propriétés du point de vue topologique.
Si on considère $\widehat K$ on peut regarder la $\widehat K$-algèbre
$B_L=L\otimes_K \widehat K$. En tant qu'ev, c'est de dimension $[L:K]$.
Elle a qu'un nombre fini d'idéaux maximaux, $\leq [L:K]$ car 
si on prends une suite décroissante $\m_1,\m_1\cap\m_2,\ldots,\cap_{j=1}
^i\m_i$ alors c'est une suite décroissante d'espaces vectoriels! D'où
la finitude. 

On voudrait maintenant construire une bijection entre
\[\{\textrm{Idéaux maximaux de }B_L\}\leftrightarrow\{\textrm{extensions
de $|.|_K$ à $L$}\}\]
\begin{proof}
    On commence avec $\m_0\subset B_L$, alors $B_L/\m_0$ est une 
    extension finie de $\widehat K$. Il existe une unique extension
    $|.|_{\m_0}$ de $|.|_K$ à $B_L/\m_0$. Via le morphisme canonique
    \[L\to L\otimes_K \widehat K\to B_L/\m_0\]
    le pullback de $|.|_{\m_0}$ à $L$ est une extension de $|.|_K$
    à $L$. On montre maintenant que si $\m_1$ en est un autre on obtient
    une extension différente. L'espace $L\otimes_K\widehat K$ est un
    Banach dans lequel $L$ est dense c'est assez clair. Si
    $\hat x\in \m_0\backslash\m_1$ et $L\ni a\sim \hat x$ alors 
    $|a|_{\m_0}\sim 0$ dans le quotient car $\hat x \mapsto 0$ mais
    $|a|_{\m_1}\ne 0$ parce que $\hat x\mapsto \ne 0$ dans $B_L/\m_1$.
    On a montré que $\m\mapsto |.|_{\m}$ est injective. À l'inverse si
    on regarde $\widehat L = \widehat (L,|.|_L)$ où $|.|_L$ étend 
    $|.|_K$. On regarde
% https://q.uiver.app/#q=WzAsNSxbMCwyLCJLIl0sWzIsMiwiXFx3aWRlaGF0IEsiXSxbMCwwLCJMIl0sWzIsMCwiXFx3aWRlaGF0IEwiXSxbMSwxLCJMXFxvdGltZXNfSyBcXHdpZGVoYXQgSyJdLFswLDJdLFswLDFdLFsxLDNdLFsyLDNdLFsxLDRdLFsyLDRdLFs0LDMsIiIsMix7InN0eWxlIjp7ImJvZHkiOnsibmFtZSI6ImRhc2hlZCJ9fX1dLFswLDRdXQ==
\[\begin{tikzcd}
	L && {\widehat L} \\
	& {L\otimes_K \widehat K} \\
	K && {\widehat K}
	\arrow[from=1-1, to=1-3]
	\arrow[from=1-1, to=2-2]
	\arrow[dashed, from=2-2, to=1-3]
	\arrow[from=3-1, to=1-1]
	\arrow[from=3-1, to=2-2]
	\arrow[from=3-1, to=3-3]
	\arrow[from=3-3, to=1-3]
	\arrow[from=3-3, to=2-2]
\end{tikzcd}\]
    Où $B_L\to L$ est le produit. Faut montrer que c'est bien défini,
    on sait que l'image est dense car $L$ est dense, et l'image est
    fermée, simplement par example parce que e.v. de dim finie sur 
    $\widehat K$ donc Banach donc complet donc fermé. On déf ensuite
    $\m=\ker(L\otimes_K\widehat K\to \widehat L)$ alors $\widehat L =
    B_L/\m$. ($\m$ dépend de la clôture choisie donc c'est bon)
\end{proof}

\begin{ex}
    Si on pose $L=\Q[\sqrt 2]$ et qu'on regarde $|.|_p$, on a 
    \[L\otimes_K \widehat K=\Q_p[X]/(X^2-2)\]
    dans $\Q_2$, $X^2-2$ est irréductible (regarder la valuation), 
    de sorte que $|.|_2$ s'étend de manière unique à $\Q[\sqrt 2]$!
    Dans $\Q_7$ ça split et on à deux extensions de $|.|_7$ a 
    $\Q[\sqrt 2]$.
\end{ex}

\section{Cas archimédien}
On appelle $r_1, 2r_2$ le nombre de plongement rééls, complexes de
$L/\Q$ dans $\C$. Si $L=\Q[\theta]$ et $P$ est le polynome minimal de
$\theta$ alors $r_1$ est le nombre de racines réelles et $2r_2$ le 
nombre de racines complexes de $P$.
\begin{prop}
    Il y'a exactement $r_1+r_2$ extensions de $|.|_{\infty}$ à $L$.
\end{prop}
\begin{ex}
    Si $L=\Q[i]$ alors $r_2=1$ et $r_1=0$. Faut montrer que la
    conjugaison complexe change pas la valeur absolue dans ce cas,
    on dirait qu'il faut étendre à $\C$ puis remarquer que la  
    restriction à $\R$ n'est est invariante. 
    Faut aussi utiliser que $z\mapsto \bar z$ est linéaire donc
    continue (\textbf{à faire}).
\end{ex}

\section{Corps local}
Si $(K,|.|)$ est complet ultramétrique. On veut montrer que si le corps
résiduel est fini alors $K$ est localement compact, i.e. $\Or_K$ est
compact. On suppose $|.|$ non triviale. 

\begin{proof}
    Soit $0<|t|<1$, on considère $\Or_K$ comme d'hab et $\Sigma\subset
    \Or_K$ un système de représentants de $\Or_K/t\Or_K$. On considère
    l'ensemble $\mathcal E\subset \Sigma^{\Z}$ des suites éventuellement
    nulle à gauche. On met la topologie discrète sur $\Sigma$ et 
    produit sur $\Sigma^{\Z}$. 
    \begin{prop}
        On a un homéomorphisme 
        \[f\colon \mathcal E\to K;(x_n)_n\mapsto \sum x_nt^n\]
    \end{prop}
    \begin{proof}[Preuve de la proposition]
        Faut montrer que c'est bien déf, suffit de voir que
        $|x_n||t|^n\leq |t|^n\to 0$. Ensuite la "norme" de $f$, 
        si $x\ne y \in \mathcal E$ on déf 
        $|f(x)-f(y)|$ comme le plus petit entier t.q 
        $x_{n_0}\ne y_{n_0}$. On montre que 
        $|f(x)-f(y)|=|x_{n_0}-y_{n_0}||t|^{n_0}$, on peut pas supposer
        $y=0$ parce que $x_{n_i}-y_{n_i}$ est pas nécessairement
        un représentant. 
        Il suffit de remarquer que comme $x_{n_0}-y_{n_0}\ne 0$ dans le
        quotient, alors $|x_{n_0}-y_{n_0}|>|t|$. Ensuite on factorise,
        $(x_{n_0}-y_{n_0})t^{n_0}$ et on conclut. 

        Soit maintenant $\alpha\in K^*$, on a un $n_0$ t.q. $|t|^{n_0+1}
        <|\alpha|\leq |t|^{n_0}$. On a $|t|<|\alpha/t^{n_0}\leq 1$ d'où
        dans $\Or_K$, il existe $x_{n_0}\in \Sigma$ t.q 
        \[|\alpha/t^{n_0}-x_{n_0}|\leq |t|\]
        d'où 
        \[|\alpha-x_{n_0}t^{n_0}|\leq |t|^{n_0+1}\]
        on peut alors construire de manière inductive $\alpha$. On
        a montré la bijection. Enfin, l'histoire des normes montre que
        c'est continu et même un homéo par la topologie produit (faire).
    \end{proof}
    \begin{cor}
        Si $\Or_K/t\Or_K$ est fini, alors $\Or_K$ est compact et 
        $K$ est localement compact.
    \end{cor}
    \begin{proof}
        On a $\mathcal E_0=\Sigma^\N\subset \mathcal E$. On doit étudier
        les suites de $\mathcal E_0=f^{-1}\Or_K$. On a
        $|f(x)|=|x_{n_0}||t|^{n_0}>|t|^{n_0+1}$ si $x_{n_0}\ne 0$ car
        $|x_{n_0}|>|t|$ d'où $1\geq |f(x)|>|t|^{n_0+1}\implies n_0\geq 0$. 
        Maintenant,  $\Sigma$ est compact puis $\Sigma^{\N}$ est compact.
        Enfin $K=\cup (\lambda +\Or_K)$.
    \end{proof}
\end{proof}
\chapter{Théorie algébrique}
On utilise des anneaux de valuations discrètes plutôt que des 
valeurs absolues.

\section{Anneaux de Dedekind}
Si on prends deux anneaux intègres $A\subset B$, $B$ 
intégralement clos alors la clôture intégrale de $A$
dans $B$ est intégralement close. Si on note $\tilde A$
la clôture intégral de $A$ dans $B$ et 
$\lambda\in Frac(\tilde A)$, entier sur $\tilde A\subset B$
alors entier sur $A$. Maintenant, $\lambda\in Frac(B)$ entier
sur $B$ implique $\lambda \in B\cap \tilde A=\tilde A$.
\begin{prop}
    Si $b\in B$ algébrique sur $Frac(A)=K$ et $P$ son
    polynôme minimal est unitaire. Alors $b$ est entier
    sur $A$ si $P[X]\in A[X]$. L'inverse est vrai si $A$ 
    est intégralement clos.
\end{prop}
\begin{proof}
    On prouve l'inverse, si $b$ est entier sur $A$ alors
    toute les racines sont entières sur $A$ par galois
    sur l'équation. D'où les coefficients sont entiers
    sur $A$ et dans $Frac(A)$ on conclut avec l'hyptothèse
    en plus sur $A$.
\end{proof}
\begin{cor}
    Si $b\in B$ est entier sur $A$ intégralement clos,
    $N_{L/K}(b), Tr_{L/K}(b)\in A$ où $L$ est juste une
    extension finie tq $b\in L$ et $K=Frac(A)$.
\end{cor}
\begin{proof}
    On regarde $N_{K[b]/K}(b)^{[L:K[b]]}, Tr(...)$.
\end{proof}
\begin{thm}
    Soit $A$ un anneau noethérien intègralement
    clos. Et soit une extension séparable finie 
    $L$ de $Frac(A)=K$. Alors la clôture intégrale
    de $A$ dans $L$, $B$, est un $A$-module de 
    type fini, en particulier noethérien.
\end{thm}
\begin{proof}
    On regarde les traces $Tr_{L/K}\colon L\times L\to K$
    qui sont bilinéaires sur $K$. Maintenant si $L/K$
    est séparable alors $Tr_{L/K}$ est non dégénérée
    (voir dans le Samuel). Soit $e_1,\ldots, e_n$ une
    base de $L$ sur $K$. On peut prendre $e_i\in B$
    via le coefficient dominant de leur polynôme minimal.
    On prend les bases duales \textbf{pour L}
    $e_i^*\in L$ et on regarde
    \[Tr_{L/K}(e_ie_j^*)=\begin{cases}1,i=j\\0\end{cases}\]
    Maintenant soit $b\in B$, on a $b=\sum \lambda_i e_i$
    avec $\lambda_i\in K$. En plus, $\lambda_i=Tr_{L/K}(be_i^*)$
    et $e_i^*$ étant algébrique il existe $t\in A\backslash 0$
    t.q $te_i^*\in B$ pour chaque $i$. Maintenant, 
    \[\lambda_i=(1/t)Tr_{L/K}(b(te_i^*))\in (1/t) A\]
    d'où $b\in \sum_{i=1}^n (1/t)Ae_i$ qui est un $A$-module
    de type fini, or $A$ est noethérien donc ce truc est 
    noethérien donc $B$ est de type fini sur $A$ car inclut
    dedans.
\end{proof}

\section{Anneaux de Dedekind}
\begin{defn}
    Un anneau de dedekind $A$ est un anneau t.q
    \begin{itemize}
        \item $A$ est noethérien.
        \item $A$ est intégralement clos.
        \item $A$ est de dimension $1$.
    \end{itemize}
\end{defn}
\begin{thm}[Décomposition en idéaux premiers]
    Il existe une décomposition unique de $(0)\ne I\subset A$ en idéaux 
    premiers \[I=\prod_{i=1}^r \m_i^{r_i}\]
    où les $r_i\geq 1$.
\end{thm}
Comment on obtient des anneaux de Dedekind maintenant ? On peut localiser
un anneau de Dedekind. Ca reste un anneau de Dedekind.
\begin{rem}
    Si on regarde topologiquement, $\Spec(S^{-1}A)\subset \Spec(A)$ 
    d'où la dimension est $\leq$.
\end{rem}

\begin{cor}
    Si $A$ est de dedekind, la localisation en $\p$ est un 
    anneau de valuation discrète.
\end{cor}

\begin{defn}
    Un anneau de valuation discrète est un anneau de Dedekind local
    qui n'est pas un corps.
\end{defn}
\begin{prop}
    Un anneau de valuation discrète est un anneau local principal
    qui n'est pas un corps.
\end{prop}
\begin{proof}
    Comme on admet le théorème de décomposition : si $\m\ne 0$ alors 
    $\m\ne \m^2$ par Nakayama. Ensuite si $t\in \m -\m^2$ alors 
    $tA=\m^k$ par le théorème de décomposition mdr, $t\notin \m^2$ 
    implique $k\leq 1$.
\end{proof}

on peut construire une valuation sur $(A,\m)$ puis sur $Frac(A)^*$. 
On réobtient $A=\{v(K^*)\geq 0\}$.

\begin{thm}
    Soit $A$ un anneau de Dedekind et $L/K=Frac(A)$ finie séparable.
    Alors la cloture de $A$ dans $L$, $B$, est un anneau de Dedekind 
    fini sur $A$.
\end{thm}
\begin{proof}
    Via l'exo du td $B$ est entier sur $A$ implique tout.
\end{proof}
\begin{ex}
    Si on prends $L/k(T)$, la clôture intégrale dans $L$ de $k[T]$
    fournit un morphisme fini $C\to \P^1$. Pareil, si on prends 
    $k((t))$ c'est transcendant sur $k(t)$. Si $k=\F_q$ alors $k(t)$
    est dénombrable contrairement à $k((t))$. On prends $s$ transcendant
    et la valuation $t$-adique. On regarde 
    \[k(t)\subset K=k(t,s^p)\subset L=k(t,s)\subset k((t))\]

    On a $[L:K]=p$, $\m_K=t\Or_K$ et $\m_L=t\Or_L$ de corps résiduels 
    $k$ avec les valuations restreintes. Alors, 
    \[\Or_L\textrm{ est un a.v.d, pas fini sur} \Or_K\]
    On verra la preuve plus tard. Le point c'est que c'est inséparable
    donc la dimension est pas finie.
\end{ex}
\begin{thm}[Krull-Akizuki]
    Si $A$ est de Dedekind et $L/K=Frac(A)$ finie. Alors la clôture
    de $A$ dans $L$ est de Dedekind. 
\end{thm}
\begin{thm}
    Si $A$ est un a.v.d avec les hyp du dessus. ALors la clôture est
    d'Artin.
\end{thm}
\begin{lem}
    Si $L/K$ est purement insép finie. Alors on obtient un anneau de
    valuation discrète.
\end{lem}
\begin{proof}
    Soit $p=char(K)>0$. Il existe $e>0$, $L^{p^e}\subset K$. On note
    $A=\Or_K$ et $\Or_L$ la clôture dans $L$. On construit une valuation
    sur $L^*$ via $v(x^{p^e})/p^e$. C'est une valuation. Et maintenant
    $v_L(x)\geq 0$ est équivalent à $x^{p^e}\in \Or_K$ est entier sur
    $\Or_K$. À l'inverse si $x\in \Or_L$ t.q. $x^{p^e}\in K$ est entier
    sur $\Or_K$ comme $\Or_K$ est de dedekind on a fini.
\end{proof}

\chapter{Ramification}
On étudie les extensions d'anneaux de valuations discrètes. On prends
$\Or_K$ un d.v.r, $k$ son corps résiduel, $K$ son corps de fractions. Et
$\pi_K$ une uniformisante.

\section{Définitions}
\begin{defn}
    Une extension de d.v.r est un morphisme d'anneaux locaux.
\end{defn}

\begin{defn}[Indice de ramification]
    C'est l'exposant $e_{L/K}$ tel que $\m_K\Or_L=\m_L^{e_{L/K}}$.
    On peut le voir sur les uniformisantes.
\end{defn}

On définit, pour $|.|_K$ associée à $\Or_K$ et étendue à $L$. La
valuation normalisée $v_K\colon K^*\to \Z$ via $v_K(\pi_K)=1$. Alors
\[v_L(\pi_K)=e\].
\begin{defn}[Degré résiduel]
    On déf $f_{L/K}:=[\Or_L/\m_L : \Or_K/\m_K]$ la dimension de
    l'extension résiduelle.
\end{defn}
\begin{ex}
    Si on pose $K=\C(z)$ et $L=K[t]$ avec $t^d=z$. Alors $\C[t]$ est
    entier sur $\C[z]$ et un PID donc la clôture intégrale de $\C[z]$
    dans $L$. Les idéaux au dessus, $(z-c)\C[z]$, y'a deux cas :
    \begin{itemize}
        \item $c\ne 0$ et la le polynôme se scinde à racines simples
            et indice de ramification $1$.
        \item $c=0$, les racines de l'unité sont inversibles. Donc
            seulement $(t)$ contient $(z)$ et indice de ramification
            $d$.
    \end{itemize}
    Si $c$ est proche de $0$, les branches se "ramifient".
\end{ex}
\begin{ex}
    Si on regarde $K=\Q[\sqrt d]$ avec $d$ sans facteurs carrés. On
    peut voir avec la trace que $\Z[\sqrt d]$ est d'indice au plus
    $2$ dans $\Or_K$.
\end{ex}

\begin{prop}
    L'indice de ramification et le degré résiduel sont multiplicatifs
    pour les sous-extensions.
\end{prop}
\begin{prop}
    L'indice de ramification et le degré résiduel sont invariants
    par complétions.
\end{prop}
\begin{thm}
    Pour une extension finie $L/K$. Avec $\tilde \Or_K$ la clôture de 
    $\Or_K$ dans $L$. En notant $\m_1,\ldots,\m_n$ les idéaux maximaux
    de $\tilde \Or_K$, $\Or_K\to (\tilde\Or_K)_{\m_i}$ est une
    extension de d.v.r. Alors 
    \[\sum_i e_if_i\leq [L:K]\]
    et les propriétés suivantes sont équivalentes 
    \begin{enumerate}
        \item l'inégalité est une égalité.
        \item $\tilde \Or_K$ est fini sur $\Or_K$.
        \item $L\otimes_K \widehat K$ est réduite.
    \end{enumerate}
    En particulier, on a une égalité si $L/K$ est séparable ou
    si $K$ est complet.
\end{thm}
\begin{lem}
    Si $\Or_K\to \Or_L$ est une extension de d.v.r alors
    \[dim_k \Or_L/\m_K\Or_L =e.f \in \N\cup\{\infty\}\]
\end{lem}
\begin{proof}
    Voir cours. En gros on considère la famille 
    $\{\pi_L^i b_j | 0\leq i\leq e-1,1\leq j\leq r\}$
    pour $b_1,\ldots, b_r\in \Or_L$ une famille libre dans le quotient. 
    On montre que la famille est libre puis que $r=f$ quand les quantités
    sont finies. Pour ça on lift une base de $k_L$, $b_1,\ldots, b_f$,
    et on a $\Or_L\subset \sum_j b_j\Or_K +\pi_L\Or_L$. On a plus 
qu'à itérer $e$ fois. On obtient \[\Or_L\subset\left( \sum_{0\leq i\leq e-1;1\leq j\leq f} b_j\pi_L^i\Or_K\right)+\pi_K\Or_L\]. Si on appelle
$M$ le gros truc à gauche. En quotientant par $\pi_K\Or_L$ on obtient
une famille génératrice du quotient de taille $ef$.
\end{proof}

\begin{rem}
    Si $\Or_L$ est fini sur $\Or_K$ alors $\Or_L=M$ par Nakayama!
\end{rem}
\begin{lem}
    \[dim_k \tilde \Or_K/\m_K\tilde \Or_K\leq [L:K]\]
\end{lem}
\begin{note}
    Dans le premier lemme, on regarde qu'un idéal, là on les regarde
    tous.
\end{note}
\begin{proof}
    Voir cours. Je crois que c'est juste qu'une famille libre dans le
    quotient doit être libre au dessus. D'où le résultat.
\end{proof}

\begin{lem}
    Si $K$ est complet :
    \[dim_k \tilde \Or_K/\m_K\tilde \Or_K= [L:K]\]
\end{lem}
\begin{proof}
    Faut faire attention, $\Or_L$ est pas nécessairement fini sur
    $\Or_K$. Mais on a $\Or_L\subset M+\pi_K\Or_L$ et en itérant, 
    $\Or_L\subset M+\pi_K^N\Or_L$ pour tout $N$. Donc $M$ est dense 
    dans $\Or_L$. Donc $M\otimes_{\Or_K}K$ est dense dans $L$
    de dimension finie sur $K$ complet donc fermé + complet dans $L$.
    D'où on a l'égalité.
\end{proof}

\begin{lem}
    La flèche induite $A/\m^r \to A_{\m}/\m^rA_{\m}$ est un isomorphisme
    dès que $\m$ est maximal.
\end{lem}
\begin{thm}
    On se place dans $L/K$ finie, $\Or_K$ un d.v.r et $\tilde\Or_K$
    la clôture de $\Or_K$ dans $L$. On considère $\m_1,\ldots, \m_n$
    les idéaux maximaux de $\tilde\Or_K$. Puis via les
    $(\tilde\Or_K)_{\m_i}/\Or_K$ on à les $e_i,f_i$ et alors 
    \[\sum_i e_if_i\leq [L:K]\]
    En plus on a égalité ssi $\tilde\Or_K$ est fini sur $\Or_K$ ssi
    $L\otimes_K \widehat K$ est réduite.
\end{thm}
\begin{proof}[Preuve de l'inégalité]
    Comme on est dans un anneau de Dedekind on a 
    \[\m_K\tilde\Or_K=\prod_i \m_i^{r_i}\]
    et par localisation on voit que $r_i=e_i$. Enfin par CRT
    et le dernier lemme on a 
    $\tilde\Or_K/\m_K \simeq \prod (\tilde\Or_K)_{\m_i}/\m_i^{e_i}(\tilde\Or_K)_{\m_i}$.
    avec $\m_i^{e_i}(\tilde\Or_K)_{\m_i}=\m_K(\tilde\Or_K)_{\m_i}$.
    D'où par le premier lemme et le deuxième lemme on obtient 
    l'inégalité.
\end{proof}

\begin{proof}[Preuve des équivalences]
    Si maintenant on a l'égalité on veut montrer que $\tilde\Or_K$ 
    est fini sur $\Or_K$. Il montre qu'une base relevée de
    $\tilde\Or_K/\m_K$ est libre sur $\Or_K$ et $K$, donc
    une base de $L$. En plus par unicité de la décomposition et via
    l'écriture dans le quotient c'est une base de $\tilde\Or_K$ sur
    $\Or_K$. Si $\tilde\Or_K$ est fini à l'inverse, alors il est libre
    car d.v.r sont principaux et son rang est le même après $\otimes K$.
\end{proof}

\section{Trouver les indices de ramification/degrés résiduels}
\begin{prop}
    On suppose que $\tilde\Or_K=\Or_K[\alpha]$ pour un $\alpha$. Soit
    $P\in K[X]$ le polynôme minimal unitaire de $\alpha$. Alors 
    $P\in \Or_K[X]$ et soit 
    $\bar P(X)=\prod_i p_i(X)^{r_i}\mod \tilde\m_K$ la décomposition
    irréductible dans $k[X]$. Soit $P_i = p_i \mod \tilde \m_K$ des
    lifts alors
    \begin{enumerate}
        \item Les idéaux maximaux de $\tilde \Or_K$ sont exactement
            les $\m_i$ où $\m_i=(\m_K, P_i(\alpha))$.
        \item $(\tilde\Or_K)_{\m_i}/(\m_i)\simeq k[X]/(P_i(X))$.
        \item $e_i=e_{(\tilde\Or_K)_{\m_i}/\Or_K}=r_i$.
    \end{enumerate}
\end{prop}
\begin{proof}
    Les idéaux maximaux de $(\tilde\Or_K)$ contiennent tous $\m_K$
    d'où on a une bijection avec les idéaux du quotient $\tilde k$. 
    On conclut en prenant les images inverses qui sont maximales parce
    que le quotient par elles est un corps. Ça prouve $1.$ et $2.$, pour
    $3.$ on a $P(X)=\prod P_i(X)^{r_i} +\epsilon(X)$ avec $\epsilon(X)
    \in \m_K[X]$. Alors 
    $0=P(\alpha)=\prod P_i(X)^{r_i}+\epsilon(\alpha)$. D'où
    $\prod_i P_i(\alpha)^{r_i}=-\epsilon(X)\in \m_K\tilde\Or_K$. 
    Ensuite 
    \[\prod \m_i^{r_i} \subset \m_k\tilde\Or_K+ P_i(\alpha)^{r_i}\tilde\Or_K\]
    d'où $\prod \m_i^{r_i}\subset \m_K\tilde \tilde\Or_K$. Comme 
    on est dans un anneau de Dedekind $e_i\leq r_i$. Ensuite
    comme $\sum r_i f_i \geq \sum e_i f_i =[L:K]$, et qu'à gauche
    on a le degré de $\bar P$ qui est monique donc pareil que le degré
    de $P=[K[\alpha]:K]$.
\end{proof}
\begin{note}
    Voir pq y contiennent tous $\m_K$ c'était dans le dernier cours.
\end{note}
On suppose tjr maintenant que $\tilde\Or_K$ est finie sur $\Or_K$.
\begin{defn}
    $\Or_K\to \Or_L$ une extension de d.v.r est non ramifiée si 
    $\begin{cases} e=1 \\ k_L/k\textrm{ est finie séparable}\end{cases}$
    Dans le cas général, $L/K$ est non ramifiée si $\Or_K\to(\tilde\Or_K)_{\m_i}$ est non ramifiée pour tout $\m\subset\tilde\Or_K$.
\end{defn}
Topologiquement, ca ressemble à un revêtement.
\begin{cor}
    On suppose que $L=K[\alpha]$ et $P$ le polynôme minimal de $\alpha$
    est dans $\Or_K[X]$ ainsi que $\bar P$ est séparable. Alors
    $\tilde \Or_K =\Or_K[\alpha]$ et $L/K$ est non ramifiée (si $k$
    est parfait).
\end{cor}
\begin{proof}
    On pose $B=\Or_K[X]/(P(X))\simeq \Or_K[\alpha]\subset\tilde\Or_K$.
    Alors $B\otimes_{\Or_K} K=K[\alpha]=L$ est intégralement clos,
    le quotient $B/\m_K B=k[X]/(\bar P(X))$ est réduit car $\bar P(X)$
    est séparable. Par l'exercice $1$ du td4, $B=\Or_K[\alpha]$ est
    intégralement clos d'où l'égalité. Enfin par la prop précédente
    la séparabilité de $\bar P$ force $e_i=1$.
\end{proof}
\begin{ex}
    On considère pour $p>2$ : $\Q_p\to L=\Q_p[\zeta_p]$ avec 
    $\zeta_p\in \C_p$. On note $\tilde \Q_p$ la clôture de $\Z_p$
    dans $L$, c'est un d.v.r $\Or_L$. On a 
    $\lambda_p=\zeta_p -1\in \m_L$ puis 
    \[(\lambda_p+1)^p=1\]
    et \[\m_L^{p-1}\ni\lambda_p^{p-1}=p(-1-\lambda_p c)\in\m_L^{e}\]
    avec $c\in \Or_L$. Comme $\lambda_p\in\m_L$, le terme à droite
    est inversible, on obtient
    $e_{\Or_L/\Z_p}\geq p-1$. Comme $ef\leq [L:\Q_p]\leq p-1$, $f=1$
    et on a pas de racines $p$-èmes de l'unité dans $\Q_p$.
\end{ex}
\begin{ex}
    Si $p$ est premier et $n$ est premier à $p$. En notant 
    $\phi_n(X)$ le $n$-ème polynôme cyclotomique et $F_n$ le polynôme
    minimal de $\zeta_n$ sur $\Q_p$ on a $F_n\mid \phi_n$ d'où
    $F_n$ est dans $\Z_p[X]$ monique car $\phi_n\in \Z[X]$ est monique. 
    En plus, $\bar F_n$ est séparable car $\phi_n$ l'est. En particulier
    $\Or_L=\Z_p[\zeta_n]$ et est non ramifiée sur $\Z_p$ par le
    corollaire précédent.
\end{ex}


\section{Extensions monogènes}
À rattraper : étant donné $K-L$ finie de corps valués discrets.
Supp. $\Or_K-\tilde\Or_K$ est finie. Sous des hypothèses modestes,
$\tilde\Or_K$ est monogène sur $\Or_K$. Par exemple,

\begin{enumerate}
    \item Si $K$ est complet et $L$ est totalement ramifiée 
        ($[L:K]=e$, et l'extension résiduelle est triviale.).
        Alors $\Or_L=\Or_K[\alpha]$ pour un $\alpha$ 
        annulé par un polynôme d'Eisenstein ($v(a_i)>0,v(a_0)=1$).
        Si en plus $L/K$ est modérée, ($e_{L/K}\wedge char(k)=1$
        ou $char(k)=0$. On peut prendre $\alpha^n-\pi_K=0$).
\end{enumerate}
\begin{lem}
    Si $K$ est complet, et on a une extension
% https://q.uiver.app/#q=WzAsNSxbMCwwLCJLIl0sWzIsMCwiTCJdLFswLDEsImsiXSxbMSwxLCJrJyJdLFsyLDEsImtfTCJdLFswLDFdLFsyLDNdLFszLDRdLFswLDIsIiIsMSx7InN0eWxlIjp7ImhlYWQiOnsibmFtZSI6Im5vbmUifX19XSxbMSw0LCIiLDEseyJzdHlsZSI6eyJoZWFkIjp7Im5hbWUiOiJub25lIn19fV1d
\[\begin{tikzcd}
	K && L \\
	k & {k'} & {k_L}
	\arrow[from=1-1, to=1-3]
	\arrow[no head, from=1-1, to=2-1]
	\arrow[no head, from=1-3, to=2-3]
	\arrow[from=2-1, to=2-2]
	\arrow[from=2-2, to=2-3]
\end{tikzcd}\]
    avec $k'/k$ séparable. Alors il existe une
    unique extension $L/K'/K$ t.q $k_{K'}=k'$.
    Avec $K'/K$ non ramifiée.  
\end{lem}
\begin{lem}
    On prends maintenant $e'|e_{L/K}$,
    avec $e'$ premier à $char(k)$. Avec
    $k_K-k_L$ purement inséparable. Alors
    il existe un lift unique $L/K'/K$ t.q.
    $k_K=k_{K'}$ et $K'/K$ est de degré $e'$
    totalement ramifiée.

\end{lem}
\begin{thm}
    Soit $K$ un corps complet et $L/K$ finie.
    On considère
% https://q.uiver.app/#q=WzAsNSxbMCwwLCJLIl0sWzIsMCwiTCJdLFswLDEsImsiXSxbMSwxLCJrJyJdLFsyLDEsImtfTCJdLFswLDFdLFsyLDMsInNlcCIsMl0sWzMsNF0sWzAsMiwiIiwxLHsic3R5bGUiOnsiaGVhZCI6eyJuYW1lIjoibm9uZSJ9fX1dLFsxLDQsIiIsMSx7InN0eWxlIjp7ImhlYWQiOnsibmFtZSI6Im5vbmUifX19XV0=
\[\begin{tikzcd}
	K && L \\
	k & {k'} & {k_L}
	\arrow[from=1-1, to=1-3]
	\arrow[no head, from=1-1, to=2-1]
	\arrow[no head, from=1-3, to=2-3]
	\arrow["sep"', from=2-1, to=2-2]
	\arrow[from=2-2, to=2-3]
\end{tikzcd}\]
    On a 
    \begin{enumerate}
        \item Pour tout $k'\subset k_L$ séparable sur $k$
            il existe une unique $K-K'-L$ une unique
            sous extension non ramifiée $K-K'-L$ de corps
            résiduel $=k'$.
        \item Il existe une plus grande sous-extension non ramifiée
            $K\subset K^u\subset L$ de $K$.
        \item Il existe une plus grande sous-extension modérément 
            ramifiée de $L$.
    \end{enumerate}
    On obtient $K-K^{un}-K^{tam}-L$ avec $[L:K^{tam}]$ une 
    puissance de $p$.
\end{thm}
\begin{proof}[Preuve de 1.]
    \textbf{ Existence} : on trouve un élément primitif de 
    $k'/k$ séparable finie.
    D'où $k'=k(\theta)$, on trouve $P\in \Or_K[X]$ t.q
    $\theta$ est une racine de $\bar P$. Comme $P$ irréd
    sur $k$. Alors irréd sur $\Or_K$ (TD6). Comme $\bar P$ 
    est séparable, en regardant $P\in \Or_L[X]$ on peut 
    lift $\theta$ par Hensel. Disons en $\alpha$.
    On a en plus $K'=K[\alpha]/K$ et $\Or_K[\alpha]=\Or_{K'}$
    car non ramifiée et $k'=k(\theta)=k_{K'}$.

    \textbf{Unicité :} Si $K''/K$ est une autre extension t.q $k_{K''}=k'=k(\theta)$.
    Alors $\theta$ lift en une racine $\alpha'\in K''\subset L$.
    Le lemme dans Hensel dans $L$ fournit le même $\alpha'=\alpha$.
    Maintenant $K'$ et $K''$ ont la même dimension, 
    \[[K':K]=[k':k]=[K'':K]\]
\end{proof}
\begin{proof}[Preuve de 2.]
    Soit $k'=k^s$ la clôture séparable de $k$
    dans $k_L$. Alors on pose $K^{un}$ l'unique
    extension non ramifiée t.q. $k_{K^{un}}=k^s$.
    Si $K-K'-L$ est une extension non ramifiée, alors 
    $k\subset k_{K'}\subset k^s\subset k_L$. Il existe 
    \[K\subset K''\subset K^{un}\]
    t.q. $k_{K''}=k_{K'}$. On obtient deux sous-extensions
    non ramifiée de corps résiduels $k_{K'}$ on conclut
    par $1.$.
\end{proof}
\begin{proof}[Preuve de 3.]
    On a tjr $e_{L/K}=e_{L/K^{un}}$. On supp 
    $char(k)>0$ sinon $K^{tam}=L$. Soit $e'$
    le plus grand diviseur de $e$ non divisible
    par $p$. Par le lemme précédent on a 
% https://q.uiver.app/#q=WzAsNixbMCwwLCJLXnt1bn0iXSxbMiwwLCJMIl0sWzAsMSwia15zIl0sWzEsMSwia15zIl0sWzIsMSwia19MIl0sWzEsMCwiS157dGFtfSJdLFsyLDNdLFszLDRdLFsxLDQsIiIsMSx7InN0eWxlIjp7ImhlYWQiOnsibmFtZSI6Im5vbmUifX19XSxbMCw1LCJlJyIsMV0sWzUsMV0sWzAsMiwiIiwxLHsic3R5bGUiOnsiaGVhZCI6eyJuYW1lIjoibm9uZSJ9fX1dXQ==
\[\begin{tikzcd}
	{K^{un}} & {K^{tam}} & L \\
	{k^s} & {k^s} & {k_L}
	\arrow["{e'}"{description}, from=1-1, to=1-2]
	\arrow[no head, from=1-1, to=2-1]
	\arrow[from=1-2, to=1-3]
	\arrow[no head, from=1-3, to=2-3]
	\arrow[from=2-1, to=2-2]
	\arrow[from=2-2, to=2-3]
\end{tikzcd}\]
    On a $K^{tam}/K^{un}$ est totalement modérément
    ramifiée. Et $K^{un}/K$ est non ramifiée sur $K$.
    D'où $K^{tam}/K$ est modérément ramifiée.
    On montre que c'est la plus grande, si on a $F/K$
    modérée et $F\subset L$ on considère $F^{un}/K$
    non ramfiée et remplacer $F$ par $F^{un}$. On
    a $k_F=k^s$ et par $1.$ on obtient 
    \[K^{un}\subset F\subset L\]
    par la preuve précédente le lift de $\theta$ le 
    générateur $K^{un}$ est dans $F$ (on peut prendre
    le pol a coeff dans $\Or_F[X]$.)

    Maintenant la preuve découle de l'unicité dans le 
    lemme précédent.
\end{proof}

\begin{cor}
    Soit $L/K$ finie et $K$ complet. On a 
% https://q.uiver.app/#q=WzAsNixbMiwwLCJLXnt1bn0iXSxbNiwwLCJMIl0sWzQsMCwiS157dGFtfSJdLFswLDAsIksiXSxbNCwxLCJrX3tLXnt0YW19fSJdLFs2LDEsImtfTCJdLFswLDIsIm1vZCt0b3QiLDJdLFsyLDEsIltMOktee3RhbX1dPXBeciIsMl0sWzMsMCwibm9ufnJhbWlmacOpZSIsMl0sWzQsNSwidG90fmluc8OpcCIsMV0sWzIsNCwiIiwxLHsic3R5bGUiOnsiaGVhZCI6eyJuYW1lIjoibm9uZSJ9fX1dLFsxLDUsIiIsMSx7InN0eWxlIjp7ImhlYWQiOnsibmFtZSI6Im5vbmUifX19XV0=
\[\begin{tikzcd}
	K && {K^{un}} && {K^{tam}} && L \\
	&&&& {k_{K^{tam}}} && {k_L}
	\arrow["{non~ramifiée}"', from=1-1, to=1-3]
	\arrow["{mod+tot}"', from=1-3, to=1-5]
	\arrow["{[L:K^{tam}]=p^r}"', from=1-5, to=1-7]
	\arrow[no head, from=1-5, to=2-5]
	\arrow[no head, from=1-7, to=2-7]
	\arrow["{tot~insép}"{description}, from=2-5, to=2-7]
\end{tikzcd}\]
\end{cor}

\begin{ex}
    Soit $K=\Q_p$,$L=\Q_p(\zeta_{p^r})$ et $r\geq 1$.
    On a $\Q_p-\Q_p(\zeta_p)$ est modérément totalement
    ramifiée de degré $p-1$. Maintenant par l'exam
    \[[L:\Q_p(\zeta_p)]=p^{r-1}\]
    est sauvagement totalement ramifiée. On pouvait 
    écrire explicitement 
    $\lambda_{p^r}^{p^{r-1}}=\lambda_p(1+\epsilon)$
    avec $1+\epsilon$ une unité.

    On obtient $\Q_p^{un}=\Q_p$ et $\Q_p^{tam}=\Q_p(\zeta_p)$.
    Le polynôme d'Eisenstein est donné par 
    \[1+\lambda_p=(1+\lambda_{p^r})^{r-1}\]
\end{ex}

\section{Théorie de Galois}
Si on a $L/F/K$ avec $L/F$ galoisienne 
alors $F/K$ est galoisienne ssi $Gal(L/K).F\subset F$.
Et cette dernière condition fournie la suite exacte
\[1\to Gal(L/F)\to Gal(L/K)\to Gal(F/K)\to 1\]
Maintenant $L/K$ est normale ssi $K^s/K$ est galoisienne.
Et $Aut_K(L)=Gal(K^s/K)$. Aussi $K\subset L\subset \tilde L$
avec $\tilde L/K$ normale est galoisienne si $L/K$ est séparable.
Il existe de telles extensions minimales.

Maintenant si on regarde une extension galoisienne
finie $L/K$ et $K$ de valuation discrète. Alors
\[\Or_K\to \tilde\Or_K\]
et finie et $\tilde\Or_K$ est semi-local de Dedekind.
Maintenant $G=Gal(L/K)$ agit sur $Spm(\tilde\Or_K)$
et on a
\begin{prop}
    $G$ agit transitivement sur $\tilde\Or_K$.
\end{prop}
\begin{lem}[Lemme d'évitement]
    Soit $A$ un anneau commutatif et 
    $\p_1,\ldots,\p_n$ des idéaux premiers.
    Supp $\p_1\nsubset \p_i$ pour tout $i\geq2$.
    Alors $\p_1\nsubset \cup_i \p_i$.
\end{lem}
\begin{proof}[Preuve du lemme]
    Dans le cas des idéaux maximaux, c'est un
    CRT direct. Le cas $n=3$ est plutôt simple ?
    Sinon, par induction.
\end{proof}
\begin{proof}[Preuve de la proposition]
    On fixe un idéal maximal $\m$ et on suppose 
    qu'il existe $\m'$ t.q. $\m\ne \sigma(\m')$ pour tout 
    $\sigma$. On obtient $x\in\m-\cup \sigma(\m')$.
    Maintenant $N_{L/K}(x)\in \m_K\subset \m\cap \sigma(\m')$
    d'où on a pour un $\m'$, $\sigma_0(x)\in \m'$ d'où
    $x\in \sigma_0^{-1}(\m')$. Ce qui contredit l'hyptohèse.
\end{proof}

\begin{defn}
    Soit $\m$ un idéal maximal de $\tilde \Or_K$, on 
    définit $D_\m=\{\sigma(\m)=\m\}$ le groupe de décom-
    position de $\m$. 
\end{defn}
On remarque que $D_\m$ agit sur $(\tilde\Or_K)_{\m}$
d'où sur le corps résiduel $\tilde\Or_K/\m=:k_\m$.
Si $\sigma\in D_\m$ et $\alpha\in\tilde\Or_K$ alors
\[\sigma(\bar\alpha)=\bar(\sigma(\alpha))\]
est bien défini. On obtient
$D_\m\to Aut_k(k_\m)$ et on note 
$I_\m$ le noyau tel que 
\[1\to I_\m\to D_\m\to Aut_k(k_\m)\]
c'est le groupe d'inertie en $\m$.
\begin{cor}
    Il existe une bijection entre les idéaux maximaux
    et les coset $G/D_\m$ donnée par 
    \[G\to Spm(\tilde\Or_K)\]
    \[\sigma\mapsto \sigma(\m)\]
\end{cor}
\begin{rem}
    Si $\sigma_0(\m)=\m'$ alors 
    \[D_{\m'}=\sigma_0D_\m\sigma_0^{-1}\]
    et
    \[I_{\m'}=\sigma_0I_\m\sigma_0^{-1}\]
\end{rem}
\begin{cor}
    Les indices de ramifications et les degrés
    résiduels sont égaux à tout les
    premiers. 
\end{cor}
On va montrer qu'en fait 
\[1\to I_\m\to D_\m\to Aut_k(k_\m)\]
s'étend en 
\[1\to I_\m\to D_\m\to Aut_k(k_\m)\to 1\]
\begin{lem}
    L'extension $k_\m/k$ est normale.
\end{lem}
\begin{proof}
    On montre que pour tout $\theta\in k_\m$, ses 
    conjugués sont dans $k_\m$. On lift en un $\alpha$
    dans $\tilde\Or_K$. G agit sur le polynôme min $P$ 
    d'où $P\in \Or_K[X]$. Puis $\bar P\in k[X]$ (car unitaire).
    D'où le polynome minimal de $\theta$ divise $P$ qui est 
    split dans $k_\m$ par réduction.
\end{proof}
\begin{prop}
    On montre que la s.e.c est exacte.
\end{prop}
\begin{proof}
    On a vu que $k_\m/k$ est normale. On
    note $k_\m^s$ la clôture séparable. Alors
    $k_\m^s/k$ est galoisienne et $Aut_k(k_\m)=
    Gal(k_\m^s/k)$. On écrit $\k_\m^s=k[\theta]$.
    Soit $\tau\in Gal(k_\m^s/k)$, $\tau$ est 
    déterminé par $\tau(\theta)$. On note
    $r\colon D_\m\to Aut_k(k_\m)$, on montre
    qu'il existe $\sigma\in D_\m$ tq 
    \[r(\sigma)(\theta)=\tau(\theta)\]. On lift
    $\theta$ en $\alpha\in \tilde\Or_K$. Alors
    $r(\sigma)(\alpha)=\overline{\sigma(\alpha)}$.


    On prends $P=\prod (X-\sigma(\alpha))\in \Or_K[X]$,
    et on a $\bar P(\theta)=0$. Maintenant les conjugués
    de $\theta$ sont des racines de $\bar P$, qui sont
    des conjugués de $\alpha$. On a $\tau(\theta)=\sigma(\alpha)$
    pour un $\sigma\in G$, reste à trouver $\sigma\in D_\m$. 
    On peut le trouver en choisissant $\alpha$ mieux.

    Plutôt que la flèche du haut dans 
    % https://q.uiver.app/#q=WzAsNCxbMCwwLCJcXHRpbGRlXFxPcl9LIl0sWzEsMCwia19cXG0iXSxbMCwxLCJcXHRpbGRlXFxPcl9LIl0sWzEsMSwiXFxwcm9kX1xcbSBcXHRpbGRlXFxPcl9LL1xcbSJdLFswLDFdLFsyLDNdXQ==
\[\begin{tikzcd}
	{\tilde\Or_K} & {k_\m} \\
	{\tilde\Or_K} & {\prod_\m \tilde\Or_K/\m}
	\arrow[from=1-1, to=1-2]
	\arrow[from=2-1, to=2-2]
\end{tikzcd}\]
    on la factorise par la flèche du bas. Et on prends 
    $\alpha$ tel que $\alpha=0$ modulo les autres premiers.
    Reste à voir que les $\sigma\in G$ tels que 
    $\overline{\sigma(\alpha)}=\tau(\theta)$ sont dans $D_\m$.
    Par construction, $\alpha\in \m'$ pour $\m'\ne\m$. D'où
    si $\sigma\notin D_\m$ alors $\sigma^{-1}\m=\m'\ne\m$ puis
    $\sigma^{-1}\sigma\alpha\in \m'$, d'où $\sigma(\alpha)=0\mod \m$.
    Ce qui est contradictoire si $k_\m^s\ne k$.
\end{proof}
Maintenant on peut regarder la tour 
\[K-L^{D_\m}-L^{I_\m}-L\]
et $\m$ correspond a une valeur absolue de $K$ à $L$,
$|.|_\m$. On note les deux restrictions par $|.|_D$ et
$|.|_I$.
\begin{thm}
    On a 
    \begin{enumerate}
        \item $(K,|.|_K)-(L^D,|.|_D)$ a indice de 
            ramification $1$ et une extension de 
            corps résiduels trivial.
        \item $(L^D,|.|_D)-(L^I,|.|_I)$ est non
            ramifiée et d'extension de corps 
            résiduels $k-k_\m$.
        \item $(L^I,|.|_I)-(L,|.|_L)$ est d'indice
            de ramification $|D_\m|$ et l'extension
            de corps résiduels est purement
            inséparable.
    \end{enumerate}
\end{thm}
\begin{rem}
    $L^I/K$ est alors non ramifiée et $L^I/L^D$
    est galoisienne non ramifiée.
\end{rem}



\printbibliography
\end{document}


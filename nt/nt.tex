\documentclass[a4paper,12pt]{book}
\usepackage{amsmath,  amsthm,enumerate}
\usepackage{csquotes}
\usepackage[provide=*,french]{babel}
\usepackage[dvipsnames]{xcolor}
\usepackage{quiver, tikz}

%symbole caligraphique
\usepackage{mathrsfs}

%hyperliens
\usepackage{hyperref}

%pseudo-code
\usepackage{algpseudocode}
\usepackage{algorithm}
\makeatletter
  \renewcommand{\ALG@name}{Algorithme}
  \makeatother
\usepackage{fancyhdr}

\pagestyle{fancy}
\addtolength{\headwidth}{\marginparsep}
\addtolength{\headwidth}{\marginparwidth}
\renewcommand{\chaptermark}[1]{\markboth{#1}{}}
\renewcommand{\sectionmark}[1]{\markright{\thesection\ #1}}
\fancyhf{}
\fancyfoot[C]{\thepage}
\fancyhead[LO]{\textit \leftmark}
\fancyhead[RE]{\textit \rightmark}
\renewcommand{\headrulewidth}{0pt} % and the line
\fancypagestyle{plain}{%
    \fancyhead{} % get rid of headers
}

%bibliographie
\usepackage[
backend=biber,
style=alphabetic,
sorting=ynt
]{biblatex}

\addbibresource{bib.bib}

\usepackage{appendix}
\renewcommand{\appendixpagename}{Annexe}

\definecolor{wgrey}{RGB}{148, 38, 55}

\setlength\parindent{24pt}

\newcommand{\Z}{\mathbb{Z}}
\newcommand{\R}{\mathbb{R}}
\newcommand{\rel}{\omathcal{R}}
\newcommand{\Q}{\mathbb{Q}}
\newcommand{\C}{\mathbb{C}}
\newcommand{\N}{\mathbb{N}}
\newcommand{\K}{\mathbb{K}}
\newcommand{\A}{\mathbb{A}}
\newcommand{\B}{\mathcal{B}}
\newcommand{\Or}{\mathcal{O}}
\newcommand{\F}{\mathbb F}
\newcommand{\m}{\mathfrak m}
\renewcommand{\a}{\mathfrak a}
\newcommand{\I}{\mathfrak I}
\newcommand{\Hom}{\textrm{Hom}}
\newcommand{\disc}{\textrm{disc}}
\newcommand{\Pic}{\textrm{Pic}}
\newcommand{\End}{\textrm{End}}
\newcommand{\Spec}{\textrm{Spec}}

\newcommand{\cL}{\mathscr{L}}
\newcommand{\G}{\mathscr{G}}
\newcommand{\D}{\mathscr{D}}
\newcommand{\E}{\mathscr{E}}

\theoremstyle{plain}
\newtheorem{thm}[subsection]{Théoreme}
\newtheorem{lem}[subsection]{Lemme}
\newtheorem{prop}[subsection]{Proposition}
\newtheorem{cor}[subsection]{Corollaire}
\newtheorem{heur}{Heuristique}
\newtheorem{rem}{Remarque}
\newtheorem{rembis}{Remarque}
\newtheorem{note}{Note}

\theoremstyle{definition}
\newtheorem{conj}{Conjecture}
\newtheorem{prob}{Problème}
\newtheorem{quest}{Question}
\newtheorem{prot}{Protocole}
\newtheorem{algo}{Algorithme}
\newtheorem{defn}[subsection]{Définition}
\newtheorem{defnbis}{Définition}
\newtheorem{exmp}[subsection]{Exemples}
\newtheorem{exo}[subsection]{Exercices}

\theoremstyle{remark}

\definecolor{wgrey}{RGB}{148, 38, 55}
\definecolor{wgreen}{RGB}{100, 200,0} 
\hypersetup{
    colorlinks=true,
    linkcolor=wgreen,
    urlcolor=wgrey,
    filecolor=wgrey
}

\title{Corps locaux \\ \small{(Par Qing Liu)}}
\date{}

\begin{document}
\maketitle
\chapter*{Introduction}
Le but c'est d'abord de décrire les extensions finies de corps locaux
comment dans \cite{cassels}.
\chapter{Corps complets}
\section{Valuations et valeurs absolues}
Le but là c'est de classifier les valuations, d'abord sur des corps
premiers comme $\Q$. D'abord l'équivalence entre valuations et 
valeurs absolues. 
\begin{defn}[Valuation]
    Une valuation $v$ (de rang $1$) est un morphisme de groupe 
    (multiplicatif vers additif) \[K\to \R\cup\{\infty\}\]
    tel que $v(x+y)\geq min\{v(x), v(y)\}$ quand $x+y\ne 0$ avec
    $v(x+y)=\infty$.
\end{defn}
\begin{defn}[Valeur absolue]
    Une valeur absolue $|.|$ est un morphisme de groupes multiplicatifs
    $K\to \R$ étendu avec $0\mapsto 0$ et qui vérifie une inégalité 
    triangulaire.
\end{defn}
Le passage aux valeurs absolues : étant donné un réel $0<t<1$ on peut
déf une valeur absolue 
\[x\mapsto t^{v(x)}\]

Bon ensuite la caractérisation archimédienne :
\begin{defn}[Valeur absolue archimédienne]
    Un corps valué $(K,|.|)$ est archimédien si pour tout 
    $(x,c)\in K\times \R$ il existe $n$ t.q \[c\leq |n.x|\]
    et pas $n|x|$ attention.
\end{defn}
Ensuite la première partie de la caractérisation :
\begin{prop}
    Être archimédien c'est équivalent à 
    \begin{itemize}
        \item $(|n|)_{n\in \Z}$  est bornée.
        \item $(|n|)_{n\in \Z}$  est $\leq 1$.
        \item $|.|$ est ultramétrique!
    \end{itemize}
\end{prop}
\begin{proof}[Preuve]
    La première équivalence est directe. La deuxième aussi via 
    $x\mapsto x^k$ et la troisième se ramène à la deuxieme en 
    regardant $|1+x|\leq 1$ et la flèche $x\mapsto x^k$ à nouveau !
    (Ca caractérise les élements inferieurs à $1$)
\end{proof}
\begin{prop}[Valuations et valeurs absolues]
    Il y a une bijection entre valeurs absolues ultramétriques
    et valuations. L'inverse de la flèche du dessus est juste 
    \[|.|\mapsto \frac{1}{\log(1/2)}\log(|.|)\]
\end{prop}

On s'approche du théorème d'Ostrowski. Avant on déf les équivalences
de valeurs absolues.
\begin{prop}[La topologie induite]
    On obtient une distance à partir d'une valeur absolue puis la 
    topologie de la distance.
\end{prop}
Dans le cas ultramétrique c'est très bizarre.
\begin{prop}
    On a les propriétés suivantes.
    \begin{itemize}
        \item Si $|x|\ne |y|$ alors $|x+y|=max(|x|,|y|)$.
        \item Avec l'inégalité ultramétrique, les boules ont un seul
            centre.
        \item Puis deux disques qui s'intersectent forment en fait une
            chaine.
    \end{itemize}
\end{prop}
\begin{prop}[Anneau de valuation]
    On peut déf l'anneau de valuation de $(K,|.|)$ par 
    $\Or:=\bar D(0,1)$ les éléments de valuations plus petites que $1$.
    Et son idéal maximal la boule ouverte $D(0,1)$. Les inversibles
    sont dans la frontière.
\end{prop}
\begin{defn}[Équivalence de valeurs absolues]
    Deux valeurs absolues sont équivalentes si elles définissent la 
    même topologie.
\end{defn}
\begin{lem}[Caractérisation]
    Deux valeur sabsolues sont équivalentes ssi il existe 
    $s\in \R^*_+$ t.q $|.|_1=|.|_2^s$.
\end{lem}
\begin{proof}[Preuve]
    La bonne idée c'est la caractérisation de $D(0,1)$ par les trucs
    qui tendent vers $0$ par $x\mapsto x^k$. Faut comprendre que 
    prendre des limites dans $\R$ donne la même chose pour 
    $|.|_1$ et $|.|_2$.
\end{proof}

Ça suffit à prouver le théorème d'Ostrowski :
\begin{thm}[Ostrowski] 
    Une valeur absolue non triviale sur $\Q$ est équivalente soit
    à $|.|_{\infty}$ ou à une valeur absolue $p$-adique $|.|_p$.
\end{thm}
\begin{proof}[Preuve]
    Pour la partie non archimédienne, on regarde $\mathfrak m\cap \Z$
    pour obtenir un $p$. Ensuite, on montre via $|uk|=|1-vp|$ que
    $|k|=1$ dés que $k\wedge p=1$, $|vp|<1$ parce que 
    $vp\in \mathfrak m$. Ensuite y reste à comparer $|p|$ et $|p|_{p}$.
    Vu que $|x|=|p|^r|k|=|p|^r$. 

    Pour le cas archimédien l'idée c'est d'écrire $a$ en base $b$ pour
    comparer $a^n$ et $b^l$ où $l\sim n$. En particulier montrer que
    $|a|\geq 1$ implique $b\geq1$ dès que $a\geq 2$ ET 
    $|a|=|b|^{\log(a)/\log(b)}$, via la comparaison et par symétrie.
    Puis remplacer $b$ par $2$. On a montré que seul $|2|$ détermine
    le lien entre $|.|$ et $|.|_{\infty}$.
\end{proof}
\begin{defn}[Places]
    On définit une place comme une classe d'équivalence de valeurs
    absolues. Une place est finie/infinie si elle est 
    ultramétrique/archimédienne.
\end{defn}

\section{Complétions et corps complets}
Dès qu'on a un corps valué $(K,|.|)$, on peut construire une distance
\[d_{|.|}\colon(x,y)\mapsto |x-y|\]
puis définir des suites de Cauchy. 
\begin{defn}
    Un corps valué $(K,|.|)$ est complet si les suites de Cauchy pour 
    $d_{|.|}$ convergent dans $K$.
\end{defn}
\begin{defn}[Complétion]
    Pour tout corps valué $(K,|.|)$ il existe un corps valué complet 
    $(\hat K,|.|)$ tel que $|.|$ s'étend à $\hat K$.
\end{defn}
\begin{proof}[Preuve]
    L'idée c'est de regarder l'ensemble des suites de Cauchy 
    $\mathscr C(K)\subset K^{\N}$, de voir que c'est un anneau 
    puis de quotienter par l'idéal maximal des suites qui convergent
    vers $0$.
\end{proof}

\begin{defn}[Morphismes dans la catégorie des corps valués]
    Un morphisme de corps valués est un morphisme d'anneau et une
    isométrie.
\end{defn}

On peut étendre les morphismes vers des corps complets en morphismes
de corps valués unique à unique isomorphisme près.

\begin{rem}
    Si $(K,|.|)$ est ultramétrique et non triviale, $\hat K$ peut-être
    construit algébriquement. On fixe $t\in K$ t.q $0<|t|<1$. Dans
    l'anneau de valuation, on regarde 
    \[\widehat O_K=\varprojlim_n(O_K/t^nO_K)\]
    C'est un anneau intègre muni d'une valuation t.q
    \[\widehat v((x_n)_n)=v(y)\]
    où $(x_n)=\pi(y)$ et 
    $\pi\colon O_K \to \widehat O_K;~x\mapsto (x)_n$. C'est
    une valuation par densité. On l'étend au corps de fraction de la 
    manière évidente. On peut montrer que c'est complet \textbf{à faire}.
    Pareil en général \textbf{$I\subset A$}.
\end{rem}
\begin{exo}
    Rayon de convergence de $exp(z)=\sum_n \frac{z^n}{n!}$ dans 
    $\Q_p$? Sachant que en métrique $p$-adique 
    $1/n!\to_{n\to\infty}+\infty$. Faut calculer la valuation de 
    $n!$.
\end{exo}
Plus généralement, si $A$ est intègre et $\m a=tA$ est maximal
on peut définir la valuation $t$-adique associée et l'étendre au 
corps de fractions et c'est une valuation discrète. Je crois que l'idée 
c'est que 
\[I=\cap t^n A=\{0\}\]
parce que $tI=I$ puis Nakayama dans un anneau noethérien.

Apparemment il fait une construction de $\R$ sans complétion parce que
c'est tautologique? Ah bah oui les valuations c'est dans $\R$, mais
en fait on peut d'abord juste les prendre dans $\Q$.


Il regarde un corps $K$ avec un ordre total 
compatible archimédien et la topologie de l'ordre.

Alors on demande
enfin que $\iota\colon \Q\to K$ soit dense, on a nécessairemment que
$\iota$ est croissante. Par densité de $\Q$ dans $K$, on peut remplacer
toute les suites de Cauchy dans $K$ par des suites de Cauchy dans $\Q$.


En particulier, y'a équivalence entre les suites de Cauchy de $K$ et 
$\Q$. On regarde maintenant $C(\Q)$ les suites de Cauchy dans $\Q$ et
$I(\Q)$ les suites qui tendent vers $0$. Dans
\[\pi\colon C(\Q)\to C(\Q)/I(\Q)\]
on définit un ordre à droite via $\pi(x_n)_n\geq \pi(y_n)_n$ ssi
on a égalité où il existe $r\in \Q^*_+$ t.q $x_n\geq y_n+r$. On obtient
un corps totalement ordonné $C(\Q)/I(\Q)$ (totalement car 
$\pm\pi(x_n)_n\geq 0$). Bon maintenant faut juste conclure en montrant
que le quotient est complet. \textbf{à faire?}

\begin{rem}
    On a bel et bien utilisé que $\Q$.
\end{rem}
\begin{prop}
    $\R$ est unique à unique isomorphisme près en tant que
    corps totalement ordonné complet archimédien où $\Q$ est dense.
\end{prop}

\section{Espaces de Banach}
Une norme $K$ sur un espace vectoriel $V$ est une flèche $x\mapsto ||x||$
qui est nulle qu'en $0$, a une inégalité triangulaire et qui transforme
l'action de $K$ en action de $\R$ via $|.|$.

\begin{defn}
    Un e.v.n est un Banach si il est complet pour sa norme.
\end{defn}
\begin{rem}
    Un $k$-espace vectoriel de dimension finie sur un corps complet
    est un Banach. L'inverse est faux!
\end{rem}
\begin{defn}
    Deux normes sont équivalentes si elles définissent la même topologie.
    Ou immédiatemment si il existe $c_1,c_2$ t.q 
    \[c_1||x||_1\leq ||x||_2\leq c_2||x||_1\]
\end{defn}
L'intérêt c'est maintenant qu'étant donné une extension de corps $L/k$
et une valeur absolue sur $k$, qu'est-ce qu'il se passe quand on l'étend
à $L$? En fait c'est des normes sur $L$ en tant que $k$-ev, et elles
sont équivalentes en tant que norme ssi elles le sont en tant que v.a 
par définition!
\begin{thm}
    Si $k$ est complet, et $V$ est un $k$-ev de dimension finie, alors
    toutes les normes sur $V$ sont équivalentes et $V$ est un Banach.
\end{thm}
\begin{proof}
    La norme du max donne une structure de Banach grâce à la convergence
    normale. Il reste à montrer que toutes les normes sont équivalentes
    à la norme du max $||.||_{\infty}$. Un côté est simple, l'autre 
    par induction \textbf{Faire? Ah la preuve est non triviale mdr}.
\end{proof}

Soit $(K,|.|)$ un corps complet ultramétrique. On note $k=\Or_K/\m$
le corps résiduel. 
\begin{lem}[Lemme de Hensel]
    Soit $P(X)\in \Or_K[X]$, on suppose que $P\equiv f.g\mod \m$ tels que
    $gcd(f,g)=1$. Alors il existe $F,G\in \Or_K[X]$ tels que 
    \[P=F.G\]
    et $F\equiv f\mod \m$, $G\equiv g\mod\m$, avec $\deg(F)=\deg(f)$.
    La décomposition est unique à inversible près.
\end{lem}
\begin{proof}
    On commence par prendre un lift $F_0$ de $f$ et $G_0$ de $g$ de 
    degrés minimaux. On a $(f,g )=k[X]$ d'où $1\in (F_0,G_0)+\m[X]$,
    il existe donc $t\in K$ t.q $0<|t|<1$ et
    $
    \begin{cases}
        P-F_0G_0\in t\Or_K[X]\\
        1\in (F_0,G_0)+t\Or_K[X]\\
    \end{cases}
    $. On a déjà augmenté la précision. Soit $F_1=F_0+tV_1$ et 
    $G_1=G_0+tU_1$. On veut $P-F_1G_1\in t^2 \Or_K[X]$ t.q $\deg(F_1)=m$
    et $\deg(G_1)\leq d-m$. On regarde 
    \begin{align*}
        P-(F_0+tV_1)(G_0+tU_1)&=P-(F_0G_0+t(F_0U_1+G_0V_1)+t^2U_1V_1)\\
                              &=(P-F_0G_0)+t(F_0U_1+G_0V_1)+t^2*\\
                              &=tE_0+t(\_)+t^2*\\
                              &=t(E_0+F_0U_1+G_0V_1)+t^2*
    \end{align*}
    On prends $E_0=H_0F_0+R_0G_0+t*$.
    \textbf{Revoir la preuve ailleurs.}
\end{proof}




\printbibliography

\end{document}


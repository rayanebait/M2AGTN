\documentclass[a4paper,12pt]{book}
\usepackage{amsmath,  amsthm,enumerate}
\usepackage{csquotes}
\usepackage[provide=*,french]{babel}
\usepackage[dvipsnames]{xcolor}
\usepackage{quiver, tikz}

%symbole caligraphique
\usepackage{mathrsfs}

%hyperliens
\usepackage{hyperref}

%pseudo-code
\usepackage{algpseudocode}
\usepackage{algorithm}
\makeatletter
  \renewcommand{\ALG@name}{Algorithme}
  \makeatother
\usepackage{fancyhdr}

\pagestyle{fancy}
\addtolength{\headwidth}{\marginparsep}
\addtolength{\headwidth}{\marginparwidth}
\renewcommand{\chaptermark}[1]{\markboth{#1}{}}
\renewcommand{\sectionmark}[1]{\markright{\thesection\ #1}}
\fancyhf{}
\fancyfoot[C]{\thepage}
\fancyhead[LO]{\textit \leftmark}
\fancyhead[RE]{\textit \rightmark}
\renewcommand{\headrulewidth}{0pt} % and the line
\fancypagestyle{plain}{%
    \fancyhead{} % get rid of headers
}

%bibliographie
\usepackage[
backend=biber,
style=alphabetic,
sorting=ynt
]{biblatex}

\addbibresource{bib.bib}

\usepackage{appendix}
\renewcommand{\appendixpagename}{Annexe}

\definecolor{wgrey}{RGB}{148, 38, 55}

\setlength\parindent{24pt}

\newcommand{\Z}{\mathbb{Z}}
\newcommand{\R}{\mathbb{R}}
\newcommand{\rel}{\omathcal{R}}
\newcommand{\Q}{\mathbb{Q}}
\newcommand{\C}{\mathbb{C}}
\newcommand{\N}{\mathbb{N}}
\newcommand{\K}{\mathbb{K}}
\newcommand{\A}{\mathbb{A}}
\newcommand{\B}{\mathcal{B}}
\newcommand{\Or}{\mathcal{O}}
\newcommand{\F}{\mathbb F}
\newcommand{\m}{\mathfrak m}
\renewcommand{\a}{\mathfrak a}
\newcommand{\I}{\mathfrak I}
\newcommand{\Hom}{\textrm{Hom}}
\newcommand{\disc}{\textrm{disc}}
\newcommand{\Pic}{\textrm{Pic}}
\newcommand{\End}{\textrm{End}}
\newcommand{\Spec}{\textrm{Spec}}

\newcommand{\cL}{\mathscr{L}}
\newcommand{\G}{\mathscr{G}}
\newcommand{\D}{\mathscr{D}}
\newcommand{\E}{\mathscr{E}}

\theoremstyle{plain}
\newtheorem{thm}{Théoreme}
\newtheorem{lem}{Lemme}
\newtheorem{prop}{Proposition}
\newtheorem{cor}{Corollaire}
\newtheorem{heur}{Heuristique}
\newtheorem{rem}{Remarque}
\newtheorem{rembis}{Remarque}
\newtheorem{note}{Note}

\theoremstyle{definition}
\newtheorem{conj}{Conjecture}
\newtheorem{prob}{Problème}
\newtheorem{quest}{Question}
\newtheorem{prot}{Protocole}
\newtheorem{algo}{Algorithme}
\newtheorem{defn}{Définition}
\newtheorem{defnbis}{Définition}
\newtheorem{ex}{Exemple}
\newtheorem{exo}{Exercices}

\theoremstyle{remark}

\definecolor{wgrey}{RGB}{148, 38, 55}
\definecolor{wgreen}{RGB}{100, 200,0} 
\hypersetup{
    colorlinks=true,
    linkcolor=wgreen,
    urlcolor=wgrey,
    filecolor=wgrey
}

\title{Point sur le cours de corps locaux }
\date{}

\begin{document}
\maketitle

\section{2e point sur les cours, 09/10/2024}


\section{1er point sur les cours, 04/10/2024}
\subsection{Extensions sur corps non complets}
Si $(K, |.|_K)$ est ultramétrique et $E/K$ finie. On peut regarder
$Aut(L/K)$ pour construire
\[|.|_L\mapsto |\sigma(.)|_L\]
elles étendent toutes $|.|_K$ et on en a $\#Aut(L/K)/ef$ dans le cas
galoisien par exemple. À l'inverse, 
étant donné $|.|_L$ qui étend $|.|_K$ on peut obtenir un $\sigma$ en
trouvant $\tau\colon \widehat L\to (\widehat{K})^c$ d'où
$|.|_{\widehat L}= |.|_c\circ\tau$. En fait on regarde 
% https://q.uiver.app/#q=WzAsNSxbMCwxLCJLIl0sWzEsMSwiXFxvdmVybGluZSBLXFxzaW1lcSBcXHdpZGVoYXQgSyJdLFsyLDEsIihcXHdpZGVoYXQgSyleYyJdLFswLDAsIkwiXSxbMSwwLCJMLlxcb3ZlcmxpbmUgSz1cXHdpZGVoYXQgTCJdLFswLDEsIiIsMix7InN0eWxlIjp7InRhaWwiOnsibmFtZSI6Imhvb2siLCJzaWRlIjoidG9wIn19fV0sWzEsMiwiIiwyLHsic3R5bGUiOnsidGFpbCI6eyJuYW1lIjoiaG9vayIsInNpZGUiOiJ0b3AifX19XSxbMCwzXSxbMyw0XSxbMSw0XSxbNCwyLCIiLDAseyJzdHlsZSI6eyJib2R5Ijp7Im5hbWUiOiJkYXNoZWQifX19XV0=
\[\begin{tikzcd}
	L & {L.\overline K=\widehat L} \\
	K & {\overline K\simeq \widehat K} & {(\widehat K)^c}
	\arrow[from=1-1, to=1-2]
	\arrow[dashed, from=1-2, to=2-3]
	\arrow[from=2-1, to=1-1]
	\arrow[hook, from=2-1, to=2-2]
	\arrow[from=2-2, to=1-2]
	\arrow[hook, from=2-2, to=2-3]
\end{tikzcd}\]
où $\overline K$ est l'adhérence de $K$ dans $\widehat L$. L'égalité
montre que $\widehat L$ est de dimension finie sur $\widehat K$. En
particulier on obtient $\tau \colon \widehat L\to (\widehat K)$. Ensuite,
$\tau(\widehat L)\ni\tau(x)\mapsto |x|_{\widehat L}$ est une valeur 
absolue sur $\tau(\widehat L)\subset (\widehat K)^c$ qui étend celle
de $\overline K\simeq \widehat K$ d'où par unicité sur la clôture on a
\[|x|_{\widehat L}=|\tau(x)|_c\]
et on peut faire redescendre sur $\tau|_L\colon L\to K^c$. 
\begin{rem}
    Y'a plusieurs plongements en général $\widehat L \to (\widehat K)^c$
    penser au groupe de galois sur $\Q$ qui devient le groupe de 
    décomposition sur $\Q_p$. Et y définissent tous la même valeur
    absolue sur $\widehat L$.
\end{rem}
\subsection{Avec les anneaux artiniens}
Là y'a un truc cool qui montre clairement la décomposition des premiers
dans des extensions! On a une correspondance entre 


\[\{\textrm{Idéaux maximaux de }L\otimes_K\widehat K\}\leftrightarrow\{\textrm{extensions de $|.|_K$ à $L$}\}\]
Pour la surjectivité on utilise % https://q.uiver.app/#q=WzAsNSxbMCwyLCJLIl0sWzIsMiwiXFx3aWRlaGF0IEsiXSxbMCwwLCJMIl0sWzIsMCwiXFx3aWRlaGF0IEwiXSxbMSwxLCJMXFxvdGltZXNfSyBcXHdpZGVoYXQgSyJdLFswLDJdLFswLDFdLFsxLDNdLFsyLDNdLFsxLDRdLFsyLDRdLFs0LDMsIiIsMix7InN0eWxlIjp7ImJvZHkiOnsibmFtZSI6ImRhc2hlZCJ9fX1dLFswLDRdXQ==
\[\begin{tikzcd}
	L && {\widehat L} \\
	& {L\otimes_K \widehat K} \\
	K && {\widehat K}
	\arrow[from=1-1, to=1-3]
	\arrow[from=1-1, to=2-2]
	\arrow[dashed, from=2-2, to=1-3]
	\arrow[from=3-1, to=1-1]
	\arrow[from=3-1, to=2-2]
	\arrow[from=3-1, to=3-3]
	\arrow[from=3-3, to=1-3]
	\arrow[from=3-3, to=2-2]
\end{tikzcd}\]
et on pose $\m_{|.|_L}=\ker(L\otimes_K \widehat K)\to \widehat L$. Pour
l'injectivité c'est assez direct.

\subsection{Extensions de valuations et corps complets}
Le cas archimédien est assez particulier puisque y'a que $\R$ et $\C$.
C'est l'exercice $8$ de la feuille où on peut montrer que tout corps
archimédien complet qui contient $i$ est isomorphe à $\C$. Je
vais regarder surtout le cas non archimédien.
\begin{thm}
    Si $C(K)$ est l'ensemble des suites de Cauchy sur $K$ et $\m(K)$
    celles qui tendent vers $0$ alors $\m(K)$ est maximal et
    $C(K)/\m(K)$ est un corps complet minimal où $K$ est dense.
\end{thm}
Pour les extensions maintenant on a besoin de Hensel et du fait que
\begin{cor}
    Si $K$ est complet ultramétrique, le max des coefficients de $P(X)$
    irréductible sur $K$ est atteint par $P^*(0)$ (coeff dominant)
    ou $P(0)$. En particulier si $P^*(0)=1$ et $|P(0)|\leq 1$ alors
    $P$ est dans $\Or_K[X]$.
\end{cor}
mais j'en parlerai a un autre moment. Donc maintenant qu'on sait c'est
quoi les corps complets premiers en caractéristique $0$ on regarde
leurs extensions finies. L'idée c'est que c'est des Banach de dimension
finie sur un $\Q_p$ donc
\begin{itemize}
    \item Toutes les normes sont équivalentes.
\end{itemize}
en particulier les valeurs absolues sont équivalentes, donc y'en a 
qu'une qui étend $|.|_p$.

\begin{proof}[Construction de la valeur absolue]
    
Étant donné $x\in L/K$ on regarde 
l'endomorphisme de multiplication sur $L$ par $x$, $m_x\in End_K(L)$.
On déf 
\[N_{L/K}(x)=\det(m_x)\]
comme $N_{L/K}(y\in K)=y^{[L:K]}$ on déf 
\[|.|_L:=|N_{L/K}(.)^{1/[L:K]}|\]
pour vérifier qu'elle est ultramétrique, on se rappelle que $\det(m_x)$
est le coefficient constant de $\chi_{m_x}(X)$ le polynôme 
caractéristique de $m_x$ à coefficient dans $K$. En plus 
$x\mapsto m_x$ est un homomorphisme d'anneau. D'où 
\[N_{L/K}(x)=(-1)^{[L:K]}\chi_{m_x}(0)\]
si on remarque que $m_x$ est diagonale par bloc avec $[L:K(x)]$
blocs. On peut affiner en remarquant que 
\[\chi_{m_x}=\mu_{x}^{[L:K(x)]}\]
avec $\mu_x$ le polynôme minimal de $x$. Maintenant pour montrer que
c'est ultramétrique faut montrer que 
\[|1+\alpha|_L\leq 1\]
Mais $\mu_{1+x}(X)=\mu_{x}(X-1)$ et $\mu_x\in \Or_K$ par le corollaire 
puis $\mu_x(-1)\in \Or_K$ d'où le résultat. Les autres propriétés sont
claires.
\end{proof}
On peut maintenant l'étendre a une clôture algébrique $K^c$ via
\[x\mapsto |N_{K(x)/K}(x)^{1/[K(x):K]}|\]
et il y'a unicité.

\subsection{Classification sur $\Q$}
En théorie des nombres, on a regardé grosso modo Ostrowski et les 
équivalences de valeurs absolues ($|.|^s$) pour définir la même 
topologie.
% https://q.uiver.app/#q=WzAsOSxbMywxLCIoXFxtYXRoYmIgUSx8LnwpIl0sWzIsMiwiXFxtYXRoYmIgUV97cF8yfSJdLFsxLDIsIlxcbWF0aGJiIFFfe3BfMX0iXSxbMywyLCJcXG1hdGhiYiBRX3twXzN9Il0sWzQsMiwiXFxtYXRoYmIgUV97cF80fSJdLFs1LDIsIlxcbWF0aGJiIFFfe3BfNX0iXSxbMywwLCJcXG1hdGhiYiBSIl0sWzYsMiwiXFxsZG90cyJdLFswLDIsIlxcbGRvdHMiXSxbMCw2XSxbMCwyXSxbMCwxXSxbMCwzXSxbMCw0XSxbMCw1XV0=
\[\begin{tikzcd}
	&&& {\mathbb R} \\
	&&& {(\mathbb Q,|.|)} \\
    {\mathbb Q_{2}} & {\mathbb Q_{3}} & {\mathbb Q_{5}} & {\mathbb Q_{7}} & {\mathbb Q_{11}} & {\mathbb Q_{13}} & \ldots
	\arrow[from=2-4, to=1-4]
	\arrow[from=2-4, to=3-1]
	\arrow[from=2-4, to=3-2]
	\arrow[from=2-4, to=3-3]
	\arrow[from=2-4, to=3-4]
	\arrow[from=2-4, to=3-5]
	\arrow[from=2-4, to=3-6]
\end{tikzcd}\]
Les grosses idées a retenir pour moi : étant donné une $|.|$, si
elle est non archimédienne $\to$ on regarde la valuation associée $v$. 
Ensuite $\m\cap \Z=p\Z$, et ensuite sur $\Z$ si $x\wedge p=1$
$|ux|=|1-yp|=1$ d'où suffit de regarder $|p|$. Le cas archimédien est 
un peu plus compliqué mais assez étonnant. En fait si on regarde un corps
archimédien complet dont la v.a. étend celle de $\Q$ contenant $i$.
Alors on peut plonger $\C$, disons $i\colon \C\to K$. L'idée c'est de
regarder pour $a\in K$ l'inf de,
\[f_a\colon z\mapsto |z-a|_K\]
si c'est $r>0$ on peut faire bouger $r$ comme on veut. Ensuite faut 
regarder pour $|\gamma_0 - a|=r$ et $|\gamma - \gamma_0|<r$ on peut 
calculer 
\[(\gamma_0-\gamma)^n - (\gamma_0 - a)^n\]
le spécificité de $\C$ analytiquement c'est alors d'avoir toutes
les racines de l'unités. Ce qui permet de factoriser le truc du dessus.


\end{document}


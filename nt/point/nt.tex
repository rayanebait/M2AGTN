\documentclass[a4paper,12pt]{book}
\usepackage{amsmath,  amsthm,enumerate}
\usepackage{csquotes}
\usepackage[provide=*,french]{babel}
\usepackage[dvipsnames]{xcolor}
\usepackage{quiver, tikz}

%symbole caligraphique
\usepackage{mathrsfs}

%hyperliens
\usepackage{hyperref}

%pseudo-code
\usepackage{algpseudocode}
\usepackage{algorithm}
\makeatletter
  \renewcommand{\ALG@name}{Algorithme}
  \makeatother
\usepackage{fancyhdr}

\pagestyle{fancy}
\addtolength{\headwidth}{\marginparsep}
\addtolength{\headwidth}{\marginparwidth}
\renewcommand{\chaptermark}[1]{\markboth{#1}{}}
\renewcommand{\sectionmark}[1]{\markright{\thesection\ #1}}
\fancyhf{}
\fancyfoot[C]{\thepage}
\fancyhead[LO]{\textit \leftmark}
\fancyhead[RE]{\textit \rightmark}
\renewcommand{\headrulewidth}{0pt} % and the line
\fancypagestyle{plain}{%
    \fancyhead{} % get rid of headers
}

%bibliographie
\usepackage[
backend=biber,
style=alphabetic,
sorting=ynt
]{biblatex}

\addbibresource{bib.bib}

\usepackage{appendix}
\renewcommand{\appendixpagename}{Annexe}

\definecolor{wgrey}{RGB}{148, 38, 55}

\setlength\parindent{24pt}

\newcommand{\Z}{\mathbb{Z}}
\newcommand{\R}{\mathbb{R}}
\newcommand{\rel}{\omathcal{R}}
\newcommand{\Q}{\mathbb{Q}}
\newcommand{\C}{\mathbb{C}}
\newcommand{\N}{\mathbb{N}}
\newcommand{\K}{\mathbb{K}}
\newcommand{\A}{\mathbb{A}}
\newcommand{\B}{\mathcal{B}}
\newcommand{\Or}{\mathcal{O}}
\newcommand{\F}{\mathbb F}
\newcommand{\m}{\mathfrak m}
\renewcommand{\b}{\mathfrak b}
\renewcommand{\a}{\mathfrak a}
\newcommand{\p}{\mathfrak p}
\newcommand{\I}{\mathfrak I}
\newcommand{\Hom}{\textrm{Hom}}
\newcommand{\disc}{\textrm{disc}}
\newcommand{\Pic}{\textrm{Pic}}
\newcommand{\End}{\textrm{End}}
\newcommand{\Spec}{\textrm{Spec}}

\newcommand{\cL}{\mathscr{L}}
\newcommand{\G}{\mathscr{G}}
\newcommand{\D}{\mathscr{D}}
\newcommand{\E}{\mathscr{E}}

\theoremstyle{plain}
\newtheorem{thm}{Théoreme}
\newtheorem{lem}{Lemme}
\newtheorem{prop}{Proposition}
\newtheorem{cor}{Corollaire}
\newtheorem{heur}{Heuristique}
\newtheorem{rem}{Remarque}
\newtheorem{rembis}{Remarque}
\newtheorem{note}{Note}

\theoremstyle{definition}
\newtheorem{conj}{Conjecture}
\newtheorem*{eq}{Équivalences}
\newtheorem{prob}{Problème}
\newtheorem{quest}{Question}
\newtheorem{prot}{Protocole}
\newtheorem{algo}{Algorithme}
\newtheorem{defn}{Définition}
\newtheorem{defnbis}{Définition}
\newtheorem{ex}{Exemple}
\newtheorem{exo}{Exercices}

\theoremstyle{remark}

\definecolor{wgrey}{RGB}{148, 38, 55}
\definecolor{wgreen}{RGB}{100, 200,0} 
\hypersetup{
    colorlinks=true,
    linkcolor=wgreen,
    urlcolor=wgrey,
    filecolor=wgrey
}

\title{Point sur le cours de corps locaux }
\date{}

\begin{document}
\maketitle
\chapter{Des choses à faire}
Construire des extensions explicites! Y'a l'air d'avoir beaucoup de 
critères là dans le chapitre. 

\chapter{4e point sur les cours, 22/10/2024}
Dernier point avant l'exam !


\section{Ramification 2}
On se place \textbf{toujours} dans le cadre où on a $\Or_K$ de valuation 
\textbf{discrète}. Le cadre en gros c'est 
% https://q.uiver.app/#q=WzAsNixbMCwwLCJcXE9yX0siXSxbMSwwLCJcXHRpbGRlXFxPcl9LIl0sWzIsMCwiKFxcdGlsZGVcXE9yX0spX3tcXG1faX0iXSxbMywwLCJcXGxlZnQoXFxPcl9MXFxyaWdodCkiXSxbMCwxLCJrX0siXSxbMiwxLCJrX0wiXSxbMCwxLCIiLDAseyJzdHlsZSI6eyJoZWFkIjp7Im5hbWUiOiJub25lIn19fV0sWzEsMiwiXFxzdWJzZXRlcSIsMSx7InN0eWxlIjp7ImJvZHkiOnsibmFtZSI6Im5vbmUifSwiaGVhZCI6eyJuYW1lIjoibm9uZSJ9fX1dLFsyLDMsIj8iLDEseyJsZXZlbCI6Miwic3R5bGUiOnsiaGVhZCI6eyJuYW1lIjoibm9uZSJ9fX1dLFswLDRdLFsyLDVdLFs0LDVdXQ==
\[\begin{tikzcd}
	{\Or_K} & {\tilde\Or_K} & {(\tilde\Or_K)_{\m_i}} & {\left(\Or_L\right)} \\
	{k_K} && {k_L}
	\arrow[no head, from=1-1, to=1-2]
	\arrow[from=1-1, to=2-1]
	\arrow["\subseteq"{description}, draw=none, from=1-2, to=1-3]
	\arrow["{?}"{description}, Rightarrow, no head, from=1-3, to=1-4]
	\arrow[from=1-3, to=2-3]
	\arrow[from=2-1, to=2-3]
\end{tikzcd}\]
C'est à dire qu'on prends la clôture intégrale, on regarde ses idéaux
maximaux et on obtient des extensions de d.v.r. Quand $K$ est complet 
ou quand on fixe une valuation (un premier $\m_i$) sur $L$, $\Or_L$ fait
sens.
\subsection{Calcul générique}
    Pour calculer maintenant en fait une marche à suivre c'est
\begin{center}
    \textit{On sait le faire dans $\Or_K[\alpha]$.}
\end{center}
On aurait besoin de critères pour que $\tilde\Or_K=\Or_K[\alpha]$. Si
c'est le cas alors :
\begin{enumerate}
    \item La factorisation de $P$ 


	dans $k_K[X]$ donne la ramification
	et les idéaux maximaux de $\tilde\Or_K$!
    \item Plus précisément, si 
\[\bar P=\prod_i p_i^{r_i}\in k_K[X]\]
	alors $\m_i=(\m_K,p_i(\alpha))$.
\end{enumerate}
Le point \textbf{important} c'est la ramification, on relève
\[P(\alpha)=\prod_iP_i^{r_i}(\alpha) + \epsilon(\alpha)\]
ce qui donne par le deuxième point 
\[\prod_i \m_i^{r_i}=\prod_i (\m_K, P_i(\alpha))^{r_i}\subset \m_K\tilde\Or_K=\prod_i \m_i^{e_i}\]
On en déduit $r_i\geq e_i$ pour tout $i$ et on conclut directement
avec \[\sum r_i f_i = \deg \bar P=\deg P= [L:K]=\sum e_if_i\]
On a utilisé que $\tilde\Or_K$ est fini sur $\Or_K$ pour l'égalité
$\deg \bar P=\deg P$ \textbf{et} la dimension $[L:K]=\sum e_i f_i$.

\subsection{Cas primitif}

Un cas intéressant quand on a $L=K(\alpha)$, par exemple
si $L/K$ est séparable on peut dire des choses fortes. Si $\bar P$ est
séparable, alors $\tilde\Or_K=\Or_K[\alpha]$ et on peut appliquer la 
section d'avant!  

On a un problème quand l'extension résiduelle est 
inséparable, on se place dans le diagramme
% https://q.uiver.app/#q=WzAsOSxbMCwxLCJcXE9yX0siXSxbMCwwLCJLIl0sWzIsMCwiTCJdLFsxLDAsIksoXFxhbHBoYSkiXSxbMSwxLCJcXHRpbGRlXFxPcl9LJyJdLFsyLDEsIlxcdGlsZGVcXE9yX0siXSxbMCwyLCJrX0siXSxbMSwyLCJrX3tLKFxcYWxwaGEpfSJdLFsyLDIsImtfTCJdLFsxLDMsIiIsMSx7InN0eWxlIjp7ImhlYWQiOnsibmFtZSI6Im5vbmUifX19XSxbMywyLCIiLDEseyJzdHlsZSI6eyJoZWFkIjp7Im5hbWUiOiJub25lIn19fV0sWzAsMSwiIiwxLHsic3R5bGUiOnsiaGVhZCI6eyJuYW1lIjoibm9uZSJ9fX1dLFswLDQsIiIsMSx7InN0eWxlIjp7ImhlYWQiOnsibmFtZSI6Im5vbmUifX19XSxbNCw1LCIiLDEseyJzdHlsZSI6eyJoZWFkIjp7Im5hbWUiOiJub25lIn19fV0sWzQsMywiIiwxLHsic3R5bGUiOnsiaGVhZCI6eyJuYW1lIjoibm9uZSJ9fX1dLFs1LDIsIiIsMSx7InN0eWxlIjp7ImhlYWQiOnsibmFtZSI6Im5vbmUifX19XSxbNiw3LCIiLDEseyJzdHlsZSI6eyJoZWFkIjp7Im5hbWUiOiJub25lIn19fV0sWzcsOCwiIiwxLHsic3R5bGUiOnsiaGVhZCI6eyJuYW1lIjoibm9uZSJ9fX1dLFswLDZdLFs0LDddLFs1LDhdXQ==
\[\begin{tikzcd}
	K & {K(\alpha)} & L \\
	{\Or_K} & {\tilde\Or_K'} & {\tilde\Or_K} \\
	{k_K} & {k_{K(\alpha)}} & {k_L}
	\arrow[no head, from=1-1, to=1-2]
	\arrow[no head, from=1-2, to=1-3]
	\arrow[no head, from=2-1, to=1-1]
	\arrow[no head, from=2-1, to=2-2]
	\arrow[from=2-1, to=3-1]
	\arrow[no head, from=2-2, to=1-2]
	\arrow[no head, from=2-2, to=2-3]
	\arrow[from=2-2, to=3-2]
	\arrow[no head, from=2-3, to=1-3]
	\arrow[from=2-3, to=3-3]
	\arrow[no head, from=3-1, to=3-2]
	\arrow[no head, from=3-2, to=3-3]
\end{tikzcd}\]


\subsection{Cas complet}
On a une équivalence entre :
\begin{enumerate}
    \item L'extension $L/K$ est non ramifiée (par déf non ramifiée et 
$k_{K(\alpha)}/k_K$ est séparable).
    \item Il existe $\alpha$ : $L=K(\alpha)$ et $P$ le pol min 
de $\alpha$ sur $K$ est séparable sur $k_K$.
\end{enumerate}
L'idée c'est juste que la formule $ef=[L:K]$ est vraie. Et on peut 
relever une base de l'extension résiduelle !
En gros ça donne une réciproque à la section d'avant. 

Dans le cas $p$-adique, les corps finis sont parfaits et on a toujours
des extensions séparables (c'est immédiat de la déf)! En particulier,
si $\bar P$ est inséparable c'est qu'il est scindé. Ça se voit bien
par Hensel :
\begin{enumerate}
    \item On a toujours $\bar P=F^d$ et en réécrivant $d\deg F=
	\deg P=e.f$ sachant que $\deg F\mid f$ (à vérifier mais ça se
	voit) on obtient $e\mid d$. (l'égalité c'est qu'on suppose $P$ 
	unitaire)
\end{enumerate}

Conclure là dessus, ajouter une discussion des cassages d'extensions 
de $\Q_p$ est totalement ramifiée et non ramifiée (le faire). Et aussi
faire le lien entre ramification sur $\Q$ est sur des complétions
$\Q_p$. Aussi conclure le cas primitif avec des divisibilités.

\section{Ramification 1}
J'vais parler de ramification ici. Le lemme clé c'est que dans une
extensions de d.v.r $\Or_K - \Or_L$. Si 
\[k_K - k_L\]
est de dimension $f\in\N\cup \infty$. Alors 
\[dim_k \Or_L/\m_K\Or_L = e.f\]
avec $\m_K=\m_L^e$. Ensuite, si $\tilde\Or_K$ est la fermeture 
intégrale de $\Or_K$ dans $L$ alors 
\[\sum_{i} e_i f_i \leq [L:K]\]
où on écrit $\m_K\tilde\Or_K=\prod_i \m_i^{e_i}$ et 
$f_i=[\tilde\Or_K/\m_i : k_K]$. Ça c'est par le lemme chinois! Pour
utiliser le résultat de juste avant faut aussi montrer que 
\[(\tilde\Or_K)_{\m_i}/\m_i^r(\tilde\Or_K)_{\m_i}\simeq \tilde\Or_K/\m_i^r\tilde\Or_K\]
Pour $\m$ maximal (ça se fait à la main). On a l'égalité dans
plusieurs cas :
\begin{enumerate}
    \item $K$ est complet, car alors $\tilde\Or_K=\Or_L$.
    \item $L/K$ est séparable, car alors $\tilde\Or_K$ est fini sur 
	$\Or_K$.
    \item Plus généralement, si $\tilde\Or_K$ est fini sur $\Or_K$.
    \item $L\otimes_K \widehat K$ est réduite. Regarder le lien
	entre les nilpotents et la séparabilité.
\end{enumerate}
Maintenant faut la calculer.


\chapter{3e point sur les cours, 21/10/2024}
Faut faire un peu plus attention qu'au 2e point.

\section{Extensions de valuations bis}
En conséquence de l'autre section on peut reformuler la bijection
\[\{\textrm{Ideaux maximaux de }L\otimes_K\widehat K\}\leftrightarrow
\{\textrm{ Extensions de $|.|_K$ à $L$}\}\]
en 
\[\{\textrm{Idéaux maximaux de $\tilde\Or_K$}\}\leftrightarrow \{\textrm{
Extensions de $|.|_K$ à $L$}\}\]

Dans le cadre où $(K,|.|_K)$ est donné par une valuation discrète, $L/K$
est finie et $\tilde\Or_K$ est la clôture intégrale de $\Or_v$ dans $L$.
Le point c'est que d'un idéal maximal on peut tjr étendre la valuation
et étant donné une extension de valuation c'est un d.v.r qui contient
$\tilde\Or_K$. On obtient les extensions de d.v.r via
\[\Or_K - (\tilde\Or_K)_\m\]
La seule différence c'est que là on traite que du cas ultramétrique et
de valuation discrète.

\section{Extensions d'anneaux de Dedekind}
Y'a des propositions clés pour tout c'est que si $A\to B$ est une
extension entière d'anneaux intègres alors :
\begin{enumerate}
    \item $A$ est un corps ssi $B$ est un corps.
    \item $\p$ un premier de $B$ est maximal dans $B$ ssi $\p\cap A$ 
	est maximal dans $A$.
    \item $A$ est de dimension $1$ implique $B$ est de dimension $1$
	(on est dans un cas intègre).
\end{enumerate}
Ducoup ça dit directement que \textbf{si} :
\begin{itemize}
    \item $B$ est la clôture intégrale de $A$
un anneau de Dedekind dans $L/Frac(A)$
    \item $B/A$ est entière.
\end{itemize}
Alors $B$ est de dimension $1$ par le truc d'avant, entier
sur $A$ et intégralement clos par construction. La question c'est 
\[\textrm{Quand est-ce que B est noethérien?}\]
C'est vrai si $B/A$ est finie et $L/K$ séparable.
\begin{center}
    Et si $B/A$ est pas finie?
\end{center}
Y'a un théorème de \textbf{Krull-Akizuki} qui prouve que c'est toujours
vrai si $L/K$ est juste finie.

On peut le faire dans le cas semi-local fini. En gros y suffit de 
séparer en une extension séparable et une extension purement inséparable,
 y suffit de traiter le cas purement inséparable (On a tjr $B$ entier
 sur $a$):
\begin{enumerate}
    \item Cas purement inséparable : Si $A$ est un d.v.r sa clôture est
	un d.v.r simplement parce qu'on peut déf $v(x) = v(x^{p^e})/p^e$.
	($\tilde\Or_K^{p^e}\subset \Or_K$)
    \item Si $B_{\m}$ est noethérien pour tout $\m\subset A$ alors
	$B$ est noethérien en relevant les générateurs!
	($A$ semi-local noetherien) (L'argument est intéressant et montre
	que $\cap B_{\m}$ ca marche pas très bien.)
\end{enumerate}
Un peu plus précisément
\begin{enumerate}
    \item Au dessus ça prouve que $B$ est de Dedekind, en fait il est
	aussi semi-local.
\end{enumerate}

Ça couvre ce qui nous intéresse : Les extensions de $\Q_p$ par exemple
où $\Z_p$ est un d.v.r puis ses extensions sont semi-locales.

\chapter{2e point sur les cours, 11/10/2024}

\section{Conséquences}
Ducoup comme la fermeture intégrale commute avec la localisation on
a directement 
\begin{description}
    \item[Les idéaux sont tous inversibles :] ça c'est direct, en 
	particulier les idéaux fractionnaires forment un groupe.
    \item[La decomposition en idéaux premiers :] avec une petite 
	application de la noethérianité.
\end{description}
ensuite, si $L/K=Frac(A)$ est séparable finie, on prouve que 
\begin{enumerate}
    \item La trace $Tr(x\_)$ est non dégénérée (via le discriminant),
	et c'est ptet là l'hypothèse séparable.
    \item La fermeture de $A$ dans $L$ est de Dedekind et de type fini 
	sur $A$.
\end{enumerate}
On obtient directement les décompositions, 
\[\p B=\prod_{\mathfrak P\mid \p} \mathfrak P^{e_{\mathfrak P}}\]
et la formule 
\[[L:K]=\sum e_{\mathfrak P}f_{\mathfrak P}\]
avec le lemme chinois.


\section{Sur les anneaux de Dedekind}
Le truc cool que j'avais jamais vu c'est que si $A$ est intègre 
noethérien
\begin{eq}
    On a :
    \begin{center}
	$A$ est de dimension $1$ et intégralement clos.
    \end{center}
    équivaut à
    \begin{center}
	Tout les localisés $A_{\p}$ sont de valuations discrètes.
    \end{center}
\end{eq}
L'équivalence est assez directe. On obtient la déf d'anneau de Dedekind.
\begin{rem}
    En fait ça prouve que localiser et passer à la fermeture intégrale
    commutent. I.e. $(\a.\b)_{\p}=\a_{\p}.\b_{\p}$ et 
    $(\a+\b)_{\p}=\a_{\p}+\b_{\p}$.
    \begin{center}
	\textbf{Ça prouve directement que 
	les idéaux sont tous inversibles dans un anneau de Dedekind}.
    \end{center}
    Faut juste se rappeler de la correspondance entre idéaux de 
    $A$ et $S^{-1}A$.
\end{rem}
\section{Sur les anneaux de valuations discrètes}
J'aimerais donner seulement des idées de preuves et équivalences.
J'regarde toujours des anneaux commutatifs.

\begin{eq}
    Dans un anneau noethérien, 
    \begin{center}
	DVR $\equiv$ local, noethérien, $\m=(\pi)$ non nilpotent.
    \end{center}
\end{eq}
\begin{proof}[Idées]
    On veut écrire $x=\pi^n u$ de manière unique. Il faut montrer que
    $\cap \m^n = 0$! Pour ça on utilise $\mathfrak u = \{x|\exists n, \pi^n x=0\}$. Ensuite c'est fini.
\end{proof}

\begin{eq}
    Dans un anneau noethérien intègre, 
    \begin{center}
	DVR $\equiv$ un seul idéal premier $\ne 0$, intégralement clos.
    \end{center}
\end{eq}
\begin{proof}
    Le point c'est de montrer que $\m$ est inversible, alors $\m$ est
    principal. On note $\m'=\{x\in K|x\m\subset A\}$. On a 
    \begin{center}
	$\m\m'\subset A$ et $A\subset \m'$ implique $\m\subset \m\m'$.
    \end{center}
    d'où $\m\m'=\m$ ou $A$.
    Maintenant en fait
    \begin{center}
	si $\m\m'=A$ alors $\sum x_iy_i=1$ d'où 
	$u=x_{i_0}y_{i_0}\in A-\m=A^{\times}$ 
    \end{center}
    en particulier tout 
    $z\in\m$ se réécrit $z=x_{i_0}(u^{-1}yz)$.
    Le reste est page $20$ du Corps locaux.
\end{proof}
Je veux mentionner la deuxième partie de la preuve, où $\m\m^{-1}=\m$
implique $\m^{-1}=A$. On prend $x\in \m^{-1}$, si $x\notin A$ alors on
a pas nécessairemment $x^n\in \m^{-1}$. Faut le montrer avant de regarder
les $A$-module $[1,x,\ldots,x^n]$ dans $\m^{-1}$ de type fini.

Je veux aussi mentionner la troisème, la preuve est assez instructive
sur les méthodes au final. L'idée c'est de montrer que y'a une fraction
dans $\m^{-1}$ si $A$ a un seul idéal premier non nul (dans le 1.).
Pour ça
\begin{enumerate}
    \item On montre que $K=A_x$ pour un $x\in \m$. ($1/z=y/x^n$)
    \item Alors si $z\in A-0$, $\m\subset\sqrt{zA}$. On peut prendre
	$z\in \m$.
    \item Prendre $y\in\m^{n-1}-\m^n$ avec $\m^n\subset zA$ minimal.
\end{enumerate}

\chapter{1er point sur les cours, 04/10/2024}
\section{Extensions sur corps non complets}
Si $(K, |.|_K)$ est ultramétrique et $E/K$ finie. On peut regarder
$Aut(L/K)$ pour construire
\[|.|_L\mapsto |\sigma(.)|_L\]
elles étendent toutes $|.|_K$ et on en a $\#Aut(L/K)/ef$ dans le cas
galoisien par exemple. À l'inverse, 
étant donné $|.|_L$ qui étend $|.|_K$ on peut obtenir un $\sigma$ en
trouvant $\tau\colon \widehat L\to (\widehat{K})^c$ d'où
$|.|_{\widehat L}= |.|_c\circ\tau$. En fait on regarde 
% https://q.uiver.app/#q=WzAsNSxbMCwxLCJLIl0sWzEsMSwiXFxvdmVybGluZSBLXFxzaW1lcSBcXHdpZGVoYXQgSyJdLFsyLDEsIihcXHdpZGVoYXQgSyleYyJdLFswLDAsIkwiXSxbMSwwLCJMLlxcb3ZlcmxpbmUgSz1cXHdpZGVoYXQgTCJdLFswLDEsIiIsMix7InN0eWxlIjp7InRhaWwiOnsibmFtZSI6Imhvb2siLCJzaWRlIjoidG9wIn19fV0sWzEsMiwiIiwyLHsic3R5bGUiOnsidGFpbCI6eyJuYW1lIjoiaG9vayIsInNpZGUiOiJ0b3AifX19XSxbMCwzXSxbMyw0XSxbMSw0XSxbNCwyLCIiLDAseyJzdHlsZSI6eyJib2R5Ijp7Im5hbWUiOiJkYXNoZWQifX19XV0=
\[\begin{tikzcd}
	L & {L.\overline K=\widehat L} \\
	K & {\overline K\simeq \widehat K} & {(\widehat K)^c}
	\arrow[from=1-1, to=1-2]
	\arrow[dashed, from=1-2, to=2-3]
	\arrow[from=2-1, to=1-1]
	\arrow[hook, from=2-1, to=2-2]
	\arrow[from=2-2, to=1-2]
	\arrow[hook, from=2-2, to=2-3]
\end{tikzcd}\]
où $\overline K$ est l'adhérence de $K$ dans $\widehat L$. L'égalité
montre que $\widehat L$ est de dimension finie sur $\widehat K$. En
particulier on obtient $\tau \colon \widehat L\to (\widehat K)$. Ensuite,
$\tau(\widehat L)\ni\tau(x)\mapsto |x|_{\widehat L}$ est une valeur 
absolue sur $\tau(\widehat L)\subset (\widehat K)^c$ qui étend celle
de $\overline K\simeq \widehat K$ d'où par unicité sur la clôture on a
\[|x|_{\widehat L}=|\tau(x)|_c\]
et on peut faire redescendre sur $\tau|_L\colon L\to K^c$. 
\begin{rem}
    Y'a plusieurs plongements en général $\widehat L \to (\widehat K)^c$
    penser au groupe de galois sur $\Q$ qui devient le groupe de 
    décomposition sur $\Q_p$. Et y définissent tous la même valeur
    absolue sur $\widehat L$.
\end{rem}
\section{Avec les anneaux artiniens}
Là y'a un truc cool qui montre clairement la décomposition des premiers
dans des extensions! On a une correspondance entre 


\[\{\textrm{Idéaux maximaux de }L\otimes_K\widehat K\}\leftrightarrow\{\textrm{extensions de $|.|_K$ à $L$}\}\]
Pour la surjectivité on utilise % https://q.uiver.app/#q=WzAsNSxbMCwyLCJLIl0sWzIsMiwiXFx3aWRlaGF0IEsiXSxbMCwwLCJMIl0sWzIsMCwiXFx3aWRlaGF0IEwiXSxbMSwxLCJMXFxvdGltZXNfSyBcXHdpZGVoYXQgSyJdLFswLDJdLFswLDFdLFsxLDNdLFsyLDNdLFsxLDRdLFsyLDRdLFs0LDMsIiIsMix7InN0eWxlIjp7ImJvZHkiOnsibmFtZSI6ImRhc2hlZCJ9fX1dLFswLDRdXQ==
\[\begin{tikzcd}
	L && {\widehat L} \\
	& {L\otimes_K \widehat K} \\
	K && {\widehat K}
	\arrow[from=1-1, to=1-3]
	\arrow[from=1-1, to=2-2]
	\arrow[dashed, from=2-2, to=1-3]
	\arrow[from=3-1, to=1-1]
	\arrow[from=3-1, to=2-2]
	\arrow[from=3-1, to=3-3]
	\arrow[from=3-3, to=1-3]
	\arrow[from=3-3, to=2-2]
\end{tikzcd}\]
et on pose $\m_{|.|_L}=\ker(L\otimes_K \widehat K)\to \widehat L$. Pour
l'injectivité c'est assez direct.

\section{Extensions de valuations et corps complets}
Le cas archimédien est assez particulier puisque y'a que $\R$ et $\C$.
C'est l'exercice $8$ de la feuille où on peut montrer que tout corps
archimédien complet qui contient $i$ est isomorphe à $\C$. Je
vais regarder surtout le cas non archimédien.
\begin{thm}
    Si $C(K)$ est l'ensemble des suites de Cauchy sur $K$ et $\m(K)$
    celles qui tendent vers $0$ alors $\m(K)$ est maximal et
    $C(K)/\m(K)$ est un corps complet minimal où $K$ est dense.
\end{thm}
Pour les extensions maintenant on a besoin de Hensel et du fait que
\begin{cor}
    Si $K$ est complet ultramétrique, le max des coefficients de $P(X)$
    irréductible sur $K$ est atteint par $P^*(0)$ (coeff dominant)
    ou $P(0)$. En particulier si $P^*(0)=1$ et $|P(0)|\leq 1$ alors
    $P$ est dans $\Or_K[X]$.
\end{cor}
mais j'en parlerai a un autre moment. Donc maintenant qu'on sait c'est
quoi les corps complets premiers en caractéristique $0$ on regarde
leurs extensions finies. L'idée c'est que c'est des Banach de dimension
finie sur un $\Q_p$ donc
\begin{itemize}
    \item Toutes les normes sont équivalentes.
\end{itemize}
en particulier les valeurs absolues sont équivalentes, donc y'en a 
qu'une qui étend $|.|_p$.

\begin{proof}[Construction de la valeur absolue]
    
Étant donné $x\in L/K$ on regarde 
l'endomorphisme de multiplication sur $L$ par $x$, $m_x\in End_K(L)$.
On déf 
\[N_{L/K}(x)=\det(m_x)\]
comme $N_{L/K}(y\in K)=y^{[L:K]}$ on déf 
\[|.|_L:=|N_{L/K}(.)^{1/[L:K]}|\]
pour vérifier qu'elle est ultramétrique, on se rappelle que $\det(m_x)$
est le coefficient constant de $\chi_{m_x}(X)$ le polynôme 
caractéristique de $m_x$ à coefficient dans $K$. En plus 
$x\mapsto m_x$ est un homomorphisme d'anneau. D'où 
\[N_{L/K}(x)=(-1)^{[L:K]}\chi_{m_x}(0)\]
si on remarque que $m_x$ est diagonale par bloc avec $[L:K(x)]$
blocs. On peut affiner en remarquant que 
\[\chi_{m_x}=\mu_{x}^{[L:K(x)]}\]
avec $\mu_x$ le polynôme minimal de $x$. Maintenant pour montrer que
c'est ultramétrique faut montrer que 
\[|1+\alpha|_L\leq 1\]
Mais $\mu_{1+x}(X)=\mu_{x}(X-1)$ et $\mu_x\in \Or_K$ par le corollaire 
puis $\mu_x(-1)\in \Or_K$ d'où le résultat. Les autres propriétés sont
claires.
\end{proof}
On peut maintenant l'étendre a une clôture algébrique $K^c$ via
\[x\mapsto |N_{K(x)/K}(x)^{1/[K(x):K]}|\]
et il y'a unicité.

\section{Classification sur $\Q$}
En théorie des nombres, on a regardé grosso modo Ostrowski et les 
équivalences de valeurs absolues ($|.|^s$) pour définir la même 
topologie.
% https://q.uiver.app/#q=WzAsOSxbMywxLCIoXFxtYXRoYmIgUSx8LnwpIl0sWzIsMiwiXFxtYXRoYmIgUV97cF8yfSJdLFsxLDIsIlxcbWF0aGJiIFFfe3BfMX0iXSxbMywyLCJcXG1hdGhiYiBRX3twXzN9Il0sWzQsMiwiXFxtYXRoYmIgUV97cF80fSJdLFs1LDIsIlxcbWF0aGJiIFFfe3BfNX0iXSxbMywwLCJcXG1hdGhiYiBSIl0sWzYsMiwiXFxsZG90cyJdLFswLDIsIlxcbGRvdHMiXSxbMCw2XSxbMCwyXSxbMCwxXSxbMCwzXSxbMCw0XSxbMCw1XV0=
\[\begin{tikzcd}
	&&& {\mathbb R} \\
	&&& {(\mathbb Q,|.|)} \\
    {\mathbb Q_{2}} & {\mathbb Q_{3}} & {\mathbb Q_{5}} & {\mathbb Q_{7}} & {\mathbb Q_{11}} & {\mathbb Q_{13}} & \ldots
	\arrow[from=2-4, to=1-4]
	\arrow[from=2-4, to=3-1]
	\arrow[from=2-4, to=3-2]
	\arrow[from=2-4, to=3-3]
	\arrow[from=2-4, to=3-4]
	\arrow[from=2-4, to=3-5]
	\arrow[from=2-4, to=3-6]
\end{tikzcd}\]
Les grosses idées a retenir pour moi : étant donné une $|.|$, si
elle est non archimédienne $\to$ on regarde la valuation associée $v$. 
Ensuite $\m\cap \Z=p\Z$, et ensuite sur $\Z$ si $x\wedge p=1$
$|ux|=|1-yp|=1$ d'où suffit de regarder $|p|$. Le cas archimédien est 
un peu plus compliqué mais assez étonnant. En fait si on regarde un corps
archimédien complet dont la v.a. étend celle de $\Q$ contenant $i$.
Alors on peut plonger $\C$, disons $i\colon \C\to K$. L'idée c'est de
regarder pour $a\in K$ l'inf de,
\[f_a\colon z\mapsto |z-a|_K\]
si c'est $r>0$ on peut faire bouger $r$ comme on veut. Ensuite faut 
regarder pour $|\gamma_0 - a|=r$ et $|\gamma - \gamma_0|<r$ on peut 
calculer 
\[(\gamma_0-\gamma)^n - (\gamma_0 - a)^n\]
le spécificité de $\C$ analytiquement c'est alors d'avoir toutes
les racines de l'unités. Ce qui permet de factoriser le truc du dessus.


\end{document}


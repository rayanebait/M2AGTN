\documentclass[a4paper,12pt]{article}
\usepackage{amsmath,  amsthm,enumerate}
\usepackage{csquotes}
\usepackage[provide=*,french]{babel}
\usepackage[dvipsnames]{xcolor}
\usepackage{quiver, tikz}

%symbole caligraphique
\usepackage{mathrsfs}

%hyperliens
\usepackage{hyperref}

%pseudo-code
\usepackage{algpseudocode}
\usepackage{algorithm}
\makeatletter
  \renewcommand{\ALG@name}{Algorithme}
  \makeatother
\usepackage{fancyhdr}

%
%\pagestyle{fancy}
%\addtolength{\headwidth}{\marginparsep}
%\addtolength{\headwidth}{\marginparwidth}
%\renewcommand{\chaptermark}[1]{\markboth{#1}{}}
%\renewcommand{\sectionmark}[1]{\markright{\thesection\ #1}}
%\fancyhf{}
%\fancyfoot[C]{\thepage}
%\fancyhead[LO]{\textit \leftmark}
%\fancyhead[RE]{\textit \rightmark}
%\renewcommand{\headrulewidth}{0pt} % and the line
%\fancypagestyle{plain}{%
%\fancyhead{} % get rid of headers
%}
%

%bibliographie
\usepackage[
backend=biber,
style=alphabetic,
sorting=ynt
]{biblatex}

\addbibresource{bib.bib}

\usepackage{appendix}
\renewcommand{\appendixpagename}{Annexe}

\definecolor{wgrey}{RGB}{148, 38, 55}

\setlength\parindent{24pt}

\newcommand{\Z}{\mathbb{Z}}
\newcommand{\R}{\mathbb{R}}
\newcommand{\rel}{\omathcal{R}}
\newcommand{\Q}{\mathbb{Q}}
\newcommand{\C}{\mathbb{C}}
\newcommand{\N}{\mathbb{N}}
\newcommand{\K}{\mathbb{K}}
\newcommand{\A}{\mathbb{A}}
\newcommand{\B}{\mathcal{B}}
\newcommand{\Or}{\mathcal{O}}
\newcommand{\F}{\mathbb F}
\newcommand{\m}{\mathfrak m}
\renewcommand{\b}{\mathfrak b}
\renewcommand{\a}{\mathfrak a}
\newcommand{\p}{\mathfrak p}
\newcommand{\I}{\mathfrak I}
\newcommand{\Hom}{\textrm{Hom}}
\newcommand{\disc}{\textrm{disc}}
\newcommand{\Pic}{\textrm{Pic}}
\newcommand{\End}{\textrm{End}}
\newcommand{\Spec}{\textrm{Spec}}

\newcommand{\cL}{\mathscr{L}}
\newcommand{\G}{\mathscr{G}}
\newcommand{\D}{\mathscr{D}}
\newcommand{\E}{\mathscr{E}}

\theoremstyle{plain}
\newtheorem{thm}{Théoreme}
\newtheorem{lem}{Lemme}
\newtheorem{prop}{Proposition}
\newtheorem{cor}{Corollaire}
\newtheorem{heur}{Heuristique}
\newtheorem{rem}{Remarque}
\newtheorem{rembis}{Remarque}
\newtheorem{note}{Note}

\theoremstyle{definition}
\newtheorem{conj}{Conjecture}
\newtheorem*{eq}{Équivalences}
\newtheorem{prob}{Problème}
\newtheorem{quest}{Question}
\newtheorem{prot}{Protocole}
\newtheorem{algo}{Algorithme}
\newtheorem{defn}{Définition}
\newtheorem{defnbis}{Définition}
\newtheorem{ex}{Exemple}
\newtheorem{exo}{Exercices}

\theoremstyle{remark}

\definecolor{wgrey}{RGB}{148, 38, 55}
\definecolor{wgreen}{RGB}{100, 200,0} 
\hypersetup{
    colorlinks=true,
    linkcolor=wgreen,
    urlcolor=wgrey,
    filecolor=wgrey
}

\title{Produits tensoriels}
\date{}

\begin{document}
\maketitle
Je veux discuter à nouveau les produits tensoriels. Une très grande
remarque, bilinéaire implique pas linéaire ! Et inversement.

\section{Cadre}
On se place toujours dans le cadre où on a des $R$-modules $M$ et 
$N$ ou des $R$-algèbres
\[R\to A\]
et
\[R\to B\]
dans le cas d'anneaux $M=A$ et $N=B$ bah c'est des $R$-algèbres.
\section{Construction}
Comme d'hab, on prends un gros quotient de $A\times B$.
\section{Propriété universelle}
Voir produits fibrés.

\section{Codiagonale, ou pas}
Étant donné $id_{A\times B}$, on pourrait dire qu'on obtient
$c\colon A\otimes_C B \to
A\times B$ tel que $c\circ \otimes =id_{A\times B}$. Mais en fait
ça aurait pas de sens vu que $c(a\otimes b)=(a,b)$ voudrait dire
que $(ra,b)=r(a,b)=(a,rb)$. Le problème c'est que $id_{A\times B}$
est pas bilinéaire, vu que 
\[(a+a',2b)=(a,b)+(a',b)\ne (a+a',b)\]
mais elle est bien linéaire.
\begin{rem}
    De cette manière on voit que la bilinéarité c'est vachement
    différent de la linéarité en un sens.
\end{rem}

\section{Exactitude}
Là c'est croustillant.  On regarde
% https://q.uiver.app/#q=WzAsOSxbMCwwLCIwIl0sWzEsMCwiQSJdLFsyLDAsIkQiXSxbMywwLCJFIl0sWzQsMCwiMCJdLFsxLDEsIkFcXG90aW1lc19SIEIiXSxbMiwxLCJEXFxvdGltZXNfUiBCIl0sWzMsMSwiRVxcb3RpbWVzX1IgQiJdLFs0LDEsIjAiXSxbMCwxXSxbMSwyLCJmIl0sWzIsMywiZyJdLFszLDRdLFs3LDhdLFs2LDcsImZcXG90aW1lcyBpZF9CIl0sWzUsNiwiZlxcb3RpbWVzIGlkX0IiXV0=
\[\begin{tikzcd}
	0 & A & D & E & 0 \\
	& {A\otimes_R B} & {D\otimes_R B} & {E\otimes_R B} & 0
	\arrow[from=1-1, to=1-2]
	\arrow["f", from=1-2, to=1-3]
	\arrow["g", from=1-3, to=1-4]
	\arrow[from=1-4, to=1-5]
	\arrow["{f\otimes id_B}", from=2-2, to=2-3]
	\arrow["{f\otimes id_B}", from=2-3, to=2-4]
	\arrow[from=2-4, to=2-5]
\end{tikzcd}\]
La surjectivité à droite est immédiate ! Pour l'exactitude au
milieu, pour un élément du noyau
$g\otimes id_B (\sum d_i\otimes b_i)=0$
on remarque que la flèche bilinéaire (!!)
\[D\times B\to E\times B\to E\]
se factorise en
% https://q.uiver.app/#q=WzAsNCxbMCwwLCJEXFx0aW1lcyBCIl0sWzEsMCwiRVxcdGltZXMgQiJdLFsyLDAsIkUiXSxbMSwxLCJFXFxvdGltZXNfQyBCIl0sWzEsMiwicF8xIl0sWzAsMV0sWzMsMiwiIiwxLHsic3R5bGUiOnsiYm9keSI6eyJuYW1lIjoiZG90dGVkIn19fV0sWzEsM11d
\[\begin{tikzcd}
	{D\times B} & {E\times B} & E \\
	& {E\otimes_C B}
	\arrow[from=1-1, to=1-2]
	\arrow["{p_1}", from=1-2, to=1-3]
	\arrow[from=1-2, to=2-2]
	\arrow[dotted, from=2-2, to=1-3]
\end{tikzcd}\]
d'où le noyau de $D\otimes_C B\to E\otimes_C B$ est
contenu dans $\ker(D\to E)\otimes_C B$. Et l'inverse est clair.

\begin{rem}
    Pas besoin de parler de $R$-algèbre, on aurait juste pu
    raccourcir la preuve en regardant les $(d,1)$ quoi.
\end{rem}

Maintenant pourquoi c'est pas nécessairemment injectif à gauche? 
En fait la propriété universelle pour $A\otimes_R B$ est pas
entièrement vérifiée. C'est à dire que $\sum f(a_i)\times b_i$ est
d'image nulle pour tout $D\times B\to F$, mais c'est pas assez pour
que ça implique que $v=\sum (a_i,b_i)$ soit nul. Simplement parce
que $v$ est d'image nulle seulement pour les $A\times B\to F$ qui
se factorisent par $A\times B\to D\times B\to F$. Il en manque.


\section{Produits de corps}
Un jour je regarderai bien  
\href{https://mathoverflow.net/questions/82083/when-is-the-tensor-product-of-two-fields-a-field}{ça} sur MO.
Apparemment étant donné $L/k, E/k$ deux extensions de $k$. On a
\[\dim_{Krull}(L\otimes_k K)=\min(dimtr_k(L),dimtr_k(K))\]
c'est trop marrant. Donc pour que le produit soit un corps
y faut forcément que l'un des deux corps soit algébrique sur $k$.
Ça discute une condition suffisante aussi!

\section{Produits d'algèbres}
Si on remplace $A\times B\to F$ un morphisme bilinéaire de 
$R$-modules. Par un morphisme de $R$-algèbres en plus d'être 
bilinéaire. Le morphisme $A\otimes_R B\to F$ devient un morphisme
de $R$-algèbres.
\subsection{Polynomes}
En particulier si on a $R\to A$ alors
\[R[X]\otimes_R A=A[X]\]
parce que $A[X]$ représente $B\mapsto B$ le foncteur 
\[Mod_A\to Set\]
c'est à dire que suffit de connaître l'image de $X$. 
Je détaille parce que c'est pas entièrement clair.
Faut aussi montrer que étant donné $b\in B$ on peut
toujours définir entièrement $f\colon R[X]\times A\to B$ 
à partir de $f(X,1)$. Mais on les envoie où les
$(1,a)$? Bah $B$ a forcément une structure de $A$-algèbre
ici. D'où 
\[f(1,a)=af(1,1)=a\]
dans $A$-$Alg$.

\section{Remarques}
Si on étudie plus en détail des familles génératrices dans 
$A\times B$ comparées à $A\otimes_C B$ on se rend compte de 
plusieurs trucs. On "perd" des éléments dans $A\otimes_C B$ mais
ça force la dimension à augmenter ! Je parle du fait qu'on perd
après parce que c'est bizarre.

\subsection{"On perd"}
On a une flèche surjective $A\times B\to A\otimes_C B$ ce qui est
étonnant vu les histoires de dimension. On a un noyau qui contient
$0\times B$ et $A\times 0$ par exemple. Le truc c'est que c'est
une flèche bilinéaire mais pas linéaire! ($f(ra,rb)=r^2f(a,b)$)

\subsection{Ça force la dimension à augmenter!}
Par exemple si on a 
$(a_i)\in A$ et $(b_j)\in B$ des familles alors 
\[(a_i, b_j)\]
sont engendrés par les $(a_i,0)$ et $(0,b_j)$. Sauf que par 
bilinéarité $a_i\otimes 0=0\otimes b_j=0$. On est obligés de
considérer les 
\[a_i\otimes b_j\]
d'où si $C=k$ un corps par exemple on a 
\[\dim_k A\times B=\dim_k (A)+\dim_k(B)\]
tandis que
\[\dim_k A\otimes_k B=\dim_k(A).\dim_k(B)!\]
Pour prouver le deuxième on peut juste remarque que
les $(a_i\otimes b_j)_j$ sont libres en indexant à droite. D'où
la liberté se ramène à une liberté terme à terme.
\section{Résumé}
Pour la construction, on peut le voir comme le quotient du 
$R$-module libre sur $A\times B$, $E$, où on quotiente par les 
conditions nécessaires à ce que toute $f\colon A\times B\to F$
bilinéaire s'étende en $E\to F$ par $\bar f(e_{(a,b)})=f(a,b)$.
En particulier c'est engendré par 
\[e_{ax+by,z}-ae_{x,z}-be_{y,z}=e_{x,az+bt}-ae_{x,z}-be_{x,t}=0\]
c'est là qu'on voit que le produit est engendré par les
\[x\otimes y\]


\subsection{Même définitions}
Donc la construction en tant que $R$-module est exactement
la même que la construction en tant que $R$-algèbres, juste 
on rajoute le produit terme à terme (loi de composition
pas quotient).
\subsection{Deux propriétés universelles}
En tant que $R$-modules : les $R$-flèche bilinéaires
\[A\times B\to C\]
se factorisent par $A\otimes_R B\to F$ en flèche linéaire de 
manière unique une fois $\otimes$ fixé.
En tant que $R$-algèbres : Toutes deux $R$-flèches d'algèbres
\[A\to C, B\to C\]
fournissent une unique flèche $A\times_R B\to C$ de $R$-algèbres.
C'est le produit fibré. 

\begin{rem}
    La dernière remarque c'est que les $R$-flèches doivent
    coincider, faut donc former le carré du push-out. La flèche
    bilinéaire fait rentrer ça dans sa définition. La différence
    c'est qu'un $R$-module c'est pas qu'un morphisme $R\to A$ 
    nécessairemment! C'est plutôt $R\to End(A,A)$ (qui a une
    structure d'anneau via celle de $A$) j'ai
    l'impression.
\end{rem}

\end{document}


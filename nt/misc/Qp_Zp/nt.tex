\documentclass[a4paper,12pt]{book}
\usepackage{amsmath,  amsthm,enumerate}
\usepackage{csquotes}
\usepackage[provide=*,french]{babel}
\usepackage[dvipsnames]{xcolor}
\usepackage{quiver, tikz}

%symbole caligraphique
\usepackage{mathrsfs}

%hyperliens
\usepackage{hyperref}

%pseudo-code
\usepackage{algpseudocode}
\usepackage{algorithm}
\makeatletter
  \renewcommand{\ALG@name}{Algorithme}
  \makeatother
\usepackage{fancyhdr}

\pagestyle{fancy}
\addtolength{\headwidth}{\marginparsep}
\addtolength{\headwidth}{\marginparwidth}
\renewcommand{\chaptermark}[1]{\markboth{#1}{}}
\renewcommand{\sectionmark}[1]{\markright{\thesection\ #1}}
\fancyhf{}
\fancyfoot[C]{\thepage}
\fancyhead[LO]{\textit \leftmark}
\fancyhead[RE]{\textit \rightmark}
\renewcommand{\headrulewidth}{0pt} % and the line
\fancypagestyle{plain}{%
    \fancyhead{} % get rid of headers
}

%bibliographie
\usepackage[
backend=biber,
style=alphabetic,
sorting=ynt
]{biblatex}

\addbibresource{bib.bib}

\usepackage{appendix}
\renewcommand{\appendixpagename}{Annexe}

\definecolor{wgrey}{RGB}{148, 38, 55}

\setlength\parindent{24pt}

\newcommand{\Z}{\mathbb{Z}}
\newcommand{\R}{\mathbb{R}}
\newcommand{\rel}{\omathcal{R}}
\newcommand{\Q}{\mathbb{Q}}
\newcommand{\C}{\mathbb{C}}
\newcommand{\N}{\mathbb{N}}
\newcommand{\K}{\mathbb{K}}
\newcommand{\A}{\mathbb{A}}
\newcommand{\B}{\mathcal{B}}
\newcommand{\Or}{\mathcal{O}}
\newcommand{\F}{\mathbb F}
\newcommand{\m}{\mathfrak m}
\renewcommand{\b}{\mathfrak b}
\renewcommand{\a}{\mathfrak a}
\newcommand{\p}{\mathfrak p}
\newcommand{\I}{\mathfrak I}
\newcommand{\Hom}{\textrm{Hom}}
\newcommand{\disc}{\textrm{disc}}
\newcommand{\Pic}{\textrm{Pic}}
\newcommand{\End}{\textrm{End}}
\newcommand{\Spec}{\textrm{Spec}}
\newcommand{\Frac}{\textrm{Frac}}

\newcommand{\cL}{\mathscr{L}}
\newcommand{\G}{\mathscr{G}}
\newcommand{\D}{\mathscr{D}}
\newcommand{\E}{\mathscr{E}}

\theoremstyle{plain}
\newtheorem{thm}{Théoreme}
\newtheorem{lem}{Lemme}
\newtheorem{prop}{Proposition}
\newtheorem{cor}{Corollaire}
\newtheorem{heur}{Heuristique}
\newtheorem{rem}{Remarque}
\newtheorem{rembis}{Remarque}
\newtheorem{note}{Note}

\theoremstyle{definition}
\newtheorem{conj}{Conjecture}
\newtheorem*{eq}{Équivalences}
\newtheorem{prob}{Problème}
\newtheorem{quest}{Question}
\newtheorem{prot}{Protocole}
\newtheorem{algo}{Algorithme}
\newtheorem{defn}{Définition}
\newtheorem{defnbis}{Définition}
\newtheorem{ex}{Exemple}
\newtheorem{exo}{Exercices}

\theoremstyle{remark}

\definecolor{wgrey}{RGB}{148, 38, 55}
\definecolor{wgreen}{RGB}{100, 200,0} 
\hypersetup{
    colorlinks=true,
    linkcolor=wgreen,
    urlcolor=wgrey,
    filecolor=wgrey
}

\title{Complétions}
\date{}

\begin{document}
\maketitle


\section{Développement $p$-adique.}
Pour rappel (fichier complétions), étant donné $x=(x_n)\in \Q_p$ : 
\begin{enumerate}
    \item Si $|x|\leq 1$, on peut supposer $x_n\in \Z$ pour tout
        $n$.
    \item La valuation se stabilise ($|K|=|\hat K|$ dans le cas
        ultramétrique).
    \item En écrivant $x_n=\sum a_{i,n}p^i$, le fait que
        $|x_n-x_{n+k}|\to 0$ force les premiers termes à 
        stabiliser.
    \item On peut remplacer $x_n$ par $\sum_{i=0}^n a_i p^i$.
\end{enumerate}
On peut alors écrire $x=\sum_{i\geq0} a_ip^i$.
\section{Topologie}
\subsection{Densité de $\Z$ dans $\Z_p$}
Par l'écriture en série entière. Mais on a contourné l'idée
que si $u\in \Z$ et $v_p(u)=0$, $1/u$ a pas toujours de
développement $p$-adique fini. 

\subsection{$\Z_p$ est compact.}
On peut remarquer que $\Z_p$ est séparé 
(c'est pas totalement clair, prendre les $x+p\Z_p$). 

\subsection{$\Z_p[p^{-1}]=\Q_p$}
On a $\Z_{(p)}\subseteq \Z_p$ et 
\[\Z_{(p)}[p^{-1}]=\Q\subset \Z_p[p^{-1}]\]
maintenant le truc de droite est complet donc fermé. Pour le voir
on remarque que une suite $(x_n)$ de ce truc converge dans $\Q_p$
et sa valuation se fixe. En particulier il existe $N$ tel que 
$(p^Nx_n)\in \Z_p$ et le résultat.

Le deuxième on peut le voir comme juste qu'on coupe la partie 
positive du développement, donc c'est assez clair qu'il reste que
une somme finie dans $\Z[p^{-1}]$.

\end{document}


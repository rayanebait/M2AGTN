\documentclass[a4paper,12pt]{book}
\usepackage{amsmath,  amsthm,enumerate}
\usepackage{csquotes}
\usepackage[provide=*,french]{babel}
\usepackage[dvipsnames]{xcolor}
\usepackage{quiver, tikz}

%symbole caligraphique
\usepackage{mathrsfs}

%hyperliens
\usepackage{hyperref}

%pseudo-code
\usepackage{algpseudocode}
\usepackage{algorithm}
\makeatletter
  \renewcommand{\ALG@name}{Algorithme}
  \makeatother
\usepackage{fancyhdr}

\pagestyle{fancy}
\addtolength{\headwidth}{\marginparsep}
\addtolength{\headwidth}{\marginparwidth}
\renewcommand{\chaptermark}[1]{\markboth{#1}{}}
\renewcommand{\sectionmark}[1]{\markright{\thesection\ #1}}
\fancyhf{}
\fancyfoot[C]{\thepage}
\fancyhead[LO]{\textit \leftmark}
\fancyhead[RE]{\textit \rightmark}
\renewcommand{\headrulewidth}{0pt} % and the line
\fancypagestyle{plain}{%
    \fancyhead{} % get rid of headers
}

%bibliographie
\usepackage[
backend=biber,
style=alphabetic,
sorting=ynt
]{biblatex}

\addbibresource{bib.bib}

\usepackage{appendix}
\renewcommand{\appendixpagename}{Annexe}

\definecolor{wgrey}{RGB}{148, 38, 55}

\setlength\parindent{24pt}

\newcommand{\Z}{\mathbb{Z}}
\newcommand{\R}{\mathbb{R}}
\newcommand{\rel}{\omathcal{R}}
\newcommand{\Q}{\mathbb{Q}}
\newcommand{\C}{\mathbb{C}}
\newcommand{\N}{\mathbb{N}}
\newcommand{\K}{\mathbb{K}}
\newcommand{\A}{\mathbb{A}}
\newcommand{\B}{\mathcal{B}}
\newcommand{\Or}{\mathcal{O}}
\newcommand{\F}{\mathbb F}
\newcommand{\m}{\mathfrak m}
\renewcommand{\b}{\mathfrak b}
\renewcommand{\a}{\mathfrak a}
\newcommand{\p}{\mathfrak p}
\newcommand{\I}{\mathfrak I}
\newcommand{\Hom}{\textrm{Hom}}
\newcommand{\disc}{\textrm{disc}}
\newcommand{\Pic}{\textrm{Pic}}
\newcommand{\End}{\textrm{End}}
\newcommand{\Spec}{\textrm{Spec}}

\newcommand{\cL}{\mathscr{L}}
\newcommand{\G}{\mathscr{G}}
\newcommand{\D}{\mathscr{D}}
\newcommand{\E}{\mathscr{E}}

\theoremstyle{plain}
\newtheorem{thm}{Théoreme}
\newtheorem{lem}{Lemme}
\newtheorem{prop}{Proposition}
\newtheorem{cor}{Corollaire}
\newtheorem{heur}{Heuristique}
\newtheorem{rem}{Remarque}
\newtheorem{rembis}{Remarque}
\newtheorem{note}{Note}

\theoremstyle{definition}
\newtheorem{conj}{Conjecture}
\newtheorem*{eq}{Équivalences}
\newtheorem{prob}{Problème}
\newtheorem{quest}{Question}
\newtheorem{prot}{Protocole}
\newtheorem{algo}{Algorithme}
\newtheorem{defn}{Définition}
\newtheorem{defnbis}{Définition}
\newtheorem{ex}{Exemple}
\newtheorem{exo}{Exercices}

\theoremstyle{remark}

\definecolor{wgrey}{RGB}{148, 38, 55}
\definecolor{wgreen}{RGB}{100, 200,0} 
\hypersetup{
    colorlinks=true,
    linkcolor=wgreen,
    urlcolor=wgrey,
    filecolor=wgrey
}

\title{Les extensions de corps résiduels et conséquences}
\date{}

\begin{document}
\maketitle


\section{Cas générique}
On se place \textbf{toujours} dans le cadre où on a $\Or_K$ de valuation 
\textbf{discrète}. Le cadre en gros c'est 
% https://q.uiver.app/#q=WzAsNixbMCwwLCJcXE9yX0siXSxbMSwwLCJcXHRpbGRlXFxPcl9LIl0sWzIsMCwiKFxcdGlsZGVcXE9yX0spX3tcXG1faX0iXSxbMywwLCJcXGxlZnQoXFxPcl9MXFxyaWdodCkiXSxbMCwxLCJrX0siXSxbMiwxLCJrX0wiXSxbMCwxLCIiLDAseyJzdHlsZSI6eyJoZWFkIjp7Im5hbWUiOiJub25lIn19fV0sWzEsMiwiXFxzdWJzZXRlcSIsMSx7InN0eWxlIjp7ImJvZHkiOnsibmFtZSI6Im5vbmUifSwiaGVhZCI6eyJuYW1lIjoibm9uZSJ9fX1dLFsyLDMsIj8iLDEseyJsZXZlbCI6Miwic3R5bGUiOnsiaGVhZCI6eyJuYW1lIjoibm9uZSJ9fX1dLFswLDRdLFsyLDVdLFs0LDVdXQ==
\[\begin{tikzcd}
	{\Or_K} & {\tilde\Or_K} & {(\tilde\Or_K)_{\m_i}} & {\left(\Or_L\right)} \\
	{k_K} && {k_L}
	\arrow[no head, from=1-1, to=1-2]
	\arrow[from=1-1, to=2-1]
	\arrow["\subseteq"{description}, draw=none, from=1-2, to=1-3]
	\arrow["{?}"{description}, Rightarrow, no head, from=1-3, to=1-4]
	\arrow[from=1-3, to=2-3]
	\arrow[from=2-1, to=2-3]
\end{tikzcd}\]

C'est à dire qu'on prends la clôture intégrale, on regarde ses idéaux
maximaux et on obtient des extensions de d.v.r. Quand $K$ est complet 
ou quand on fixe une valuation (un premier $\m_i$) sur $L$, $\Or_L$ fait
sens.
\subsection{Calcul dans le cas monogène}
    Pour calculer maintenant en fait une marche à suivre c'est
\begin{center}
    \textit{On sait le faire dans $\Or_K[\alpha]$.}
\end{center}
Si c'est le cas alors :
\begin{enumerate}
    \item La factorisation de $P$ 
	dans $k_K[X]$ donne la ramification
	et les idéaux maximaux de $\tilde\Or_K$!
    \item Plus précisément, si 
\[\bar P=\prod_i p_i^{r_i}\in k_K[X]\]
	alors $\m_i=(\m_K,p_i(\alpha))$.
\end{enumerate}
Le point \textbf{important} c'est la ramification, on relève
\[P(\alpha)=\prod_iP_i^{r_i}(\alpha) + \epsilon(\alpha)\]
ce qui donne par le deuxième point 
\[\prod_i \m_i^{r_i}=\prod_i (\m_K, P_i(\alpha))^{r_i}\subset \m_K\tilde\Or_K=\prod_i \m_i^{e_i}\]
On en déduit $r_i\geq e_i$ pour tout $i$ et on conclut directement
avec \[\sum r_i f_i = \deg \bar P=\deg P= [L:K]=\sum e_if_i\]
On a utilisé que $\tilde\Or_K$ est fini sur $\Or_K$ pour l'égalité
$\deg \bar P=\deg P$ \textbf{et} la dimension $[L:K]=\sum e_i f_i$.

\subsection{Cas primitif}

On suppose $L=K(\alpha)$ (par exemple si $L/K$ est séparable).

Si $\bar P$ est séparable, alors $\tilde\Or_K=\Or_K[\alpha]$
et on peut appliquer la section d'avant!  

On a un problème quand l'extension résiduelle est 
inséparable, on se place dans le diagramme
% https://q.uiver.app/#q=WzAsOSxbMCwxLCJcXE9yX0siXSxbMCwwLCJLIl0sWzIsMCwiTCJdLFsxLDAsIksoXFxhbHBoYSkiXSxbMSwxLCJcXHRpbGRlXFxPcl9LJyJdLFsyLDEsIlxcdGlsZGVcXE9yX0siXSxbMCwyLCJrX0siXSxbMSwyLCJrX3tLKFxcYWxwaGEpfSJdLFsyLDIsImtfTCJdLFsxLDMsIiIsMSx7InN0eWxlIjp7ImhlYWQiOnsibmFtZSI6Im5vbmUifX19XSxbMywyLCIiLDEseyJzdHlsZSI6eyJoZWFkIjp7Im5hbWUiOiJub25lIn19fV0sWzAsMSwiIiwxLHsic3R5bGUiOnsiaGVhZCI6eyJuYW1lIjoibm9uZSJ9fX1dLFswLDQsIiIsMSx7InN0eWxlIjp7ImhlYWQiOnsibmFtZSI6Im5vbmUifX19XSxbNCw1LCIiLDEseyJzdHlsZSI6eyJoZWFkIjp7Im5hbWUiOiJub25lIn19fV0sWzQsMywiIiwxLHsic3R5bGUiOnsiaGVhZCI6eyJuYW1lIjoibm9uZSJ9fX1dLFs1LDIsIiIsMSx7InN0eWxlIjp7ImhlYWQiOnsibmFtZSI6Im5vbmUifX19XSxbNiw3LCIiLDEseyJzdHlsZSI6eyJoZWFkIjp7Im5hbWUiOiJub25lIn19fV0sWzcsOCwiIiwxLHsic3R5bGUiOnsiaGVhZCI6eyJuYW1lIjoibm9uZSJ9fX1dLFswLDZdLFs0LDddLFs1LDhdXQ==
\[\begin{tikzcd}
	K & {K(\alpha)} & L \\
	{\Or_K} & {\tilde\Or_K'} & {\tilde\Or_K} \\
	{k_K} & {k_{K(\alpha)}} & {k_L}
	\arrow[no head, from=1-1, to=1-2]
	\arrow[no head, from=1-2, to=1-3]
	\arrow[no head, from=2-1, to=1-1]
	\arrow[no head, from=2-1, to=2-2]
	\arrow[from=2-1, to=3-1]
	\arrow[no head, from=2-2, to=1-2]
	\arrow[no head, from=2-2, to=2-3]
	\arrow[from=2-2, to=3-2]
	\arrow[no head, from=2-3, to=1-3]
	\arrow[from=2-3, to=3-3]
	\arrow[no head, from=3-1, to=3-2]
	\arrow[no head, from=3-2, to=3-3]
\end{tikzcd}\]


\subsection{Cas complet}
On a une équivalence entre :
\begin{enumerate}
    \item L'extension $L/K$ est non ramifiée (par déf non ramifiée et 
$k_{K(\alpha)}/k_K$ est séparable).
    \item Il existe $\alpha$ : $L=K(\alpha)$ et $P$ le pol min 
de $\alpha$ sur $K$ est séparable sur $k_K$.
\end{enumerate}
L'idée c'est juste que la formule $ef=[L:K]$ est vraie. Et on peut 
relever une base de l'extension résiduelle !
En gros ça donne une réciproque à la section d'avant. 

Dans le cas $p$-adique, les corps finis sont parfaits et on a toujours
des extensions séparables (c'est immédiat de la déf)! En particulier,
si $\bar P$ est inséparable c'est qu'il est scindé. Ça se voit bien
par Hensel :
\begin{enumerate}
    \item On a toujours $\bar P=F^d$ et en réécrivant $d\deg F=
	\deg P=e.f$ sachant que $\deg F\mid f$ (à vérifier mais ça se
	voit) on obtient $e\mid d$. (l'égalité c'est qu'on suppose $P$ 
	unitaire)
\end{enumerate}

Conclure là dessus, ajouter une discussion des cassages d'extensions 
de $\Q_p$ est totalement ramifiée et non ramifiée (le faire). Et aussi
faire le lien entre ramification sur $\Q$ est sur des complétions
$\Q_p$. Aussi conclure le cas primitif avec des divisibilités.

\section{Ramification 1}
J'vais parler de ramification ici. Le lemme clé c'est que dans une
extensions de d.v.r $\Or_K - \Or_L$. Si 
\[k_K - k_L\]
est de dimension $f\in\N\cup \infty$. Alors 
\[dim_k \Or_L/\m_K\Or_L = e.f\]
avec $\m_K=\m_L^e$. Ensuite, si $\tilde\Or_K$ est la fermeture 
intégrale de $\Or_K$ dans $L$ alors 
\[\sum_{i} e_i f_i \leq [L:K]\]
où on écrit $\m_K\tilde\Or_K=\prod_i \m_i^{e_i}$ et 
$f_i=[\tilde\Or_K/\m_i : k_K]$. Ça c'est par le lemme chinois! Pour
utiliser le résultat de juste avant faut aussi montrer que 
\[(\tilde\Or_K)_{\m_i}/\m_i^r(\tilde\Or_K)_{\m_i}\simeq \tilde\Or_K/\m_i^r\tilde\Or_K\]
Pour $\m$ maximal (ça se fait à la main). On a l'égalité dans
plusieurs cas :
\begin{enumerate}
    \item $K$ est complet, car alors $\tilde\Or_K=\Or_L$.
    \item $L/K$ est séparable, car alors $\tilde\Or_K$ est fini sur 
	$\Or_K$.
    \item Plus généralement, si $\tilde\Or_K$ est fini sur $\Or_K$.
    \item $L\otimes_K \widehat K$ est réduite. Regarder le lien
	entre les nilpotents et la séparabilité.
\end{enumerate}
Maintenant faut la calculer.


\end{document}


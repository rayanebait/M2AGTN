\documentclass[a4paper,12pt]{book}
\usepackage{amsmath,  amsthm,enumerate}
\usepackage{csquotes}
\usepackage[provide=*,french]{babel}
\usepackage[dvipsnames]{xcolor}
\usepackage{quiver, tikz}

%symbole caligraphique
\usepackage{mathrsfs}

%hyperliens
\usepackage{hyperref}

%pseudo-code
\usepackage{algpseudocode}
\usepackage{algorithm}
\makeatletter
  \renewcommand{\ALG@name}{Algorithme}
  \makeatother
\usepackage{fancyhdr}

\pagestyle{fancy}
\addtolength{\headwidth}{\marginparsep}
\addtolength{\headwidth}{\marginparwidth}
\renewcommand{\chaptermark}[1]{\markboth{#1}{}}
\renewcommand{\sectionmark}[1]{\markright{\thesection\ #1}}
\fancyhf{}
\fancyfoot[C]{\thepage}
\fancyhead[LO]{\textit \leftmark}
\fancyhead[RE]{\textit \rightmark}
\renewcommand{\headrulewidth}{0pt} % and the line
\fancypagestyle{plain}{%
    \fancyhead{} % get rid of headers
}

%bibliographie
\usepackage[
backend=biber,
style=alphabetic,
sorting=ynt
]{biblatex}

\addbibresource{bib.bib}

\usepackage{appendix}
\renewcommand{\appendixpagename}{Annexe}

\definecolor{wgrey}{RGB}{148, 38, 55}

\setlength\parindent{24pt}

\newcommand{\Z}{\mathbb{Z}}
\newcommand{\R}{\mathbb{R}}
\newcommand{\rel}{\omathcal{R}}
\newcommand{\Q}{\mathbb{Q}}
\newcommand{\C}{\mathbb{C}}
\newcommand{\N}{\mathbb{N}}
\newcommand{\K}{\mathbb{K}}
\newcommand{\A}{\mathbb{A}}
\newcommand{\B}{\mathcal{B}}
\newcommand{\Or}{\mathcal{O}}
\newcommand{\F}{\mathbb F}
\newcommand{\m}{\mathfrak m}
\renewcommand{\b}{\mathfrak b}
\renewcommand{\a}{\mathfrak a}
\newcommand{\p}{\mathfrak p}
\newcommand{\I}{\mathfrak I}
\newcommand{\Hom}{\textrm{Hom}}
\newcommand{\disc}{\textrm{disc}}
\newcommand{\Pic}{\textrm{Pic}}
\newcommand{\End}{\textrm{End}}
\newcommand{\Spec}{\textrm{Spec}}
\newcommand{\Frac}{\textrm{Frac}}

\newcommand{\cL}{\mathscr{L}}
\newcommand{\G}{\mathscr{G}}
\newcommand{\D}{\mathscr{D}}
\newcommand{\E}{\mathscr{E}}

\theoremstyle{plain}
\newtheorem{thm}{Théoreme}
\newtheorem{lem}{Lemme}
\newtheorem{prop}{Proposition}
\newtheorem{cor}{Corollaire}
\newtheorem{heur}{Heuristique}
\newtheorem{rem}{Remarque}
\newtheorem{rembis}{Remarque}
\newtheorem{note}{Note}

\theoremstyle{definition}
\newtheorem{conj}{Conjecture}
\newtheorem*{eq}{Équivalences}
\newtheorem{prob}{Problème}
\newtheorem{quest}{Question}
\newtheorem{prot}{Protocole}
\newtheorem{algo}{Algorithme}
\newtheorem{defn}{Définition}
\newtheorem{defnbis}{Définition}
\newtheorem{ex}{Exemple}
\newtheorem{exo}{Exercices}

\theoremstyle{remark}

\definecolor{wgrey}{RGB}{148, 38, 55}
\definecolor{wgreen}{RGB}{100, 200,0} 
\hypersetup{
    colorlinks=true,
    linkcolor=wgreen,
    urlcolor=wgrey,
    filecolor=wgrey
}

\title{Complétions}
\date{}

\begin{document}
\maketitle

C'est pas mal du chapitre $10$ de Atiyaah-McDonald.

\chapter{Groupes abéliens topologiques}
Les opérations du groupes sont continues de sorte que si
$U$ est un voisinage de $0$, alors $x+U$ un voisinage de $x$.
\section{Séparation}
Étant donné $H=\cap_{0\in U} U$, on a $H=\{0\}$, et si $x\in H$,
alors 
\[0\in U-x\]
pour tout $0\in U$. D'où $-x\in H$. Pareil si $x,y\in H$, alors
pour tout voisinage de $0$ :
\[x\in U-y\]
d'où $x+y\in U$. D'où $H$ est un sous-groupe de $G$.
On déduit rapidement que $x+H$ est fermé pour tout $x$ d'où
\[G/H\]
est séparé. Alors 
\[G\textrm{ est séparé ssi }H=0\]


\chapter{Complétions via les suites de Cauchy}
On se met dans un groupe abélien topologique.
Étant donné une base de voisinage dénombrable de $0$. On peut
construire les suites de Cauchy en disant 
\[\textrm{Pour tout $U\ni 0$ il existe un entier $s(U)$}\]
tel que 
\[x_n-x_m\in U\textrm{ dés que} $n,m\geq s(U)$\]
\begin{rem}
    La limite est définie à l'adhérence près!
\end{rem}
\section{Complété de $G$}
Deux suites de Cauchy sont équivalentes si $x_n-y_n$ tend
vers $0$. Au sens défini d'avant. Maintenant on quotiente comme
d'hab et la somme est bien définie. On obtient 
\[\hat G:=C(G)/\m.\]

Maintenant un point intéressant, on a 
\[i\colon G\to \hat G\]
donné par les suites constantes. Mais 
\[\ker i = H\]
ça c'est rigolo. 
\begin{rem}
    Dans le cas d'un corps ou d'un anneau de valuation discrète
    $H=0$. En fait par le chapitre d'avant, quand $G$ est
    séparé.
\end{rem}

\chapter{Complétions algébriques}
J'aimerai bien aller jusqu'à Artin-Rees.
\section{Cadre}
On suppose qu'on a une base de voisinages de $0$ qui sont des
groupes. Ca évince $\R$ et la topologie habituelle. Par contre
pas $\Q$ et ses topologies $p$-adiques.

\begin{rem}
    Automatiquement, les ouverts de la base sont ouverts
    fermés. Car si $g\in G_n$, comme $g+G_n\subset G_n$ est un
    ouvert $G_n$ est ouvert. À l'inverse, pour $h\notin G_n$,
    $h+G_n\cap G_n=\emptyset$ et 
    $h+G_n$
    est ouvert d'où
    \[\cup{h\in G-G_n} h+G_n = G-G_n\]
    est ouvert puis $G_n$ est fermé.
\end{rem}


\section{}




\[\textrm{}\]
\end{document}


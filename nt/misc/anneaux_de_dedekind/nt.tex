\documentclass[a4paper,12pt]{book}
\usepackage{amsmath,  amsthm,enumerate}
\usepackage{csquotes}
\usepackage[provide=*,french]{babel}
\usepackage[dvipsnames]{xcolor}
\usepackage{quiver, tikz}

%symbole caligraphique
\usepackage{mathrsfs}

%hyperliens
\usepackage{hyperref}

%pseudo-code
\usepackage{algpseudocode}
\usepackage{algorithm}
\makeatletter
  \renewcommand{\ALG@name}{Algorithme}
  \makeatother
\usepackage{fancyhdr}

\pagestyle{fancy}
\addtolength{\headwidth}{\marginparsep}
\addtolength{\headwidth}{\marginparwidth}
\renewcommand{\chaptermark}[1]{\markboth{#1}{}}
\renewcommand{\sectionmark}[1]{\markright{\thesection\ #1}}
\fancyhf{}
\fancyfoot[C]{\thepage}
\fancyhead[LO]{\textit \leftmark}
\fancyhead[RE]{\textit \rightmark}
\renewcommand{\headrulewidth}{0pt} % and the line
\fancypagestyle{plain}{%
    \fancyhead{} % get rid of headers
}

%bibliographie
\usepackage[
backend=biber,
style=alphabetic,
sorting=ynt
]{biblatex}

\addbibresource{bib.bib}

\usepackage{appendix}
\renewcommand{\appendixpagename}{Annexe}

\definecolor{wgrey}{RGB}{148, 38, 55}

\setlength\parindent{24pt}

\newcommand{\Z}{\mathbb{Z}}
\newcommand{\R}{\mathbb{R}}
\newcommand{\rel}{\omathcal{R}}
\newcommand{\Q}{\mathbb{Q}}
\newcommand{\C}{\mathbb{C}}
\newcommand{\N}{\mathbb{N}}
\newcommand{\K}{\mathbb{K}}
\newcommand{\A}{\mathbb{A}}
\newcommand{\B}{\mathcal{B}}
\newcommand{\Or}{\mathcal{O}}
\newcommand{\F}{\mathbb F}
\newcommand{\m}{\mathfrak m}
\renewcommand{\b}{\mathfrak b}
\renewcommand{\a}{\mathfrak a}
\newcommand{\p}{\mathfrak p}
\newcommand{\I}{\mathfrak I}
\newcommand{\Hom}{\textrm{Hom}}
\newcommand{\disc}{\textrm{disc}}
\newcommand{\Pic}{\textrm{Pic}}
\newcommand{\End}{\textrm{End}}
\newcommand{\Spec}{\textrm{Spec}}
\newcommand{\Frac}{\textrm{Frac}}

\newcommand{\cL}{\mathscr{L}}
\newcommand{\G}{\mathscr{G}}
\newcommand{\D}{\mathscr{D}}
\newcommand{\E}{\mathscr{E}}

\theoremstyle{plain}
\newtheorem{thm}{Théoreme}
\newtheorem{lem}{Lemme}
\newtheorem{prop}{Proposition}
\newtheorem{cor}{Corollaire}
\newtheorem{heur}{Heuristique}
\newtheorem{rem}{Remarque}
\newtheorem{rembis}{Remarque}
\newtheorem{note}{Note}

\theoremstyle{definition}
\newtheorem{conj}{Conjecture}
\newtheorem*{eq}{Équivalences}
\newtheorem{prob}{Problème}
\newtheorem{quest}{Question}
\newtheorem{prot}{Protocole}
\newtheorem{algo}{Algorithme}
\newtheorem{defn}{Définition}
\newtheorem{defnbis}{Définition}
\newtheorem{ex}{Exemple}
\newtheorem{exo}{Exercices}

\theoremstyle{remark}

\definecolor{wgrey}{RGB}{148, 38, 55}
\definecolor{wgreen}{RGB}{100, 200,0} 
\hypersetup{
    colorlinks=true,
    linkcolor=wgreen,
    urlcolor=wgrey,
    filecolor=wgrey
}

\title{Anneaux de Dedekind}
\date{}

\begin{document}
\maketitle

\chapter{Anneaux de valuation discrète et de Dedekind.}
\section{Anneaux de valuation discrète}
Pour le contexte, moi je m'intéresse au cas intègre déjà et
au cas où le DVR est un $A_\p$ pour un anneau noethérien intègre
de dimension $1$. Sa clôture intégrale dans $\Frac(A)$ devient
un anneau de Dedekind.
\subsection{Les 2 définitions.}
Y'a deux manières de les voir :
\begin{enumerate}
    \item Dans un corps $(K,v)$ muni d'une valuation discrète.
	Avec $A=\{x\in K|v(x)\geq 0\}$. 
	($v(x)=0\implies v(x^{-1})=0$)
    \item Comme un anneau principal (donc intègre) ayant un seul
	idéal premier non nul.
\end{enumerate}
L'implication $1.$ implique $2.$ consiste juste à se placer
dans l'espace ambiant $K$.

L'autre côté consiste à construire une valuation par l'absurde.

\subsection{Première caractérisation}
\begin{eq}
    Dans un anneau noethérien, 
    \begin{center}
	DVR $\equiv$ local, noethérien, $\m=(\pi)$ non nilpotent.
    \end{center}
\end{eq}
On veut juste écrire $x=\pi^n u$ de manière unique, on peut juste
utiliser que $t\in A$ implique $t\in \m$ ou $t\in A^\times$.

\subsection{Deuxième caractérisation}
\begin{eq}
    \begin{center}
	DVR $\equiv$ Noethérien, intégralement clos, un seul idéal
	premier $\ne 0$ (local mais pas un corps ? Non! Tu
	verras pq.).
    \end{center}
\end{eq}
Là c'est un peu plus dur. Le point c'est de montrer que $\m$ est 
inversible, alors $\m$ est principal. On note 
$\m'=\{x\in K|x\m\subset A\}$. On a 
\begin{center}
    $\m\m'\subset A$ et $A\subset \m'$ implique 
    $\m\subset \m\m'$.
\end{center}
d'où $\m\m'=\m$ ou $A$. Maintenant en fait
\begin{center}
    si $\m\m'=A$ alors $\sum x_iy_i=1$ d'où 
    $u=x_{i_0}y_{i_0}\in A-\m=A^{\times}$ 
\end{center}
par l'absurde. En particulier tout $z\in\m$ se réécrit 
\[z=x_{i_0}(u^{-1}y_{i_0}z)\]
parce que $x_{i_0}y_{i_0}u^{-1}=1$ et $y_{i_0}z\in A$!

\subsection{Résumé de preuve}
On montre en trois temps que 
\begin{enumerate}
    \item Si $\m$ est inversible alors il est principal.
    \item Si $\m\m'=\m$ et $A$ est intégralement clos alors
	$\m'=A$.
    \item Si $A$ a un seul idéal non nul (et pas juste local)
	alors $\m'\ne A$.
\end{enumerate}
Le premier je l'ai expliqué avant. Le deuxième c'est que 
si $z\in \m'$ alors $A[z]\subset \m'$ qui est t.f. D'où le 
résultat. Le troisème utilise que si $x\in \m-0$, alors 
\begin{enumerate}
    \item $A_x=K$
    \item En faisant varier $x$, $\m^N\subset zA$ pour un $N$
	minimal et $z\in A-0$.
    \item Pour $z\in \m$, on a $y\in \m^{N-1}-zA$ puis $y/z\m'-A$.
\end{enumerate}
\begin{note}
Le troisième point se traduit en $\Spec(K)=D(x)\subset \Spec(A)$ et
    \begin{align*}
    A_x&=\Or_{D(x)}(D(x))\\
       &=\Or_{\Spec(A)}(\Spec(A))|_{(0)}\\
       &=\Or_{\Spec(A),(0)}\\
       &= (A\backslash 0)^{-1}A\\
       &=K
    \end{align*}
\end{note}
Aussi, cette histoire de $y\in \m^{n-1}-zA$ et $y\m\subset zA$
ça fait remarquer de l'arithmétique plus habituelle.

\subsection{But de ces caractérisations}
Celle qui nous intéresse c'est la deuxième qui permet de montrer
que $\m$ est principal. Alors on peut utiliser la première
pour montrer que c'est un DVR.

\section{Anneaux de Dedekind}
On montre que 
\begin{eq}
    \begin{center}
	$A_\p$ est un dvr pour tout $\p\equiv$ $A$ est
	noethérien intégralement clos de dimension $1$.
    \end{center}
\end{eq}
On sait que un $DVR\equiv$ noethérien intègre avec un seul
idéal premier non nul. I.e. de dimension $\leq 1$. Donc faut juste
Que intégralement clos équivaut à tout les $A_\p$ sont 
intégralement clos.
\begin{rem}
    Gauche à droite c'est l'équivalence habituelle et droite à
    gauche c'est que le dénominateur est dans aucun idéal maximal.
    Donc est inversible.
\end{rem}


\subsection{Relever l'inversibilité pour les premiers}
De l'inversibilité de $\p$ dans $A_\p$ je veux relever dans $A$.
En fait ça vient du fait que $\_\otimes_A A_\p$ est un morphisme
de monoides $Mod_{A,\subset K}\to Mod_{A_\p,\subset K}$ additif
et multiplicatif !

\subsection{$\Spec(A)$ est de dimension $1$, i.e. $V(I)$ est fini
pour $I\ne (0)$.}

Si on prends $x\in A$, psq si $\p_1,\ldots,\p_k,\ldots$ le
contiennent alors on a une chaine 
\[A\subset (A:\p_1)\subset (A:\p_1\cap \p_2)\subset \ldots\subset
(A:\p_1\cap\ldots\cap \p_k)\subset \ldots\subset x^{-1}A\]
d'où ça stationne et c'est fini. 

Comme $I$ est de type fini on obtient direct que $V(I)$ est fini.
\subsection{Valuations et décomposition}
Étant donné $I\subset A$, on a $I_\p=\p^m$ et on peut définir
$v_\p(I)=m$, on l'étend à $I\subset K$ par $v_\p(I.(A_\p:I))=0$.

\subsection{Relever l'inversibilité en général}

Il transforme aussi $(I : J)$ en $(I_\p:J_\p)$.
Concrètement :
\[(\p.(A:\p))\otimes_A A_\p = (\p A_\p.(A_\p:\p A_\p))=A_\p\]
sauf que à gauche $\p.(A:\p)\subseteq A$ est un idéal et y'a
que $A$ qui a pour image $A_\p$.
\begin{rem}
    ATTENTION. On a utilisé que $\p \mapsto \p A_\p$ est injectif !
\end{rem}
\begin{rem}
    Ca se généralise à $I$ un idéal fractionnaire car
    $I\otimes_A A_\p=\p^n$ d'où $(I\p^{-n})_\p=A_\p$. On peut 
    pas conclure que $\p^{-n}=(A:I)$ parce que si $\p'\mid I$ 
    alors $\p'\otimes_A A_\p=A_\p$. 
\end{rem}
Y faut maintenant remarquer que 
$(A:\prod \p^{v_\p(I)})_\p=(A:I)_\p$ pour tout $\p$ et conclure.

\begin{note}
    Le foncteur $\_\otimes_A A_\p$ de $Mod_A$ dans $Mod_{A_\p}$
    a surement les mêmes propriétés pour les bonnes définitions
    de produits et sommes.
\end{note}
\subsection{Décomposition en idéaux}
De la même manière que pour l'inversibilité on remarque que 
$\prod \p^{v_\p(I)}$ coincide avec $I$ dans
tout les $A_\p$. Faut montrer que donc il est égal à $I$.

\begin{note}
    Faire une section sur la décomposition en idéaux primaires.
\end{note}


\end{document}


\documentclass[a4paper,12pt]{book}
\usepackage{amsmath,  amsthm,enumerate}
\usepackage{csquotes}
\usepackage[provide=*,french]{babel}
\usepackage[dvipsnames]{xcolor}
\usepackage{quiver, tikz}

%symbole caligraphique
\usepackage{mathrsfs}

%hyperliens
\usepackage{hyperref}

%pseudo-code
\usepackage{algpseudocode}
\usepackage{algorithm}
\makeatletter
  \renewcommand{\ALG@name}{Algorithme}
  \makeatother
\usepackage{fancyhdr}

\pagestyle{fancy}
\addtolength{\headwidth}{\marginparsep}
\addtolength{\headwidth}{\marginparwidth}
\renewcommand{\chaptermark}[1]{\markboth{#1}{}}
\renewcommand{\sectionmark}[1]{\markright{\thesection\ #1}}
\fancyhf{}
\fancyfoot[C]{\thepage}
\fancyhead[LO]{\textit \leftmark}
\fancyhead[RE]{\textit \rightmark}
\renewcommand{\headrulewidth}{0pt} % and the line
\fancypagestyle{plain}{%
    \fancyhead{} % get rid of headers
}

%bibliographie
\usepackage[
backend=biber,
style=alphabetic,
sorting=ynt
]{biblatex}

\addbibresource{bib.bib}

\usepackage{appendix}
\renewcommand{\appendixpagename}{Annexe}

\definecolor{wgrey}{RGB}{148, 38, 55}

\setlength\parindent{24pt}

\newcommand{\Z}{\mathbb{Z}}
\newcommand{\R}{\mathbb{R}}
\newcommand{\rel}{\omathcal{R}}
\newcommand{\Q}{\mathbb{Q}}
\newcommand{\C}{\mathbb{C}}
\newcommand{\N}{\mathbb{N}}
\newcommand{\K}{\mathbb{K}}
\newcommand{\A}{\mathbb{A}}
\newcommand{\B}{\mathcal{B}}
\newcommand{\Or}{\mathcal{O}}
\newcommand{\F}{\mathbb F}
\newcommand{\m}{\mathfrak m}
\renewcommand{\b}{\mathfrak b}
\renewcommand{\a}{\mathfrak a}
\newcommand{\p}{\mathfrak p}
\newcommand{\I}{\mathfrak I}
\newcommand{\Hom}{\textrm{Hom}}
\newcommand{\disc}{\textrm{disc}}
\newcommand{\Pic}{\textrm{Pic}}
\newcommand{\End}{\textrm{End}}
\newcommand{\Spec}{\textrm{Spec}}

\newcommand{\cL}{\mathscr{L}}
\newcommand{\G}{\mathscr{G}}
\newcommand{\D}{\mathscr{D}}
\newcommand{\E}{\mathscr{E}}

\theoremstyle{plain}
\newtheorem{thm}{Théoreme}
\newtheorem{lem}{Lemme}
\newtheorem{prop}{Proposition}
\newtheorem{cor}{Corollaire}
\newtheorem{heur}{Heuristique}
\newtheorem{rem}{Remarque}
\newtheorem{rembis}{Remarque}
\newtheorem{note}{Note}

\theoremstyle{definition}
\newtheorem{conj}{Conjecture}
\newtheorem*{eq}{Équivalences}
\newtheorem{prob}{Problème}
\newtheorem{quest}{Question}
\newtheorem{prot}{Protocole}
\newtheorem{algo}{Algorithme}
\newtheorem{defn}{Définition}
\newtheorem{defnbis}{Définition}
\newtheorem{ex}{Exemple}
\newtheorem{exo}{Exercices}

\theoremstyle{remark}

\definecolor{wgrey}{RGB}{148, 38, 55}
\definecolor{wgreen}{RGB}{100, 200,0} 
\hypersetup{
    colorlinks=true,
    linkcolor=wgreen,
    urlcolor=wgrey,
    filecolor=wgrey
}

\title{Point sur le cours de corps locaux }
\date{}

\begin{document}
\maketitle
\chapter{Hauteurs}
C'est un petit récap rapide des hauteurs. Déjà les cent milles 
définitions, si on a $(K,|.|_v)$
\[||x||_v=|.|_v^{\lambda_v}\]
et on étend $|.|_v$ en $(L,|.|_w)$, on a 
\[|N_{L_w/K_v}(.)|^{1/[L_w/K_v]}=|.|_w\]
puis on veut que la formule du produit soit préservée sur $L$, i.e.
sur $L/K/E$ si $x\in K$, on a 
\[\prod_{w\mid u} ||x||_w= \prod_{v\mid u}\prod_{w\mid v}||x||_w=
\prod_{v\mid u} ||x||_v\]
la formule clée c'est
\[N_{L/K}(x)=\prod_{w\mid v} N_{L_w/K_v}(x)\qquad (*)\]
d'où 
\[\prod_{w\mid v}||x||_w=||x||_v\]
et 
\[\prod_{w\mid v}|x|_w^{\lambda_w}=|x|_v^{\lambda_v}\]
puis comme 
\[\prod_{w\mid v}|N_{L_w/K_v}(x)|_v^{\lambda_w/[L_w/K_v]}=
|x|_v^{\lambda_v}\]
on obtient $\sum_{w\mid v} \lambda_w=\lambda_v$ en prenant $x\in K$. 
Ensuite en général on a mieux, en essayant d'appliquer la formule $(*)$,
on peut poser $\lambda_w=\lambda_v[L_w/K_v]/[L:K]$. Ça donne
\[\prod_{w\mid v}|N_{L_w/K_v}(x)|_v^{\lambda_w/[L_w/K_v]}=
\prod_{w\mid v}|N_{L_w/K_v}(x)|_v^{\lambda_v/[L:K]}=|N_{L/K}(x)|_v^{\lambda_v/[L:K]}\]
et le dernier terme vaut $|x|_v$ si $x\in K$.


\section{Résumé}
On pose 
\[||x||_v=|x|_v\]
et sur $(L,w)-(K,v)$ :
\[||x||_w=||N_{L_w/K_v}(x)||_v^{1/[L:K]}\]
puis \[\lambda_w = \lambda_v [L_w:K_v]/[L:K].\]

\section{Norme de Gauss}
Le théorème de Northcott montre semble bien montrer un lien avec la norme
de Gauss. Simplement le fait que $||X-a||=max(1,||a||)=H([1,a])$.
C'était le sujet du td du jour.


\end{document}


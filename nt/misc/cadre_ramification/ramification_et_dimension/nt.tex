\documentclass[a4paper,12pt]{book}
\usepackage{amsmath,  amsthm,enumerate}
\usepackage{csquotes}
\usepackage[provide=*,french]{babel}
\usepackage[dvipsnames]{xcolor}
\usepackage{quiver, tikz}

%symbole caligraphique
\usepackage{mathrsfs}

%hyperliens
\usepackage{hyperref}

%pseudo-code
\usepackage{algpseudocode}
\usepackage{algorithm}
\makeatletter
  \renewcommand{\ALG@name}{Algorithme}
  \makeatother
\usepackage{fancyhdr}

\pagestyle{fancy}
\addtolength{\headwidth}{\marginparsep}
\addtolength{\headwidth}{\marginparwidth}
\renewcommand{\chaptermark}[1]{\markboth{#1}{}}
\renewcommand{\sectionmark}[1]{\markright{\thesection\ #1}}
\fancyhf{}
\fancyfoot[C]{\thepage}
\fancyhead[LO]{\textit \leftmark}
\fancyhead[RE]{\textit \rightmark}
\renewcommand{\headrulewidth}{0pt} % and the line
\fancypagestyle{plain}{%
    \fancyhead{} % get rid of headers
}

%bibliographie
\usepackage[
backend=biber,
style=alphabetic,
sorting=ynt
]{biblatex}

\addbibresource{bib.bib}

\usepackage{appendix}
\renewcommand{\appendixpagename}{Annexe}

\definecolor{wgrey}{RGB}{148, 38, 55}

\setlength\parindent{24pt}

\newcommand{\Z}{\mathbb{Z}}
\newcommand{\R}{\mathbb{R}}
\newcommand{\rel}{\omathcal{R}}
\newcommand{\Q}{\mathbb{Q}}
\newcommand{\C}{\mathbb{C}}
\newcommand{\N}{\mathbb{N}}
\newcommand{\K}{\mathbb{K}}
\newcommand{\A}{\mathbb{A}}
\newcommand{\B}{\mathcal{B}}
\newcommand{\Or}{\mathcal{O}}
\newcommand{\F}{\mathbb F}
\newcommand{\m}{\mathfrak m}
\renewcommand{\b}{\mathfrak b}
\renewcommand{\a}{\mathfrak a}
\newcommand{\p}{\mathfrak p}
\newcommand{\I}{\mathfrak I}
\newcommand{\Hom}{\textrm{Hom}}
\newcommand{\disc}{\textrm{disc}}
\newcommand{\Pic}{\textrm{Pic}}
\newcommand{\End}{\textrm{End}}
\newcommand{\Spec}{\textrm{Spec}}

\newcommand{\cL}{\mathscr{L}}
\newcommand{\G}{\mathscr{G}}
\newcommand{\D}{\mathscr{D}}
\newcommand{\E}{\mathscr{E}}

\theoremstyle{plain}
\newtheorem{thm}{Théoreme}
\newtheorem{lem}{Lemme}
\newtheorem{prop}{Proposition}
\newtheorem{cor}{Corollaire}
\newtheorem{heur}{Heuristique}
\newtheorem{rem}{Remarque}
\newtheorem{rembis}{Remarque}
\newtheorem{note}{Note}

\theoremstyle{definition}
\newtheorem{conj}{Conjecture}
\newtheorem*{eq}{Équivalences}
\newtheorem{prob}{Problème}
\newtheorem{quest}{Question}
\newtheorem{prot}{Protocole}
\newtheorem{algo}{Algorithme}
\newtheorem{defn}{Définition}
\newtheorem{defnbis}{Définition}
\newtheorem{ex}{Exemple}
\newtheorem{exo}{Exercices}

\theoremstyle{remark}

\definecolor{wgrey}{RGB}{148, 38, 55}
\definecolor{wgreen}{RGB}{100, 200,0} 
\hypersetup{
    colorlinks=true,
    linkcolor=wgreen,
    urlcolor=wgrey,
    filecolor=wgrey
}

\title{Quand est-ce que $\sum e_if_i=[L:K]$?}
\date{}

\begin{document}
\maketitle


\section{Manipuler}
Le point c'est qu'on a toujours
\[\sum e_if_i\leq [L:K]\]
(voir lemme 3.6 vu que dans le cas des dvrs on sait pas si
$f=\infty$)
avec égalité ssi $\tilde\Or_K-\Or_K$ est finie ssi 
$L\otimes_K \hat K$ est réduite. Un gros détail, les transitions
de dimensions se font entre $k_K-k_L$ et $K-L$. Autrement dit
on s'en fout de la finitude de $\tilde\Or_K$ sur $\Or_K$. Si
$[L:K]$ est finie alors $f_i$ aussi pour tout $i$.
\begin{rem}
    Cette histoire de $B_L$ réduite vient du fait que
    \[B_L\to \prod L_i\]
    a un noyau nilpotent vu que c'est $\cap \m_i$ et qu'on
    est sur une $\hat K$-algèbre de type fini à gauche (car
    de dimension finie $\to$) et à gauche c'est de dim
    $\dim_{\hat K}B_L=[L:K]$ alors qu'à droite c'est 
    $\sum e_if_i$.
\end{rem}

\section{Prérequis}
Quelques prérequis nécessaire à l'étude : si $\Or_K$ est de
Dedekind, quand est-ce que 
\begin{enumerate}
    \item $\tilde\Or_K$ est de Dedekind.
    \item $\tilde\Or_K$ est fini sur $\Or_K$.
\end{enumerate}
Pour la première question :
\begin{enumerate}
    \item Si $\Or_K$ est semi-local ca se fait bien parce que 
	$\tilde\Or_K$ est noethérien sur $\Or_K$ ssi
	$\tilde\Or_K\otimes\Or_K (\Or_K)_{\m_i}$ est noethérien
	pour tout les premiers (faut en avoir un nb fini).
    \item Plus généralement si $L/K$ est finie par Krull-Akizuki.
\end{enumerate}
Pour la deuxième : dès que $\sum e_if_i=[L:K]$ d'où si
\begin{enumerate}
    \item $K$ est complet, par densité de $\sum_{i,j} e_j\pi_L^i\Or_K$
	dans $\tilde\Or_K$.
    \item $L/K$ est séparable via le disriminant non nul et la trace
	non dégénérée.
    \item Évidemment si $\tilde\Or_K=\Or_K[\alpha]$ est monogène.
\end{enumerate}



\chapter{Cadre}
\section{Objets}
On se place \textbf{toujours} dans le cadre où on a $\Or_K$ de valuation 
\textbf{discrète}. Le cadre en gros c'est 
% https://q.uiver.app/#q=WzAsNixbMCwwLCJcXE9yX0siXSxbMSwwLCJcXHRpbGRlXFxPcl9LIl0sWzIsMCwiKFxcdGlsZGVcXE9yX0spX3tcXG1faX0iXSxbMywwLCJcXGxlZnQoXFxPcl9MXFxyaWdodCkiXSxbMCwxLCJrX0siXSxbMiwxLCJrX0wiXSxbMCwxLCIiLDAseyJzdHlsZSI6eyJoZWFkIjp7Im5hbWUiOiJub25lIn19fV0sWzEsMiwiXFxzdWJzZXRlcSIsMSx7InN0eWxlIjp7ImJvZHkiOnsibmFtZSI6Im5vbmUifSwiaGVhZCI6eyJuYW1lIjoibm9uZSJ9fX1dLFsyLDMsIj8iLDEseyJsZXZlbCI6Miwic3R5bGUiOnsiaGVhZCI6eyJuYW1lIjoibm9uZSJ9fX1dLFswLDRdLFsyLDVdLFs0LDVdXQ==
\[\begin{tikzcd}
	{\Or_K} & {\tilde\Or_K} & {(\tilde\Or_K)_{\m_i}} & {\left(\Or_L\right)} \\
	{k_K} && {k_L}
	\arrow[no head, from=1-1, to=1-2]
	\arrow[from=1-1, to=2-1]
	\arrow["\subseteq"{description}, draw=none, from=1-2, to=1-3]
	\arrow["{?}"{description}, Rightarrow, no head, from=1-3, to=1-4]
	\arrow[from=1-3, to=2-3]
	\arrow[from=2-1, to=2-3]
\end{tikzcd}\]
C'est à dire qu'on prends la clôture intégrale, on regarde ses idéaux
maximaux et on obtient des extensions de d.v.r. Quand $K$ est complet 
ou quand on fixe une valuation (un premier $\m_i$) sur $L$, $\Or_L$ fait
sens.

\section{Les cadres successifs}
On regarde d'abord $\Or_K-\Or_L$ une extension de DVR. De sorte
à montrer que 
\[e.f=\dim \Or_L/\m_K\Or_L\]
à l'aide du module $M$. Ensuite on regarde $\Or_K-\tilde\Or_K$.
Et on montre que 
\[\dim_{k_K} \tilde\Or_K/\m_K\tilde\Or_K\leq [L:K]\]
enfin on montre que dans le même cas que
\[ \dim_{k_K} \tilde\Or_K/\m_K\tilde\Or_K=\sum e_if_i \leq [L:K]\]
avec égalité quand (de manière équivalente)
\begin{enumerate}
    \item $\tilde\Or_K$ est fini sur $\Or_K$.
    \item $L\otimes_K \hat K$ est réduite.
\end{enumerate}

\section{Extensions de dvrs.}
Étant donné une extension $\Or_K-(\tilde\Or_K)_\m=\Or_L$, y'a une 
inclusion à regarder, si $k_K-k_L$ contient une famille libre et
génératrice $(e_i)_i$ :

\[\Or_L\subset \sum e_i\Or_K + \pi_L\Or_L\]
puis en itérant
\[\Or_L\subset \sum e_i\pi_L^j\Or_K + \pi_K\Or_L\]
et même pour tout $n\geq 1$ 
\[\Or_L\subset \sum_i\sum_{j=0,\ldots, e-1} e_i\pi_L^j\Or_K +
\pi_K^n\Or_L\]
car $\pi_L^e\in\Or_K$. Donc une densité de $M$ dans $\Or_L$. Je note 
\[M=\sum_{i=1,\ldots, f}\sum_{j=0,\ldots,e-1} e_i\pi_L^j\Or_K.\]

Ça montre que $\Or_L/\m_K\Or_L$ est de dimension au plus $e.f$.
L'autre est un peu technique mais pas dur, y s'agit de jouer
sur la valuation. 

\begin{rem}
    Là on a juste utilisé que $k_L$ est de dimension finie sur 
    $k_K$. On peut écrire des doubles inégalités même en général.
\end{rem}
On a construit un $\Or_K$-module libre dense dans $\Or_L$.


\section{Cas canonique}
On a directement 
$\dim_{k_K}\tilde\Or_K/\m_K\tilde\Or_K=\sum e_if_i$. Par le lemme
chinois et le cas des dvrs.
\section{Cas complet}
On se retrouve dans le cas des dvrs. Et on a 
\[M\otimes_{\Or_K} K=L\]
parce que dense dans $\Or_L\otimes_{\Or_K}K=L$ et complet donc fermé.
Donc on obtient le cas d'égalité $e.f\geq[L:K]$.
\begin{rem}
    Le fait que $M\otimes_{\Or_K}K$ soit un $K-$e.v dense
    dans $L$ force pas de même dimension ! Ça peut arriver
    qu'une ligne soit dense en dimension $2$. Par exemple
    $\Q$ dans $\Q(\sqrt 2)$ avec la norme infinie, ou toutes
    les $v_p$ avec $p\ne 2$ je crois.
\end{rem}



\section{Équivalences}
Seulement de "égalité" équivaut à $\tilde\Or_K$ fini sur $\Or_K$.
De droite à gauche c'est que $\Or_K$ est principal donc fini 
implique libre ici, la dimension se voit bien d'où l'égalité.
L'autre côté c'est que on obtient une base de $L$ sur $K$ et
on fait redescendre les relations.



\end{document}


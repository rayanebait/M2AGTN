\documentclass[a4paper,12pt]{book}
\usepackage{amsmath,  amsthm,enumerate}
\usepackage{csquotes}
\usepackage[provide=*,french]{babel}
\usepackage[dvipsnames]{xcolor}
\usepackage{quiver, tikz}

%symbole caligraphique
\usepackage{mathrsfs}

%hyperliens
\usepackage{hyperref}

%pseudo-code
\usepackage{algpseudocode}
\usepackage{algorithm}
\makeatletter
  \renewcommand{\ALG@name}{Algorithme}
  \makeatother
\usepackage{fancyhdr}

\pagestyle{fancy}
\addtolength{\headwidth}{\marginparsep}
\addtolength{\headwidth}{\marginparwidth}
\renewcommand{\chaptermark}[1]{\markboth{#1}{}}
\renewcommand{\sectionmark}[1]{\markright{\thesection\ #1}}
\fancyhf{}
\fancyfoot[C]{\thepage}
\fancyhead[LO]{\textit \leftmark}
\fancyhead[RE]{\textit \rightmark}
\renewcommand{\headrulewidth}{0pt} % and the line
\fancypagestyle{plain}{%
    \fancyhead{} % get rid of headers
}

%bibliographie
\usepackage[
backend=biber,
style=alphabetic,
sorting=ynt
]{biblatex}

\addbibresource{bib.bib}

\usepackage{appendix}
\renewcommand{\appendixpagename}{Annexe}

\definecolor{wgrey}{RGB}{148, 38, 55}

\setlength\parindent{24pt}

\newcommand{\Z}{\mathbb{Z}}
\newcommand{\R}{\mathbb{R}}
\newcommand{\rel}{\omathcal{R}}
\newcommand{\Q}{\mathbb{Q}}
\newcommand{\C}{\mathbb{C}}
\newcommand{\N}{\mathbb{N}}
\newcommand{\K}{\mathbb{K}}
\newcommand{\A}{\mathbb{A}}
\newcommand{\B}{\mathcal{B}}
\newcommand{\Or}{\mathcal{O}}
\newcommand{\F}{\mathbb F}
\newcommand{\m}{\mathfrak m}
\renewcommand{\b}{\mathfrak b}
\renewcommand{\a}{\mathfrak a}
\newcommand{\p}{\mathfrak p}
\newcommand{\I}{\mathfrak I}
\newcommand{\Hom}{\textrm{Hom}}
\newcommand{\disc}{\textrm{disc}}
\newcommand{\Pic}{\textrm{Pic}}
\newcommand{\End}{\textrm{End}}
\newcommand{\Spec}{\textrm{Spec}}
\newcommand{\Frac}{\textrm{Frac}}

\newcommand{\cL}{\mathscr{L}}
\newcommand{\G}{\mathscr{G}}
\newcommand{\D}{\mathscr{D}}
\newcommand{\E}{\mathscr{E}}

\theoremstyle{plain}
\newtheorem{thm}{Théoreme}
\newtheorem{lem}{Lemme}
\newtheorem{prop}{Proposition}
\newtheorem{cor}{Corollaire}
\newtheorem{heur}{Heuristique}
\newtheorem{rem}{Remarque}
\newtheorem{rembis}{Remarque}
\newtheorem{note}{Note}

\theoremstyle{definition}
\newtheorem{conj}{Conjecture}
\newtheorem*{eq}{Équivalences}
\newtheorem{prob}{Problème}
\newtheorem{quest}{Question}
\newtheorem{prot}{Protocole}
\newtheorem{algo}{Algorithme}
\newtheorem{defn}{Définition}
\newtheorem{defnbis}{Définition}
\newtheorem{ex}{Exemple}
\newtheorem{exo}{Exercices}

\theoremstyle{remark}

\definecolor{wgrey}{RGB}{148, 38, 55}
\definecolor{wgreen}{RGB}{100, 200,0} 
\hypersetup{
    colorlinks=true,
    linkcolor=wgreen,
    urlcolor=wgrey,
    filecolor=wgrey
}

\title{Extensions de valuation}
\date{}

\begin{document}
\maketitle


Ici, on s'en fout de savoir si $\tilde \Or_K$ est fini
sur $\Or_K$.

\chapter{Cas archimédien}
On se place donc en caractéristique $0$ et sur $K=\Q$. Parce
que un corps archimédien algébriquement clos complet est isométrique
à $\C$.

\section{Cas complet}
On est soit $\C$ soit $\R$. 

\section{En général}
On a $r_1+r_2$ extensions de $|.|_\infty$ de $\Q$ à $L$.
Avec $r_1+2r_2=[L:\Q]$ et les complétés pour $r_1$ c'est $\R$.
Pour $r_2$ c'est $\C$.


\chapter{Extension via les complétions : cas ultramétrique}
\section{Unicité sur $(\hat K)^c$}
Étant donné un corps complet $K$, les extensions
$L$ de $K$ sont des corps complets et les normes
sont équivalentes.
\subsection{Extension pour les corps complets}
Ce truc 
\[x\mapsto |N_{L/K}(x)|^{1/[L:K]}\]
est une valeur absolue qui étend $|.|$ donc l'unique.
\subsection{Détails}
L'équivalence de normes force l'équivalence de valeur
absolues (passer par la topologie!!).
\subsection{Passer à la clôture algébrique}
D'une v.a sur $K^c$ suffit de restreindre à $L/K$
on garde une v.a donc l'unique!

\section{Extensions en général par plongements}

Y'a deux manières de faire. Soit regarder
$K$ dans $\hat L$. Soit regarder $L$ dans
$(\hat K)^c$. La première à l'avantage de 
sous-entendre la valeur absolue.
\subsection{Deuxième manière}
On regarde $K^c$ dans $(\hat K)^c$.
Alors si $|.|_L$ étend $|.|$ et
\[\tau \colon L\to K^c\] est un plongement.
On peut étendre $\tau$ à $\hat K.\tau(L)$ un
corps complet sur $\hat K$ de dimension finie.
En plus la valeur absolue s'étend aussi 
directement et étend bien celle de $\hat K$
par limite. En particulier, pour $x\in L$
$|\tau(x)|_{\hat K.\tau(L)}=|\tau(x)|_c$ est
uniquement determinée par $\tau$. À l'inverse
n'importe quel plongement $L\to K^c$ fournit
une valeur absolue.

\section{Extensions en général via les idéaux
maximaux de $L\otimes_K\hat K$ (topologie)}
On note $B_L=L\otimes_K \hat K$. En tant qu'espace
vectoriel c'est de dimension 
\[[L:K]\]
le produit est sur $K$. C'est un Banach et on a 
des flèches
\[L\to B_L\]
via $x\mapsto x\otimes 1$ et pour tout corps complet
$L_i$ où $L$ est dense dans 
$(\hat K)^C$ :
\[B_L\to L_i\]
via $x\otimes y\mapsto xy$. À gauche ça a une structure
d'algèbre et donc le noyau est un idéal maximal si c'est
non nul. C'est clairement non nul par la dimension.

\subsection{Interprétation : pourquoi a pas $L_i\to B_L$}
La norme $|.|_i$ sur $L_i$ mesure que la partie associée
à $\m_i$ de $L$ sur $K$. Tandis que la norme naturelle sur
$L\otimes_K \hat K$ est vraiment la norme terme à terme
sur $K$. En gros, si par exemple $L/K$ est totalement
décomposée au dessus de $|.|$ et $(x_i)$ est une base
normale. Alors $|.|_i$ force $|x_j|_i=|x_i|_i$ avec $j\ne i$.
Tandis qu'une norme sur $B_L$ ne mesure pas les $x_i$. 
Seulement via $\hat K$.



\section{Calcul par la deuxième version}
Comment savoir combien on a d'extension ? Il faut calculer
$B_L$ explicitement ! Par exemple :
\[\Q_p\otimes_\Q \Q(\sqrt 2)\simeq \Q_p[X]/(X^2-2)\]
et $\begin{pmatrix} 2\\ p\end{pmatrix}=1 $ si et 
seulement si $p^2\equiv 1\mod 8$. Donc $|.|$ a une
ou deux extensions en fonction de $p$ via Hensel.

\chapter{Extension par les anneaux de valuations discrètes}
Étant donné un corps $(K,v)$ de valuation discrète. Une 
extension $(L,w)$ finie de $v$. L'anneau $\tilde\Or_K=\tilde\Or_v$
est semi-local sur $\Or_K=\Or_v$. Le but c'est de dire
que $\Or_w=(\tilde\Or_K)_\m$ pour un idéal max de $\Or_K$
et que chaque idéal maximal de $\tilde\Or_K$ correspond
à une valuation.

Ici, on s'en fout de savoir si $\tilde \Or_K$ est fini
sur $\Or_K$.
\section{Extensions par les idéaux maximaux}
Étant donné seulement $L-K$ finie. un ideal maximal
$\m$ de $\tilde\Or_K$ fournit un dvr $(\tilde\Or_K)_\m$
et une extension de dvrs. Alors
\[w_\m:=v_\m/v(\pi_K)\]
étend $v$ et est une valuation discrète sur $L$. C'est
clair que différents idéaux fournissent des valuations
différentes.
\section{Idéal maximal associé à une valuation}
À l'inverse si $w$ étend $v$. Alors $\Or_w$ contient
$\tilde\Or_K(w)$ en étudiant la valuation d'une relation.
Ensuite faut montrer que 
\[(\tilde\Or_K)_{\m_w\cap \tilde\Or_K}=\Or_w\]
À faire c'est pas dur.




\chapter{À éclaircir}
Bon bah simplement le calcul explicit de $B_L$ en
général! I.e. si $L/K$ est monogène, $B_L$ est un
produit de corps ou chaque corps 







\[\textrm{}\]
\end{document}


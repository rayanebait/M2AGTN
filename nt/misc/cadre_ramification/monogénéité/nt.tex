\documentclass[a4paper,12pt]{book}
\usepackage{amsmath,  amsthm,enumerate}
\usepackage{csquotes}
\usepackage[provide=*,french]{babel}
\usepackage[dvipsnames]{xcolor}
\usepackage{quiver, tikz}

%symbole caligraphique
\usepackage{mathrsfs}

%hyperliens
\usepackage{hyperref}

%pseudo-code
\usepackage{algpseudocode}
\usepackage{algorithm}
\makeatletter
  \renewcommand{\ALG@name}{Algorithme}
  \makeatother
\usepackage{fancyhdr}

\pagestyle{fancy}
\addtolength{\headwidth}{\marginparsep}
\addtolength{\headwidth}{\marginparwidth}
\renewcommand{\chaptermark}[1]{\markboth{#1}{}}
\renewcommand{\sectionmark}[1]{\markright{\thesection\ #1}}
\fancyhf{}
\fancyfoot[C]{\thepage}
\fancyhead[LO]{\textit \leftmark}
\fancyhead[RE]{\textit \rightmark}
\renewcommand{\headrulewidth}{0pt} % and the line
\fancypagestyle{plain}{%
    \fancyhead{} % get rid of headers
}

%bibliographie
\usepackage[
backend=biber,
style=alphabetic,
sorting=ynt
]{biblatex}

\addbibresource{bib.bib}

\usepackage{appendix}
\renewcommand{\appendixpagename}{Annexe}

\definecolor{wgrey}{RGB}{148, 38, 55}

\setlength\parindent{24pt}

\newcommand{\Z}{\mathbb{Z}}
\newcommand{\R}{\mathbb{R}}
\newcommand{\rel}{\omathcal{R}}
\newcommand{\Q}{\mathbb{Q}}
\newcommand{\C}{\mathbb{C}}
\newcommand{\N}{\mathbb{N}}
\newcommand{\K}{\mathbb{K}}
\newcommand{\A}{\mathbb{A}}
\newcommand{\B}{\mathcal{B}}
\newcommand{\Or}{\mathcal{O}}
\newcommand{\F}{\mathbb F}
\newcommand{\m}{\mathfrak m}
\renewcommand{\b}{\mathfrak b}
\renewcommand{\a}{\mathfrak a}
\newcommand{\p}{\mathfrak p}
\newcommand{\I}{\mathfrak I}
\newcommand{\Hom}{\textrm{Hom}}
\newcommand{\disc}{\textrm{disc}}
\newcommand{\Pic}{\textrm{Pic}}
\newcommand{\End}{\textrm{End}}
\newcommand{\Spec}{\textrm{Spec}}

\newcommand{\cL}{\mathscr{L}}
\newcommand{\G}{\mathscr{G}}
\newcommand{\D}{\mathscr{D}}
\newcommand{\E}{\mathscr{E}}

\theoremstyle{plain}
\newtheorem{thm}{Théoreme}
\newtheorem{lem}{Lemme}
\newtheorem{prop}{Proposition}
\newtheorem{cor}{Corollaire}
\newtheorem{heur}{Heuristique}
\newtheorem{rem}{Remarque}
\newtheorem{rembis}{Remarque}
\newtheorem{note}{Note}

\theoremstyle{definition}
\newtheorem{conj}{Conjecture}
\newtheorem*{eq}{Équivalences}
\newtheorem{prob}{Problème}
\newtheorem{quest}{Question}
\newtheorem{prot}{Protocole}
\newtheorem{algo}{Algorithme}
\newtheorem{defn}{Définition}
\newtheorem{defnbis}{Définition}
\newtheorem{ex}{Exemple}
\newtheorem{exo}{Exercices}

\theoremstyle{remark}

\definecolor{wgrey}{RGB}{148, 38, 55}
\definecolor{wgreen}{RGB}{100, 200,0} 
\hypersetup{
    colorlinks=true,
    linkcolor=wgreen,
    urlcolor=wgrey,
    filecolor=wgrey
}

\title{Anneaux d'entiers monogènes et ramification}
\date{}

\begin{document}
\maketitle


\section{Prérequis}
Quelques prérequis nécessaire à l'étude : si $\Or_K$ est de Dedekind,
quand est-ce que 
\begin{enumerate}
    \item $\tilde\Or_K$ est de Dedekind.
    \item $\tilde\Or_K$ est fini sur $\Or_K$.
\end{enumerate}
Pour la première question :
\begin{enumerate}
    \item Si $\Or_K$ est semi-local ca se fait bien parce que 
	$\tilde\Or_K$ est noethérien sur $\Or_K$ ssi
	$\tilde\Or_K\otimes\Or_K =(\Or_K)_{\m_i}$ est noethérien
	pour tout les premiers (faut en avoir un nb fini).
    \item Plus généralement si $L/K$ est finie par Krull-Akizuki.
\end{enumerate}
Pour la deuxième : dès que $\sum e_if_i=[L:K]$ d'où si
\begin{enumerate}
    \item $K$ est complet, par densité de $\sum_{i,j} e_j\pi_L^i\Or_K$
	dans $\tilde\Or_K$.
    \item $L/K$ est séparable via le disriminant non nul et la trace
	non dégénérée.
    \item Évidemment si $\tilde\Or_K=\Or_K[\alpha]$ est monogène.
\end{enumerate}



\chapter{Cadre}
\section{Objets}
On se place \textbf{toujours} dans le cadre où on a $\Or_K$ de valuation 
\textbf{discrète}. Le cadre en gros c'est 
% https://q.uiver.app/#q=WzAsNixbMCwwLCJcXE9yX0siXSxbMSwwLCJcXHRpbGRlXFxPcl9LIl0sWzIsMCwiKFxcdGlsZGVcXE9yX0spX3tcXG1faX0iXSxbMywwLCJcXGxlZnQoXFxPcl9MXFxyaWdodCkiXSxbMCwxLCJrX0siXSxbMiwxLCJrX0wiXSxbMCwxLCIiLDAseyJzdHlsZSI6eyJoZWFkIjp7Im5hbWUiOiJub25lIn19fV0sWzEsMiwiXFxzdWJzZXRlcSIsMSx7InN0eWxlIjp7ImJvZHkiOnsibmFtZSI6Im5vbmUifSwiaGVhZCI6eyJuYW1lIjoibm9uZSJ9fX1dLFsyLDMsIj8iLDEseyJsZXZlbCI6Miwic3R5bGUiOnsiaGVhZCI6eyJuYW1lIjoibm9uZSJ9fX1dLFswLDRdLFsyLDVdLFs0LDVdXQ==
\[\begin{tikzcd}
	{\Or_K} & {\tilde\Or_K} & {(\tilde\Or_K)_{\m_i}} & {\left(\Or_L\right)} \\
	{k_K} && {k_L}
	\arrow[no head, from=1-1, to=1-2]
	\arrow[from=1-1, to=2-1]
	\arrow["\subseteq"{description}, draw=none, from=1-2, to=1-3]
	\arrow["{?}"{description}, Rightarrow, no head, from=1-3, to=1-4]
	\arrow[from=1-3, to=2-3]
	\arrow[from=2-1, to=2-3]
\end{tikzcd}\]
C'est à dire qu'on prends la clôture intégrale, on regarde ses idéaux
maximaux et on obtient des extensions de d.v.r. Quand $K$ est complet 
ou quand on fixe une valuation (un premier $\m_i$) sur $L$, $\Or_L$ fait
sens.
\section{Pourquoi on cherche des extensions monogènes}
    Pour calculer en fait une marche à suivre c'est
\begin{center}
    \textit{On sait le faire dans $\Or_K[\alpha]$.}
\end{center}
Si c'est le cas alors :
\begin{enumerate}
    \item La factorisation de $P$ 
	dans $k_K[X]$ donne la ramification
	et les idéaux maximaux de $\tilde\Or_K$!
    \item Plus précisément, si 
\[\bar P=\prod_i p_i^{r_i}\in k_K[X]\]
	alors $\m_i=(\m_K,p_i(\alpha))$.
\end{enumerate}
Le point \textbf{important} c'est la ramification, on relève
\[P(\alpha)=\prod_iP_i^{r_i}(\alpha) + \epsilon(\alpha)\]
ce qui donne par le deuxième point 
\[\prod_i \m_i^{r_i}=\prod_i (\m_K, P_i(\alpha))^{r_i}\subset \m_K\tilde\Or_K=\prod_i \m_i^{e_i}\]
On en déduit $r_i\geq e_i$ pour tout $i$ et on conclut directement
avec \[\sum r_i f_i = \deg \bar P=\deg P= [L:K]=\sum e_if_i\]
On a utilisé que $\tilde\Or_K$ est fini sur $\Or_K$ pour l'égalité
$\deg \bar P=\deg P$ \textbf{et} la dimension $[L:K]=\sum e_i f_i$.

\chapter{Des cas où on sait que c'est monogène}
D'abord un cadre intéressant. 
\section{Le cas canonique}
Étant donné une extension $\Or_K-(\tilde\Or_K)_\m=\Or_L$, y'a une 
inclusion à regarder, si $k_K-k_L$ contient une famille libre et
génératrice $(e_i)_i$ :

\[\Or_L\subset \sum e_i\Or_K + \pi_L\Or_L\]
puis en itérant
\[\Or_L\subset \sum e_i\pi_L^j\Or_K + \pi_K\Or_L\]
et même pour tout $n\geq 1$ 
\[\Or_L\subset \sum_i\sum_{j=0,\ldots, e-1} e_i\pi_L^j\Or_K +
\pi_K^n\Or_L\]
car $\pi_L^e\in\Or_K$. Donc une densité de $M$ dans $\Or_L$. Je note 
\[M=\sum_{i=1,\ldots, f}\sum_{j=0,\ldots,e-1} e_i\pi_L^j\Or_K.\]

En gros, dès que $L=K[\alpha]$, par exemple :
Il suffit de l'extension résiduel soit séparable, i.e. 
\[\textrm{$\bar P$ est séparable.}\]

\begin{rem}
    Là on a juste utilisé que $k_L$ est de dimension finie sur $k_K$.
\end{rem}

\section{Cas primitif, $L=K[\alpha]$}

On suppose $L=K(\alpha)$. Un premier critère de monogénéité :
\[\textrm{Si $\bar P$ est séparable, alors $\tilde\Or_K=\Or_K[\alpha]$}\]
Une preuve rapide c'est que \[\Or_K[\alpha]_{\pi_K}=K[\alpha]=L\] est 
intégralement
clos et \[\Or_K[\alpha]/(\pi_K)=k[\alpha]\] est réduit. D'où
\[\Or_K[\alpha]\]
est intégralement clos et dans $\tilde\Or_K$. 
\begin{rem}
    On peut étudier les nilpotents du quotient quand c'est pas réduit.
\end{rem}
Ce cas arrive dans les cas où 
\begin{itemize}
    \item On est en caractéristique $0$, car séparable donc primitif.
    \item Dans le cas non ramifié complet.
    \item Dans le cas modérément ramifié complet.
\end{itemize}
Si on veut on peut aussi utiliser le cas canonique pour le prouver en
montrant que $\pi_L=P(\alpha).u$ pour un $P$ bien choisi, je le montre
en section sur la ramification modérée.

\section{Cas non ramifié complet}
On a une équivalence entre :
\begin{enumerate}
    \item L'extension $L/K$ est non ramifiée (par déf non ramifiée et 
$k_{K(\alpha)}/k_K$ est séparable).
    \item Il existe $\alpha$ : $L=K(\alpha)$ et $P$ le pol min 
de $\alpha$ sur $K$ est séparable sur $k_K$.
\end{enumerate}
L'idée c'est juste que la formule $ef=[L:K]$ est vraie. Et on peut 
relever une base de l'extension résiduelle !
En gros ça donne une réciproque à la section d'avant. I.e. si 
$L/K$ est non ramifiée alors 
\[\textrm{$L=K[\alpha]$, $\bar P$ est séparable d'où $\Or_K[\alpha]=\tilde\Or_K$}\]

En caractéristique $0$, $L/K$ finie implique séparable implique 
$L$ est monogène sur $K$.

\section{Cas $p$-adique}
Dans le cas $p$-adique, les corps finis sont parfaits et on a toujours
des extensions séparables (c'est immédiat de la déf)! En particulier,
si $\bar P$ est inséparable c'est qu'il est scindé. Ça se voit bien
par Hensel :
\begin{enumerate}
    \item On a toujours $\bar P=F^d$ et en réécrivant $d\deg F=
	\deg P=e.f$ sachant que $\deg F\mid f$ (à vérifier mais ça se
	voit) on obtient $e\mid d$. (l'égalité c'est qu'on suppose $P$ 
	unitaire)
\end{enumerate}

En fait on a beaucoup mieux mais j'en parlerai dans une autre note.

\section{Cas où $k_L$ séparable sur $k_K$}
On regarde $L/K$ finie. Si
\begin{enumerate}
    \item les $\Or_L/\m_L=k_L/k_K$ est séparable
\end{enumerate}
alors \[(Or_L)_{\m_L}/\m_K\Or_L\] 
est monogène sur $k_K$. On peut réécrire $M$, via $e_i\in \Or_K[\alpha]$,
d'où \[M\subset\sum \Or_K[\alpha]\pi_L^j\]
alors y suffit de montrer que $\pi_L\in \Or_K[\alpha]$. Si
\[k_L=k_K(\bar\alpha)\]
et $\bar P$ le pol min de $\bar\alpha$ alors $(P(\alpha))=\m_L$ ou 
$(P(\alpha+\pi_L))=\m_L$. Via $\bar P$ est séparable d'où 
$\bar P'\ne 0$ :
\[P(\alpha+\pi_L)-P(\alpha)=\pi_L(P'(\alpha)+\pi_L\beta)\]
d'où 
\[k(\bar \alpha)=(\tilde\Or_K)_{\m_i}/\m_K(\tilde\Or_K)_{\m_i}.\] 
\begin{rem}
    On obtient un début preuve du cas primitif.
\end{rem}
\begin{rem}
    On a pas supposé que c'est non ramifié ni que $\tilde\Or_K$ est fini
    sur $\Or_K$.
\end{rem}
\section{Cas modérément ramifié général}
Dans le cas $\Or_K-\tilde\Or_K$, si on part du principe que
\begin{enumerate}
    \item Les $k_K-k_i$ sont séparables.
    \item $\prod k_i=\prod \tilde\Or_K/\m_i\tilde\Or_K$ est 
	monogène sur $k$. (Ça se déduit du 1. dans plusieurs cas)
    \item $\tilde\Or_K$ est fini sur $\Or_K$.
\end{enumerate}
\begin{rem}
    Pour $2.$ c'est $2.$ puis $3.$ ($k$ fini, $card(k)>max(n,3)$
    avec $n$ le nombre d'idéaux) ou $1.$ ($k$ infini) du
    lemme 3.18.
\end{rem}
Donc on utilise que un produit d'algèbres monogènes finies est 
monogène fini ssi sa version réduite l'est (2. du lemme 3.18)
d'où
\[\tilde\Or_K/\m_K\tilde\Or_K\]
est monogène sur $k$ (CRT) si le produit de corps l'est disons
\[\tilde\Or_K/\m_K\tilde\Or_K=k_K(\theta)\]
Maintenant si
\begin{enumerate}
    \item[2.] $\tilde\Or_K$ est fini sur $\Or_K$
\end{enumerate}
alors il est monogène via Nakayama parce que
$\tilde\Or_K\subset \Or_K[\theta] +\m_K\tilde\Or_K$.
\begin{defn}
    Petit rappel de définition, $L/K$ est non ramifiée si toutes
    les extensions de DVRS ont $e\wedge char(k_K)=1$ et 
    $k_K-k_{L_i}$ est séparable.
\end{defn}
\begin{rem}
    Donc ca a pas exactement attrait à la ramification? Je vois
    pas où on utilise tame et pas juste que les extensions 
    résiduelles sont séparables. Après relecture j'ai vraiment
    l'impression que ça se précise.
\end{rem}
Donc maintenant si $L/K$ est modérément ramifiée, d'où les
extensions résiduelles sont séparables (ça c'est vraiment
nécessaire pour $(P(\alpha))=\m_L$) d'où $\prod k_{L_i}$ est
monogène dans les cas adéquats, par exemple
\begin{enumerate}
    \item $\tilde \Or_K$ local, par exemple cas totalement ramifié.
	($\tilde\Or_K$ local par déf et $e=[L:K]$)
    \item Cas où $k_K$ est infini.
\end{enumerate}
\begin{rem}
    Quasi-sur que les conditions sont loin d'être optimales.
\end{rme}

\section{Cas modérément ramifié complet}
Dans le cas $L/K$ est modérément ramifiée, si 
$K$ est complet alors $\tilde\Or_K=\Or_L$ est local et on a 
l'hypothèse, si $k_K$ est infini alors $\prod k_i$ est monogène
et on a l'hypothèse.

\section{Cas totalement ramifié}
\section{Cas totalement ramifié complet}

\section{Corps de fonctions}



\end{document}

